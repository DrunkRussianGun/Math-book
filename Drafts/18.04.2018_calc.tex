\chapter{18.04.2018, математический анализ}
\subsection{Уравнение Бернулли}
\textbf{Уравнением Бернулли} называется уравнение вида $y' = a(x) y + b(x) y^n$, где~$n \neq 1$.
Пусть $\frac1{y^{n-1}} = z$, тогда
\begin{equation*}
y' = a(x) y + b(x) y^n \Leftrightarrow
\frac{y'}{y^n} = a(x) y^{1-n} + b(x) \Leftrightarrow
\frac{z'}{1 - n} = a(x) z + b(x)
\end{equation*}

Т.\,о., решение уравнения Бернулли сводится к решению линейного уравнения.

\begin{theorem}
Пусть $y'(x) = f(x, y(x)) \lAnd y(x_0) = y_0$, причём в некоторой окрестности $\exists M > 0 \colon |f(x, y_1) - f(x, y_2)| \leqslant M |y_1 - y_2|$, тогда уравнение имеет единственное решение в окрестности~$(x_0 - d; x_0 + d)$: $y(x) = \int\limits_{x_0}^x f(x, y(x))\,dx + y_0$.
\end{theorem}

Это уравнение можно решить методом итераций:\\
$y_1(x) = \int\limits_{x_0}^x f(x, y_0)\,dx + y_0$\\
$y_n(x) = \int\limits_{x_0}^x f(x, y_{n-1}(x))\,dx + y_0$
\begin{proof}
\begin{equation*}
y_n(x) - y_{n-1}(x) = \int\limits_{x_0}^x (f(x, y_{n-1}(x)) - f(x, y_{n-2}(x)))\,dx
\end{equation*}

$|\int\limits_a^b g(x)\,dx| \leqslant |b - a| \max_{x \in [a; b]} |g(x)|$, тогда
\begin{equation*}
\max_{|x - x_0| < d} |y_n(x) - y_{n-1}(x)| \leqslant
d \max_{|x - x_0| < d} |f(x, y_{n-1}(x)) - f(x, y_{n-2}(x))| \leqslant
d M \max_{|x - x_0| < d} |y_{n-1}(x) - y_{n-2}(x)|
\end{equation*}

Пусть $q = dM < 1$, тогда
\begin{equation*}
\max_{|x - x_0| < d} |y_n(x) - y_{n-1}(x)| \leqslant
q \max_{|x - x_0| < d} |y_{n-1}(x) - y_{n-2}(x)| \leqslant
q^{n-1} \max_{|x - x_0| < d} |y_1(x) - y_0|
\end{equation*}

Тогда
\begin{equation*}
\forall k \geqslant 1 \
\max_{x \in [a; b]} |y_{n+k}(x) - y_n(x)| =
\max_{|x - x_0| < d} |(y_{n+k}(x) - y_{n+k-1}(x)) + (y_{n+k-1}(x) + y_{n+k-2}(x)) + \ldots + (y_{n+1}(x) - y_n(x))| \leqslant
\max_{|x - x_0| < d} |y_{n+k}(x) - y_{n+k-1}(x)| +
\max_{|x - x_0| < d} |y_{n+k-1}(x) - y_{n+k-2}(x)| + \ldots +
\max_{|x - x_0| < d} |y_{n+1}(x) - y_n(x)| \leqslant
(q^{n+k-1} + q^{n+k-2} + \ldots + q^n) \max_{|x - x_0| < d} |y_1(x) - y_0| =
q^n \cdot \frac{1 - q^k}{1 - q} < \frac{q^n}{1 - q}
\end{equation*}

По признаку Коши получим
\begin{equation*}
\forall \varepsilon > 0 \ \exists N \in \mathbb N \colon \forall n > N \ \forall k \geqslant 1 \
\max_{x \in [a; b]} |y_{n+k}(x) - y_n(x)| < \varepsilon \Rightarrow
\exists \lim_{n \to \infty} y_n(x) = \tilde y(x)
\end{equation*}

Докажем единственность:
\begin{equation*}
|y(x) - \tilde y(x)| =
|\int_{x_0}^x (f(x, y(x)) - f(x, \tilde y(x)))\,dx| \Rightarrow
\max_{|x - x_0| < d} |y(x) - \tilde y(x)| \leqslant
|x - x_0| \cdot \max_{|x - x_0| < d} |f(x, y(x)) - f(x, \tilde y(x))| <
d M \max_{|x - x_0| < d} |y(x) - \tilde y(x)| \Rightarrow
\max_{|x - x_0| < d} |y(x) - \tilde y(x)| < q \max_{|x - x_0| < d} |y(x) - \tilde y(x)| \Rightarrow
(1 - q) \max_{|x - x_0| < d} |y(x) - \tilde y(x)| < 0
\end{equation*}

Противоречие, значит, решение единственно.
\end{proof}