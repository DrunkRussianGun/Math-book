\chapter{27.04.2018, дискретная математика}
\section{Автоматы с магазинной памятью}
\textbf{Автоматом с магазинной памятью} называется набор~$(S, X, \Gamma, \delta, s_1, Z_0, F)$, где\\
$S$~--- конечное множество состояний;\\
$X$~--- алфавит;\\
$\Gamma$~--- конечное множество магазинных символов;\\
$\delta \colon S \times (X \cup \{ \lambda \}) \times \Gamma \to 2^S \times \Gamma^*$~--- функция перехода;\\
$s_1$~--- начальное состояние;\\
$Z_0$~--- начальный магазинный символ \textbf{(маркер дна)};\\
$F$~--- множество допускающих состояний.

$L(A) = \{ \alpha \in X^* \mid \delta(s_1, \alpha, Z_0) = (s_f, \gamma), \ s_f \in F, \gamma \in \Gamma^* \}$

$N(A) = \{ \alpha \in X^* \mid \delta(s_1, \alpha, Z_0) = (s, \lambda, \lambda), \ s \in S \}$

Если $\delta(s, a \alpha, X) = (s', \xi)$, то $(s, a \alpha, x n) \Rightarrow (s', \alpha, \xi n)$.