\chapter{25.05.2018, дискретная математика}
\section{Недетерминированная машина Тьюринга}
\begin{statement}[тезис Чёрча]
Всё, что можно сделать алгоритмом, можно сделать и на машине Тьюринга.
\end{statement}

\begin{statement}
Языки, распознаваемые недетерминированной машиной Тьюринга, распознаются детерминированной машиной Тьюринга.
\end{statement}
\begin{proof}
При каждом переходе в недетерминированной машине Тьюринга получаются сразу несколько конфигураций.
В детерминированной машине Тьюринга эти конфигурации можно записывать друг за другом.
\end{proof}

\textbf{Временем работы на входных данных размера~$n$} называется количество шагов, необходимое для получения ответа, и обозначается~$T(n)$.

Если время решения задачи на детерминированной машине Тьюринга выражается полиномом, то её относят к \textbf{классу P-задач}, обозначаемому~$P$.

Если время решения задачи на недетерминированной машине Тьюринга выражается полиномом, то её относят к \textbf{классу NP-задач}, обозначаемому~$NP$.

Если решение задачи~$z_1$ сводится к решению задачи~$z_2$, то говорят, что \textbf{$z_1$ сводится к $z_2$}, и обозначают $z_1 \leqslant z_2$.

Задача~$z$ называется \textbf{NP-полной}, если $z \in NP \lAnd \forall z' \in NP \ z' \leqslant z$.