\chapter{14.03.2018, математический анализ}
\begin{theorem}[интегральный признак Коши]
Пусть $\series a_k$~--- знакоположительный ряд.
Если существует монотонная функция~$f(x) colon f(n) = a_n \lAnd \lim\limits_{x \to +\infty} f(x) = 0$, то $\series a_k$ сходится $\Leftrightarrow$ $\exists \int\limits_1^{+\infty} f(x)\,dx$.
\end{theorem}
\begin{proof}
Площадь заштрихованной фигуры равна $1 a_1 + 1 a_2 + \ldots + 1 a_n$, а криволинейной трапеции~--- $\int\limits_1^{n+1} f(x)\,dx$, тогда
$S_n > \int\limits_1^{n+1} f(x)\,dx$, $S_{n+1} - a_1 < \int\limits_1^{n+1} f(x)\,dx$.
\begin{enumerate}
	\item $\Leftarrow$. $\forall n \in \mathbb N \ S_n < a_1 + \int\limits_1^{n+1} f(x)\,dx < a_1 + \int\limits_1^{+\infty} f(x)\,dx$ $\Rightarrow$ $\series a_k$ сходится.
	\item $\Rightarrow$. $\forall n \in \mathbb N \ \int_1^{n+1} f(x)\,dx < S_n < S \Rightarrow \exists \int\limits_1^{+\infty} f(x)\,dx$.
\end{enumerate}
\end{proof}

\subsection{Знакопеременные ряды}
\index{Ряд!знакопеременный}

\begin{theorem}
Если $\series |a_k|$ сходится, то $\series a_k$ тоже сходится.
\end{theorem}
\begin{proof}
Пусть $(S_n)$ и $(\sigma_n)$~--- частичные суммы рядов $\series a_k$ и $\series |a_k|$ соответственно.
По \hyperref[th:Cauchy_criterion]{критерию Коши}
\begin{equation*}
\forall \varepsilon > 0 \ \exists N \in \mathbb N \colon \forall m > N \ \forall k \geqslant 1 \ |\sigma_{m+k} - \sigma_m| < \varepsilon
\end{equation*}

Тогда
\begin{equation*}
|S_{m+k} - S_m| = |\sum_{i=1}^k a_{m+i}| \leqslant \sum_{i=1}^k |a_{m+i}| = |\sigma_{m+k} - \sigma_m| < \varepsilon
\end{equation*}

Значит, $\series a_k$ сходится.
\end{proof}

\begin{theorem}[признак Лейбница]
Пусть $\series (-1)^{k-1} a_k$~--- знакочередующийся ряд.
Если $\lim\limits_{n \to \infty} a_n = 0$, причём $(a_n)$~--- монотонная последовательность, то ряд сходится.
\end{theorem}
\begin{proof}
\begin{equation*}
S_{2n} = (a_1 - a_2) + (a_3 - a_4) + \ldots + (a_{2n-1} - a_{2n}) < S_{2n+2}
\end{equation*}
\begin{equation*}
S_{2n} = a_1 - (a_2 - a_3) - (a_4 - a_5) - \ldots - (a_{2n-2} - a_{2n-1}) - a_{2n} < a_1
\end{equation*}

Тогда по свойству~\ref{st:monotonic_bounded_sequence} предела последовательности
\begin{equation*}
\lim_{n \to \infty} S_{2n} = S \Rightarrow
\lim_{n \to \infty} S_{2n+1} =
\lim_{n \to \infty} S_{2n} + \lim_{n \to \infty} a_{2n+1} =
S \Rightarrow
\lim_{n \to \infty} S_n = S
\end{equation*}

Значит, ряд сходится.
\end{proof}

\section{Функциональные ряды}
\index{Ряд!функциональный} 

\begin{theorem}[мажорантный признак сходимости]
Если существует последовательность~$(a_n) \colon \forall n \in \mathbb N \ |f_n(x)| \leqslant a_n$, то $\series a_k$ сходится $\Rightarrow$ $\series f_k(x)$ сходится абсолютно.
\end{theorem}

\subsection{Степенные ряды}
\index{Ряд!степенной} \textbf{Степенным} называется ряд вида $\series[0] c_k (x - x_0)^k$.

\begin{theorem}[Абеля]
\begin{itemize}
	\item Если ряд $\series[0] c_k x^k$ сходится при $x = x_0$, то $\forall x \colon |x| < |x_0|$ он сходится абсолютно.
	\item Если ряд $\series[0] c_k x^k$ расходится при $x = x_0$, то $\forall x \colon |x| > |x_0|$ он расходится.
\end{itemize}
\end{theorem}
\begin{proof}
\begin{itemize}
	\item $\lim\limits_{n \to \infty} c_n x_0^n = 0 \Rightarrow
	\exists M > 0 \colon |c_n x_0^n| \leqslant M \Rightarrow
	|c_n x^n| \leqslant M \left| \frac{x}{x_0} \right|^n$.
	
	Тогда $\series[0] M \left| \frac{x}{x_0} \right|^k$ сходится $\Rightarrow$ $\series[0] c_k x^k$ сходится абсолютно.
	
	\item Если бы $\exists x_1 \colon |x_1| > |x_0|$ $\lAnd$ $\series[0] c_k x_1^k$ сходится, то $\series[0] c_k x_0^k$ сходился бы, что противоречит условию.
\end{itemize}
\end{proof}

\begin{consequent}
$\exists R > 0 \colon$ при $|x| < R$ $\series[0] c_k x^k$ сходится, а при $|x| > R$ $\series[0] c_k x^k$ расходится.
\end{consequent}

$R$ называется \textbf{радиусом сходимости степенного ряда}.

Рассмотрим ряд~$\series[0] |c_k x^k|$.
По признаку д'Аламбера $\lim\limits_{n \to \infty} = \lim\limits_{n \to \infty} \left| \frac{c_{n+1}}{c_n} \cdot |x| \right| \ldots$