\chapter{16.03.2018, дискретная математика}
По индукции можно задать функцию перехода $\delta^* \colon S \times X^* \to S$:
\begin{itemize}
	\item $\delta^*(s, x) = \delta(s, x)$;
	\item $\delta^*(s, \alpha x) = \delta(\delta^*(s, \alpha), x)$.
\end{itemize}

Такой функции перехода соответствует ориентированный путь в графе.

Запись $\delta^*(s, \alpha)$ несколько громоздка, поэтому вместо неё может использоваться запись $s \delta(\alpha)$.

\index{Автомат конечный!настроенный}Конечный автомат называется \textbf{настроенным}, если для него указаны начальное состояние~$s_1$ и множество~$F$ допускающих состояний.
Т.\,е. настроенный автомат задаётся набором~$(S, X, \delta, s_1, F)$.

Настроенный автомат~$A$ \textbf{распознаёт язык~$L$}, если $\alpha \in L \Leftrightarrow s_1 \delta(\alpha) \in F$.

\begin{statement}
Любой конечный язык распознаётся конечным автоматом.
\end{statement}
\begin{proof}
Пусть $L$~--- конечный язык, множество~$S$ состояний состоит из префиксов слов~$L$, а также включает дополнительное состояние~$s'$, $\alpha \delta(x) =
\begin{cases}
\alpha x, \ \alpha x \in S \\
s', \ \alpha x \notin S
\end{cases}$

Рассмотрим автомат~$(S, X, \delta, \lambda, L)$.
\begin{equation*}
\lambda \delta(\alpha) =
\begin{cases}
\alpha, \ \alpha \in S \setminus s' \\
s', \ \alpha \notin S \setminus s'
\end{cases} \Rightarrow
(s_1 \delta(\alpha) \in F \Leftrightarrow \alpha \in L)
\end{equation*}
\end{proof}

\begin{theorem}
Язык~$L = \{ a^k b^k \mid k \geqslant 0 \}$ не распознаётся конечным автоматом.
\end{theorem}
\begin{proofcontra}
Пусть $L$ распознаётся конечным автоматом~$A = (S, X, \delta, s_1, F)$ с $n$~состояниями.
Тогда какие-то из состояний $s_1, s_1 \delta(a), s_1 \delta(aa), \ldots, s_1 \delta(a^{n-1}), s_1 \delta(a^n)$ совпадают.
Пусть $s_1 \delta(a^i) = s_1 \delta(a^j)$, тогда $s_1 \delta(a^i) \delta(b^i) \in F \Rightarrow
s_1 \delta(a^j) \delta(b^i) \in F$.
Значит, $a^j b^i \in L$.
Противоречие.
\end{proofcontra}

Некоторое отношение~$\sim$ называется \textbf{отношением эквивалентности}, если оно удовлетворяет условиям:
\begin{enumerate}
	\item Рефлексивность: $a \sim a$.
	\item Симметричность: $a \sim b \Rightarrow b \sim a$.
	\item Транзитивность: $a \sim b \lAnd b \sim c \Rightarrow a \sim c$.
\end{enumerate}

\textbf{Классом эквивалентности}, или \textbf{фактор"=классом}, \textbf{элемента~$x$} называется множество~$[x] = \{ y \mid y \sim x \}$.

\textbf{Фактор"=множеством} называется множество различных фактор"=классов.

Слова $\alpha$ и $\beta$ называются \textbf{различимыми словом~$\gamma \in X^*$ относительно языка~$L$}, если $\alpha \gamma \in L \lAnd \beta \gamma \notin L \lOr \alpha \gamma \notin L \lAnd \beta \gamma \in L$.
Различимость обозначается $\alpha \not\sim_L \beta$.

Слова $\alpha$ и $\beta$ называются \textbf{неразличимыми относительно языка~$L$}, если $\forall \gamma \in X^* \ \alpha \gamma \in L \Leftrightarrow \beta \gamma \in L$.
Неразличимость обозначается $\alpha \sim_L \beta$.

\begin{statement}
Отношение неразличимости слов относительно языка является отношением эквивалентности.
\end{statement}
\begin{proof}
Очевидно, что $\alpha \sim \alpha$ и $\alpha \sim \beta \Rightarrow \beta \sim \alpha$.

Пусть $\alpha \sim \beta \lAnd \beta \sim \gamma$, тогда $\forall \Theta \in X^* \ 
\alpha \Theta \in L \Leftrightarrow
\beta \Theta \in L \Leftrightarrow
\gamma \Theta \in L \Rightarrow
\alpha \sim \gamma$.
\end{proof}

\begin{statement}
$\alpha \sim \beta \Rightarrow \forall \gamma \in X^* \ \alpha \gamma \sim \beta \gamma$.
\end{statement}
\begin{proof}
\begin{equation*}
\forall \Theta \in X^* \ (\alpha \gamma) \Theta \in L \Leftrightarrow
\alpha (\gamma \Theta) \in L \Leftrightarrow
\beta (\gamma \Theta) \in L \Leftrightarrow
(\beta \gamma) \Theta \in L \Rightarrow
\alpha \gamma \sim \beta \gamma
\end{equation*}
\end{proof}

\textbf{Рангом языка~$L$} называется количество элементов в фактор"=множестве относительно неразличимости слов относительно~$L$ и обозначается~$\rank L$.

\begin{statement}
Если $A$~--- автомат с $n$~состояниями, распознающий язык~$L$, то $n \geqslant \rank L$.
\end{statement}