\chapter{20.04.2018, дискретная математика}
\subsection{Алгоритм Бржозовского}
Назовём язык, распознаваемый автоматом~$(S, X, \delta, s_1, \{ s' \})$, \textbf{левым языком~$L_l(s')$}, а язык, распознаваемый автоматом~$(S, X, \delta, s', F)$~--- \textbf{правым языком~$L_r(s')$}.

$L = \bigcup\limits_{s' \in S} L_l(s')$

Пусть $A = (S, X, \delta, s_1, F)$.
Рассмотрим произвольное состояние~$s'$.
\begin{enumerate}
	\item Автомат детерминирован $\Leftrightarrow$ все его левые языки не пересекаются.
	\begin{proofcontra}
	Пусть $\alpha \in L_l(s') \cap L_l(s'')$, тогда по~$\alpha$ можно прийти и в~$s'$, и в $s''$ $\Leftrightarrow$ автомат недетерминирован.
	\end{proofcontra}
	
	\item Если $L_l(s')$~--- левый язык автомата~$A$, то $r(L_l(s'))$~--- правый язык~$r(A)$.
	
	\item Автомат минимальный $\Leftrightarrow$ все его правые языки различны и все вершины достижимы.
\end{enumerate}

\begin{theorem}
Автомат~$\mathop{drdr}(A)$ минимален, где $A$~--- автомат.
\end{theorem}
\begin{proof}
Т.\,к. обращений было два, то автомат распознаёт тот же язык.
Кроме того, он детерминированный.

Левые языки автомата~$dr(A)$ не пересекаются $\Rightarrow$ правые языки $rdr(A)$ не пересекаются $\Rightarrow$ правые языки $drdr(A)$ различны $\Rightarrow$ $drdr(A)$ минимален.
\end{proof}

\begin{enumerate}
	\item $\varnothing$, $\lambda$, $x$ называются \textbf{регулярными выражениями}.
	Они определяют языки $\varnothing$, $\{ \lambda \}$, $\{ x \}$.
	
	\item $L \cup M$, $LM$, $L^*$ называются \textbf{регулярными выражениями}, где $L, M$~--- регулярные выражения.
\end{enumerate}

Приоритет операций в порядке убывания: $*$, $\cdot$, $\cup$.
При записи регулярных выражений объединение может обозначаться знаком~$+$.

\begin{theorem}[Клини]
Язык над алфавитом~$X$ распознаётся конечным автоматом $\Leftrightarrow$ он может быть выражен через языки $\varnothing$, $\{ \lambda \}$, $\{ x \}$, где $x \in X$, и операции $\cup$, $\cdot$, $*$.
\end{theorem}
\begin{proof}
\begin{enumerate}
	\item $\Leftarrow$. Очевидно.
	\item $\Rightarrow$. Пусть вершины пронумерованы от $1$ до $n$.
	Обозначим $L_{ij}^{(k)} = \{ \alpha \mid i \delta(\alpha) = j \lAnd \text{промежуточные состояния имеют номер не больше } k \}$.
		\indbase $L_{ij}^{(0)} =
		\begin{cases}
		x_1 + x_2 + \ldots + x_k, \ i \neq j \\
		x_1 + x_2 + \ldots + x_k + \lambda, \ i = j
		\end{cases}$
		\indstep $L_{ij}^{(k+1)} = L_{ij}^{(k)} + L_{i\,k+1}^{(k)} (L_{k+1\,k+1}^{(k)})^* L_{k+1\,j}^{(k)}$
		\indend
\end{enumerate}
\end{proof}