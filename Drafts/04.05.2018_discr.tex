\chapter{04.05.2018, дискретная математика}
\section{Контекстно-свободные грамматики}
\textbf{Контекстно-свободной грамматикой} называется набор~$(V, X, P, S)$, где\\
$X$~--- конечное множество \textbf{терминальных символов}, или \textbf{терминалов};\\
$V$~--- конечное множество \textbf{переменных}, представляющих языки;\\
$S$~--- \textbf{стартовый символ}~--- переменная, представляющая определяемый язык;\\
$P$~--- конечное множество \textbf{продукций}~--- правил вывода вида~$A \to B$, в результате которого $A$ заменяется на~$B$.

\begin{theorem}
Язык $\{ a^n b^n \mid n \geqslant 0 \}$~--- нерегулярный и контекстно-свободный.
\end{theorem}
\begin{proof}
Такой язык описывается контекстно-свободной грамматикой~$(\{ s \}, \{ a, b \}, \{ S \to \lambda, S \to aSb \}, S)$.
\end{proof}

\textbf{Деревом разбора для контекстно-свободной грамматики~$G = (V, X, P, S)$} называется дерево такое, что:
\begin{enumerate}
	\item Каждый внутренний узел отмечен переменной из~$V$.
	\item Каждый лист отмечен либо переменной из~$V$, либо терминалом, либо $\lambda$.
	Если он отмечен $\lambda$, то он должен быть единственным потомком своего родителя.
	\item Если внутренний узел отмечен $A$, а его потомки~--- $X_1, X_2, \ldots, X_k$ слева направо, то $(A \to X_1 X_2 \ldots X_k) \in P$.
\end{enumerate}

Контекстно-свободная грамматика называется \textbf{неоднозначной}, если она содержит слово, для которого можно построить несколько деревьев разбора, иначе~--- \textbf{однозначной}.

\begin{theorem}
Все контекстно-свободные языки распознаются автоматами с магазинной памятью, и все автоматы с магазинной памятью задают контекстно-свободные языки.
\end{theorem}