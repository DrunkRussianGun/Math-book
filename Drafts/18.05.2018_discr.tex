\chapter{18.05.2018, дискретная математика}
Язык называется \textbf{вычислимо-перечислимым}, если на словах из языка машина останавливается на непустом символе, а на словах не из языка работает бесконечно долго.

\begin{statement}
Языки $L$ и $\overline L$ вычислимо-перечислимы $\Leftrightarrow$ $L$ вычислим.
\end{statement}
\begin{proof}
\begin{enumerate}
	\item $\Rightarrow$.
	Можно объединить две машины, на нечётных шагах выполняя команды первой машины, а на чётных~--- второй.
	На ленте изначально будет два одинаковых слова, своё для каждой машины.
	Тогда получим машину, которая запускает на слове обе машины.
	Одна из них всегда остановится.
	
	\item $\Leftarrow$. 
\end{enumerate}
\end{proof}

\begin{statement}
$L$ вычислим $\Leftrightarrow$ $\overline L$ вычислим.
\end{statement}

\begin{theorem}
Язык $L_0 = \{ i \mid \alpha_i \in L_{M_i} \}$ вычислимо-перечислим, но невычислим.
\end{theorem}
\begin{proof}
Пусть команда $c = q_i x_j q_k x_l d$ кодируется числом $|c| = 2^{i+1} \cdot 3^{j+1} \cdot 5^{k+1} \cdot 7^{l+1} \cdot 11^{d+1}$.
Тогда машина Тьюринга с $r$~командами кодируется числом $2^{|c_1| + 1} \cdot 3^{|c_2| + 1} \cdot \ldots \cdot p_r^{|c_r| + 1}$.
Также можно закодировать слова.

Пусть $\alpha_i$~--- слово с кодом $i$, $L_{M_i}$~--- язык, распознаваемый машиной Тьюринга с кодом~$i$.

\begin{enumerate}
	\item Если $L_0$ вычислим, то $\overline L_0$ также вычислим, тогда существует машина Тьюринга $M_k$ с кодом~$k$, вычисляющая $\overline L_0$.
	Значит, $\alpha_k \in L_{M_k} \Leftrightarrow
	k \in L_0 \Leftrightarrow
	k \notin \overline L_0 \Leftrightarrow
	\alpha_k \notin L_{M_k}$.
	Противоречие, тогда $L_0$ не вычислим.
	
	\item На первом шаге запускаем $M_1$, на втором~--- $M_2$ и т.\,д.
	Если какая-то из машин остановилась, то соответствующее слово принадлежит $L_0$.
\end{enumerate}
\end{proof}

\begin{consequent}
$\overline L_0$ не вычислимо-перечислим.
\end{consequent}

\begin{theorem}
$L_1 = \{ (i, j) \mid \alpha_i \in L_{M_j} \}$
\end{theorem}