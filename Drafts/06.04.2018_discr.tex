\chapter{06.04.2018, дискретная математика}
\begin{statement}
Если непустой язык распознаётся автоматом с $n$~состояниями, то он содержит слово длины не больше~$n$.
\end{statement}
\begin{proof}
Если автомат распознаёт слово, то в соответствующем графе есть путь из начального состояния в допускающее, а значит, есть и простой путь.
В графе $n$~вершин, тогда длина простого пути не больше~$n$, а ему соответствует слово длины не больше~$n$.
\end{proof}

\begin{statement}
Пусть $L_1$ и $L_2$~--- языки, распознаваемые автоматами с $n_1$ и $n_2$ состояниями соответственно.
$L_1 = L_2$ $\Leftrightarrow$ все слова длин, не больших $n_1 n_2$, 
\end{statement}

\textbf{Конкатенацией языков $L_1$ и $L_2$} называется язык~$\{ \alpha \beta | \alpha \in L_1 \lAnd \beta \in L_2 \}$.

\textbf{Недетерминированным конечным автоматом} называется набор~$(S, X, \delta, s_1, F)$, где
$S$~--- конечное множество состояний;
$X$~--- конечный алфавит;
$\delta \colon S \times (X \cup \{ \lambda \}) \to Y$~--- функция перехода, где $Y \subseteq 2^S$;
$s_1 \in S$~--- начальное состояние;
$F \subseteq S$~--- множество допускающих состояний.

Недетерминированный автомат \textbf{распознаёт язык~$L$}, если при чтении слова из языка~$L$ хотя бы один из получившихся путей приводит в допускающее состояние.

Если языки $L_1$ и $L_2$ распознаются автоматами $(S_1, X, \delta_1, s_1, F_1)$ и $(S_2, X, \delta_2, s_2, F_2)$ соответственно, тогда автомат~$(S_1 \cup S_2, X, \delta, s_1, F_2)$ распознаёт язык $L_1 L_2$, где
$s \delta(x) =
\begin{cases}
s \delta_1(x), \ s \in S_1 \\
s \delta_2(x), \ s \in S_2 \\
s_2, \ s \in F_1 \lAnd x = \lambda
\end{cases}$, т.\,к.
\begin{equation*}
s_1 \delta(\alpha \beta) =
(s_1 \delta(\alpha)) \delta(\beta) =
s_1 \delta(\alpha) \delta(\lambda) \delta(\beta) =
s_2 \delta(\beta)
\end{equation*}

$L^n = \underbrace{L L \ldots L}_n$, причём $L^0 = \{ \lambda \}$
$L^* = \bigcup\limits_{n=0}^\infty L^n$, $X^* = \bigcup\limits_{n=0}^\infty X^n$~--- \textbf{звёздочка Клини}.

Пусть $L$ распознаётся автоматом~$(S, X, \delta, s_1, F)$, тогда $(S \cup \{ s_0 \}, X, \delta_1, s_0, F \cup \{ s_0 \})$ распознаёт $L^*$, где
$\delta_1 =
\begin{cases}
s \delta(x), \ s \in S \\
s_1, \ s \in S \lAnd x = \lambda
\end{cases}$