\chapter{28.03.2018, математический анализ}
\subsection{Тригонометрические ряды Фурье}
Представим функцию~$f(x)$ на отрезке~$\left[ -\frac{T}2; \frac{T}2 \right]$ в виде
\begin{equation*}
f(x) = \frac{a_0}2 + \series a_n \cos \frac{2 \pi n}T\,x + \series b_n \sin \frac{2 \pi n}T\,x
\end{equation*}

\textbf{Скалярным произведением функций $f(x)$ и $g(x)$} называется $\langle f(x), g(x) \rangle = \int\limits_{-\frac{T}2}^{\frac{T}2} f(x)g(x)\,dx$.

Проверим, что функции $1$, $\cos \frac{2 \pi n}T\,x$ и $\sin \frac{2 \pi n}T\,x$ попарно ортогональны:
\begin{itemize}
	\item $\displaystyle \int_{-\frac{T}2}^{\frac{T}2} \cos \frac{2 \pi n}T\,x\,dx =
	\frac{T}{2 \pi n} \left. \sin \frac{2 \pi n}T\,x \right|_{-\frac{T}2}^{\frac{T}2} = 0$
	
	\item $\displaystyle \int_{-\frac{T}2}^{\frac{T}2} \sin \frac{2 \pi n}T\,x\,dx =
	-\frac{T}{2 \pi n} \left. \cos \frac{2 \pi n}T\,x \right|_{-\frac{T}2}^{\frac{T}2} = 0$
	
	\item $\displaystyle \int_{-\frac{T}2}^{\frac{T}2} \cos \frac{2 \pi m}T\,x \cos \frac{2 \pi n}T\,x\,dx =
	\frac12 \int_{-\frac{T}2}^{\frac{T}2} \cos \frac{2 \pi (m - n)}T\,x +
	\frac12 \int_{-\frac{T}2}^{\frac{T}2} \cos \frac{2 \pi (m + n)}T\,x = 0$, $m \neq n$
	
	\item $\displaystyle \int_{-\frac{T}2}^{\frac{T}2} \sin \frac{2 \pi m}T\,x \sin \frac{2 \pi n}T\,x\,dx =
	\frac12 \int_{-\frac{T}2}^{\frac{T}2} \cos \frac{2 \pi (m - n)}T\,x -
	\frac12 \int_{-\frac{T}2}^{\frac{T}2} \cos \frac{2 \pi (m + n)}T\,x = 0$, $m \neq n$
	
	\item $\displaystyle \int_{-\frac{T}2}^{\frac{T}2} \cos \frac{2 \pi m}T\,x \sin \frac{2 \pi n}T\,x\,dx =
	\frac12 \int_{-\frac{T}2}^{\frac{T}2} \sin \frac{2 \pi (m + n)}T\,x +
	\frac12 \int_{-\frac{T}2}^{\frac{T}2} \sin \frac{2 \pi (m - n)}T\,x = 0$
\end{itemize}

Найдём квадраты этих функций:
\begin{itemize}
	\item $\displaystyle \int_{-\frac{T}2}^{\frac{T}2} 1^2\,dx = T$
	\item $\displaystyle \int_{-\frac{T}2}^{\frac{T}2} \cos^2 \frac{2 \pi n}T\,x\,dx = \frac{T}2$
	\item $\displaystyle \int_{-\frac{T}2}^{\frac{T}2} \sin^2 \frac{2 \pi n}T\,x\,dx = \frac{T}2$
\end{itemize}

Тогда можно найти коэффициенты:
\begin{itemize}
	\item $\displaystyle a_k = \frac2T \int_{-\frac{T}2}^{\frac{T}2} f(x) \cos \frac{2 \pi k}T\,x\,dx$
	\item $\displaystyle b_k = \frac2T \int_{-\frac{T}2}^{\frac{T}2} f(x) \sin \frac{2 \pi k}T\,x\,dx$
\end{itemize}

Причём если $f(x)$ чётна, то $b_k = 0$.
Если же $f(x)$ нечётна, то $a_k = 0$.