\chapter{11.04.2018, математический анализ}
\subsection{Уравнение в полных дифференциалах}
\textbf{Уравнением в полных дифференциалах} называется уравнение вида $y' = -\frac{M(x, y)}{N(x, y)}$.
Если $M_y' = N_x'$, то $M(x, y) = F_x' \lAnd N(x, y) = F_y'$, тогда
\begin{equation*}
y' = -\frac{M(x, y)}{N(x, y)} \Leftrightarrow
M(x, y)\,dx + N(x, y)\,dy = 0 \Leftrightarrow
dF(x, y) = 0 \Leftrightarrow
F(x, y) = C
\end{equation*}

Пусть $M_y' \neq N_x'$, но $\exists \mu(x, y) \colon (\mu \cdot M)_y' = (\mu \cdot N)_x'$, тогда
\begin{equation*}
M(x, y)\,dx + N(x, y)\,dy = 0 \Leftrightarrow
\mu(x, y) M(x, y)\,dx + \mu(x, y) N(x, y)\,dy = 0 \Leftrightarrow
dF(x, y) = 0 \Leftrightarrow
F(x, y) = C
\end{equation*}

$\mu(x, y)$ называется \textbf{интегрирующим множителем}.

\section{Линейное уравнение первого порядка}
\textbf{Линейным дифференциальным уравнением первого порядка} называется уравнение вида $y' = a(x) y + b(x)$.

Рассмотрим \textbf{метод вариации произвольной постоянной}.
\begin{enumerate}
	\item Решим уравнение
	\begin{equation*}
	y_0' = a(x) y_0 \Leftrightarrow
	\frac{dy_0}{y_0} = a(x)\,dx \Rightarrow
	\ln y_0 = \varphi(x) + \ln C \Rightarrow
	y_0 = C e^{\varphi(x)}
	\end{equation*}
	
	\item Подставим $y = C(x) e^{\varphi(x)}$ в исходное уравнение:
	\begin{equation*}
	C'(x) e^{\varphi(x)} + C(x) e^{\varphi(x)} \varphi'(x) = a(x) C(x) e^{\varphi(x)} + b(x)
	\end{equation*}
	
	$y_0 = Ce^{\varphi(x)} \Rightarrow C e^{\varphi(x)} \varphi'(x) = a(x) C e^{\varphi(x)}$, тогда получим
	\begin{equation*}
	C'(x) e^{\varphi(x)} = b(x) \Leftrightarrow
	C'(x) = b(x) e^{-\varphi(x)} \Leftrightarrow
	C(x) = \int b(x) e^{-\varphi(x)}
	\end{equation*}
\end{enumerate}

Тогда $y = C(x) e^{\varphi(x)}$.