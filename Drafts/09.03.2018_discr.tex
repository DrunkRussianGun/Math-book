\chapter{09.03.2018, дискретная математика}
\subsection{Алгоритм Эдмондса"--~Карпа}
Путь~$(s = v_1, \ldots, v_{k+1} = t)$ в сети~$(V, E)$ называется \textbf{увеличивающим}, если $\forall i \in \{ 1, \ldots, k \} \ \varepsilon(v_i, v_{i+1}) > 0$, где $\varepsilon(v_i, v_{i+1}) = \begin{cases}
q(v_i, v_{i+1}) - p(v_i, v_{i+1}), \ (v_i, v_{i+1}) \in E \\
p(v_{i+1}, v_i), \ (v_{i+1}, v_i) \in E
\end{cases}$.
Введём $\displaystyle \delta = \min_{0 \leqslant i \leqslant k} \varepsilon(v_i, v_{i+1})$, тогда новое значение потока в сети равно $\begin{cases}
p(v_i, v_{i+1}) + \delta, \ (v_i, v_{i+1}) \in E \\
p(v_{i+1}, v_i) - \delta, \ (v_{i+1}, v_i) \in E
\end{cases}$.

Рассмотрим случай, когда до~$t$ не существует увеличивающего пути.
Пусть $X$~--- множество вершин, до которых существует увеличивающий путь, $u \in X$, $v \notin X$.
Если $(u, v) \in E$, то $q(u, v) = p(u, v)$, а если $(v, u) \in E$, то $p(v, u) = 0$, тогда
\begin{equation*}
p(X, \overline X) =
\sum_{u \in V, v \in \overline X} p(u, v) - \sum_{u \in V, v \in \overline X} p(v, u) =
\sum_{u \in V, v \in \overline X} q(u, v)
\end{equation*}

\begin{lemma}
В ходе работы алгоритма Эдмондса"--~Карпа кратчайший $(s, t)$"=путь не уменьшается.
\end{lemma}
\begin{proofcontra}
Рассмотрим самую близкую к~$s$ вершину~$v$, для которой кратчайший путь~$(s, \ldots, u, v)$ уменьшается, тогда для вершины~$u$ кратчайший $(s, u)$"=путь не уменьшается.
Пусть $d_u$ и $d_v$~--- длины кратчайших $(s, u)$- и $(s, v)$"=путей соответственно на предыдущем шаге, а $d_u'$ и $d_v'$~--- на текущем.
\begin{equation*}
d_v > d_v' = d_u' + 1 \geqslant d_u + 1 \Rightarrow d_v \geqslant d_u + 2
\end{equation*}

Значит, на предыдущем шаге не было дуги~$(u, v)$, тогда не было и кратчайшего $(s, v)$"=пути.
Противоречие.
\end{proofcontra}

Назовём дугу~$(v_i, v_{i+1})$ \textbf{критической}, если $e(v_i, v_{i+1}) = \delta$.

\begin{lemma}
Каждая дуга может быть критической на увеличивающем пути порядка $\frac{|V|}2$~раз.
\end{lemma}
\begin{proof}
Пусть дуга~$(u, v)$ критическая на шагах $t_1$ и $t_2$.
Если она была использована как прямая два раза, то между этими использованиями она должна была быть использована как обратная (на шаге~$t_3$), тогда
\begin{equation*}
d_v(t_2) = d_u(t_2) + 1 \geqslant
d_u(t_3) + 1 = d_v(t_3) + 2 \geqslant
d_v(t_1) + 2
\end{equation*}
\end{proof}

\section{Конечные автоматы}
Назовём \textbf{алфавитом} конечное непустое множество и обозначим через~$X$.
Его элементы называются \textbf{буквами}.
Конечная последовательность букв называется \textbf{словом}, а его \textbf{длиной}~--- количество букв в слове с учётом повторений.

Слово, не содержащее букв, называется \textbf{пустым} и обозначается~$\lambda$.

Множество из всех слов алфавита~$X$ обозначается~$X^*$.

\textbf{Конкатенацией слов} $\alpha = x_1 x_2 \ldots x_n$ и $\beta = y_1 y_2 \ldots y_m$ называется слово $\alpha \cdot \beta = x_1 \ldots x_n y_1 \ldots y_m$.

\textbf{Степенью слова}~$\alpha = x_1 \ldots x_n$ называется слово~$\alpha^n = \alpha \cdot \alpha \cdot \ldots \cdot \alpha$, где $n \in \mathbb N$.
$\alpha^0 = \lambda$.

Языком называется множество $L \subseteq X^*$.

\textbf{Конечным автоматом} называется набор~$(X, S, \delta)$, где $X$~--- алфавит, $S$~--- конечное множество \textbf{состояний}, $\delta \colon S \times X \to S$~--- функция перехода.

Если задан орграф, в котором каждой дуге соответствует буква, то по нему можно построить конечный автомат.