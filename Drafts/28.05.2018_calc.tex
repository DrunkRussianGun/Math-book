\subsection{Метод Эйлера}
$y_{k+1} = y_k + f(x_k, y_k) (x_{k+1} - x_k)$

Можно улучшить точность:
$y_{k+1} = y_k + f($

Метод приближённого решения дифференциального уравнения высшего порядка заключается в его сведении к системе линейных уравнений.

\subsection{Графический метод}
Приближённые решения уравнения вида $y' = f(x, y)$ можно получить графическим методом, находя изоклины~--- линии, на которых производная функции не меняет значение.
По ним можно получить представление о том, какую форму имеет кривая.

Рассмотрим \textbf{динамическую систему}
\begin{equation*}
\begin{cases}
\dot x = f(x, y, t) \\
\dot y = g(x, y, t)
\end{cases}
\end{equation*}

Если
$\begin{cases}
f(a, b, t) = 0 \\
g(a, b, t) = 0
\end{cases}$, то $(a, b)$ называется \textbf{точкой покоя}, или \textbf{положением равновесия}.

Исследуем систему
\begin{equation*}
\begin{cases}
\dot x = ax + by \\
\dot y = cx + dy
\end{cases}
\end{equation*}

Для неё $(0, 0)$~--- точка покоя.

Решая уравнение
$\begin{vmatrix}
a - k & b \\
c & d - k
\end{vmatrix} = 0$, получим корни $k_1$ и $k_2$.

Тогда
\begin{enumerate}
	\item Если $k_1, k_2 \in \mathbb R$, то
	\begin{equation*}
	x = C_1 e^{k_1 t} + C_2 e^{k_2 t} \\
	y = C_1\,\frac{k_1 - a}b\,e^{k_1 t} + C_2\,\frac{k_2 - a}b\,e^{k_2 t}
	\end{equation*}
	
	\begin{itemize}
		\item Если $k_1, k_2 < 0$, то $(0, 0)$~--- устойчивый узел.
		\item Если $k_1, k_2 > 0$, то $(0, 0)$~--- неустойчивый узел.
		\item Если $k_1 < 0 < k_2$, то $(0, 0)$~--- седло.
	\end{itemize}
	
	\item Если $k_{1,2} = \alpha \pm i \beta$, то
	\begin{equation*}
	\begin{cases}
	x = C_1 e^{\alpha t} \cos \beta t + C_2 e^{\alpha t} \sin \beta t \\
	y = \left( \frac{\alpha - a}b\,C_1 + \frac\beta{b}\,C_2 \right) e^{\alpha t} \cos \beta t + C_2 e^{\alpha t} \sin \beta t
	\end{cases}
	\end{equation*}
	
	\begin{itemize}
		\item Если $\alpha < 0$, то $(0, 0)$~--- устойчивый фокус.
		\item Если $\alpha > 0$, то $(0, 0)$~--- неустойчивый фокус.
		\item Если $\alpha = 0$, то $(0, 0)$~--- центр.
	\end{itemize}
	
	\item Если $k_1 = k_2 = \alpha \in R$, то
	\begin{equation*}
	x = 
	\end{equation*}
\end{enumerate}