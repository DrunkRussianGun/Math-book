\section{Многочлены от одной переменной}
\index{Одночлен}\index{Моном} \textbf{Одночленом}, или \textbf{мономом}, называется произведение числового множителя и нуля и более переменных, взятых каждая в~неотрицательной степени.

\textbf{Степенью одночлена} называется сумма степеней входящих в~него переменных.
Степень тождественного нуля равна~$-\infty$.

\index{Многочлен}\index{Полином} \textbf{Многочленом}, или \textbf{полиномом}, \textbf{от одной переменной} называется сумма вида
\begin{equation*}
a_0 + a_1 x + a_2 x^2 + \ldots + a_n x^n
\end{equation*}
где $x_1, \ldots, x_n$~--- переменные.

\textbf{Степенью многочлена} называется максимальная из степеней его одночленов.
Многочлен $1$\nobreakdash-й степени называется \textbf{линейным}, $2$\nobreakdash-й степени~--- \textbf{квадратным}.

\begin{lemma}
Пусть $f$ и $g$~--- многочлены, тогда $\deg fg = \deg f + \deg g$.
\end{lemma}