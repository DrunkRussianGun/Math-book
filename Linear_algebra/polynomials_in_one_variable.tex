\section{Многочлены от одной переменной}
\textbf{Одночленом}, или \textbf{мономом}, называется произведение числового множителя и~нуля и~более переменных, взятых каждая в~неотрицательной степени.

\textbf{Степенью} одночлена называется сумма степеней входящих в~него переменных. Степень тождественного нуля равна $-\infty$.

\textbf{Многочленом}, или \textbf{полиномом}, от~одной переменной называется сумма вида
\begin{equation*}
a_0 + a_1 x + a_2 x^2 + \ldots + a_n x^n
\end{equation*}
где $x_1, \ldots, x_n$~--- переменные.

\textbf{Степенью} многочлена называется максимальная из~степеней его одночленов.

\begin{lemma}
Пусть $f$ и~$g$~--- многочлены, тогда $\deg fg = \deg f + \deg g$.
\end{lemma}