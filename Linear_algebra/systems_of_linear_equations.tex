% beta
\section{Системы линейных алгебраических уравнений}
Система линейных алгебраических уравнений имеет вид
\begin{equation*}
\begin{cases}
a_{11}x_1 + a_{12}x_2 + \dots + a_{1n}x_n = b_1 \\
a_{21}x_1 + a_{22}x_2 + \dots + a_{2n}x_n = b_2 \\
\vdots \\
a_{m1}x_1 + a_{m2}x_2 + \dots + a_{mn}x_n = b_m
\end{cases}
\end{equation*}
где $x_1, \ldots, x_n$~--- переменные.

$a_{11}, a_{12}, \ldots, a_{mn}$ называются \textbf{коэффициентами при~переменных}, $b_1, b_2, \dots, b_m$~--- \textbf{свободными членами}.

Система линейных уравнений называется \textbf{однородной}, если все её свободные члены равны 0, иначе~--- \textbf{неоднородной}.

Система линейных уравнений называется \textbf{совместной}, если она имеет хотя~бы одно решение, иначе~--- \textbf{несовместной}.

Система линейных уравнений называется \textbf{определённой}, если она имеет единственное решение.
Если система имеет более одного решения, то она называется \textbf{неопределённой}.

Две системы линейных уравнений называются \textbf{эквивалентными}, если их решения совпадают или~обе не~имеют решений.