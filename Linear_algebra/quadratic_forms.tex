% beta
\section{Квадратные формы}
\textbf{Квадратичной формой} называется многочлен, все одночлены в~котором второй степени:
\begin{equation*}
f(x_1, \ldots, x_n) = \sum_{i=1}^n \sum_{j=1}^n a_{ij} x_i x_j
\end{equation*}
Для~определённости полагают $a_{ij} = a_{ji}$.

Квадратичной форме можно сопоставить \textbf{матрицу квадратичной формы}, составленную из~коэффициентов:
\begin{equation*}
\begin{Vmatrix}
a_{11} & a_{12} & \cdots & a_{1n} \\
a_{21} & a_{22} & \cdots & a_{2n} \\
\vdots & \vdots & \ddots & \vdots \\
a_{n1} & a_{n2} & \cdots & a_{nn}
\end{Vmatrix} =
\begin{Vmatrix}
a_{11} & a_{12} & \cdots & a_{1n} \\
a_{12} & a_{22} & \cdots & a_{2n} \\
\vdots & \vdots & \ddots & \vdots \\
a_{1n} & a_{2n} & \cdots & a_{nn}
\end{Vmatrix}
\end{equation*}

\textbf{Каноническим видом} квадратичной формы называется её представление в~виде суммы квадратов с~некоторыми коэффициентами.

\begin{theorem}[метод Лагранжа]
Любая квадратичная форма может быть приведена к~каноническому виду.
\end{theorem}
\begin{proof}

\end{proof}

\textbf{Нормальным видом} квадратичной формы называется её канонический вид, коэффициенты в~котором равны $-1$ или~$1$.

\textbf{Рангом} квадратичной формы называется количество переменных в~её каноническом виде.
Количество положительных коэффициентов в~каноническом виде квадратичной формы называется её \textbf{положительным индексом}, а отрицательных~--- \textbf{отрицательным индексом}.
\textbf{Сигнатурой} квадратичной формы называется модуль разности положительного и~отрицательного индексов.

Ранг, положительный и~отрицательный индексы и~сигнатура одинаковы для~всех канонических видов квадратичной формы.