\section{Векторные пространства}
\index{Векторное пространство} $n$\nobreakdash-мерным векторным пространством над полем вещественных чисел называется множество
\begin{equation*}
V_n = \mathbb R^n = \{ (x_1, \ldots, x_n) \mid x_1, \ldots, x_n \in \mathbb R \}
\end{equation*}
элементы которого называются \textbf{векторами}. Над ними определены операции сложения и умножения на число, удовлетворяющие аксиомам:
\begin{enumerate}
	\item Коммутативность сложения:
	\begin{equation*}
	\forall \overline a, \overline b \in V_n \
	\overline a + \overline b = \overline b + \overline a
	\end{equation*}
	
	\item Ассоциативность сложения:
	\begin{equation*}
	\forall \overline a, \overline b, \overline c \in V_n \
	\overline a + (\overline b + \overline c) = (\overline a + \overline b) + \overline c
	\end{equation*}
	
	\item Существование \textbf{нулевого} вектора, или \textbf{нуля}, обозначаемого $\overline 0$:
	\begin{equation*}
	\exists \overline 0 \in V_n \colon \forall \overline a \in V \
	\overline a + \overline 0 = \overline 0 + \overline a = \overline a
	\end{equation*}
	
	\item Существование \textbf{противоположного} вектора:
	\begin{equation*}
	\forall \overline a \in V_n \
	\exists (-\overline a) \in V_n \colon
	\overline a + (-\overline a) = \overline 0
	\end{equation*}
	
	\item Ассоциативность умножения на число:
	\begin{equation*}
	\forall \alpha, \beta \in \mathbb R, \
	\forall \overline a \in V_n \
	\alpha (\beta \overline a) = (\alpha \beta) \overline a
	\end{equation*}
	
	\item Дистрибутивность умножения на число относительно сложения векторов:
	\begin{equation*}
	\forall \alpha \in \mathbb R, \
	\forall \overline a, \overline b \in V_n \
	\alpha (\overline a + \overline b) = \alpha \overline a + \alpha \overline b
	\end{equation*}
	
	\item Дистрибутивность умножения на число относительно сложения чисел:
	\begin{equation*}
	\forall \alpha, \beta \in \mathbb R, \
	\forall \overline a \in V_n \
	(\alpha + \beta) \overline a = \alpha \overline a + \beta \overline a
	\end{equation*}
	
	\item Существование \textbf{единицы}:
	\begin{equation*}
	\forall \overline a \in V_n \
	1 \cdot \overline a = \overline a
	\end{equation*}
\end{enumerate}