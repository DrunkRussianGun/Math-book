% beta
\subsection{Симметрические многочлены}
Многочлен называется \textbf{симметрическим}, если при перестановке переменных он не изменяется.

\begin{statement}
Если $f(x_1, \ldots, x_n) = a x_1^{i_1} x_2^{i_2} \cdot \ldots \cdot x_n^{i_n} + \ldots$~--- симметрический многочлен, то $i_1 \geqslant i_2 \geqslant \ldots \geqslant i_n$.
\end{statement}
\begin{proofcontra}
Пусть $\exists r < q \colon i_r < i_q$, тогда $f$ содержит
$b x_1^{i_1} \* x_2^{i_2} \* \ldots \* x_r^{i_q} \* \ldots \* x_q^{i_r} \* \ldots \* x_n^{i_n}$, который старше, чем
$a x_1^{i_1} \* x_2^{i_2} \* \ldots \* x_n^{i_n}$. Противоречие.
\end{proofcontra}

\textbf{Элементарными} симметрическими многочленами от~$n$~переменных называются многочлены
\begin{equation*}
\sigma_1(x_1, \ldots, x_n) = \sum_{i=1}^n x_i
\end{equation*}
\begin{equation*}
\sigma_2(x_1, \ldots, x_n) = \sum_{i=1}^n \sum_{j=i+1}^n x_i x_j
\end{equation*}
\begin{equation*}
\vdots
\end{equation*}
\begin{equation*}
\sigma_n(x_1, \ldots, x_n) = x_1 x_2 \cdot \ldots \cdot x_n 
\end{equation*}

\begin{theorem}[основная теорема о симметрических многочленах]
Любой симметрический многочлен может быть представлен в виде многочлена от элементарных симметрических многочленов.
\end{theorem}
\begin{proof}

\end{proof}