\subsection{Матричная форма системы линейных уравнений}
Систему линейных уравнений можно представить в матричной форме:
\begin{equation*}
\begin{Vmatrix}
a_{11}x_1 + a_{12}x_2 + \dots + a_{1n}x_n \\
a_{21}x_1 + a_{22}x_2 + \dots + a_{2n}x_n \\
\vdots \\
a_{m1}x_1 + a_{m2}x_2 + \dots + a_{mn}x_n
\end{Vmatrix} =
\begin{Vmatrix}
b_1 \\
b_2 \\
\vdots \\
b_m
\end{Vmatrix}
\Leftrightarrow
\end{equation*}
\begin{equation*}
\Leftrightarrow
\begin{Vmatrix}
a_{11} & a_{12} & \cdots & a_{1n} \\
a_{21} & a_{22} & \cdots & a_{2n} \\
\vdots & \vdots & \ddots & \vdots \\
a_{m1} & a_{m2} & \cdots & a_{mn}
\end{Vmatrix} \cdot
\begin{Vmatrix}
x_1 \\
x_2 \\
\vdots \\
x_n
\end{Vmatrix} =
\begin{Vmatrix}
b_1 \\
b_2 \\
\vdots \\
b_m
\end{Vmatrix}
\Leftrightarrow
\end{equation*}
\begin{equation*}
\Leftrightarrow
A \cdot X = B
\end{equation*}

$A$ называется \textbf{основной матрицей системы}, $X$~--- \textbf{столбцом переменных}, $B$~--- \textbf{столбцом свободных членов}.
Если к основной матрице справа приписать столбец свободных членов, то получится \textbf{расширенная матрица системы}:
\begin{equation*}
\begin{Vmatrix}
a_{11} & a_{12} & \cdots & a_{1n} & \vline & b_1 \\
a_{21} & a_{22} & \cdots & a_{2n} & \vline & b_2 \\
\vdots & \vdots & \ddots & \vdots & \vline & \vdots \\
a_{m1} & a_{m2} & \cdots & a_{mn} & \vline & b_m
\end{Vmatrix}
\end{equation*}