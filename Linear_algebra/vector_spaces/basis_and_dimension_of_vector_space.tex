\subsection{Базис и размерность векторного пространства}
Множество векторов~$\overline{a_1}, \ldots, \overline{a_n}$ называется \textbf{линейно зависимым}, если
\begin{equation*}
\exists \alpha_1, \ldots, \alpha_n \colon
\sum_{i=1}^n \alpha_i \overline{a_i} = \overline 0, \
\sum_{i=1}^n \alpha_i^2 \neq 0
\end{equation*}
иначе~--- \textbf{линейно независимым}.

Множество линейно независимых векторов~$\overline{e_1}, \ldots, \overline{e_n}$ векторного пространства~$V$ называется \textbf{базисом} этого \textbf{пространства}, если
\begin{equation*}
\forall \overline x \in V \
\exists \alpha_1, \ldots, \alpha_n \colon
\overline x = \sum_{i=1}^n \alpha_i \overline{e_i}
\end{equation*}

Приведённое равенство называется \textbf{разложением вектора~$\overline x$ по базису~$\overline{e_1}, \ldots, \overline{e_n}$.}

\begin{theorem}[о~базисе]
Любой вектор~$\overline x$ может быть разложен по базису~$\overline{e_1}, \ldots, \overline{e_n}$ единственным образом.
\end{theorem}
\begin{proof}
Пусть
\begin{equation*}
\overline x = a_1 \overline{e_1} + \ldots + a_n \overline{e_n}
\end{equation*}
\begin{equation*}
\overline x = b_1 \overline{e_1} + \ldots + b_n \overline{e_n}
\end{equation*}

Вычитанием одного равенства из другого получим:
\begin{equation*}
(a_1 - b_1) \overline{e_1} + \ldots + (a_n - b_n) \overline{e_n} = \overline 0
\end{equation*}

В силу линейной независимости векторов~$\overline{e_1}, \ldots, \overline{e_n}$
\begin{equation*}
\begin{cases}
a_1 - b_1 = 0 \\
\ldots \\
a_n - b_n = 0
\end{cases}
\Leftrightarrow
\begin{cases}
a_1 = b_1 \\
\ldots \\
a_n = b_n
\end{cases}
\end{equation*}
\end{proof}

\textbf{Размерностью векторного пространства} называется максимальное количество линейно независимых векторов.

\begin{theorem}
В~векторном пространстве~$V$ размерности~$n$ любые $n$ линейно независимых векторов образуют его базис.
\end{theorem}
\begin{proof}
Рассмотрим множество векторов~$\overline{e_1}, \ldots, \overline{e_n} \in V$.
Для любого вектора~$\overline x \in V$ множество векторов~$\overline{e_1}, \ldots, \overline{e_n}, \overline x$ линейно зависимо, т.\,к. размерность~$V$ равна~$n$, тогда
\begin{equation*}
\exists \alpha_1, \ldots, \alpha_n, \alpha_{n+1} \colon
\sum_{i=1}^n \alpha_i \overline{e_i} + \alpha_{n+1} \overline x = \overline 0, \
\alpha_{n+1} \neq 0 \Rightarrow
\overline x = \sum_{i=1}^n -\frac{\alpha_i}{\alpha_{n+1}} \overline{e_i}
\end{equation*}

Значит, векторы~$\overline{e_1}, \ldots, \overline{e_n}$ образуют базис пространства~$V$.
\end{proof}

\begin{theorem}
Если векторное пространство~$V$ имеет базис из $n$~векторов, то его размерность равна~$n$.
\end{theorem}
\begin{proof}
Рассмотрим базис, состоящий из векторов~$\overline{e_1}, \ldots, \overline{e_n} \in V$.
\begin{equation*}
\forall \overline{e_{n+1}} \in V \
\exists \alpha_1, \ldots, \alpha_n \colon
\overline{e_{n+1}} = \sum_{i=1}^n \alpha_i \overline{e_i}
\end{equation*}

Значит, базис из $n + 1$~векторов не существует, тогда размерность пространства~$V$ равна~$n$.
\end{proof}