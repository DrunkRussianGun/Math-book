\section{Матрицы}
\index{Матрица} \textbf{Матрицей} называется прямоугольная таблица из чисел, содержащая $m$~строк и $n$~столбцов, и обозначается
\begin{equation*}
A = (a_{ij})_{\begin{smallmatrix}
i = \overline{1, m} \\
j = \overline{1, n}
\end{smallmatrix}} =
\begin{pmatrix}
a_{11} & a_{12} & \cdots & a_{1n} \\
a_{21} & a_{22} & \cdots & a_{2n} \\
\vdots & \vdots & \ddots & \vdots \\
a_{m1} & a_{m2} & \cdots & a_{mn}
\end{pmatrix} =
\begin{Vmatrix}
a_{11} & a_{12} & \cdots & a_{1n} \\
a_{21} & a_{22} & \cdots & a_{2n} \\
\vdots & \vdots & \ddots & \vdots \\
a_{m1} & a_{m2} & \cdots & a_{mn}
\end{Vmatrix} =
\|a_{ij}\|_{\begin{smallmatrix}
i = \overline{1, m} \\
j = \overline{1, n}
\end{smallmatrix}}
\end{equation*}

Числа $m$ и $n$ называются \textbf{порядками} матрицы.

Если $m = n$, то матрица называется \textbf{квадратной}, а число~$m = n$~--- её \textbf{порядком}.
\textbf{Главной} называется диагональ квадратной матрицы, состоящая из элементов $a_{11}, a_{22}, \ldots, a_{nn}$, а \textbf{побочной}~--- состоящая из элементов $a_{n1}, a_{n-1\, 2}, \ldots, a_{1n}$.

$i$-я строка матрицы обозначается $A_i$, $j$-й столбец~--- $A^j$.

Две матрицы называются \textbf{равными}, если их порядки и соответствующие элементы совпадают, иначе~--- \textbf{неравными}.