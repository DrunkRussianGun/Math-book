\subsection{Определитель матрицы}
\index{Определитель} \textbf{Определителем порядка~$n$} квадратной {матрицы~$A$} порядка~$n$, называется число, равное
\begin{equation}
\label{eq:determinant}
\Delta = \det A =
\begin{vmatrix}
a_{11} & a_{12} & \cdots & a_{1n} \\
a_{21} & a_{22} & \cdots & a_{2n} \\
\vdots & \vdots & \ddots & \vdots \\
a_{n1} & a_{n2} & \cdots & a_{nn}
\end{vmatrix} =
\sum_{\sigma = (i_1; \ldots; i_n) \in S_n} (-1)^{|\sigma|} a_{1\, i_1} a_{2\, i_2} \cdot \ldots \cdot a_{n\, i_n}, \ 
|\sigma| =
\begin{cases}
0, \sigma \text{ чётная} \\
1, \sigma \text{ нечётная}
\end{cases}
\end{equation}
где $S_n$~--- множество всех перестановок $n$-элементного множества.

Матрица называется \textbf{вырожденной}, если её определитель равен~$0$, иначе~--- \textbf{невырожденной}.

Свойства определителя:
\begin{itemize}
	\item Если элементы какой-либо строки или столбца определителя имеют общий множитель~$\lambda$, то его можно вынести за знак определителя.
	\begin{proof}
	\begin{equation*}
	\Delta = \sum (-1)^{|\sigma|} a_{1\, i_1} a_{2\, i_2} \cdot \ldots \cdot a_{n\, i_n}
	\end{equation*}
	Каждое слагаемое имеет множитель из каждой строки, а также из каждого столбца, т.\,к. $\sigma$ является перестановкой и содержит все номера столбцов от $1$ до $n$ включительно.
	Тогда все слагаемые имеют общий множитель~$\lambda$, поэтому его можно вынести за скобки.
	\end{proof}
	
	\item Если какая-либо строка или столбец определителя состоит из нулей, то он равен~$0$.
	
	\item \begin{equation*}
	\begin{vmatrix}
	a_{11} & a_{12} & \cdots & a_{1n} \\
	\vdots & \vdots & \ddots & \vdots \\
	a_{i1} + b_{i1} & a_{i2} + b_{i2} & \cdots & a_{in} + b_{in} \\
	\vdots & \vdots & \ddots & \vdots \\
	a_{n1} & a_{n2} & \cdots & a_{nn}
	\end{vmatrix} =
	\begin{vmatrix}
	a_{11} & a_{12} & \cdots & a_{1n} \\
	\vdots & \vdots & \ddots & \vdots \\
	a_{i1} & a_{i2} & \cdots & a_{in} \\
	\vdots & \vdots & \ddots & \vdots \\
	a_{n1} & a_{n2} & \cdots & a_{nn}
	\end{vmatrix} +
	\begin{vmatrix}
	a_{11} & a_{12} & \cdots & a_{1n} \\
	\vdots & \vdots & \ddots & \vdots \\
	b_{i1} & b_{i2} & \cdots & b_{in} \\
	\vdots & \vdots & \ddots & \vdots \\
	a_{n1} & a_{n2} & \cdots & a_{nn}
	\end{vmatrix}
	\end{equation*}
	Свойство для столбцов аналогично.
	\begin{proof}
	\begin{equation*}
	\Delta = \begin{vmatrix}
	a_{11} & a_{12} & \cdots & a_{1n} \\
	\vdots & \vdots & \ddots & \vdots \\
	a_{i1} + b_{i1} & a_{i2} + b_{i2} & \cdots & a_{in} + b_{in} \\
	\vdots & \vdots & \ddots & \vdots \\
	a_{n1} & a_{n2} & \cdots & a_{nn}
	\end{vmatrix} =
	\sum (-1)^{|\sigma|} a_{1\, i_1} \cdot \ldots \cdot a_{n\, i_n} =
	\end{equation*}
	\begin{equation*}
	\left| \text{ Каждое слагаемое содержит ровно~$1$ элемент из $i$\nobreakdash-й строки и поэтому имеет вид } \right|
	\end{equation*}
	\begin{equation*}
	= \sum (-1)^{|\sigma|} a_{1\, i_1} \cdot \ldots \cdot a_{k-1\, i_{k-1}} (a_{k\, i_k} + b_{k\, i_k}) a_{k+1\, i_{k+1}} \cdot \ldots \cdot a_{n\, i_n} =
	\end{equation*}
	\begin{equation*}
	= \sum (-1)^{|\sigma|} a_{1\, i_1} \cdot \ldots \cdot a_{k\, i_k} \cdot \ldots \cdot a_{n\, i_n} +
	\sum (-1)^{|\sigma|} a_{1\, i_1} \cdot \ldots \cdot b_{k\, i_k} \cdot \ldots \cdot a_{n\, i_n} =
	\end{equation*}
	\begin{equation*}
	= \begin{vmatrix}
	a_{11} & a_{12} & \cdots & a_{1n} \\
	\vdots & \vdots & \ddots & \vdots \\
	a_{i1} & a_{i2} & \cdots & a_{in} \\
	\vdots & \vdots & \ddots & \vdots \\
	a_{n1} & a_{n2} & \cdots & a_{nn}
	\end{vmatrix} +
	\begin{vmatrix}
	a_{11} & a_{12} & \cdots & a_{1n} \\
	\vdots & \vdots & \ddots & \vdots \\
	b_{i1} & b_{i2} & \cdots & b_{in} \\
	\vdots & \vdots & \ddots & \vdots \\
	a_{n1} & a_{n2} & \cdots & a_{nn}
	\end{vmatrix}
	\end{equation*}
	Свойство для столбцов доказывается аналогично.
	\end{proof}
	
	\item Если в~определителе поменять две строки или два столбца местами, то он изменит знак.
	\begin{proof}
	При перестановке строк или столбцов местами все перестановки в формуле~(\ref{eq:determinant}) меняют чётность, значит, каждое слагаемое меняет знак, тогда и определитель меняет знак.
	\end{proof}
	
	\item Если в~определителе две строки или два столбца совпадают, то он равен~$0$.
	\begin{proof}
	Если поменять местами совпадающие строки или столбцы, то он, с~одной стороны, не изменится, а с~другой, поменяет знак. Значит, определитель равен~$0$.
	\end{proof}
	
	\item \begin{equation*}
	\begin{vmatrix}
	a_{11} & a_{12} & \cdots & a_{1n} \\
	\vdots & \vdots & \ddots & \vdots \\
	a_{i1} & a_{i2} & \cdots & a_{in} \\
	\vdots & \vdots & \ddots & \vdots \\
	a_{j1} & a_{j2} & \cdots & a_{jn} \\
	\vdots & \vdots & \ddots & \vdots \\
	a_{n1} & a_{n2} & \cdots & a_{nn}
	\end{vmatrix} =
	\begin{vmatrix}
	a_{11} & a_{12} & \cdots & a_{1n} \\
	\vdots & \vdots & \ddots & \vdots \\
	a_{i1} & a_{i2} & \cdots & a_{in} \\
	\vdots & \vdots & \ddots & \vdots \\
	\lambda a_{i1} + a_{j1} & \lambda a_{i2} + a_{j2} & \cdots & \lambda a_{in} + a_{jn} \\
	\vdots & \vdots & \ddots & \vdots \\
	a_{n1} & a_{n2} & \cdots & a_{nn}
	\end{vmatrix}
	\end{equation*}
	Свойство для столбцов аналогично.
	\begin{proof}
	\begin{equation*}
	\begin{vmatrix}
	a_{11} & a_{12} & \cdots & a_{1n} \\
	\vdots & \vdots & \ddots & \vdots \\
	a_{i1} & a_{i2} & \cdots & a_{in} \\
	\vdots & \vdots & \ddots & \vdots \\
	a_{j1} & a_{j2} & \cdots & a_{jn} \\
	\vdots & \vdots & \ddots & \vdots \\
	a_{n1} & a_{n2} & \cdots & a_{nn}
	\end{vmatrix} =
	\begin{vmatrix}
	a_{11} & a_{12} & \cdots & a_{1n} \\
	\vdots & \vdots & \ddots & \vdots \\
	a_{i1} & a_{i2} & \cdots & a_{in} \\
	\vdots & \vdots & \ddots & \vdots \\
	\lambda a_{i1} & \lambda a_{i2} & \cdots & \lambda a_{in} \\
	\vdots & \vdots & \ddots & \vdots \\
	a_{n1} & a_{n2} & \cdots & a_{nn}
	\end{vmatrix} +
	\begin{vmatrix}
	a_{11} & a_{12} & \cdots & a_{1n} \\
	\vdots & \vdots & \ddots & \vdots \\
	a_{i1} & a_{i2} & \cdots & a_{in} \\
	\vdots & \vdots & \ddots & \vdots \\
	a_{j1} & a_{j2} & \cdots & a_{jn} \\
	\vdots & \vdots & \ddots & \vdots \\
	a_{n1} & a_{n2} & \cdots & a_{nn}
	\end{vmatrix} =
	\end{equation*}
	\begin{equation*}
	= \begin{vmatrix}
	a_{11} & a_{12} & \cdots & a_{1n} \\
	\vdots & \vdots & \ddots & \vdots \\
	a_{i1} & a_{i2} & \cdots & a_{in} \\
	\vdots & \vdots & \ddots & \vdots \\
	\lambda a_{i1} + a_{j1} & \lambda a_{i2} + a_{j2} & \cdots & \lambda a_{in} + a_{jn} \\
	\vdots & \vdots & \ddots & \vdots \\
	a_{n1} & a_{n2} & \cdots & a_{nn}
	\end{vmatrix}
	\end{equation*}
	Свойство для столбцов доказывается аналогично.
	\end{proof}
\end{itemize}

Пусть дана матрица
\begin{equation*}
\begin{Vmatrix}
a_{11} & a_{12} & \cdots & a_{1n} \\
a_{21} & a_{22} & \cdots & a_{2n} \\
\vdots & \vdots & \ddots & \vdots \\
a_{n1} & a_{n2} & \cdots & a_{nn}
\end{Vmatrix}
\end{equation*}

\textbf{Алгебраическим дополнением элемента~$a_{ij}$} называется число, равное
\begin{equation*}
A_{ij} = (-1)^{i+j} \cdot
\begin{vmatrix}
a_{11} & \cdots & a_{1\, j-1} & a_{1\, j+1} & \cdots & a_{1n} \\
\vdots & \ddots & \vdots & \vdots & \ddots & \vdots \\
a_{i-1\, 1} & \cdots & a_{i-1\, j-1} & a_{i-1\, j+1} & \cdots & a_{i-1\, n} \\
a_{i+1\, 1} & \cdots & a_{i+1\, j-1} & a_{i+1\, j+1} & \cdots & a_{i+1\, n} \\
\vdots & \ddots & \vdots & \vdots & \ddots & \vdots \\
a_{n1} & \cdots & a_{n\, j-1} & a_{n\, j+1} & \cdots & a_{nn}
\end{vmatrix}
\end{equation*}

\begin{lemma}
\begin{equation*}
\begin{vmatrix}
a & 0 & \cdots & 0 \\
a_{21} & a_{22} & \cdots & a_{2n} \\
\vdots & \vdots & \ddots & \vdots \\
a_{n1} & a_{n2} & \cdots & a_{nn}
\end{vmatrix} = a \cdot
\begin{vmatrix}
a_{22} & \cdots & a_{2n} \\
\vdots & \ddots & \vdots \\
a_{n2} & \cdots & a_{nn}
\end{vmatrix}
\end{equation*}
\end{lemma}
\begin{proof}
\begin{equation*}
\begin{vmatrix}
a & 0 & \cdots & 0 \\
a_{21} & a_{22} & \cdots & a_{2n} \\
\vdots & \vdots & \ddots & \vdots \\
a_{n1} & a_{n2} & \cdots & a_{nn}
\end{vmatrix} =
\end{equation*}
\begin{equation*}
= \sum a \cdot a_{2\, i_2} \cdot \ldots \cdot a_{n\, i_n} +
\sum 0 \cdot a_{2\, i_2} \cdot \ldots \cdot a_{n\, i_n} + \ldots +
\sum 0 \cdot a_{2\, i_2} \cdot \ldots \cdot a_{n\, i_n} =
\end{equation*}
\begin{equation*}
= a \sum a_{2\, i_2} \cdot \ldots \cdot a_{n\, i_n} = a \cdot
\begin{vmatrix}
a_{22} & \cdots & a_{2n} \\
\vdots & \ddots & \vdots \\
a_{n2} & \cdots & a_{nn}
\end{vmatrix}
\end{equation*}
\end{proof}

\begin{theorem}
\label{th:determinant_expansion}
Любой определитель можно \textbf{разложить по элементам} произвольной строки или столбца:
\begin{equation*}
\begin{vmatrix}
a_{11} & a_{12} & \cdots & a_{1n} \\
a_{21} & a_{22} & \cdots & a_{2n} \\
\vdots & \vdots & \ddots & \vdots \\
a_{n1} & a_{n2} & \cdots & a_{nn}
\end{vmatrix}
= \sum_{j=1}^n a_{ij} A_{ij}
= \sum_{i=1}^n a_{ij} A_{ij}
\end{equation*}
где $A_{ij}$~--- алгебраическое дополнение элемента~$a_{ij}$.
\end{theorem}
\begin{proof}
\begin{equation*}
\begin{vmatrix}
a_{11} & a_{12} & \cdots & a_{1n} \\
\vdots & \vdots & \ddots & \vdots \\
a_{i1} & a_{i2} & \cdots & a_{in} \\
\vdots & \vdots & \ddots & \vdots \\
a_{n1} & a_{n2} & \cdots & a_{nn}
\end{vmatrix} = (-1)^{i-1} \cdot
\begin{vmatrix}
a_{i1} & a_{i2} & \cdots & a_{in} \\
a_{11} & a_{12} & \cdots & a_{1n} \\
\vdots & \vdots & \ddots & \vdots \\
a_{i-1\, 1} & a_{i-1\, 2} & \cdots & a_{i-1\, n} \\
a_{i+1\, 1} & a_{i+1\, 2} & \cdots & a_{i+1\, n} \\
\vdots & \vdots & \ddots & \vdots \\
a_{n1} & a_{n2} & \cdots & a_{nn}
\end{vmatrix} = (-1)^{i+1} \cdot
\end{equation*}
\begin{equation*}
\cdot \left(
\begin{vmatrix}
a_{i1} & 0 & \cdots & 0 \\
a_{11} & a_{12} & \cdots & a_{1n} \\
\vdots & \vdots & \ddots & \vdots \\
a_{i-1\, 1} & a_{i-1\, 2} & \cdots & a_{i-1\, n} \\
a_{i+1\, 1} & a_{i+1\, 2} & \cdots & a_{i+1\, n} \\
\vdots & \vdots & \ddots & \vdots \\
a_{n1} & a_{n2} & \cdots & a_{nn}
\end{vmatrix} +
\begin{vmatrix}
0 & a_{i2} & \cdots & 0 \\
a_{11} & a_{12} & \cdots & a_{1n} \\
\vdots & \vdots & \ddots & \vdots \\
a_{i-1\, 1} & a_{i-1\, 2} & \cdots & a_{i-1\, n} \\
a_{i+1\, 1} & a_{i+1\, 2} & \cdots & a_{i+1\, n} \\
\vdots & \vdots & \ddots & \vdots \\
a_{n1} & a_{n2} & \cdots & a_{nn}
\end{vmatrix} + \ldots + 
\begin{vmatrix}
0 & 0 & \cdots & a_{in} \\
a_{11} & a_{12} & \cdots & a_{1n} \\
\vdots & \vdots & \ddots & \vdots \\
a_{i-1\, 1} & a_{i-1\, 2} & \cdots & a_{i-1\, n} \\
a_{i+1\, 1} & a_{i+1\, 2} & \cdots & a_{i+1\, n} \\
\vdots & \vdots & \ddots & \vdots \\
a_{n1} & a_{n2} & \cdots & a_{nn}
\end{vmatrix}
\right) =
\end{equation*}
\begin{equation*}
= (-1)^{i+1} \cdot \left(
\begin{vmatrix}
a_{i1} & 0 & \cdots & 0 \\
a_{11} & a_{12} & \cdots & a_{1n} \\
\vdots & \vdots & \ddots & \vdots \\
a_{i-1\, 1} & a_{i-1\, 2} & \cdots & a_{i-1\, n} \\
a_{i+1\, 1} & a_{i+1\, 2} & \cdots & a_{i+1\, n} \\
\vdots & \vdots & \ddots & \vdots \\
a_{n1} & a_{n2} & \cdots & a_{nn}
\end{vmatrix} -
\begin{vmatrix}
a_{i2} & 0 & 0 & \cdots & 0 \\
a_{12} & a_{11} & a_{13} & \cdots & a_{1n} \\
\vdots & \vdots & \vdots & \ddots & \vdots \\
a_{i-1\, 2} & a_{i-1\, 1} & a_{i-1\, 3} & \cdots & a_{i-1\, n} \\
a_{i+1\, 2} & a_{i+1\, 1} & a_{i+1\, 3} & \cdots & a_{i+1\, n} \\
\vdots & \vdots & \vdots & \ddots & \vdots \\
a_{n2} & a_{n1} & a_{n3} & \cdots & a_{nn}
\end{vmatrix} + \ldots + \right.
\end{equation*}
\begin{equation*}
\left. + (-1)^{n-1} \cdot
\begin{vmatrix}
a_{in} & 0 & \cdots & 0 \\
a_{1n} & a_{11} & \cdots & a_{1\, n-1} \\
\vdots & \vdots & \ddots & \vdots \\
a_{i-1\, n} & a_{i-1\, 1}  & \cdots & a_{i-1\, n-1} \\
a_{i+1\, n} & a_{i+1\, 1} & \cdots & a_{i+1\, n-1} \\
\vdots & \vdots & \ddots & \vdots \\
a_{nn} & a_{n1} & \cdots & a_{n\, n-1}
\end{vmatrix}
\right) =
\end{equation*}
\begin{equation*}
= (-1)^{i+1} a_{i1} \cdot
\begin{vmatrix}
a_{12} & \cdots & a_{1n} \\
\vdots & \ddots & \vdots \\
a_{i-1\, 2} & \cdots & a_{i-1\, n} \\
a_{i+1\, 2} & \cdots & a_{i+1\, n} \\
\vdots & \ddots & \vdots \\
a_{n2} & \cdots & a_{nn}
\end{vmatrix} + (-1)^{i+2} a_{i2} \cdot
\begin{vmatrix}
a_{11} & a_{13} & \cdots & a_{1n} \\
\vdots & \vdots & \ddots & \vdots \\
a_{i-1\, 1} & a_{i-1\, 3} & \cdots & a_{i-1\, n} \\
a_{i+1\, 1} & a_{i+1\, 3} & \cdots & a_{i+1\, n} \\
\vdots & \vdots & \ddots & \vdots \\
a_{n1} & a_{n3} & \cdots & a_{nn}
\end{vmatrix} + \ldots +
\end{equation*}
\begin{equation*}
+ (-1)^{i+n} a_{in} \cdot
\begin{vmatrix}
a_{11} & \cdots & a_{1\, n-1} \\
\vdots & \ddots & \vdots \\
a_{i-1\, 1}  & \cdots & a_{i-1\, n-1} \\
a_{i+1\, 1} & \cdots & a_{i+1\, n-1} \\
\vdots & \ddots & \vdots \\
a_{n1} & \cdots & a_{n\, n-1}
\end{vmatrix}
= \sum_{j=1}^n a_{ij} A_{ij}
\end{equation*}

Аналогично доказывается
\begin{equation*}
\begin{vmatrix}
a_{11} & a_{12} & \cdots & a_{1n} \\
a_{21} & a_{22} & \cdots & a_{2n} \\
\vdots & \vdots & \ddots & \vdots \\
a_{n1} & a_{n2} & \cdots & a_{nn}
\end{vmatrix}
= \sum_{i=1}^n a_{ij} A_{ij}
\end{equation*}
\end{proof}

\begin{statement}
Определитель транспонированной матрицы равен определителю исходной.
\end{statement}
\begin{proof}
\begin{equation*}
\begin{vmatrix}
a_{11} & a_{12} & \cdots & a_{1n} \\
a_{21} & a_{22} & \cdots & a_{2n} \\
\vdots & \vdots & \ddots & \vdots \\
a_{n1} & a_{n2} & \cdots & a_{nn}
\end{vmatrix}^T =
\begin{vmatrix}
a_{11} & a_{21} & \cdots & a_{n1} \\
a_{12} & a_{22} & \cdots & a_{n2} \\
\vdots & \vdots & \ddots & \vdots \\
a_{1n} & a_{2n} & \cdots & a_{nn}
\end{vmatrix} = \sum_{j=1}^n a_{1j} A_{1j} =
\begin{vmatrix}
a_{11} & a_{12} & \cdots & a_{1n} \\
a_{21} & a_{22} & \cdots & a_{2n} \\
\vdots & \vdots & \ddots & \vdots \\
a_{n1} & a_{n2} & \cdots & a_{nn}
\end{vmatrix}
\end{equation*}
\end{proof}