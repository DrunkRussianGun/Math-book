\subsection{Определитель матрицы}
\index{det}\index{Определитель} \textbf{Определителем порядка~$n$} квадратной {матрицы~$A$} порядка~$n$, называется число, равное
\begin{equation}
\label{eq:determinant}
\Delta = \det A = |A| =
\begin{vmatrix}
a_{11} & a_{12} & \cdots & a_{1n} \\
a_{21} & a_{22} & \cdots & a_{2n} \\
\vdots & \vdots & \ddots & \vdots \\
a_{n1} & a_{n2} & \cdots & a_{nn}
\end{vmatrix} =
\sum_{\sigma = (i_1, \ldots, i_n) \in S_n} (-1)^{|\sigma|} a_{1\, i_1} a_{2\, i_2} \cdot \ldots \cdot a_{n\, i_n}, \ 
|\sigma| =
\begin{cases}
0, \sigma \text{ чётная} \\
1, \sigma \text{ нечётная}
\end{cases}
\end{equation}
где $S_n$~--- множество всех перестановок $n$-элементного множества.

Матрица называется \textbf{вырожденной}, если её определитель равен~$0$, иначе~--- \textbf{невырожденной}.

Свойства определителя:
\begin{itemize}
	\item Если элементы какой-либо строки или столбца определителя имеют общий множитель~$\lambda$, то его можно вынести за знак определителя.
	\begin{proof}
	\begin{equation*}
	\Delta = \sum (-1)^{|\sigma|} a_{1\, i_1} a_{2\, i_2} \cdot \ldots \cdot a_{n\, i_n}
	\end{equation*}
	Каждое слагаемое имеет множитель из каждой строки, а также из каждого столбца, т.\,к. $\sigma$ является перестановкой и содержит все номера столбцов от $1$ до $n$ включительно.
	Тогда все слагаемые имеют общий множитель~$\lambda$, поэтому его можно вынести за скобки.
	\end{proof}
	
	\item Если какая-либо строка или столбец определителя состоит из нулей, то он равен~$0$.
	
	\item \begin{equation*}
	\begin{vmatrix}
	a_{11} & a_{12} & \cdots & a_{1n} \\
	\vdots & \vdots & \ddots & \vdots \\
	a_{i1} + b_{i1} & a_{i2} + b_{i2} & \cdots & a_{in} + b_{in} \\
	\vdots & \vdots & \ddots & \vdots \\
	a_{n1} & a_{n2} & \cdots & a_{nn}
	\end{vmatrix} =
	\begin{vmatrix}
	a_{11} & a_{12} & \cdots & a_{1n} \\
	\vdots & \vdots & \ddots & \vdots \\
	a_{i1} & a_{i2} & \cdots & a_{in} \\
	\vdots & \vdots & \ddots & \vdots \\
	a_{n1} & a_{n2} & \cdots & a_{nn}
	\end{vmatrix} +
	\begin{vmatrix}
	a_{11} & a_{12} & \cdots & a_{1n} \\
	\vdots & \vdots & \ddots & \vdots \\
	b_{i1} & b_{i2} & \cdots & b_{in} \\
	\vdots & \vdots & \ddots & \vdots \\
	a_{n1} & a_{n2} & \cdots & a_{nn}
	\end{vmatrix}
	\end{equation*}
	Свойство для столбцов аналогично.
	\begin{proof}
	\begin{equation*}
	\Delta = \begin{vmatrix}
	a_{11} & a_{12} & \cdots & a_{1n} \\
	\vdots & \vdots & \ddots & \vdots \\
	a_{i1} + b_{i1} & a_{i2} + b_{i2} & \cdots & a_{in} + b_{in} \\
	\vdots & \vdots & \ddots & \vdots \\
	a_{n1} & a_{n2} & \cdots & a_{nn}
	\end{vmatrix} =
	\sum (-1)^{|\sigma|} a_{1\, i_1} \cdot \ldots \cdot a_{n\, i_n} =
	\end{equation*}
	\begin{equation*}
	\left| \text{ Каждое слагаемое содержит ровно~$1$ элемент из $i$\nobreakdash-й строки и поэтому имеет вид } \right|
	\end{equation*}
	\begin{equation*}
	= \sum (-1)^{|\sigma|} a_{1\, i_1} \cdot \ldots \cdot a_{k-1\, i_{k-1}} (a_{k\, i_k} + b_{k\, i_k}) a_{k+1\, i_{k+1}} \cdot \ldots \cdot a_{n\, i_n} =
	\end{equation*}
	\begin{equation*}
	= \sum (-1)^{|\sigma|} a_{1\, i_1} \cdot \ldots \cdot a_{k\, i_k} \cdot \ldots \cdot a_{n\, i_n} +
	\sum (-1)^{|\sigma|} a_{1\, i_1} \cdot \ldots \cdot b_{k\, i_k} \cdot \ldots \cdot a_{n\, i_n} =
	\end{equation*}
	\begin{equation*}
	= \begin{vmatrix}
	a_{11} & a_{12} & \cdots & a_{1n} \\
	\vdots & \vdots & \ddots & \vdots \\
	a_{i1} & a_{i2} & \cdots & a_{in} \\
	\vdots & \vdots & \ddots & \vdots \\
	a_{n1} & a_{n2} & \cdots & a_{nn}
	\end{vmatrix} +
	\begin{vmatrix}
	a_{11} & a_{12} & \cdots & a_{1n} \\
	\vdots & \vdots & \ddots & \vdots \\
	b_{i1} & b_{i2} & \cdots & b_{in} \\
	\vdots & \vdots & \ddots & \vdots \\
	a_{n1} & a_{n2} & \cdots & a_{nn}
	\end{vmatrix}
	\end{equation*}
	Свойство для столбцов доказывается аналогично.
	\end{proof}
	
	\item Если в~определителе поменять две строки или два столбца местами, то он изменит знак.
	\begin{proof}
	При перестановке строк или столбцов местами по утверждению~\ref{st:parity_of_permutation} все перестановки в формуле~(\ref{eq:determinant}) меняют чётность, значит, каждое слагаемое меняет знак, тогда и определитель меняет знак.
	\end{proof}
	
	\item Если в~определителе две строки или два столбца совпадают, то он равен~$0$.
	\begin{proof}
	Если поменять местами совпадающие строки или столбцы, то он, с~одной стороны, не изменится, а с~другой, поменяет знак. Значит, определитель равен~$0$.
	\end{proof}
	
	\item \begin{equation*}
	\begin{vmatrix}
	a_{11} & a_{12} & \cdots & a_{1n} \\
	\vdots & \vdots & \ddots & \vdots \\
	a_{i1} & a_{i2} & \cdots & a_{in} \\
	\vdots & \vdots & \ddots & \vdots \\
	a_{j1} & a_{j2} & \cdots & a_{jn} \\
	\vdots & \vdots & \ddots & \vdots \\
	a_{n1} & a_{n2} & \cdots & a_{nn}
	\end{vmatrix} =
	\begin{vmatrix}
	a_{11} & a_{12} & \cdots & a_{1n} \\
	\vdots & \vdots & \ddots & \vdots \\
	a_{i1} & a_{i2} & \cdots & a_{in} \\
	\vdots & \vdots & \ddots & \vdots \\
	\lambda a_{i1} + a_{j1} & \lambda a_{i2} + a_{j2} & \cdots & \lambda a_{in} + a_{jn} \\
	\vdots & \vdots & \ddots & \vdots \\
	a_{n1} & a_{n2} & \cdots & a_{nn}
	\end{vmatrix}
	\end{equation*}
	Свойство для столбцов аналогично.
	\begin{proof}
	\begin{equation*}
	\begin{vmatrix}
	a_{11} & a_{12} & \cdots & a_{1n} \\
	\vdots & \vdots & \ddots & \vdots \\
	a_{i1} & a_{i2} & \cdots & a_{in} \\
	\vdots & \vdots & \ddots & \vdots \\
	a_{j1} & a_{j2} & \cdots & a_{jn} \\
	\vdots & \vdots & \ddots & \vdots \\
	a_{n1} & a_{n2} & \cdots & a_{nn}
	\end{vmatrix} =
	\begin{vmatrix}
	a_{11} & a_{12} & \cdots & a_{1n} \\
	\vdots & \vdots & \ddots & \vdots \\
	a_{i1} & a_{i2} & \cdots & a_{in} \\
	\vdots & \vdots & \ddots & \vdots \\
	\lambda a_{i1} & \lambda a_{i2} & \cdots & \lambda a_{in} \\
	\vdots & \vdots & \ddots & \vdots \\
	a_{n1} & a_{n2} & \cdots & a_{nn}
	\end{vmatrix} +
	\begin{vmatrix}
	a_{11} & a_{12} & \cdots & a_{1n} \\
	\vdots & \vdots & \ddots & \vdots \\
	a_{i1} & a_{i2} & \cdots & a_{in} \\
	\vdots & \vdots & \ddots & \vdots \\
	a_{j1} & a_{j2} & \cdots & a_{jn} \\
	\vdots & \vdots & \ddots & \vdots \\
	a_{n1} & a_{n2} & \cdots & a_{nn}
	\end{vmatrix} =
	\end{equation*}
	\begin{equation*}
	= \begin{vmatrix}
	a_{11} & a_{12} & \cdots & a_{1n} \\
	\vdots & \vdots & \ddots & \vdots \\
	a_{i1} & a_{i2} & \cdots & a_{in} \\
	\vdots & \vdots & \ddots & \vdots \\
	\lambda a_{i1} + a_{j1} & \lambda a_{i2} + a_{j2} & \cdots & \lambda a_{in} + a_{jn} \\
	\vdots & \vdots & \ddots & \vdots \\
	a_{n1} & a_{n2} & \cdots & a_{nn}
	\end{vmatrix}
	\end{equation*}
	Свойство для столбцов доказывается аналогично.
	\end{proof}
\end{itemize}

Рассмотрим квадратную матрицу~$A$ $n$-го порядка.
Пусть $1 \leqslant i_1 < i_2 < \ldots < i_k \leqslant n$, $1 \opbr\leqslant j_1 \opbr< j_2 \opbr< \ldots \opbr< j_k \opbr\leqslant n$.
\index{Минор} \textbf{Минором $k$-го порядка матрицы~$A$} называется определитель, образованный элементами матрицы, стоящими на пересечении строк с номерами~$i_1, i_2, \ldots, i_k$ и столбцов с номерами~$j_1, j_2, \ldots, j_k$, и обозначается~$M_{j_1 j_2 \ldots j_k}^{i_1 i_2 \ldots i_k}$.
\textbf{Дополнительным минором $n - k$-го порядка к минору $M_{j_1 j_2 \ldots j_k}^{i_1 i_2 \ldots i_k}$} называется определитель, полученный вычеркиванием строк с номерами~$i_1, i_2, \ldots, i_k$ и столбцов с номерами~$j_1, j_2, \ldots, j_k$ из определителя матрицы~$A$, и обозначается $\overline M_{j_1 j_2 \ldots j_k}^{i_1 i_2 \ldots i_k}$.
\index{Алгебраическое дополнение} \textbf{Алгебраическим дополнением элемента~$a_{ij}$} матрицы~$A$ называется величина, равная $(-1)^{i+j} \overline M_j^i$, и обозначается $A_{ij}$.

\begin{theorem}
\label{th:determinant_expansion}
Любой определитель можно \textbf{разложить по элементам} произвольной строки или столбца:
\begin{equation*}
\begin{vmatrix}
a_{11} & a_{12} & \cdots & a_{1n} \\
a_{21} & a_{22} & \cdots & a_{2n} \\
\vdots & \vdots & \ddots & \vdots \\
a_{n1} & a_{n2} & \cdots & a_{nn}
\end{vmatrix} =
\sum_{j=1}^n a_{ij} A_{ij} =
\sum_{i=1}^n a_{ij} A_{ij}
\end{equation*}
\end{theorem}
\begin{proof}
\begin{equation*}
\begin{vmatrix}
a_{11} & a_{12} & \cdots & a_{1n} \\
\vdots & \vdots & \ddots & \vdots \\
a_{i1} & a_{i2} & \cdots & a_{in} \\
\vdots & \vdots & \ddots & \vdots \\
a_{n1} & a_{n2} & \cdots & a_{nn}
\end{vmatrix} =
(-1)^{i-1} \cdot
\begin{vmatrix}
a_{i1} & a_{i2} & \cdots & a_{in} \\
a_{11} & a_{12} & \cdots & a_{1n} \\
\vdots & \vdots & \ddots & \vdots \\
a_{i-1\, 1} & a_{i-1\, 2} & \cdots & a_{i-1\, n} \\
a_{i+1\, 1} & a_{i+1\, 2} & \cdots & a_{i+1\, n} \\
\vdots & \vdots & \ddots & \vdots \\
a_{n1} & a_{n2} & \cdots & a_{nn}
\end{vmatrix} =
(-1)^{i+1} \cdot
\end{equation*}
\begin{equation*}
\cdot \left(
\begin{vmatrix}
a_{i1} & 0 & \cdots & 0 \\
a_{11} & a_{12} & \cdots & a_{1n} \\
\vdots & \vdots & \ddots & \vdots \\
a_{i-1\, 1} & a_{i-1\, 2} & \cdots & a_{i-1\, n} \\
a_{i+1\, 1} & a_{i+1\, 2} & \cdots & a_{i+1\, n} \\
\vdots & \vdots & \ddots & \vdots \\
a_{n1} & a_{n2} & \cdots & a_{nn}
\end{vmatrix} +
\begin{vmatrix}
0 & a_{i2} & \cdots & 0 \\
a_{11} & a_{12} & \cdots & a_{1n} \\
\vdots & \vdots & \ddots & \vdots \\
a_{i-1\, 1} & a_{i-1\, 2} & \cdots & a_{i-1\, n} \\
a_{i+1\, 1} & a_{i+1\, 2} & \cdots & a_{i+1\, n} \\
\vdots & \vdots & \ddots & \vdots \\
a_{n1} & a_{n2} & \cdots & a_{nn}
\end{vmatrix} + \ldots + 
\begin{vmatrix}
0 & 0 & \cdots & a_{in} \\
a_{11} & a_{12} & \cdots & a_{1n} \\
\vdots & \vdots & \ddots & \vdots \\
a_{i-1\, 1} & a_{i-1\, 2} & \cdots & a_{i-1\, n} \\
a_{i+1\, 1} & a_{i+1\, 2} & \cdots & a_{i+1\, n} \\
\vdots & \vdots & \ddots & \vdots \\
a_{n1} & a_{n2} & \cdots & a_{nn}
\end{vmatrix}
\right) =
\end{equation*}
\begin{equation*}
= (-1)^{i+1} \cdot \left(
\begin{vmatrix}
a_{i1} & 0 & \cdots & 0 \\
a_{11} & a_{12} & \cdots & a_{1n} \\
\vdots & \vdots & \ddots & \vdots \\
a_{i-1\, 1} & a_{i-1\, 2} & \cdots & a_{i-1\, n} \\
a_{i+1\, 1} & a_{i+1\, 2} & \cdots & a_{i+1\, n} \\
\vdots & \vdots & \ddots & \vdots \\
a_{n1} & a_{n2} & \cdots & a_{nn}
\end{vmatrix} -
\begin{vmatrix}
a_{i2} & 0 & 0 & \cdots & 0 \\
a_{12} & a_{11} & a_{13} & \cdots & a_{1n} \\
\vdots & \vdots & \vdots & \ddots & \vdots \\
a_{i-1\, 2} & a_{i-1\, 1} & a_{i-1\, 3} & \cdots & a_{i-1\, n} \\
a_{i+1\, 2} & a_{i+1\, 1} & a_{i+1\, 3} & \cdots & a_{i+1\, n} \\
\vdots & \vdots & \vdots & \ddots & \vdots \\
a_{n2} & a_{n1} & a_{n3} & \cdots & a_{nn}
\end{vmatrix} + \ldots + \right.
\end{equation*}
\begin{equation*}
\left. \vphantom1 + (-1)^{n-1} \cdot
\begin{vmatrix}
a_{in} & 0 & \cdots & 0 \\
a_{1n} & a_{11} & \cdots & a_{1\, n-1} \\
\vdots & \vdots & \ddots & \vdots \\
a_{i-1\, n} & a_{i-1\, 1}  & \cdots & a_{i-1\, n-1} \\
a_{i+1\, n} & a_{i+1\, 1} & \cdots & a_{i+1\, n-1} \\
\vdots & \vdots & \ddots & \vdots \\
a_{nn} & a_{n1} & \cdots & a_{n\, n-1}
\end{vmatrix}
\right) =
\end{equation*}
\begin{equation*}
\left| \begin{gathered}
\text{Пользуясь формулой~(\ref*{eq:determinant}), получим} \\
\begin{vmatrix}
a & 0 & \cdots & 0 \\
a_{21} & a_{22} & \cdots & a_{2n} \\
\vdots & \vdots & \ddots & \vdots \\
a_{n1} & a_{n2} & \cdots & a_{nn}
\end{vmatrix} = \\
= \sum a \cdot a_{2\, i_2} \cdot \ldots \cdot a_{n\, i_n} +
\sum 0 \cdot a_{2\, i_2} \cdot \ldots \cdot a_{n\, i_n} + \ldots +
\sum 0 \cdot a_{2\, i_2} \cdot \ldots \cdot a_{n\, i_n} = \\
= a \sum a_{2\, i_2} \cdot \ldots \cdot a_{n\, i_n} =
a \cdot
\begin{vmatrix}
a_{22} & \cdots & a_{2n} \\
\vdots & \ddots & \vdots \\
a_{n2} & \cdots & a_{nn}
\end{vmatrix}
\end{gathered} \right|
\end{equation*}
\begin{equation*}
= (-1)^{i+1} a_{i1} \cdot
\begin{vmatrix}
a_{12} & \cdots & a_{1n} \\
\vdots & \ddots & \vdots \\
a_{i-1\, 2} & \cdots & a_{i-1\, n} \\
a_{i+1\, 2} & \cdots & a_{i+1\, n} \\
\vdots & \ddots & \vdots \\
a_{n2} & \cdots & a_{nn}
\end{vmatrix} +
(-1)^{i+2} a_{i2} \cdot
\begin{vmatrix}
a_{11} & a_{13} & \cdots & a_{1n} \\
\vdots & \vdots & \ddots & \vdots \\
a_{i-1\, 1} & a_{i-1\, 3} & \cdots & a_{i-1\, n} \\
a_{i+1\, 1} & a_{i+1\, 3} & \cdots & a_{i+1\, n} \\
\vdots & \vdots & \ddots & \vdots \\
a_{n1} & a_{n3} & \cdots & a_{nn}
\end{vmatrix} + \ldots +
\end{equation*}
\begin{equation*}
\vphantom1 + (-1)^{i+n} a_{in} \cdot
\begin{vmatrix}
a_{11} & \cdots & a_{1\, n-1} \\
\vdots & \ddots & \vdots \\
a_{i-1\, 1}  & \cdots & a_{i-1\, n-1} \\
a_{i+1\, 1} & \cdots & a_{i+1\, n-1} \\
\vdots & \ddots & \vdots \\
a_{n1} & \cdots & a_{n\, n-1}
\end{vmatrix} =
\sum_{j=1}^n a_{ij} A_{ij}
\end{equation*}

Аналогично доказывается
\begin{equation*}
\begin{vmatrix}
a_{11} & a_{12} & \cdots & a_{1n} \\
a_{21} & a_{22} & \cdots & a_{2n} \\
\vdots & \vdots & \ddots & \vdots \\
a_{n1} & a_{n2} & \cdots & a_{nn}
\end{vmatrix} =
\sum_{i=1}^n a_{ij} A_{ij}
\end{equation*}
\end{proof}

\begin{consequent}[фальшивое разложение определителя]
Пусть дана квадратная матрица~$A = \|a_{ij}\|$ $n$-го порядка, тогда
\begin{equation*}
\sum_{k=1}^n a_{ik} A_{jk} = \sum_{k=1}^n a_{ki} A_{kj} = 0, \ i \neq j
\end{equation*}
\end{consequent}
\begin{proof}
\begin{equation*}
\sum_{k=1}^n a_{ik} A_{jk} =
\begin{vmatrix}
a_{11} & a_{12} & \cdots & a_{1n} \\
\vdots & \vdots & \ddots & \vdots \\
a_{i1} & a_{i2} & \cdots & a_{in} \\
\vdots & \vdots & \ddots & \vdots \\
a_{i1} & a_{i2} & \cdots & a_{in} \\
\vdots & \vdots & \ddots & \vdots \\
a_{n1} & a_{n2} & \cdots & a_{nn}
\end{vmatrix} = 0 =
\begin{vmatrix}
a_{11} & \cdots & a_{1i} & \cdots & a_{1i} & \cdots & a_{1n} \\
a_{21} & \cdots & a_{2i} & \cdots & a_{2i} & \cdots & a_{2n} \\
\vdots & \ddots & \vdots & \ddots & \vdots & \ddots & \vdots \\
a_{n1} & \cdots & a_{ni} & \cdots & a_{ni} & \cdots & a_{nn}
\end{vmatrix} =
\sum_{k=1}^n a_{ki} A_{kj}
\end{equation*}
\end{proof}

\begin{statement}
Определитель транспонированной матрицы равен определителю исходной.
\end{statement}
\begin{proof}
\begin{equation*}
\begin{vmatrix}
a_{11} & a_{12} & \cdots & a_{1n} \\
a_{21} & a_{22} & \cdots & a_{2n} \\
\vdots & \vdots & \ddots & \vdots \\
a_{n1} & a_{n2} & \cdots & a_{nn}
\end{vmatrix}^T =
\begin{vmatrix}
a_{11} & a_{21} & \cdots & a_{n1} \\
a_{12} & a_{22} & \cdots & a_{n2} \\
\vdots & \vdots & \ddots & \vdots \\
a_{1n} & a_{2n} & \cdots & a_{nn}
\end{vmatrix} =
\sum_{j=1}^n a_{1j} A_{1j} =
\begin{vmatrix}
a_{11} & a_{12} & \cdots & a_{1n} \\
a_{21} & a_{22} & \cdots & a_{2n} \\
\vdots & \vdots & \ddots & \vdots \\
a_{n1} & a_{n2} & \cdots & a_{nn}
\end{vmatrix}
\end{equation*}
\end{proof}

\begin{theorem}[Лапласа]
Пусть дана квадратная матрица~$A$ $n$-го порядка.
\begin{equation*}
\forall 0 < k < n, \ 1 \leqslant i_1 < i_2 < \ldots < i_k \leqslant n \
\det A = \sum_{1 \leqslant j_1 < \ldots < j_k \leqslant n}
(-1)^{i_1 + \ldots + i_k + j_1 + \ldots + j_k}
M_{j_1 \ldots j_k}^{i_1 \ldots i_k}
\overline M_{j_1 \ldots j_k}^{i_1 \ldots i_k}
\end{equation*}
\end{theorem}
\begin{proofmathind}
	\indbase При~$k = 1$ данная теорема эквивалентна теореме~\ref*{th:determinant_expansion}.
	\indstep Пусть теорема верна при~$k - 1$. Докажем её для~$k$.
	\begin{equation*}
	\det A = \sum_{1 \leqslant j_1 < \ldots < j_{k-1} \leqslant n}
	(-1)^{i_1 + \ldots + i_{k-1} + j_1 + \ldots + j_{k-1}}
	M_{j_1 \ldots j_{k-1}}^{i_1 \ldots i_{k-1}}
	\overline M_{j_1 \ldots j_{k-1}}^{i_1 \ldots i_{k-1}} =
	\end{equation*}
	\begin{equation*}
	\left| \text{Разложим каждый минор~$\overline M_{j_1 \ldots j_{k-1}}^{i_1 \ldots i_{k-1}}$ по строке~$A_{i_k}$, полагая, что $\Theta_{j_1 \ldots j_k}$~--- некоторое число} \right|
	\end{equation*}
	\begin{equation*}
	= \sum_{1 \leqslant j_1 < \ldots < j_k \leqslant n}
	\Theta_{j_1 \ldots j_k} \overline M_{j_1 \ldots j_k}^{i_1 \ldots i_k}
	\end{equation*}
	
	Найдём значение~$\Theta_{j_1 \ldots j_k}$.
	Заметим, что минор~$\overline M_{j_1 \ldots j_k}^{i_1 \ldots i_k}$ получается при разложении только миноров
	\begin{equation*}
	\overline M_{j_1 \ldots j_{s-1} j_{s+1} \ldots j_k}^{i_1 \ldots i_{k-1}}, \ s = 1, 2, \ldots, k
	\end{equation*}
	причём
	\begin{equation*}
	\overline M_{j_1 \ldots j_{s-1} j_{s+1} \ldots j_k}^{i_1 \ldots i_{k-1}} =
	(-1)^{i_k - (k - 1) + j_s - (s - 1)} a_{i_k j_s}
	\overline M_{j_1 \ldots j_k}^{i_1 \ldots i_k} + \ldots
	\end{equation*}
	где многоточием обозначены остальные слагаемые.
	
	Тогда
	\begin{equation*}
	\Theta_{j_1 \ldots j_k} =
	(-1)^{i_1 + \ldots + i_k + j_1 + \ldots + j_k}
	\sum_{s=1}^k (-1)^{k+s} M_{j_1 \ldots j_{s-1} j_{s+1} \ldots j_k}^{i_1 \ldots i_{k-1}} =
	(-1)^{i_1 + \ldots + i_k + j_1 + \ldots + j_k}
	M_{j_1 \ldots j_k}^{i_1 \ldots i_k}
	\end{equation*}
	\indend
\end{proofmathind}

\begin{theorem}
Если $A = \|a_{ij}\|, B = \|b_{ij}\|$~--- квадратные матрицы $n$-го порядка, то $\det AB \opbr= \det A \opbr\cdot \det B$.
\end{theorem}
\begin{proof}
Пусть $O, E$~--- нулевая и единичная соответственно квадратные матрицы $n$-го порядка, $C = AB$.
Рассмотрим блочные матрицы
\begin{equation*}
\begin{Vmatrix}
A & O \\
-E & B
\end{Vmatrix}, \
\begin{Vmatrix}
A & C \\
-E & O
\end{Vmatrix}
\end{equation*}

Раскладывая первую матрицу по первым $n$~строкам, а вторую~--- по последним $n$~строкам, получим
\begin{equation*}
\begin{vmatrix}
A & O \\
-E & B
\end{vmatrix} =
|A| |B|, \
\begin{vmatrix}
A & C \\
-E & O
\end{vmatrix} =
(-1)^{1 + \ldots + 2n}|-E| |C| =
|C|
\end{equation*}

Тогда
\begin{equation*}
\det A \cdot \det B =
\begin{vmatrix}
A & O \\
-E & B
\end{vmatrix} =
\begin{vmatrix}
a_{11} & \cdots & a_{1n} & 0 & \cdots & 0 \\
\vdots & \ddots & \vdots & \vdots & \ddots & \vdots \\
a_{n1} & \cdots & a_{nn} & 0 & \cdots & 0 \\
-1 & \cdots & 0 & b_{11} & \cdots & b_{1n} \\
\vdots & \ddots & \vdots & \vdots & \ddots & \vdots \\
0 & \cdots & -1 & b_{n1} & \cdots & b_{nn}
\end{vmatrix} =
\end{equation*}
\begin{equation*}
= \begin{vmatrix}
a_{11} & \cdots & a_{1n} & \sum\limits_{i=1}^n a_{1i} b_{i1} & \cdots & \sum\limits_{i=1}^n a_{1i} b_{in} \\
\vdots & \ddots & \vdots & \vdots & \ddots & \vdots \\
a_{n1} & \cdots & a_{nn} & \sum\limits_{i=1}^n a_{ni} b_{i1} & \cdots & \sum\limits_{i=1}^n a_{ni} b_{in} \\
-1 & \cdots & 0 & 0 & \cdots & 0 \\
\vdots & \ddots & \vdots & \vdots & \ddots & \vdots \\
0 & \cdots & -1 & 0 & \cdots & 0
\end{vmatrix} =
\begin{vmatrix}
A & C \\
-E & O
\end{vmatrix} =
\det C
\end{equation*}
\end{proof}