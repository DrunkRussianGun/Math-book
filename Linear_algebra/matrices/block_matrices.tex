\subsection{Блочные матрицы}
\index{Матрица!блочная} Если матрицу при помощи горизонтальных и вертикальных прямых разделить на прямоугольные клетки, называемые \textbf{блоками}, то получится \textbf{блочная матрица}, состоящая из блоков, которые, в~свою очередь, также являются матрицами.
Легко проверить непосредственно, что операции над блочными матрицами осуществляются так~же, как и над обычными.