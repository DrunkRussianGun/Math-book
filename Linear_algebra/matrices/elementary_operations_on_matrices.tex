\subsection{Элементарные преобразования матриц}
\textbf{Элементарными преобразованиями} называются следующие операции над матрицей:
\begin{itemize}
	\item\textbf{Перестановка строк} матрицы~--- преобразование I типа
	\item\textbf{Умножение строки на~$\lambda \neq 0$}~--- преобразование II типа
	\item\textbf{Прибавление к строке матрицы другой строки, умноженной на~$\lambda$.}
\end{itemize}

Аналогично определяются элементарные преобразования над столбцами.

\begin{theorem}
Элементарные преобразования матрицы не изменяют её ранг.
\end{theorem}
\begin{proof}
Для доказательства достаточно показать, что в результате элементарных преобразований равенство определителя с нулём сохраняется.
\begin{itemize}
	\item Перестановка строк матрицы изменяет только знак определителя.
	\item Умножение строки матрицы на ненулевое число приводит к умножению определителя на это~же число.
	\item Прибавление к строке матрицы другой строки, умноженной на некоторое число, не изменяет определитель.
\end{itemize}
\end{proof}

Матрица~$A$ имеет \textbf{ступенчатый вид}, если:
\begin{itemize}
	\item все нулевые строки стоят последними;
	\item для любой ненулевой строки~$A_p$ верно, что $\forall i > p, \ j \leqslant q \ a_{ij} = 0$, где $a_{pq}$~--- первый ненулевой элемент строки~$A_p$.
\end{itemize}

\begin{theorem}
Любую матрицу путём элементарных преобразований только над строками можно привести к ступенчатому виду.
\end{theorem}
\begin{proof}
Приведём алгоритм, преобразующий любую матрицу~$||a_{ij}||_{\begin{smallmatrix}
i = \overline{1,m} \\
j = \overline{1,n}
\end{smallmatrix}}$ к ступенчатому виду путём элементарных преобразований только над строками.
В~качестве текущего элемента возьмём $a_{11}$.
\begin{enumerate}
	\item Если текущий элемент~$a_{ij} = 0$, то переходим к шагу~2, иначе к каждой строке~$A_k$, где $k \opbr= i + 1, i + 2, \ldots, n$, добавляем строку~$-\frac{a_{kj}}{a_{ij}} A_i$.
	Если $i = m$ или $j = n$, то матрица приведена к ступенчатому виду, иначе выбираем новый текущий элемент~$a_{i+1\, j+1}$ и повторяем шаг~1.
	\item Просматриваем элементы матрицы, расположенные под текущим элементом~$a_{ij}$.
	Если $a_{kj} \neq 0$, то меняем местами строки $A_i$ и $A_k$ и переходим к шагу~1, иначе переходим к шагу~3.
	\item Пусть $a_{ij}$~--- текущий элемент.
	Если $j = n$, то матрица приведена к ступенчатому виду, иначе выбираем новый текущий элемент~$a_{i\, j + 1}$ и переходим к шагу~1.
\end{enumerate}

Матрица имеет конечные размеры, а положение текущего элемента смещается как минимум на~$1$ столбец вправо за не более, чем $3$~шага, поэтому алгоритм закончит работу за не более, чем $3n$~шагов.
\end{proof}