\subsection{Ранг матрицы}
Строка (столбец) матрицы называется \textbf{линейно зависимой}, если она является линейной комбинацией остальных строк (столбцов), иначе~--- \textbf{линейно независимой}.

\index{Ранг} \textbf{Рангом} матрицы называется максимальное количество её линейно независимых строк.

\index{Минор} \textbf{Минором} $k$-го порядка матрицы называется определитель, содержащий только те её элементы, которые стоят на пересечении некоторых $k$~строк и $k$~столбцов.
Минор наибольшего порядка, отличный от нуля, называется \textbf{базисным}.

\begin{theorem}
Ранг матрицы равен порядку базисного минора.
\end{theorem}
\begin{proof}
Пусть
\begin{equation*}
A =
\begin{Vmatrix}
a_{11} & a_{12} & \cdots & a_{1n} \\
a_{21} & a_{22} & \cdots & a_{2n} \\
\vdots & \vdots & \ddots & \vdots \\
a_{n1} & a_{n2} & \cdots & a_{nn}
\end{Vmatrix},
\end{equation*}
$M_k$~--- базисный минор $k$\nobreakdash-го порядка.
При перестановке строк и столбцов минора равенство с нулём сохраняется, значит, без ограничения общности можно считать, что
\begin{equation*}
M_k =
\begin{vmatrix}
a_{11} & a_{12} & \cdots & a_{1k} \\
a_{21} & a_{22} & \cdots & a_{2k} \\
\vdots & \vdots & \ddots & \vdots \\
a_{k1} & a_{k2} & \cdots & a_{kk}
\end{vmatrix}
\end{equation*}

$M_k \neq 0$, значит, строки~$A_1, \ldots, A_k$ линейно независимы. Пусть $M_{k+1}$~--- минор $(k + 1)$\nobreakdash-го порядка:
\begin{equation*}
M_{k+1} =
\begin{vmatrix}
a_{11} & a_{12} & \cdots & a_{1k} & a_{1j} \\
a_{21} & a_{22} & \cdots & a_{2k} & a_{2j} \\
\vdots & \vdots & \ddots & \vdots & \vdots \\
a_{k1} & a_{k2} & \cdots & a_{kk} & a_{kj} \\
a_{i1} & a_{i2} & \cdots & a_{ik} & a_{ij}
\end{vmatrix} = 0
\end{equation*}
т.\,к. $M_k$~--- базисный минор.
Тогда
\begin{equation*}
a_{1j} A_{1j} + a_{2j} A_{2j} + \ldots + a_{kj} A_{kj} + a_{ij} A_{ij} = 0, \ A_{ij} = M_k \neq 0 \Rightarrow
\end{equation*}
\begin{equation*}
\Rightarrow a_{ij} = -\frac{A_{1j}}{A_{ij}} a_{1j} - \frac{A_{2j}}{A_{ij}} a_{2j} - \ldots - \frac{A_{kj}}{A_{ij}} a_{kj}
\end{equation*}
где $A_{1j}, \ldots, A_{kj}, A_{ij}$~--- алгебраические дополнения $a_{1j}, \ldots, a_{kj}, a_{ij}$.
$A_{1j}, \ldots, A_{kj}, A_{ij}$ не зависят от~$j$, тогда $A_i$~--- линейная комбинация $A_1, \ldots, A_k$, значит, $k$~--- ранг матрицы $A$.
\end{proof}

Рангом матрицы по строкам (столбцам) называется максимальное количество её линейно независимых строк (столбцов).

\begin{consequent}
Ранг матрицы по строкам равен рангу матрицы по столбцам.
\end{consequent}%
Для доказательства достаточно заметить, что определитель транспонированной матрицы равен определителю исходной.