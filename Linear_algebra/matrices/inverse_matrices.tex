% beta
\subsection{Обратные матрицы}
Матрица~$B$ называется \textbf{левой обратной} к~квадратной матрице~$A$, если $BA = E$.

Матрица~$C$ называется \textbf{правой обратной} к~квадратной матрице~$A$, если $AC = E$.

Заметим, что обе матрицы $B$ и~$C$~--- квадратные того же порядка, что и~$A$.

\begin{statement}
Если существуют левая и~правая обратные к~$A$ матрицы $B$ и~$C$, то они совпадают.
\end{statement}
\begin{proof}
$B = BE = BAC = EC = C$
\end{proof}

Т.\,о., матрица~$A^{-1}$ называется \textbf{обратной} к~матрице~$A$, если $A^{-1} A = A A^{-1} = E$.

\begin{theorem}
Пусть даны матрицы
\begin{equation*}
A =
\begin{Vmatrix}
a_{11} & a_{12} & \cdots & a_{1n} \\
a_{21} & a_{22} & \cdots & a_{2n} \\
\vdots & \vdots & \ddots & \vdots \\
a_{n1} & a_{n2} & \cdots & a_{nn}
\end{Vmatrix}, \ 
\tilde A =
\begin{Vmatrix}
A_{11} & A_{12} & \cdots & A_{1n} \\
A_{21} & A_{22} & \cdots & A_{2n} \\
\vdots & \vdots & \ddots & \vdots \\
A_{n1} & A_{n2} & \cdots & A_{nn}
\end{Vmatrix}
\end{equation*}
где $A_{ij}$~--- алгебраическое дополнение $a_{ij}$.

Если $|A| \neq 0$, то
\begin{equation*}
A^{-1} = \frac{\tilde A^T}{|A|}
\end{equation*}
\end{theorem}
\begin{proof}

\end{proof}

\begin{theorem}
Пусть дана невырожденная матрица
\begin{equation*}
A =
\begin{Vmatrix}
a_{11} & a_{12} & \cdots & a_{1n} \\
a_{21} & a_{22} & \cdots & a_{2n} \\
\vdots & \vdots & \ddots & \vdots \\
a_{n1} & a_{n2} & \cdots & a_{nn}
\end{Vmatrix}
\end{equation*}

Присоединим к~ней единичную матрицу:
\begin{equation*}
\begin{Vmatrix}
a_{11} & a_{12} & \cdots & a_{1n} & 1 & 0 & \cdots & 0 \\
a_{21} & a_{22} & \cdots & a_{2n} & 0 & 1 & \cdots & 0 \\
\vdots & \vdots & \ddots & \vdots & \vdots & \vdots & \ddots & \vdots \\
a_{n1} & a_{n2} & \cdots & a_{nn} & 0 & 0 & \cdots & 1
\end{Vmatrix}
\end{equation*}
и~с~помощью элементарных преобразований только над~строками полученной матрицы (или только над столбцами) приведём её левую часть к~единичной матрице.
Тогда правая часть будет обратной к~$A$ матрицей.
\end{theorem}
\begin{proof}

\end{proof}