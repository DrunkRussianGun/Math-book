\section{Многочлены от нескольких переменных}
\begin{enumerate}
	\item В~многочлене $a_n x^n + a_{n-1} x^{n-1} + \ldots + a_0$ подставим $a_i = P_i(y)$~--- многочлен от~$y$.
	Получим многочлен от~$x$ и~$y$.
	
	\item Пусть имеем многочлен от~$n$~переменных.
	Подставим вместо его коэффициентов многочлен от~одной переменной, получим многочлен от~$n + 1$~переменных.
\end{enumerate}

Одночлены многочлена будем записывать в~лексикографическом порядке степеней переменных (члены с~б\'{о}льшими степенями идут раньше).

\begin{theorem}
Старший член произведения многочленов равен произведению старших членов множителей.
\end{theorem}
\begin{proof}
Перемножая члены с~наибольшими показателями старшей переменной, получим член с~наибольшим показателем при~этой переменной.
Проведя аналогичные рассуждения для~остальных переменной, придём к~выводу, что полученный член является старшим.
\end{proof}

Аналогично доказывается следующая теорема.
\begin{theorem}
Младший член произведения многочленов равен произведению младших членов множителей.
\end{theorem}