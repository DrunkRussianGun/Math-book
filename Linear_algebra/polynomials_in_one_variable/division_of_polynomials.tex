% beta
\subsection{Деление многочленов}
\begin{theorem}
Пусть $f(x)$ и~$g(x) \neq 0$~--- многочлены, тогда существуют единственные многочлены $q(x)$ и~$r(x)$ такие, что $f = qg + r$, причём либо $r = 0$, либо $\deg r < \deg g$.
\end{theorem}
\begin{proof}
\begin{enumerate}
	\item Докажем единственность.
	
	\item Докажем существование.
	
\end{enumerate}
\end{proof}

\textbf{Общим делителем} многочленов $f(x)$ и~$g(x)$ называется многочлен~$h(x)$, на который и~$f$, и~$g$ делятся нацело:
\begin{equation*}
f = ph, \ g = qh
\end{equation*}

\textbf{Наибольшим} называется общий делитель наибольшей степени и обозначается $\NOD$.

\begin{theorem}[Евклида]
Любые два многочлена имеют единственный $\NOD$.
\end{theorem}
\begin{proof}

\end{proof}