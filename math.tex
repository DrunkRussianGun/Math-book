\documentclass{report}
\usepackage[left=1.5cm,top=2cm,right=1.5cm,bottom=2cm]{geometry}
\usepackage[utf8]{inputenc}
\usepackage[T1,T2A]{fontenc}
\usepackage[russian]{babel}

\usepackage{index} % предметный указатель
\usepackage{indentfirst} % отступ красной строки
\usepackage[unicode,colorlinks=true]{hyperref} % ссылки в оглавлении
\usepackage{array} % таблицы

\usepackage{xparse}

\usepackage{tikz}
\usetikzlibrary{angles, chains, intersections, matrix, scopes, quotes}
\NewDocumentCommand\drawaxis{mmmm}{% {левая граница}{правая граница}{нижняя граница}{верхняя граница}
	\def\left_x{#1} \def\right_x{#2}
	\def\bottom_y{#3} \def\top_y{#4}
	\draw[->, >=stealth, name path=x_axis] (\left_x, 0) -- (\right_x, 0) node[below] {$x$};
	\draw[->, >=stealth, name path=y_axis] (0, \bottom_y) -- (0, \top_y) node[left] {$y$};
	\clip (\left_x, \top_y) rectangle (\right_x, \bottom_y)
}
\NewDocumentCommand\printcoordsonaxis{O{dashed}mgO{below}gO{left}}{% [параметры линий]{координаты (без скобок)}{надпись для x}[положение надписи для x]{надпись для y}[положение надписи для y]
	\draw[#1] \IfValueT{#3}{(#2) -- (#2 |- 0,0) node[#4] {$#3$}}
		\IfValueT{#5}{(#2) -- (#2 -| 0,0) node[#6] {$#5$}}
}

\usepackage{float}
\usepackage{wrapfig}

\usepackage{enumerate} % задание пользовательской нумерации для нумерованных списков

\usepackage{mathenvrus} % математическая запись

\tolerance=500 % увеличение максимально возможной разреженности строк

\makeindex % создание предметного указателя
\renewindex{default}{idx}{ind}{Предметный указатель} % переименование заголовка предметного указателя

%\includeonly{University_course/Arithmetic}
\begin{document}
\chapter*{Введение}
Здесь содержатся знания maxim4133 о~математике. Принятые обозначения:
\begin{itemize}
	\item $\forall$~--- \textbf{квантор всеобщности}.
	Обозначение условия, которое верно для~всех указанных элементов.
	Читается как~<<для всех>>, <<для каждого>>, <<для любого>> или~<<все>>, <<каждый>>, <<любой>>.
	
	\item $\exists$~--- \textbf{квантор существования}.
	Обозначение условия, которое верно хотя~бы для~одного из~указанных элементов.
	Читается как~<<существует>>, <<найдётся>>.
	
	\item $\exists!$~--- \textbf{квантор существования и единственности}.
	Обозначение условия, которое верно ровно для~одного из~указанных элементов.
	Читается как~<<существует единственный>>.
	
	\item $\colon$~--- <<что>>, <<такой (такие)>>, <<что>>, <<так, что>>, <<обладающий свойством>>.
	
	\item $\And$~--- знак конъюнкции. Читается как <<и>>.
	
	\item $\lor$~--- знак дизъюнкции. Читается как <<или>>.
	
	\item $\Rightarrow$~--- знак импликации (следствия).
	Читается как~<<если\dots, то\dots>>, <<значит>>, <<тогда>>.
	
	\item $\Leftrightarrow$~--- знак эквивалентности (равносильности).
	Читается как <<тогда и~только тогда, когда>>, <<ровно/в~точности тогда, когда>>.
	
	\item $\scriptstyle \blacksquare$~--- Q.E.D. (лат. quod erat demonstrandum, рус. что и~требовалось доказать).
	Обозначение конца доказательства.
\end{itemize}

Приоритет связок в порядке от высшего к низшему: $\And, \lor, \Rightarrow, \Leftrightarrow$.
\section*{Благодарности}
borisgk98 за~его отличные идеи по~улучшению данного сборника

\tableofcontents

\part{Школьный курс}
\chapter{Арифметика}
\section{Вещественные числа}
\index{R@$\mathbb R$} \index{Число!вещественное} Множество вещественных чисел обозначается $\mathbb R$.

\subsection{Аксиоматика вещественных чисел}
Аксиомы сложения:
\begin{enumerate}
	\item Коммутативность сложения: $\forall a, b \in \mathbb R \ a + b = b + a$
	\item Ассоциативность сложения: $\forall a, b, c \in \mathbb R \ a + (b + c) = (a + b) + c$
	\item Существование \textbf{нуля}: $\exists 0 \in \mathbb R \colon \forall a \in \mathbb R \ a + 0 = a$
	\item Существование \textbf{противоположного} числа: $\forall a \in \mathbb R \ \exists (-a) \in \mathbb R \colon a + (-a) = 0$
\end{enumerate}

Аксиомы умножения:
\begin{enumerate}
	\item Коммутативность умножения: $\forall a, b \in \mathbb R \ a \cdot b = b \cdot a$
	\item Ассоциативность умножения: $\forall a, b, c \in \mathbb R \ a \cdot (b \cdot c) = (a \cdot b) \cdot c$
	\item Дистрибутивность умножения относительно сложения: $\forall a, b, c \in \mathbb R \ a \cdot (b + c) = a \cdot b + a \cdot c$
	\item Существование \textbf{единицы}: $\exists 1 \in \mathbb R \colon \forall a \in \mathbb R \ a \cdot 1 = a$
	\item Существование \textbf{обратного} числа: $\forall a \in \mathbb R \setminus \{ 0 \} \ \exists \dfrac1a = a^{-1} \in \mathbb R \colon a \cdot a^{-1} = 1$
\end{enumerate}

Аксиомы порядка:
\begin{enumerate}
	\item Рефлексивность: $\forall a \in \mathbb R \ a \leqslant a$
	\item Антисимметричность: $\forall a, b \in \mathbb R \ a \leqslant b \lAnd b \leqslant a \Rightarrow a = b$
	\item Транзитивность: $\forall a, b, c \in \mathbb R \ a \leqslant b \lAnd b \leqslant c \Rightarrow a \leqslant c$
	\item Линейная упорядоченность: $\forall a, b \in \mathbb R \ a \leqslant b \lOr b \leqslant a$
	\item Связь сложения и порядка: $\forall a, b, c \in \mathbb R \ a \leqslant b \Rightarrow a + c \leqslant b + c$
	\item Связь умножения и порядка: $\forall a, b \in \mathbb R \ 0 \leqslant a \lAnd 0 \leqslant b \Rightarrow 0 \leqslant a \cdot b$
\end{enumerate}

Аксиома нетривиальности: $0 \neq 1$

\hypertarget{eq:continuity_axiom}{Аксиома непрерывности}: $\forall A, B \subset \mathbb R \colon
(\forall a \in A, \ b \in B \ a \leqslant b) \
\exists x \in \mathbb R \colon
a \leqslant x \leqslant b$

\subsection{Существование иррациональных чисел}
\index{I@$\mathbb I$} \index{Число!иррациональное} \textbf{Иррациональным} называется число, не являющееся рациональным.
Множество иррациональных чисел обозначается $\mathbb I$.

\begin{statement}
Существуют иррациональные числа.
\end{statement}
\begin{proofcontra}
Пусть $\exists p \in \mathbb Z, \ q \in \mathbb N \colon
\left( \dfrac{p}q \right)^2 = 2, \ \NOD(p, q) = 1$.
Тогда
\begin{equation*}
p^2 = 2q^2 \Rightarrow p^2 \mult 2 \Leftrightarrow p \mult 2
\Leftrightarrow p = 2l, \ l \in \mathbb Z \Rightarrow 2l^2 = q^2
\end{equation*}

Аналогичными рассуждениями получим $q \mult 2$.
$p \mult 2, \ q \mult 2 \Rightarrow \NOD(p, q) \neq 1$.
Противоречие.
\end{proofcontra}

\begin{statement}
Среди вещественных чисел есть иррациональные.
\end{statement}
\begin{proof}
Пусть $X = \{ x \in \mathbb R^+ \mid x^2 < 2 \}$, $Y = \{ y \in \mathbb R^+ \mid y^2 > 2 \}$.
\begin{equation*}
\forall x \in X, \ y \in Y \ y^2 - x^2 = (y - x)(y + x) > 0 \Rightarrow y > x \Rightarrow
\exists z \in \mathbb R \colon \forall x \in X, \ y \in Y \ x \leqslant z \leqslant y
\end{equation*}

\begin{enumerate}
	\item Пусть $z^2 < 2$. $z \in X$, $z > 1{,}1$, тогда
	\begin{equation*}
	0 < 2 - z^2 < 1 \Rightarrow \frac{2 - z^2}5 < 1 \Rightarrow
	\left( \frac{2 - z^2}5 \right)^2 < \frac{2 - z^2}5
	\end{equation*}
	
	\begin{itemize}
		\item $\displaystyle z + \frac{2 - z^2}5 > z \Rightarrow z + \frac{2 - z^2}5 \notin X$
		
		\item $\displaystyle \left( z + \frac{2 - z^2}5 \right)^2 =
		z^2 + 2z\cdot\frac{2 - z^2}5 + \left( \frac{2 - z^2}5 \right)^2 <
		z^2 + \frac45 (2 - z^2) + \frac{2 - z^2}5 = 2 \Rightarrow z + \frac{2 - z^2}5 \in X$
	\end{itemize}
	Противоречие, значит, $z^2 \geqslant 2$.
	
	\item Пусть $z^2 > 2$. $z \in Y$, $z < 1{,}9$, тогда
	\begin{equation*}
	0 < z^2 - 2 < 2 \Rightarrow \frac{z^2 - 2}4 < 1 \Rightarrow
	\left( \frac{z^2 - 2}4 \right)^2 < \frac{z^2 - 2}4
	\end{equation*}
	
	\begin{itemize}
		\item $\displaystyle z - \frac{z^2 - 2}4 < z \Rightarrow z + \frac{z^2 - 2}4 \notin Y$
		
		\item $\displaystyle \left( z - \frac{z^2 - 2}4 \right)^2 =
		z^2 - z\cdot\frac{z^2 - 2}2 + \left( \frac{2 - z^2}4 \right)^2 >
		z^2 - (z^2 - 2) = 2 \Rightarrow z - \frac{z^2 - 2}4 \in Y$
	\end{itemize}
	
	Противоречие, значит, $z^2 \leqslant 2$.
\end{enumerate}

Т.\,о., $z^2 \leqslant 2 \lAnd z^2 \geqslant 2 \Leftrightarrow z^2 = 2 \Rightarrow
\exists z \in \mathbb R \colon z^2 = 2 \Rightarrow z \notin \mathbb Q$.
\end{proof}

\subsection{Геометрическое представление вещественных чисел}
Самой распространённой интерпретацией множества~$\mathbb R$ является бесконечная прямая.
\begin{figure}[H] \centering
\noindent
\begin{tikzpicture}
\draw (-5, 0) -- (5, 0);
\foreach \x in {-3, ..., 3}
	\draw (\x, 0) node {$\bullet$} node[below] {$\x$};
\end{tikzpicture}
\caption{Множество~$\mathbb R$ в~виде прямой}
\end{figure}

Множество~$\mathbb R$ также можно представить в~виде окружности, одна точка которой соответствует нулю, а другая~--- бесконечности.
\begin{figure}[H] \centering
\noindent
\begin{tikzpicture}
\draw (0, 0) circle (1);
\def\angle#1{#1*30-90}
\def\drawpoint#1#2#3{% {угол}{надпись}{параметры надписи}
	\draw (#1:1) node {$\bullet$} node[#3] {$#2$}
}

\foreach \x in {-2, -1}
	\drawpoint{\angle{\x}}{\x}{below left};
\drawpoint{\angle{0}}{0}{below};
\foreach \x in {1, 2}
	\drawpoint{\angle{\x}}{\x}{below right};
\drawpoint{90}{\infty}{above};
\end{tikzpicture}
\caption{Множество~$\mathbb R$ в~виде окружности}
\end{figure}

\begin{wrapfigure}{r}{0pt} %54mm
\noindent
\begin{tikzpicture}[scale=0.5]
\draw (-5, 0) -- (0, 0) node {$\bullet$} node[below] {$0$} -- (5, 0) node[above] {$a$};
\draw (0, 1) circle (1);
\draw (0, 2) coordinate (I) node {$\bullet$} node[above] {$\infty$};
\draw (I) -- (3.75, -0.5)
	(I) -- (-2.5, -0.5);
\end{tikzpicture}
\end{wrapfigure}
Покажем, что эти интерпретации взаимозаменяемы.
Изобразим их так, чтобы точка, соответствующая нулю на прямой~$a$, совпадала с точкой, соответствующей нулю на окружности.
Теперь из точки, соответствующей бесконечности на окружности, проведём все возможные прямые.
Каждая из них пересекает одну точку на прямой~$a$ и одну точку на окружности и таким образом устанавливает взаимно однозначное соответствие, при этом $-\infty$ и $+\infty$ означают движение к одной и той~же точке на окружности, соответствующей бесконечности.
\section{Прогрессии}
\subsection{Арифметическая прогрессия}
\index{Прогрессия!арифметическая} \textbf{Арифметической прогрессией} называется последовательность~$(a_n)$, если $\exists d \colon \forall n \in \mathbb N \ a_{n+1} = a_n + d$.
$d$ называется \textbf{разностью арифметической прогрессии}.
Если $d < 0$, то прогрессия называется \textbf{убывающей}, а если $d > 0$, то \textbf{возрастающей}.

\begin{statement}
$\forall n \in \mathbb N \ a_n = a_1 + d(n - 1)$.
\end{statement}
\begin{proofmathind}
	\indbase $a_1 = a_1 + d \cdot (1 - 1)$
	\indstep Пусть $a_n = a_1 + d(n - 1)$, тогда $a_{n+1} = a_n + d = a_1 + dn$. \indend
\end{proofmathind}

\begin{theorem}[характеристическое свойство арифметической прогрессии]
$(a_n)$~--- арифметическая прогрессия $\Leftrightarrow$ $\forall n \in \mathbb N \setminus \{ 1 \} \ a_n = \dfrac{a_{n-1} + a_{n+1}}2$.
\end{theorem}
\begin{proof}
\begin{enumerate}
	\item $\Rightarrow$.
	\begin{equation*}
	a_n = \frac{a_1 + d(n - 1)}2 + \frac{a_1 + d(n - 1)}2 =
	\frac{a_1 + d(n - 2) + a_1 + dn}2 =
	\frac{a_{n-1} + a_{n+1}}2
	\end{equation*}
	
	\item $\Leftarrow$. Пусть $d = a_2 - a_1$.
	Докажем методом математической индукции, что $\forall n \in \mathbb N \ a_{n+1} - a_n = d$.
		\indbase $a_2 - a_1 = d$ по определению.
		\indstep Пусть $a_{n+1} - a_n = d$.
		\begin{equation*}
		a_{n+1} = \frac{a_n + a_{n+2}}2 \Leftrightarrow
		2 a_{n+1} = a_n + a_{n+2} \Leftrightarrow
		a_{n+2} - a_{n+1} = a_{n+1} - a_n = d
		\end{equation*}
		\indend
		
	Тогда $\forall n \in \mathbb N \ a_{n+1} = a_n + d \Rightarrow (a_n)$~--- арифметическая прогрессия.
\end{enumerate}
\end{proof}

\begin{lemma}
\begin{equation*}
\forall i, j \in \mathbb N \ i + j = n + 1 \Rightarrow a_i + a_j = a_1 + a_n
\end{equation*}
\end{lemma}
\begin{proof}
\begin{equation*}
a_i + a_j =
a_1 + d(i - 1) + a_1 + d(j - 1) =
2 a_1 + d(i + j - 2) =
2 a_1 + d(n + 1 - 2) =
a_1 + a_1 + d(n - 1) =
a_1 + a_n
\end{equation*}
\end{proof}

\begin{theorem}
\begin{equation*}
\sum_{k=1}^n a_k = \frac{a_1 + a_n}2 \cdot n
\end{equation*}
\end{theorem}
\begin{proof}
\begin{equation*}
2 \sum_{k=1}^n a_k =
\sum_{k=1}^n (a_k + a_k) =
\sum_{k=1}^n (a_k + a_{n-k+1}) \;
\left| k + (n - k + 1) = n + 1 \right| =
\sum_{k=1}^n (a_1 + a_n) =
n(a_1 + a_n) \Leftrightarrow
\end{equation*}
\begin{equation*}
\Leftrightarrow \sum_{k=1}^n a_k = \frac{a_1 + a_n}2 \cdot n
\end{equation*}
\end{proof}

\subsection{Геометрическая прогрессия}
\index{Прогрессия!геометрическая} \textbf{Геометрической прогрессией} называется последовательность~$(b_n)$, если $b_1 \neq 0 \lAnd \exists q \neq 0 \colon \forall n \in \mathbb N \ b_{n+1} = b_n q$.
$q$~называется \textbf{знаменателем геометрической прогрессии}.

\begin{statement}
$\forall n \in \mathbb N \ b_n = b_1 q^{n-1}$
\end{statement}
\begin{proofmathind}
	\indbase $b_1 = b_1 q^{1-1}$
	\indstep Пусть $b_n = b_1 q^{n-1}$, тогда $b_{n+1} = b_n q = b_1 q^n$. \indend
\end{proofmathind}

\begin{theorem}[характеристическое свойство геометрической прогрессии]
$(b_n)$~--- геометрическая прогрессия $\Leftrightarrow \forall n \in \mathbb N \setminus \{ 1 \} \ |b_n| = \sqrt{b_{n-1} b_{n+1}} \neq 0$.
\end{theorem}
\begin{proof}
\begin{enumerate}
	\item $\Rightarrow$.
	\begin{equation*}
	|b_n| = \sqrt{b_n^2} =
	\sqrt{b_1 q^{n-2} b_1 q^n} =
	\sqrt{b_{n-1} b_{n+1}}
	\end{equation*}
	
	\item $\Leftarrow$. Пусть $q = \dfrac{b_2}{b_1}$.
	Докажем методом математической индукции, что $\forall n \in \mathbb N \ \dfrac{b_{n+1}}{b_n} = q$.
		\indbase $\dfrac{b_2}{b_1} = q$ по определению.
		\indstep Пусть $\dfrac{b_{n+1}}{b_n} = q$.
		\begin{equation*}
		|b_{n+1}| = \sqrt{b_n b_{n+2}} \Leftrightarrow
		b_{n+1}^2 = b_n b_{n+2} \Leftrightarrow
		\frac{b_{n+2}}{b_{n+1}} = \frac{b_{n+1}}{b_n} = q
		\end{equation*}
		\indend
		
	Тогда $\forall n \in \mathbb N \ b_{n+1} = b_n q$ $\Rightarrow$ $(b_n)$~--- геометрическая прогрессия.
\end{enumerate}
\end{proof}

\begin{theorem}
\begin{equation*}
\sum_{k=1}^n b_k = \frac{b_1(q^n - 1)}{q - 1} = \frac{b_n q - b_1}{q - 1}, \ q \neq 1
\end{equation*}
\end{theorem}
\begin{proof}
\begin{equation*}
(q - 1) \sum_{k=1}^n b_k =
\sum_{k=1}^n (b_{k+1} - b_k) =
b_{n+1} - b_1 \Leftrightarrow
\sum_{k=1}^n b_k = \frac{b_n q - b_1}{q - 1} = \frac{b_1(q^n - 1)}{q - 1}
\end{equation*}
\end{proof}

Если $q = 1$, то очевидно, что $\sum\limits_{k=1}^n b_k = b_1 n$.

Геометрическая прогрессия называется \textbf{бесконечно убывающей}, если модуль её знаменателя меньше~$1$.
В этом случае $\sum\limits_{k=1}^\infty b_k = \frac{b_1}{1 - q}$.
\chapter{Комбинаторика}
\section{Элементы комбинаторики}
Пусть элемент~$A$ можно выбрать $m$~способами, а элемент~$B$~--- $n$~способами. Существуют основные правила комбинаторики:
\begin{enumerate}
	\item \textbf{Правило суммы:} выбор либо $A$, либо $B$ можно сделать $m + n$~способами.
	\item \textbf{Правило произведения:} выбор $A$ и $B$ можно сделать $m \cdot n$~способами.
\end{enumerate}

\index{!} \index{Факториал} \textbf{Факториалом числа~$n \in \mathbb N$} называется произведение $1 \cdot 2 \cdot \ldots \cdot n$ и обозначается~$n!$.
Также принято считать, что $0! = 1$.

\textbf{Двойным факториалом числа~$n \in \mathbb N$} называется произведение всех натуральных чисел той~же чётности, что и~$n$, и обозначается~$n!!$.
Т.\,о., $(2k)!! = 2 \cdot 4 \cdot \ldots \cdot (2k)$, $(2k + 1)!! = 1 \cdot 3 \cdot \ldots \cdot (2k + 1)$, где $k \in \mathbb N$.
Также принято считать, что $0!! = 1$.

\index{Перестановка} \textbf{Перестановкой} $n$-элементного \textbf{множества~$A$} называется биекция $f \colon \{ 1, 2, \ldots, n \} \to A$ и записывается в~виде упорядоченного набора $(a_1, a_2, \ldots, a_n)$, где $a_i = f(i), \ i = 1, 2, \ldots, n$.

\index{Размещение} \textbf{$k$-элементным размещением} $n$-элементного \textbf{множества~$A$ (размещением из $n$~элементов по $k$)} называется инъекция $f \colon \{ 1, 2, \ldots, n \} \to A$.
Количество $k$-элементных размещений $n$-элементного множества обозначается $A_n^k$.

\begin{statement}
$A_n^k = \dfrac{n!}{(n - k)!}$
\end{statement}
\begin{proofmathind}
	\indbase При $k = 1$ $A_n^1 = n = \dfrac{n!}{(n - 1)!}$.
	\indstep Пусть теорема верна при $k - 1$.
	Докажем её для $k$.
	Разобьём все $k$\nobreakdash-\hspace{0pt}элементные размещения $n$\nobreakdash-\hspace{0pt}элементного множества на группы так, чтобы во~всех размещениях одной группы на первом месте стоял один и тот~же элемент.
	Получим $n$~групп.
	В каждом размещении на оставшихся $k - 1$~местах могут стоять остальные $n - 1$ элементов.
	Т.\,о., по предположению индукции в каждой группе $A_{n-1}^{k-1}$~размещений.
	Тогда всего размещений $n \cdot A_{n-1}^{k-1} = \dfrac{n!}{(n - k)!} = A_n^k$. \indend
\end{proofmathind}

\begin{consequent}
Количество перестановок $n$-элементного множества равно $n!$.
\end{consequent}
\begin{proof}
Заметим, что искомое количество равно $A_n^n = \frac{n!}{0!} = n!$.
\end{proof}

\index{Сочетание} \textbf{$k$-элементным сочетанием} $n$-элементного \textbf{множества~$A$ (сочетанием из $n$~элементов по $k$)} называется $X \subseteq A \colon |X| = k$.
Количество $k$-элементных сочетаний $n$-элементного множества обозначается $C_n^k$, или $\binom{n}{k}$.

\begin{statement}
$C_n^k = \dfrac{n!}{k! (n - k)!}$
\end{statement}
\begin{proof}
Разобьём множество всех перестановок $n$-элементного множества на группы так, чтобы во всех перестановках одной группы на первых $k$~местах находились одни и те~же элементы (в том или ином порядке), тогда и на последних $n - k$~местах будут находиться одни и те~же элементы.
Получим $C_n^k$~групп.
В перестановках одной группы первые $k$~элементов могут находиться в произвольном порядке, т.\,е. они могут быть расположены $k!$~способами.
Аналогично последние $n - k$~элементов могут быть расположены $(n - k)!$~способами.
Тогда, учитывая правило произведения, получим $n! = C_n^k k! (n - k)! \Leftrightarrow C_n^k = \dfrac{n!}{k! (n - k)!}$.
\end{proof}

\index{Инверсия} \textbf{Инверсией в перестановке~$\pi$} называется пара индексов $i, j \colon i < j, \ \pi(i) > \pi(j)$.
\textbf{Чётность перестановки} определяется чётностью числа инверсией в ней.

\begin{statement}
\label{st:parity_of_permutation}
Если в перестановке~$(a_1, a_2, \ldots, a_n)$ поменять местами два элемента, то её чётность изменится.
\end{statement}
\begin{proof}
\begin{enumerate}
	\item Пусть переставлены соседние элементы.
	Если они образовывали инверсию, то после обмена местами не образуют, и наоборот.
	При этом наличие инверсий с остальными элементами остаётся неизменным.
	Значит, количество инверсий в перестановке изменилось на~$1$, т.\,е. чётность числа инверсий изменилась, тогда изменилась и чётность перестановки.
	
	\item Поменяем местами элементы $a_i$ и $a_{i+d}$, где $d > 0$.
	Для этого последовательно поменяем местами элементы, имеющие индексы $i+d$ и $i+d-1$, $i+d-1$ и $i+d-2$, \ldots, $i+2$ и $i+1$, $i+1$ и $i$, $i+1$ и $i+2$, $i+2$ и $i+3$, \ldots, $i+d-2$ и $i+d-1$, $i+d-1$ и $i+d$.
	Всего совершили $2d - 1$~обменов соседних элементов местами, тогда перестановка изменила чётность, т.\,к. $2d - 1 \notmult 2$.
\end{enumerate}
\end{proof}
\chapter{Математическая логика}
\section{Метод математической индукции}
\index{Метод!математической индукции} Предположим, что даны утверждения $P_1, P_2, \ldots$.
Пусть
\begin{itemize}
	\item верно $P_1$~--- \textbf{база индукции};
	\item верно, что $\forall i \in \mathbb N \ (P_i \Rightarrow P_{i+1})$~--- \textbf{шаг индукции}.
\end{itemize}

Тогда для любого $n \in \mathbb N$ верно $P_n$.
\chapter{Теория многочленов}
\section{Многочлены от одной переменной}
\index{Одночлен} \index{Моном} \textbf{Одночленом}, или \textbf{мономом}, называется произведение числового множителя и нуля и более переменных, взятых каждая в~неотрицательной степени.

\textbf{Степенью одночлена} называется сумма степеней входящих в~него переменных.
Степень тождественного нуля равна~$-\infty$.

\index{Многочлен} \index{Полином} \textbf{Многочленом}, или \textbf{полиномом}, \textbf{от одной переменной} называется сумма вида
\begin{equation*}
a_n x^n + a_{n-1} x^{n-1} + \ldots + a_1 x + a_0, \ a_n \neq 0
\end{equation*}
где $x_1, \ldots, x_n$~--- переменные.

\index{deg} \textbf{Степенью многочлена~$f$} называется максимальная из степеней его одночленов и обозначается $\deg f$.
Многочлен $1$-й степени называется \textbf{линейным}, $2$-й степени~--- \textbf{квадратным}, $3$-й степени~--- \textbf{кубическим}.

\begin{statement}
Пусть $f$ и $g$~--- многочлены, тогда $\deg (f + g) \leqslant \max \{ \deg f, \deg g \}$.
\end{statement}
\begin{statement}
Пусть $f$ и $g$~--- многочлены, тогда $\deg fg = \deg f + \deg g$.
\end{statement}
\chapter{Теория множеств}
\index{Множество} \textbf{Множество}~--- аксиоматическое понятие, не имеющее определения.
Его можно описать как совокупность различных элементов, рассматриваемую как единое целое.
\section{Отношения между множествами}
Пусть $A, B$~--- множества. Между ними определены следующие отношения:
\begin{itemize}
	\item $A$ включено в~$B$ (является \textbf{подмножеством}~$B$): $A \subseteq B \Leftrightarrow \forall a \in A \ a \in B$
	
	Нередко вместо знака~$\subseteq$ пишется знак~$\subset$\,.
	\item $A$ равно $B$: $A = B \Leftrightarrow \forall a \ (a \in A \Leftrightarrow a \in B)$
	\item $A$ строго включено в~$B$: $A \subset B \Leftrightarrow A \subseteq B \lAnd A \neq B$
\end{itemize}
\section{Операции над множествами}
Пусть $A, B$~--- множества. Над ними определены следующие операции:
\begin{itemize}
	\item \textbf{Объединение}: $A \cup B = \{ x \mid x \in A \lOr x \in B \}$
	\item \textbf{Пересечение}: $A \cap B = \{ x \mid x \in A \lAnd x \in B \}$
	\item \textbf{Разность}: $A \setminus B = \{ x \mid x \in A \lAnd x \notin B \}$
	\item \textbf{Симметрическая разность}: $A \bigtriangleup B = \{ x \mid x \in A \lAnd x \notin B \lOr x \notin A \lAnd x \in B \}$
	\item \textbf{Дополнение} до~$U$, где $A \subseteq U$: $\overline A = \{ x \in U \mid x \notin A \}$
	\item \textbf{Декартово произведение}: $A \times B = \{ (x, y) \mid x \in A \lAnd y \in B \}$
	\item \textbf{Декартова степень}: $A^n = \underbrace{A \times A \times \ldots \times A}_n$
\end{itemize}
\section{Подмножества множества \texorpdfstring{$\mathbb R$}{}}
\chapter{Элементарная алгебра}
\section{Алгебраические преобразования}
Формулы сокращённого умножения:
\begin{enumerate}
	\item Квадрат суммы:
	\begin{equation*}
	(a \pm b)^2 = (a \pm b)(a \pm b) = a^2 \pm 2ab + b^2
	\end{equation*}
	
	\item Разность квадратов:
	\begin{equation*}
	a^2 - b^2 = a^2 - ab + (ab - b^2) = (a + b)(a - b)
	\end{equation*}
	
	\item Куб суммы:
	\begin{equation*}
	(a \pm b)^3 = (a^2 \pm 2ab + b^2)(a \pm b) = a^3 \pm 3a^2 b + 3ab^2 \pm b^3
	\end{equation*}
	
	\item Сумма кубов:
	\begin{equation*}
	a^3 \pm b^3 = a^3 \mp a^2 b + ab^2 \pm a^2 b - ab^2 \pm b^3 =
	a(a^2 \mp ab + b^2) \pm b(a^2 \mp ab + b^2) = (a \pm b)(a^2 \mp ab + b^2)
	\end{equation*}
\end{enumerate}

\index{Формула!бинома Ньютона}
\begin{theorem}[формула бинома Ньютона]
\label{eq:binomial_expansion}
\begin{equation*}
\forall n \in \mathbb N \ (a + b)^n = \sum_{m=0}^n C_n^m a^{n-m} b^m
\end{equation*}
\end{theorem}
\begin{proofmathind}
	\indbase $n = 1$: $(a + b)^1 = a + b = C_1^0 a + C_1^1 b$
	\indstep Пусть формула верна для~$n$. Докажем истинность для~$n + 1$.
	\begin{equation*}
	(a + b)^{n+1} =
	(a + b)\sum_{m=0}^n C_n^m a^{n-m} b^m =
	\sum_{m=0}^n C_n^m a^{n-m+1} b^m + \sum_{m=0}^n C_n^m a^{n-m} b^{m+1} =
	\end{equation*}
	\begin{equation*}
	= a^{n+1} + \sum_{m=0}^{n-1} C_n^{m+1} a^{n-m} b^{m+1} + \sum_{m=0}^{n-1} C_n^m a^{n-m} b^{m+1} =
	a^{n+1} + \sum_{m=0}^{n-1} C_{n+1}^{m+1} a^{n-m} b^{m+1} + b^{n+1} =
	\end{equation*}
	\begin{equation*}
	= a^{n+1} + \sum_{m=1}^n C_{n+1}^m a^{n+1-m} b^m + b^{n+1} =
	\sum_{m=0}^{n+1} C_{n+1}^m a^{n+1-m} b^m
	\end{equation*}
	\indend
\end{proofmathind}
\section{Метод интервалов}
\index{Метод!интервалов} Метод интервалов~--- метод решения рациональных неравенств, т.\,е. неравенств вида $\frac{P(x)}{Q(x)} < 0$ или $\frac{P(x)}{Q(x)} > 0$, где $P(x)$ и $Q(x)$~--- многочлены, причём как строгих, так и нестрогих.
\begin{enumerate}
	\item Найдём корни многочленов $P(x)$ и $Q(x)$ и отметим их на числовой прямой, разбив её таким образом на интервалы.
	\item Найдём знак значения дроби $\frac{P(x)}{Q(x)}$ на каждом из полученных интервалов.
	Для этого достаточно найти знак значения дроби в любой точке, лежащей в рассматриваемом интервале.
	\item Включаем в ответ подходящие интервалы, а также корни многочленов, если они удовлетворяют неравенству.
\end{enumerate}

Заметим, что нахождение знаков значений дроби можно упростить.
\begin{itemize}
	\item Если старшие коэффициенты $P(x)$ и $Q(x)$ оба положительны или оба отрицательны, то дробь положительна на интервале $(a; +\infty)$, где $a$~--- наибольший из найденных корней.
	\item Дробь меняет знак при переходе через корень нечётной кратности и не меняет при переходе через корень чётной кратности.
\end{itemize}
\section{Уравнения с модулем}
\begin{enumerate}
	\item $|f(x)| = b$
	\begin{itemize}
		\item При $b < 0$ нет решений.
		\item Если $b = 0$, то $|f(x)| = 0 \Leftrightarrow f(x) = 0$.
		\item Если $b > 0$, то $|f(x)| = b \Leftrightarrow f(x) = b \lOr f(x) = -b$.
	\end{itemize}
	
	\item $|f(x)| = |g(x)| \Leftrightarrow f(x) = g(x) \lOr f(x) = -g(x)$
	
	\item $|f(x)| = g(x) \Leftrightarrow (f(x) = g(x) \lOr f(x) = -g(x)) \lAnd g(x) \geqslant 0$
\end{enumerate}
\section{Неравенства с модулем}
\index{Неравенства!с модулем}
\begin{enumerate}
	\item $|f(x)| < g(x) \Leftrightarrow -g(x) < f(x) < g(x)$
	
	\item $|f(x)| > g(x) \Leftrightarrow f(x) < -g(x) \lOr f(x) > g(x)$
	
	\item $|f(x)| < |g(x)| \Leftrightarrow (f(x) - g(x))(f(x) + g(x)) < 0$
	\begin{proof}
	\begin{equation*}
	|f(x)| < |g(x)| \Leftrightarrow
	f^2(x) < g^2(x) \Leftrightarrow
	f^2(x) - g^2(x) < 0 \Leftrightarrow
	(f(x) - g(x))(f(x) + g(x)) < 0
	\end{equation*}
	\end{proof}
	
	\item $|f(x)| > |g(x)| \Leftrightarrow (f(x) - g(x))(f(x) + g(x)) > 0$
	\begin{proof}
	\begin{equation*}
	|f(x)| > |g(x)| \Leftrightarrow
	f^2(x) > g^2(x) \Leftrightarrow
	f^2(x) - g^2(x) > 0 \Leftrightarrow
	(f(x) - g(x))(f(x) + g(x)) > 0
	\end{equation*}
	\end{proof}
\end{enumerate}
\chapter{Элементарная математика}
\section{Функции}
\index{Функция} Пусть $A$ и $B$~--- множества.
\textbf{Функцией~$f$} называется правило, ставящее в~соответствие каждому элементу~$a \in A$ единственный элемент~$f(a) \in B$.
$A$ называется \textbf{областью определения функции~$f$} и обозначается $D(f)$, $B$~--- \textbf{областью значений функции~$f$} и обозначается $E(f)$.
$a$ называется \textbf{прообразом~$f(a)$}, $f(a)$~--- \textbf{образом~$a$}.

\textbf{Нулём функции~$f \colon X \to \mathbb R$}, где $X \subseteq \mathbb R$, называется элемент $x \in X \colon f(x) = 0$.

\index{Последовательность} \textbf{Последовательностью} называется функция, заданная на множестве~$X \subseteq \mathbb N$, и обозначается~$(x_n)$.

\textbf{Подпоследовательностью последовательности~$(x_n)$} называется последовательность~$(x_{n_k})$, если $\forall k \in \mathbb N \allowbreak n_k < n_{k+1}$.

\subsection{Возрастающие и убывающие функции}
\index{Функция!возрастающая} Функция~$f$ называется \textbf{возрастающей}, или \textbf{монотонной}, \textbf{на множестве~$X$}, если $\forall x_1, x_2 \in X \ x_1 < x_2 \opbr\Rightarrow f(x_1) < f(x_2)$.

\index{Функция!убывающая} Функция~$f$ называется \textbf{убывающей}, или \textbf{монотонной}, \textbf{на множестве~$X$}, если $\forall x_1, x_2 \in X \ x_1 < x_2 \opbr\Rightarrow f(x_1) > \nobreak f(x_2)$.

Если функция возрастает (убывает) на всей области определения, то её называют \textbf{возрастающей (убывающей)}, или \textbf{монотонной}.

Свойства монотонных функций:
\begin{enumerate}
	\item Если $f(x)$ и $g(x)$~--- возрастающие (убывающие) на множестве~$X$ функции, то $h(x) = f(x) + g(x)$~--- воз\-ра\-ста\-ющ\-ая (убывающая) на~$X$ функция.
	\begin{proof}
	Пусть $f(x)$ и $g(x)$ возрастают на~$X$, $x_1, x_2 \in X \colon x_1 < x_2$.
	\begin{equation*}
	x_1 < x_2 \Rightarrow
	f(x_1) < f(x_2) \lAnd g(x_1) < g(x_2) \Rightarrow
	f(x_1) + g(x_1) < f(x_2) + g(x_2) \Rightarrow
	h(x_1) < h(x_2)
	\end{equation*}
	
	Значит, $h(x)$ возрастает на~$X$.
	
	Доказательство для случая убывания аналогично.
	\end{proof}
	
	\item Если $f(x)$~--- возрастающая (убывающая) на множестве~$X$ функция, то:
	\begin{itemize}
		\item при $C < 0$ $h(x) = Cf(x)$~--- убывающая (возрастающая) на~$X$ функция;
		\item при $C > 0$ $h(x) = Cf(x)$~--- возрастающая (убывающая) на~$X$ функция.
	\end{itemize}
	\begin{proof}
	Пусть $f(x)$ возрастает на~$X$, $C < 0$, $x_1, x_2 \in X \colon x_1 < x_2$.
	\begin{equation*}
	x_1 < x_2 \Rightarrow
	f(x_1) < f(x_2) \Rightarrow
	C f(x_1) > C f(x_2) \Rightarrow
	h(x_1) > h(x_2)
	\end{equation*}
	
	Значит, $h(x)$ убывает на~$X$.
	
	Доказательства для остальных трёх случаев аналогичны.
	\end{proof}
	
	\item Если $f(x)$ и $g(x)$~--- возрастающие (убывающие) на множестве~$X$ функции и
	\begin{itemize}
		\item $f(x), g(x) < 0$ на~$X$, то $h(x) = f(x)g(x)$~--- убывающая (возрастающая) на~$X$ функция;
		\item $f(x), g(x) > 0$ на~$X$, то $h(x) = f(x)g(x)$~--- возрастающая (убывающая) на~$X$ функция.
	\end{itemize}
	\begin{proof}
	Пусть $f(x)$ и $g(x)$ возрастают на~$X$, $f(x), g(x) < 0$ на~$X$, $x_1, x_2 \in X \colon x_1 < x_2$.
	\begin{equation*}
	x_1 < x_2 \Rightarrow
	f(x_1) < f(x_2) \lAnd g(x_1) < g(x_2) \Rightarrow
	\end{equation*}
	\begin{equation*}
	\Rightarrow -f(x_1) > -f(x_2) \And -g(x_1) > -g(x_2) \Rightarrow
	f(x_1)g(x_1) > f(x_2)g(x_2) \Rightarrow
	h(x_1) > h(x_2)
	\end{equation*}
	
	Значит, $h(x)$ убывает на~$X$.
	
	Доказательства для остальных трёх случаев аналогичны.
	\end{proof}
	
	\item Если $f(x) \neq 0$~--- функция, возрастающая (убывающая) на множестве~$X$ и сохраняющая на нём знак, то $h(x) = \frac1{f(x)}$~--- убывающая (возрастающая) на~$X$ функция.
	\begin{proof}
	Пусть $f(x)$ возрастает на~$X$, $x_1, x_2 \in X \colon x_1 < x_2$.
	\begin{equation*}
	x_1 < x_2 \Rightarrow
	f(x_1) < f(x_2) \Rightarrow
	\frac{f(x_1)}{f(x_1)f(x_2)} < \frac{f(x_2)}{f(x_1)f(x_2)} \Rightarrow
	\frac1{f(x_1)} > \frac1{f(x_2)} \Rightarrow
	h(x_1) > h(x_2)
	\end{equation*}
	
	Значит, $h(x)$ убывает на~$X$.
		
	Доказательство для случая убывания аналогично.
	\end{proof}
	
	\item Если $f(x)$ и $g(x)$~--- обе возрастающие или обе убывающие на множестве~$X$ функции, то $h(x) = g(f(x))$~--- возрастающая на~$X$ функция.
	\begin{proof}
	Пусть $f(x)$ и $g(x)$ возрастают на~$X$, $x_1, x_2 \in X \colon x_1 < x_2$.
	\begin{equation*}
	x_1 < x_2 \Rightarrow
	f(x_1) < f(x_2) \Rightarrow
	g(f(x_1)) < g(f(x_2)) \Rightarrow
	h(x_1) < h(x_2)
	\end{equation*}
	
	Значит, $h(x)$ возрастает на~$X$.
	
	Доказательство для случая убывания аналогично.
	\end{proof}
\end{enumerate}

\subsection{Чётные и нечётные функции}
\index{Функция!чётная} Функция~$f(x)$ называется \textbf{чётной}, если $\forall x \in D(f) \ {-x} \in D(f) \lAnd f(-x) = f(x)$.

\index{Функция!нечётная} Функция~$f(x)$ называется \textbf{нечётной}, если $\forall x \in D(f) \ {-x} \in D(f) \lAnd f(-x) = -f(x)$.

Свойства чётных и нечётных функций:
\begin{enumerate}
	\item Функция~$f(x) \colon \forall x \in D(f) \ {-x} \in D(f)$ единственным образом может быть представлена в виде суммы чётной и нечётной функций.
	\begin{proof}
	\begin{itemize}
		\item Докажем представимость.
		Пусть
		\begin{equation*}
		g(x) = \frac{f(x) + f(-x)}2, \ h(x) = \frac{f(x) - f(-x)}2
		\end{equation*}
		
		Тогда
		\begin{equation*}
		g(-x) = \frac{f(-x) + f(x)}2 = g(x), \ h(-x) = \frac{f(-x) - f(x)}2 = -h(x)
		\end{equation*}
		
		Значит, $g(x)$ чётна, $h(x)$ нечётна.
		\begin{equation*}
		f(x) = \frac12 f(x) + \frac12 f(x) + \frac12 f(-x) - \frac12 f(-x) =
		\frac{f(x) + f(-x)}2 + \frac{f(x) - f(-x)}2 =
		g(x) + h(x)
		\end{equation*}
		
		\item Докажем единственность представления.
		Пусть $f(x) = g(x) + h(x) = g_1(x) + h_1(x)$, где $g, g_1$~--- чётные функции, $h, h_1$~--- нечётные функции.
		\begin{equation*}
		g(x) + h(x) = g_1(x) + h_1(x) \Leftrightarrow
		g(x) - g_1(x) = h_1(x) - h(x)
		\end{equation*}
		
		Подставляя $-x$, получим
		\begin{equation*}
		g(-x) - g_1(-x) = h_1(-x) - h(-x) \Leftrightarrow
		g(x) - g_1(x) = h(x) - h_1(x)
		\end{equation*}
		
		Тогда
		\begin{equation*}
		h_1(x) - h(x) = g(x) - g_1(x) = h(x) - h_1(x) \Rightarrow
		\end{equation*}
		\begin{equation*}
		\Rightarrow g(x) - g_1(x) = 0 \And h_1(x) - h(x) = 0 \Rightarrow
		g_1(x) = g(x) \lAnd h_1(x) = h(x)
		\end{equation*}
	\end{itemize}
	\end{proof}
	
	\item Если $f(x)$ и $g(x)$~--- чётные (нечётные) функции, то $h(x) = f(x) + g(x)$ чётна (нечётна).
	\begin{proof}
	\begin{itemize}
		\item Пусть $f(x)$ и $g(x)$ чётны, тогда $h(-x) = f(-x) + g(-x) = f(x) + g(x) = h(x)$, значит, $h(x)$ чётна.
		\item Пусть $f(x)$ и $g(x)$ нечётны, тогда $h(-x) = f(-x) + g(-x) = -f(x) - g(x) = -h(x)$, значит, $h(x)$ нечётна.
	\end{itemize}
	\end{proof}
	
	\item Если $f(x)$~--- чётная (нечётная) функция, то $h(x) = C f(x)$ чётна (нечётна).
	\begin{proof}
	\begin{itemize}
		\item Пусть $f(x)$ чётна, тогда $h(-x) = C f(-x) = C f(x) = h(x)$, значит, $h(x)$ чётна.
		\item Пусть $f(x)$ нечётна, тогда $h(-x) = C f(-x) = -C f(x) = -h(x)$, значит, $h(x)$ нечётна.
	\end{itemize}
	\end{proof}
	
	\item Если $f(x)$ и $g(x)$~--- обе чётные или обе нечётные функции, то $h(x) = f(x)g(x)$ чётна.
	\begin{proof}
	\begin{itemize}
		\item Пусть $f(x)$ и $g(x)$ чётны, тогда $h(-x) = f(-x)g(-x) = f(x)g(x) = h(x)$, значит, $h(x)$ чётна.
		\item Пусть $f(x)$ и $g(x)$ нечётны, тогда $h(-x) = f(-x)g(-x) = (-f(x)) (-g(x)) = h(x)$, значит, $h(x)$ нечётна.
	\end{itemize}
	\end{proof}
	
	\item Если $f(x)$ и $g(x)$~--- нечётная и чётная функции, то $h(x) = f(x)g(x)$ нечётна.
	\begin{proof}
	$h(-x) = f(-x)g(-x) = -f(x)g(x) = -h(x)$, значит, $h(x)$ нечётна.
	\end{proof}
	
	\item Если $f(x)$ и $g(x)$~--- нечётные функции, то $h(x) = g(f(x))$ нечётна.
	\begin{proof}
	$h(-x) = g(f(-x)) = g(-f(x)) = -g(f(x)) = -h(x)$, значит, $h(x)$ нечётна.
	\end{proof}
	
	\item Если $f(x)$ и $g(x)$~--- нечётная и чётная функции соответственно, то $h(x) = g(f(x))$ чётна.
	\begin{proof}
	$h(-x) = g(f(-x)) = g(-f(x)) = g(f(x)) = h(x)$, значит, $h(x)$ чётна.
	\end{proof}
	
	\item Если $f(x)$ и $g(x)$~--- функции, причём $f(x)$ чётна, то $h(x) = g(f(x))$ чётна.
	\begin{proof}
	$h(-x) = g(f(-x)) = g(f(x)) = h(x)$, значит, $h(x)$ чётна.
	\end{proof}
\end{enumerate}

\subsection{Ограниченные функции}
\index{Функция!ограниченная} Функция~$f(x)$ называется \textbf{ограниченной сверху на множестве~$X$}, если $\exists M \colon \forall x \in X \ f(x) \leqslant M$.
Если $X \opbr= D(f)$, то $f(x)$ называется \textbf{ограниченной сверху}.

Функция~$f(x)$ называется \textbf{ограниченной снизу на множестве~$X$}, если $\exists m \colon \forall x \in X \ f(x) \geqslant m$.
Если $X = D(f)$, то $f(x)$ называется \textbf{ограниченной снизу}.

Функция называется \textbf{ограниченной на множестве}, если она ограничена и сверху, и снизу на этом множестве.
Если данное множество совпадает с областью определения этой функции, то она называется \textbf{ограниченной}.

\subsection{Обратные функции}
\index{Функция!обратимая} Функция называется \textbf{обратимой}, если $\forall x_1, x_2 \in D(f) \ (x_1 \neq x_2 \Rightarrow f(x_1) \neq f(x_2))$.

\index{Функция!обратная} Функция~$g \colon Y \to X$ называется \textbf{обратной к} обратимой \textbf{функции~$f \colon X \to Y$}, если $\forall x \in X \ g(f(x)) = x \opbr\lAnd \forall y \in Y \ f(g(y)) = y$, и обозначается $f^{-1}$.

\begin{statement}
$(f^{-1})^{-1} = f$.
\end{statement}
\begin{proof}
Пусть даны функции $f \colon X \to Y$ и $g \colon Y \to X$, причём $g = f^{-1}$, тогда
$\forall x \in Y \ f(g(x)) = x \opbr\lAnd \forall y \in X \ g(f(y)) = y \Rightarrow f = g^{-1} = (f^{-1})^{-1}$.
\end{proof}

Функции $f$ и $g = f^{-1}$ называются \textbf{взаимно обратными}.

\begin{theorem}
Графики взаимно обратных функций симметричны относительно прямой~$y = x$.
\end{theorem}
\begin{proof}
Пусть даны функции $f \colon X \to Y$ и $g \colon Y \to X$, причём $g = f^{-1}$, тогда $f(a) = b$, $g(b) \opbr= g(f(a)) \opbr= a$, где $a \in X$, $b \in Y$.
Т.\,о., точка~$(a, b)$ принадлежит графику функции~$f$ $\Leftrightarrow$ точка~$(b, a)$ принадлежит графику функции~$g$.

Найдём расстояния от точек~$(a, b)$ и $(b, a)$ до произвольной точки~$(c, c)$ прямой~$y = x$:
\begin{equation*}
d_1 = \sqrt{(a - c)^2 + (b - c)^2}, \
d_2 = \sqrt{(b - c)^2 + (a - c)^2}
\end{equation*}

$d_1 = d_2$, значит, прямая~$y = x$~--- серединный перпендикуляр к отрезку, концами которого являются точки $(a, b)$ и $(b, a)$, поэтому они симметричны относительно $y = x$.
В силу того, что $a$ может принимать любое значение из множества~$X$, графики функций $f$ и $g$ симметричны относительно прямой~$y = x$.
\end{proof}

\subsection{Линейная функция}
\index{Функция!линейная} \textbf{Линейной} называется функция вида $y = kx + b$.
$k$ называется \textbf{угловым коэффициентом}.

Очевидно, что $D(y) = E(y) = \mathbb R$.

Докажем монотонность линейной функции при $k \neq 0$.
\begin{enumerate}
	\item Пусть $k < 0, x_1 < x_2$, тогда
	\begin{equation*}
	f(x_1) - f(x_2) =
	k x_1 + b - k x_2 - b =
	k(x_1 - x_2) > 0 \Rightarrow
	f(x_1) > f(x_2)
	\end{equation*}
	
	Значит, функция убывает.
	
	\item Аналогичным образом легко доказать, что при~$k > 0$ функция возрастает.
\end{enumerate}

График линейной функции~--- прямая.
\begin{center}
$\begin{xy} /r8mm/:
(-3, 0); (3, 0) **@{-} *@{>} *++!U{x};
(0, -2); (0, 3) **@{-} *@{>} *++!R{y};
(-3, 2); (3, -1) **@{-} *+++!R{\scriptstyle y = kx + b, \ k < 0};
{(1, 0) \ellipse<2mm>:a(90),_,:a(-63.435){}} *++!LD{\alpha};
(-1, 1) = "M"; (-1, 0) **@{--} *++!U{x_0};
"M"; (0, 1) **@{--} *++!L{y_0};
\end{xy}$
$\begin{xy} /r8mm/:
(-3, 0); (3, 0) **@{-} *@{>} *++!U{x};
(0, -2); (0, 3) **@{-} *@{>} *++!R{y};
(-2, -2); (3, 2.5) **@{-} *+!DR{\scriptstyle y = kx + b, \ k > 0};
{(0.222, 0) \ellipse<3mm>:a(48.013),^,:a(90){}};
(0.9, 0.25) *{\alpha};
(1.5, 1.15) = "M"; (1.5, 0) **@{--} *++!U{x_0};
"M"; (0, 1.15) **@{--} *++!R{y_0};
\end{xy}$
\end{center}

Докажем, что $k = \tg \alpha$.
Пусть $y_0 = k x_0 + b$.
Заметим, что при $k \neq 0$ график линейной функции пересекает ось абсцисс в точке~$(-\frac{b}k, 0)$.
\begin{enumerate}
	\item Если $k < 0$, то для определённости предположим, что $x_0 < -\frac{b}k$.
	\begin{equation*}
	\tg \alpha =
	-\tg (\pi - \alpha) =
	-\frac{y_0}{-\frac{b}k - x_0} =
	\frac{k y_0}{k x_0 + b} =
	k
	\end{equation*}
	
	\item При $k = 0$ $\tg \alpha = 0$.
	
	\item Если $k > 0$, то для определённости предположим, что $x_0 > -\frac{b}k$.
	\begin{equation*}
	\tg \alpha =
	\frac{y_0}{x_0 - (-\frac{b}k)} =
	\frac{k y_0}{k x_0 + b} =
	k
	\end{equation*}
\end{enumerate}

\subsection{Квадратичная функция}
\index{Функция!квадратичная} \textbf{Квадратичной} называется функция вида $y = ax^2 + bx + c$, где $a \neq 0$.

Очевидно, что $D(y) = \mathbb R$.

Для исследования функции выделим в правой части уравнения полный квадрат:
\begin{equation*}
y = a \left( x^2 + \frac{b}a x \right) + c \Leftrightarrow
y = a \left( x + \frac{b}{2a} \right)^2 + c - \frac{b^2}{4a} \Leftrightarrow
y = a \left( x + \frac{b}{2a} \right)^2 - \frac{b^2 - 4ac}{4a}
\end{equation*}
\begin{itemize}
	\item Пусть $a < 0$.
	
	$\displaystyle E(y) = \left( -\infty; -\frac{b^2 - 4ac}{4a} \right]$
	
	Функция возрастает на~$\bigl( -\infty; -\frac{b}{2a} \bigr]$ и убывает на~$\bigl[ -\frac{b}{2a}; +\infty \bigr)$.
	
	\item Пусть $a > 0$.
	
	$\displaystyle E(y) = \left[ -\frac{b^2 - 4ac}{4a}; +\infty \right)$
	
	Функция убывает на~$\bigl( -\infty; -\frac{b}{2a} \bigr]$ и возрастает на~$\bigl[ -\frac{b}{2a}; +\infty \bigr)$.
\end{itemize}

Найдём нули функции:
\begin{equation*}
a \left( x + \frac{b}{2a} \right)^2 - \frac{b^2 - 4ac}{4a} = 0 \Leftrightarrow
\left( x + \frac{b}{2a} \right)^2 = \frac{b^2 - 4ac}{4a^2} \Leftrightarrow
x + \frac{b}{2a} = \pm\frac{\sqrt{b^2 - 4ac}}{2a} \Leftrightarrow
\end{equation*}
\begin{equation*}
\Leftrightarrow x = \frac{-b \pm \sqrt{b^2 - 4ac}}{2a}, \ b^2 - 4ac \geqslant 0
\end{equation*}

Если $b^2 - 4ac < 0$, то график квадратичной функции не пересекает ось абсцисс.
\index{Дискриминант} Выражение $D = b^2 - 4ac$ называется \textbf{дискриминантом квадратного многочлена}.

\index{Парабола} График квадратичной функции называется \textbf{параболой}.
При $a < 0$ её \textbf{ветви} направлены вниз, а при $a > 0$~--- вверх.
Точка наименьшего или наибольшего значения функции называется \textbf{вершиной параболы} (на рисунках обозначена через $M$), которая имеет координаты~$\bigl( -\frac{b}{2a}, -\frac{b^2 - 4ac}{4a} \bigr)$.
\begin{center}
$\begin{xy} /r8mm/:
(-2, 0); (2, 0) **@{-} *@{>} *++!U{x};
(0, -3); (0, 3) **@{-} *@{>} *++!R{y};
(-2, -3); (1.5, -3) **\crv{(-0.25, 5)} *++!L{\scriptstyle y = ax^2 + bx + c, \ a < 0};
(-0.25, 1) *{\bullet} *+!DR{M};
\end{xy}$
$\begin{xy} /r8mm/:
(-2, 0); (2, 0) **@{-} *@{>} *++!U{x};
(0, -3); (0, 3) **@{-} *@{>} *++!R{y};
(-1, 3); (2, 3) **\crv{(0.5, -2)} *++!L{\scriptstyle y = ax^2 + bx + c, \ a > 0};
(0.5, 0.5) *{\bullet} *+!UL{M};
\end{xy}$
\end{center}

\subsection{Рациональная функция}
\index{Функция!рациональная} \textbf{Рациональной} называется функция вида $y = \dfrac{a_n x^n + \ldots + a_1 x + a_0}{b_m x^m + \ldots + b_1 x + b_0}$.

$D(y) = \{ x \in \mathbb R \mid b_m x^m + \ldots + b_0 \neq 0 \}$

Выражение~$\frac{P(x)}{Q(x)}$, где $P(x)$ и $Q(x)$~--- многочлены, называется \textbf{правильной рациональной дробью}, если $\deg P(x) < \deg Q(x)$, иначе~--- \textbf{неправильной рациональной дробью}.

Делением многочленов любую рациональную функцию можно представить в виде суммы многочлена и правильной рациональной дроби.

\part{Университетский курс}
\chapter{Арифметика}
\section{Комплексные числа}
\index{i@$i$} \index{Мнимая единица} \textbf{Мнимой единицей} называется число $i \colon i^2 = -1$.

\index{Число!комплексное} \textbf{Комплексным} называется число вида $a + bi$, $a, b \in \mathbb R$.
\index{Число!мнимое} Если $a = 0$, то такое число называется \textbf{мнимым}, или \textbf{чисто мнимым}.
\index{C@$\mathbb C$} Множество комплексных чисел обозначается $\mathbb C$.

Если $z = a + bi$, то $\overline z = a - bi$ называется \textbf{сопряжённым к~$z$}.

Следующие операции над комплексными числами $z_1 = a_1 + b_1 i$, $z_2 = a_2 + b_2 i$, $a_1, b_1, a_2, b_2 \in \mathbb R$ осуществляются так~же, как над вещественными, и обладают теми~же свойствами:
\begin{itemize}
	\item \textbf{Сложение}: $z_1 + z_2 = (a_1 + b_1 i) + (a_2 + b_2 i) = (a_1 + a_2) + (b_1 + b_2)i$
	\item \textbf{Умножение}: $z_1 \cdot z_2 = (a_1 + b_1 i)(a_2 + b_2 i) = (a_1 a_2 - b_1 b_2) + (a_1 b_2 + a_2 b_1)i$
	\item \textbf{Деление}: $\displaystyle \frac{z_1}{z_2} = \frac{a_1 + b_1 i}{a_2 + b_2 i} =
	\frac{(a_1 + b_1 i)(a_2 - b_2 i)}{a_2^2 + b_2^2} =
	\frac{a_1 a_2 + b_1 b_2}{a_2^2 + b_2^2} + \frac{a_2 b_1 - a_1 b_2}{a_2^2 + b_2^2} i$
\end{itemize}

\subsection{Геометрическое представление комплексного числа}
Комплексное число~$a + bi$ принято изображать на координатной плоскости точкой~$(a, b)$, а также радиус"=вектором, соединяющим начало координат с этой точкой.
Такая плоскость называется \textbf{комплексной}.

\textbf{Модулем комплексного числа~$z = a + bi$}, или его \textbf{абсолютной величиной}, называется длина соответствующего радиус-вектора комплексной плоскости, равная
\begin{equation*}
|z| = \sqrt{a^2 + b^2}
\end{equation*}

\index{Arg} \textbf{Аргументом комплексного числа~$z = a + bi$} называется угол соответствующего радиус-вектора на комплексной плоскости (с точностью до $2\pi k$, $k \in \mathbb Z$):
\begin{equation*}
a = |z| \cos \Arg z, \ b = |z| \sin \Arg z
\end{equation*}
\index{arg} \textbf{Главным} аргументом называется значение~$\Arg z \in (-\pi; \pi]$ и обозначается $\arg z$.

\subsection{Тригонометрическая форма комплексного числа}
\textbf{Тригонометрической формой комплексного числа~$z$} называется его представление в~виде
\begin{equation*}
z = |z|(\cos \varphi + i\sin \varphi), \ \varphi = \Arg z
\end{equation*}

При использовании тригонометрических форм операции умножения и деления комплексных чисел
$z_1 = |z_1| \* (\cos \alpha + i\sin \alpha)$, $z_2 = |z_2|(\cos \beta + i\sin \beta)$ упрощаются:
\begin{itemize}
	\item $\displaystyle z_1 z_2 = |z_1| |z_2|(\cos \alpha \cos \beta - \sin \alpha \sin \beta + i(\sin \alpha \cos \beta + \cos \alpha \sin \beta)) =
	|z_1| |z_2|(\cos (\alpha + \beta) + i\sin (\alpha + \beta))$
	
	\item $\displaystyle \frac{z_1}{z_2} = \frac{|z_1|}{|z_2|} \cdot
	\frac{(\cos \alpha \cos \beta + \sin \alpha \sin \beta + i(\sin \alpha \cos \beta - \cos \alpha \sin \beta))}
	{\cos^2 \beta + \sin^2 \beta} =
	\frac{|z_1|}{|z_2|} (\cos (\alpha - \beta) + i\sin (\alpha - \beta))$
\end{itemize}

\index{Формула!Эйлера!в математическом анализе}
\begin{theorem}[формула Эйлера]
\begin{equation*}
\cos x + i\sin x = e^{ix}
\end{equation*}
\end{theorem}
\begin{proof}
Воспользуемся разложением $\cos x$, $\sin x$ и $e^{ix}$ в~\hyperref[eq:Maclaurin_series]{ряд Маклорена}:
\begin{equation*}
\cos x + i\sin x = 1 + \frac{ix}{1!} - \frac{x^2}{2!} - \frac{ix^3}{3!} + \frac{x^4}{4!} + \frac{ix^5}{5!} + \ldots =
1 + \frac{ix}{1!} + \frac{i^2 x^2}{2!} + \frac{i^3 x^3}{3!} + \frac{i^4 x^4}{4!} + \frac{i^5 x^5}{5!} + \ldots = e^{ix}
\end{equation*}
\end{proof}

При подстановке $x = \pi$ в~формулу Эйлера получим замечательное \textbf{тождество Эйлера}, связывающее пять фундаментальных математических констант:
\begin{equation*}
e^{i\pi} + 1 = 0
\end{equation*}

\index{Формула!Муавра}
\begin{theorem}[формула Муавра]
Если $z = |z|(\cos \varphi + i\sin \varphi)$, $n \in \mathbb R$, то
\begin{equation*}
z^n = |z|^n(\cos n\varphi + i\sin n\varphi)
\end{equation*}
\end{theorem}
\begin{proof}
Для $n \in \mathbb N$ формулу можно доказать методом математической индукции, тогда показать истинность формулы для $n \in \mathbb Z$ несложно.
Мы~же докажем формулу сразу для $n \in \mathbb R$, пользуясь формулой Эйлера:
\begin{equation*}
z^n = |z|^n(\cos \varphi + i\sin \varphi)^n =
|z|^n e^{i\varphi n} = |z|^n (\cos n\varphi + i\sin n\varphi)
\end{equation*}
\end{proof}

Пользуясь формулой Муавра, можно извлекать корни из комплексного числа~$z = |z|(\cos \varphi \opbr+ i\sin \varphi)$:
\begin{equation*}
\sqrt[n]{z} = \sqrt[n]{|z|} \left( \cos \frac{\varphi}n + i\sin \frac{\varphi}n \right)
\end{equation*}
Следует не забывать, что $\varphi$ определено с точностью до~$2\pi k, k \in \mathbb Z$, поэтому комплексный корень имеет не одно, а $n$~значений (что можно показать, пользуясь следствием~\ref{conseq:n_roots_of_polynomial}).
\chapter{Линейная алгебра}
\section{Линейные комбинации}
\index{Линейная комбинация} Выражение, построенное на множестве элементов путём сложения этих элементов, умноженных на некоторые коэффициенты, называется \textbf{линейной комбинацией}.
Если все коэффициенты линейной комбинации равны нулю, то она называется \textbf{тривиальной}, иначе~--- \textbf{нетривиальной}.
\section{Векторные пространства}
\index{Векторное пространство} $n$\nobreakdash-мерным векторным пространством над полем вещественных чисел называется множество
\begin{equation*}
V_n = \mathbb R^n = \{ (x_1, \ldots, x_n) \mid x_1, \ldots, x_n \in \mathbb R \}
\end{equation*}
элементы которого называются \textbf{векторами}. Над ними определены операции сложения и умножения на число, удовлетворяющие аксиомам:
\begin{enumerate}
	\item Коммутативность сложения: $\forall \overline a, \overline b \in V_n \
	\overline a + \overline b = \overline b + \overline a$
	\item Ассоциативность сложения: $\forall \overline a, \overline b, \overline c \in V_n \
	\overline a + (\overline b + \overline c) = (\overline a + \overline b) + \overline c$
	\item Существование \textbf{нулевого} вектора, или \textbf{нуля}: $\exists \overline 0 \in V_n \colon \forall \overline a \in V \
	\overline a + \overline 0 = \overline 0 + \overline a = \overline a$
	\item Существование \textbf{противоположного} вектора: $\forall \overline a \in V_n \
	\exists (-\overline a) \in V_n \colon
	\overline a + (-\overline a) = \overline 0$
	\item Ассоциативность умножения на число: $\forall \alpha, \beta \in \mathbb R, \
	\forall \overline a \in V_n \
	\alpha (\beta \overline a) = (\alpha \beta) \overline a$
	\item Дистрибутивность умножения на число относительно сложения векторов: $\forall \alpha \in \mathbb R, \
	\forall \overline a, \overline b \in V_n \
	\alpha (\overline a + \overline b) = \alpha \overline a + \alpha \overline b$
	\item Дистрибутивность умножения на число относительно сложения чисел: $\forall \alpha, \beta \in \mathbb R, \
	\forall \overline a \in V_n \
	(\alpha + \beta) \overline a = \alpha \overline a + \beta \overline a$
	\item Существование \textbf{единицы}: $\forall \overline a \in V_n \
	1 \cdot \overline a = \overline a$
\end{enumerate}

\subsection{Базис и размерность векторного пространства}
Множество векторов~$\overline a_1, \ldots, \overline a_n$ называется \textbf{линейно зависимым}, если
\begin{equation*}
\exists \alpha_1, \ldots, \alpha_n \colon
\sum_{i=1}^n \alpha_i \overline a_i = \overline 0 \lAnd
\sum_{i=1}^n \alpha_i^2 \neq 0
\end{equation*}
иначе~--- \textbf{линейно независимым}.

\index{Базис!векторного пространства} Множество линейно независимых векторов~$\overline e_1, \ldots, \overline e_n$ векторного пространства~$V$ называется \textbf{базисом} этого \textbf{пространства}, если
\begin{equation*}
\forall \overline x \in V \
\exists \alpha_1, \ldots, \alpha_n \colon
\overline x = \sum_{i=1}^n \alpha_i \overline e_i
\end{equation*}

Приведённое равенство называется \textbf{разложением вектора~$\overline x$ по базису~$\overline e_1, \ldots, \overline e_n$.}

\begin{theorem}[о~базисе]
Любой вектор~$\overline x$ может быть разложен по базису~$\overline e_1, \ldots, \overline e_n$ единственным образом.
\end{theorem}
\begin{proof}
Пусть
\begin{equation*}
\overline x = a_1 \overline e_1 + \ldots + a_n \overline e_n
\end{equation*}
\begin{equation*}
\overline x = b_1 \overline e_1 + \ldots + b_n \overline e_n
\end{equation*}

Вычитанием одного равенства из другого получим:
\begin{equation*}
(a_1 - b_1) \overline e_1 + \ldots + (a_n - b_n) \overline e_n = \overline 0
\end{equation*}

В силу линейной независимости векторов~$\overline e_1, \ldots, \overline e_n$
\begin{equation*}
\begin{cases}
a_1 - b_1 = 0 \\
\ldots \\
a_n - b_n = 0
\end{cases}
\Leftrightarrow
\begin{cases}
a_1 = b_1 \\
\ldots \\
a_n = b_n
\end{cases}
\end{equation*}
\end{proof}

\textbf{Размерностью векторного пространства} называется максимальное количество линейно независимых векторов.

\begin{theorem}
В~векторном пространстве~$V$ размерности~$n$ любые $n$ линейно независимых векторов образуют его базис.
\end{theorem}
\begin{proof}
Рассмотрим множество линейно независимых векторов~$\overline e_1, \ldots, \overline e_n \in V$.
Для любого вектора~$\overline x \in V$ множество векторов~$\overline e_1, \ldots, \overline e_n, \overline x$ линейно зависимо, т.\,к. размерность~$V$ равна~$n$, тогда
\begin{equation*}
\exists \alpha_1, \ldots, \alpha_n, \alpha_{n+1} \colon
\sum_{i=1}^n \alpha_i \overline e_i + \alpha_{n+1} \overline x = \overline 0 \lAnd \alpha_{n+1} \neq 0 \Rightarrow
\overline x = \sum_{i=1}^n -\frac{\alpha_i}{\alpha_{n+1}} \overline e_i
\end{equation*}

Значит, векторы~$\overline e_1, \ldots, \overline e_n$ образуют базис пространства~$V$.
\end{proof}

\begin{theorem}
Если векторное пространство~$V$ имеет базис из $n$~векторов, то его размерность равна~$n$.
\end{theorem}
\begin{proof}
Рассмотрим базис, состоящий из векторов~$\overline e_1, \ldots, \overline e_n \in V$.
\begin{equation*}
\forall \overline e_{n+1} \in V \
\exists \alpha_1, \ldots, \alpha_n \colon
\overline e_{n+1} = \sum_{i=1}^n \alpha_i \overline e_i
\end{equation*}

Значит, базис из $n + 1$~векторов не существует, тогда размерность пространства~$V$ равна~$n$.
\end{proof}
\section{Системы линейных алгебраических уравнений}
Система линейных алгебраических уравнений имеет вид
\begin{equation*}
\begin{cases}
a_{11} x_1 + a_{12} x_2 + \dots + a_{1n} x_n = b_1 \\
a_{21} x_1 + a_{22} x_2 + \dots + a_{2n} x_n = b_2 \\
\ldots \\
a_{m1} x_1 + a_{m2} x_2 + \dots + a_{mn} x_n = b_m
\end{cases}
\end{equation*}
где $x_1, \ldots, x_n$~--- переменные.

$a_{11}, a_{12}, \ldots, a_{mn}$ называются \textbf{коэффициентами при переменных}, $b_1, b_2, \dots, b_m$~--- \textbf{свободными членами}.

Система линейных уравнений называется \textbf{однородной}, если все её свободные члены равны~$0$, иначе~--- \textbf{неоднородной}.

Система линейных уравнений называется \textbf{совместной}, если она имеет хотя~бы одно решение, иначе~--- \textbf{несовместной}.

Система линейных уравнений называется \textbf{определённой}, если она имеет единственное решение.
Если система имеет более одного решения, то она называется \textbf{неопределённой}.

Две системы линейных уравнений называются \textbf{эквивалентными}, если их решения совпадают или обе не имеют решений.

Если применить к системе линейных уравнений одно из следующих преобразований, называемых \textbf{элементарными}, то получим систему, эквивалентную исходной, что элементарно проверяется подстановкой:
\begin{enumerate}
	\item Перестановка двух уравнений.
	\item Умножение одного из уравнений на ненулевое число.
	\item Сложение одного уравнения с другим, умноженным на некоторое число.
\end{enumerate}

\subsection{Матричная форма системы линейных уравнений}
Систему линейных уравнений можно представить в матричной форме:
\begin{equation*}
\begin{Vmatrix}
a_{11}x_1 + a_{12}x_2 + \dots + a_{1n}x_n \\
a_{21}x_1 + a_{22}x_2 + \dots + a_{2n}x_n \\
\vdots \\
a_{m1}x_1 + a_{m2}x_2 + \dots + a_{mn}x_n
\end{Vmatrix} =
\begin{Vmatrix}
b_1 \\
b_2 \\
\vdots \\
b_m
\end{Vmatrix}
\Leftrightarrow
\end{equation*}
\begin{equation*}
\Leftrightarrow
\begin{Vmatrix}
a_{11} & a_{12} & \cdots & a_{1n} \\
a_{21} & a_{22} & \cdots & a_{2n} \\
\vdots & \vdots & \ddots & \vdots \\
a_{m1} & a_{m2} & \cdots & a_{mn}
\end{Vmatrix} \cdot
\begin{Vmatrix}
x_1 \\
x_2 \\
\vdots \\
x_n
\end{Vmatrix} =
\begin{Vmatrix}
b_1 \\
b_2 \\
\vdots \\
b_m
\end{Vmatrix}
\Leftrightarrow
\end{equation*}
\begin{equation*}
\Leftrightarrow
A \cdot X = B
\end{equation*}

$A$ называется \textbf{основной матрицей системы}, $X$~--- \textbf{столбцом переменных}, $B$~--- \textbf{столбцом свободных членов}.
Если к основной матрице справа приписать столбец свободных членов, то получится \textbf{расширенная матрица системы}:
\begin{equation*}
\begin{Vmatrix}
a_{11} & a_{12} & \cdots & a_{1n} & \vline & b_1 \\
a_{21} & a_{22} & \cdots & a_{2n} & \vline & b_2 \\
\vdots & \vdots & \ddots & \vdots & \vline & \vdots \\
a_{m1} & a_{m2} & \cdots & a_{mn} & \vline & b_m
\end{Vmatrix}
\end{equation*}

\subsection{Линейная независимость}
Уравнение системы линейных уравнений называется \textbf{линейно зависимым}, если соответствующая ему строка расширенной матрицы является нетривиальной линейной комбинацией других строк, иначе~--- \textbf{линейно независимым}.

Система линейных уравнений называется \textbf{линейно зависимой}, если существует нетривиальная линейная комбинация строк расширенной матрицы, в~результате которой получается нулевая строка, иначе~--- \textbf{линейно независимой}.

\begin{statement}
Система линейных уравнений линейно зависима $\Leftrightarrow$ одно из её уравнений линейно зависимо.
\end{statement}
\begin{proof}
\begin{enumerate}
	\item $\Rightarrow$. Пусть система строк~$A_1, \ldots, A_n$ линейно зависима:
	\begin{equation*}
	\sum_{i=1}^n \alpha_i A_i = O \lAnd \sum_{i=1}^n \alpha_i^2 \neq 0
	\end{equation*}
	где $O$~--- нулевая строка. Без ограничения общности можно считать, что $\alpha_1 \neq 0$, тогда
	\begin{equation*}
	A_1 = -\sum_{i=2}^n \frac{\alpha_i}{\alpha_1} A_i
	\end{equation*}
	
	Значит, $A_1$~--- линейно зависимая строка.
	
	\item $\Leftarrow$. Пусть одна из строк линейно зависима:
	\begin{equation*}
	A_1 = \sum_{i=2}^n \alpha_i A_i \Leftrightarrow
	1 \cdot A_1 - \alpha_2 A_2 - \ldots - \alpha_n A_n = O
	\end{equation*}
	
	Значит, система линейно зависима.
\end{enumerate}
\end{proof}

\subsection{Решение систем линейных уравнений}
\begin{lemma}
\label{lemma:linear_independency}
Пусть система строк~$A_1, \ldots, A_n$ линейно независима и $A_{n+1}$ не является линейной комбинацией $A_1, \ldots, A_n$. Тогда система строк~$A_1, \ldots, A_n, A_{n+1}$ линейно независима.
\end{lemma}
\begin{proofcontra}
Пусть система строк~$A_1, \ldots, A_n, A_{n+1}$ линейно зависима:
\begin{equation*}
\sum_{i=1}^{n+1} \alpha_i A_i = O \lAnd
\sum_{i=1}^{n+1} \alpha_i^2 \neq 0
\end{equation*}
где $O$~--- нулевая строка.
Система строк~$A_1, \ldots, A_n$ линейно независима по условию, тогда
\begin{equation*}
\alpha_{n+1} \neq 0 \Rightarrow A_{n+1} = -\sum_{i=1}^n \frac{\alpha_i}{\alpha_{n+1}} A_i
\end{equation*}

Значит, $A_{n+1}$~--- линейная комбинация $A_1, \ldots, A_n$.
Противоречие с условием.
\end{proofcontra}

\index{Теорема!Кронекера~---~Капелли}
\begin{theorem}[Кронекера~---~Капелли]
Система линейных уравнений совместна $\Leftrightarrow$ ранг основной матрицы~$A$ совпадает с рангом расширенной матрицы.
\end{theorem}
\begin{proof}
\begin{enumerate}
	\item $\Rightarrow$. Пусть $(a_1, \ldots, a_n)$~--- решение системы, $B$~--- столбец свободных членов системы.
	Тогда $\sum\limits_{i=1}^n a_i A^i = B$, значит, $B$~--- линейная комбинация столбцов~$A^1, \ldots, A^n$, поэтому ранг расширенной матрицы совпадает с рангом основной.
	
	\item $\Leftarrow$. Пусть ранг основной матрицы равен рангу расширенной.
	Предположим, что система несовместна, тогда $B$ не является линейной комбинацией столбцов~$A^1, \ldots, A^n$, значит, по лемме~\ref*{lemma:linear_independency} система строк $A^1, \ldots, A^n, B$ линейно независима.
	Получили, что ранг расширенной матрицы больше ранга основной.
	Противоречие.
\end{enumerate}
\end{proof}

\subsubsection{Метод Гаусса}
Пусть дана система линейных уравнений
\begin{equation}
\label{eq:Gaussian_elimination(1)}
\begin{cases}
a_{11} x_1 + a_{12} x_2 + \dots + a_{1n} x_n = b_1 \\
a_{21} x_1 + a_{22} x_2 + \dots + a_{2n} x_n = b_2 \\
\ldots \\
a_{m1} x_1 + a_{m2} x_2 + \dots + a_{mn} x_n = b_m
\end{cases}
\end{equation}

Её расширенную матрицу можно привести к ступенчатому виду, т.\,е. (\ref*{eq:Gaussian_elimination(1)}) эквивалентна
\begin{equation}
\label{eq:Gaussian_elimination(2)}
\begin{cases}
a_{1\, j_1} x_{j_1} + \ldots + a_{1\, j_n} x_{j_n} = b_1 \\
a_{2\, j_2} x_{j_2} + \ldots + a_{2\, j_n} x_{j_n} = b_2 \\
\ldots \\
a_{r\, j_r} x_{j_r} + \ldots + a_{r\, j_n} x_{j_n} = b_r \\
0 = b_{r+1} \\
\ldots \\
0 = b_m
\end{cases}
\end{equation}
где $a_{1\, j_1}, \ldots, a_{r\, j_r} \neq 0$.
Без ограничения общности можно считать, что в~базисный минор основной матрицы системы~(\ref*{eq:Gaussian_elimination(2)}) входят только коэффициенты при переменных~$x_{j_1}, \ldots, x_{j_r}$, называемых \textbf{главными (зависимыми)}.
Остальные переменные называются \textbf{свободными (независимыми)}.

Если $\exists i > r \colon b_i \neq 0$, то система несовместна.
Пусть $\forall i > r \ b_i = 0$. Тогда получим систему
\begin{equation*}
\begin{cases}
\displaystyle x_{j_1} = \frac{b_1}{a_{1\, j_1}} - \frac{a_{1\, j_2}}{a_{1\, j_1}} x_{j_2} - \ldots - \frac{a_{1\, j_n}}{a_{1\, j_1}} x_{j_n} \\
\displaystyle x_{j_2} = \frac{b_2}{a_{2\, j_2}} - \frac{a_{2\, j_3}}{a_{2\, j_2}} x_{j_3} - \ldots - \frac{a_{2\, j_n}}{a_{2\, j_2}} x_{j_n} \\
\ldots \\
\displaystyle x_{j_r} = \frac{b_r}{a_{r\, j_r}} - \frac{a_{r\, j_{r+1}}}{a_{r\, j_r}} x_{j_{r+1}} - \ldots - \frac{a_{r\, j_n}}{a_{r\, j_r}} x_{j_n} \\
\end{cases}
\end{equation*}

Если свободным переменным полученной системы придавать все возможные значения и решать новую систему относительно главных неизвестных от нижнего уравнения к верхнему, то получим все решения данной системы.

\subsubsection{Метод Крамера}
\begin{theorem}[Крамера]
\label{th:Cramer}
Пусть дана система линейно независимых уравнений
\begin{equation*}
\begin{cases}
a_{11}x_1 + a_{12}x_2 + \dots + a_{1n}x_n = b_1 \\
a_{21}x_1 + a_{22}x_2 + \dots + a_{2n}x_n = b_2 \\
\ldots \\
a_{n1}x_1 + a_{n2}x_2 + \dots + a_{nn}x_n = b_n
\end{cases}
\end{equation*}

Если определитель её основной матрицы не равен~$0$, то система имеет единственное решение.
\end{theorem}
\begin{proof}
Запишем систему в~матричной форме:
\begin{equation*}
AX = B \Leftrightarrow
A^{-1}AX = A^{-1}B \Leftrightarrow
X = A^{-1}B \Leftrightarrow
\end{equation*}
\begin{equation*}
\Leftrightarrow
\begin{Vmatrix}
x_1 \\
x_2 \\
\vdots \\
x_n
\end{Vmatrix} =
\begin{Vmatrix}
\dfrac{A_{11}}{|A|} & \dfrac{A_{21}}{|A|} & \cdots & \dfrac{A_{n1}}{|A|} \\
\dfrac{A_{12}}{|A|} & \dfrac{A_{22}}{|A|} & \cdots & \dfrac{A_{n2}}{|A|} \\
\vdots & \vdots & \ddots & \vdots \\
\dfrac{A_{1n}}{|A|} & \dfrac{A_{2n}}{|A|} & \cdots & \dfrac{A_{nn}}{|A|}
\end{Vmatrix} \cdot
\begin{Vmatrix}
b_1 \\
b_2 \\
\vdots \\
b_n
\end{Vmatrix}
\end{equation*}
где $A_{ij}$~--- алгебраическое дополнение $a_{ij}$.

\index{Формула!Крамера}
Т.\,о., получим решение системы:
\begin{equation}
\label{eq:Cramer's_formula}
x_i = \frac{\displaystyle \sum_{j=1}^n A_{ji} b_j}{|A|} =
\frac1{{|A|}} \cdot
\begin{vmatrix}
a_{11} & \cdots & a_{1\, i-1} & b_{1} & a_{1\, i+1} & \cdots & a_{1n} \\
a_{21} & \cdots & a_{2\, i-1} & b_{2} & a_{2\, i+1} & \cdots & a_{2n} \\
\vdots & \ddots & \vdots & \vdots & \vdots & \ddots & \vdots \\
a_{n1} & \cdots & a_{n\, i-1} & b_{n} & a_{n\, i+1} & \cdots & a_{nn} \\
\end{vmatrix}, \ i = 1, 2, \ldots, n
\end{equation}
\end{proof}

Полученные формулы~(\ref*{eq:Cramer's_formula}) называется \textbf{формулами Крамера}.

% beta
\subsection{Фундаментальная система решений}
\begin{statement}
\label{st:homogeneous_system_sets_vector_space}
Однородная линейно независимая система уравнений
\begin{equation*}
\begin{cases}
\displaystyle \sum_{i=1}^n a_{1i} x_i = 0 \\
\displaystyle \sum_{i=1}^n a_{2i} x_i = 0 \\
\ldots \\
\displaystyle \sum_{i=1}^n a_{mi} x_i = 0
\end{cases}
\end{equation*}
задаёт векторное пространство.
\end{statement}
\begin{proof}
Пусть $(\alpha_1,  \ldots, \alpha_n), (\beta_1,  \ldots, \beta_n)$~--- решения данной системы, $\lambda \neq 0$.
\begin{itemize}
	\item \begin{equation*}
	\begin{cases}
	\displaystyle \sum_{i=1}^n a_{1i} (\alpha_i + \beta_i) = 0 \\
	\displaystyle \sum_{i=1}^n a_{2i} (\alpha_i + \beta_i) = 0 \\
	\ldots \\
	\displaystyle \sum_{i=1}^n a_{mi} (\alpha_i + \beta_i) = 0
	\end{cases}
	\Leftrightarrow
	\begin{cases}
	\displaystyle \sum_{i=1}^n a_{1i} \alpha_i + \sum_{i=1}^n a_{1i} \beta_i = 0 \\
	\displaystyle \sum_{i=1}^n a_{2i} \alpha_i + \sum_{i=1}^n a_{2i} \beta_i = 0 \\
	\ldots \\
	\displaystyle \sum_{i=1}^n a_{mi} \alpha_i + \sum_{i=1}^n a_{mi} \beta_i = 0
	\end{cases}
	\Leftrightarrow
	\begin{cases}
	0 = 0 \\
	0 = 0 \\
	\ldots \\
	0 = 0
	\end{cases}
	\end{equation*}
	
	Значит, $(\alpha_1 + \beta_1, \ldots, \alpha_n + \beta_n)$ тоже является решением системы.
	
	\item \begin{equation*}
	\begin{cases}
	\displaystyle \sum_{i=1}^n a_{1i} \lambda \alpha_i = 0 \\
	\displaystyle \sum_{i=1}^n a_{2i} \lambda \alpha_i = 0 \\
	\ldots \\
	\displaystyle \sum_{i=1}^n a_{mi} \lambda \alpha_i = 0
	\end{cases}
	\Leftrightarrow
	\begin{cases}
	\displaystyle \lambda \sum_{i=1}^n a_{1i} \alpha_i = 0 \\
	\displaystyle \lambda \sum_{i=1}^n a_{2i} \alpha_i = 0 \\
	\ldots \\
	\displaystyle \lambda \sum_{i=1}^n a_{mi} \alpha_i = 0
	\end{cases}
	\Leftrightarrow
	\begin{cases}
	0 = 0 \\
	0 = 0 \\
	\ldots \\
	0 = 0
	\end{cases}
	\end{equation*}
	
	Значит, $(\lambda \alpha_1, \ldots, \lambda \alpha_n)$ тоже является решением системы.
\end{itemize}

Тогда множество решений данной системы~--- векторное пространство.
\end{proof}

\textbf{Фундаментальной системой решений однородной системы линейных уравнений} называется базис множества всех её решений.

Пусть дана однородная линейно независимая система уравнений:
\begin{equation}
\label{eq:homogeneous_system1}
\begin{cases}
\displaystyle \sum_{i=1}^n a_{1i} x_i = 0 \\
\displaystyle \sum_{i=1}^n a_{2i} x_i = 0 \\
\ldots \\
\displaystyle \sum_{i=1}^n a_{mi} x_i = 0
\end{cases} \Leftrightarrow
\begin{cases}
\displaystyle \sum_{i=1}^m a_{1i} x_i = -\sum_{i=m+1}^n a_{1i} x_i \\
\displaystyle \sum_{i=1}^m a_{2i} x_i = -\sum_{i=m+1}^n a_{2i} x_i \\
\ldots \\
\displaystyle \sum_{i=1}^m a_{mi} x_i = -\sum_{i=m+1}^n a_{mi} x_i
\end{cases}
\end{equation}

Присваивая переменным~$x_{m+1}, \ldots, x_n$ произвольные значения, получаем систему уравнений, которая по \hyperref[th:Cramer]{теореме Крамера} имеет единственное решение.
Тогда решения системы~(\ref*{eq:homogeneous_system1})
\begin{equation*}
\overline e_1 = (x_{11}, x_{21}, \ldots, x_{m1}, 1, 0, \ldots, 0), \
\overline e_2 = (x_{12}, x_{22}, \ldots, x_{m2}, 0, 1, \ldots, 0), \ \ldots,
\end{equation*}
\begin{equation*}
\overline e_{n-m} = (x_{1\, n-m}, x_{2\, n-m}, \ldots, x_{m\, n-m}, 0, 0, \ldots, 1)
\end{equation*}

образуют фундаментальную систему решений.
\begin{proof}
Пусть $\overline r = (r_1, \ldots, r_n)$~--- решение системы~(\ref*{eq:homogeneous_system1}), $\overline p = (p_1, \ldots, p_n) = r_{m+1} \overline e_1 \opbr+ r_{m+2} \overline e_2 \opbr+ \ldots \opbr+ r_n \overline e_{n-m} \opbr- \overline r$.
По утверждению~\ref*{st:homogeneous_system_sets_vector_space} $\overline p$~--- решение системы~(\ref*{eq:homogeneous_system1}).
Легко проверить, что $p_{m+1} \opbr= \ldots \opbr= p_n \opbr= 0$.
Подставим эти значения в систему~(\ref*{eq:homogeneous_system1}), тогда по \hyperref[th:Cramer]{теореме Крамера} она имеет единственное решение~--- нулевое.
Значит, $\overline r = r_{m+1} \overline e_1 + r_{m+2} \overline e_2 + \ldots + r_n \overline e_{n-m}$, т.\,е. $\overline e_1, \ldots, \overline e_{n-m}$~--- фундаментальная система решений.
\end{proof}

\begin{theorem}
Пусть дана линейно независимая система уравнений:
\begin{equation}
\label{eq:nonhomogeneous_system}
\begin{cases}
\displaystyle \sum_{i=1}^n a_{1i} x_i = b_1 \\
\displaystyle \sum_{i=1}^n a_{2i} x_i = b_2 \\
\ldots \\
\displaystyle \sum_{i=1}^n a_{mi} x_i = b_m
\end{cases}
\end{equation}

Если $\overline e_0$~--- её решение, а векторы~$\overline e_1, \ldots, \overline e_{n-m}$~--- фундаментальная система решений системы уравнений
\begin{equation}
\label{eq:homogeneous_system2}
\begin{cases}
\displaystyle \sum_{i=1}^n a_{1i} x_i = 0 \\
\displaystyle \sum_{i=1}^n a_{2i} x_i = 0 \\
\ldots \\
\displaystyle \sum_{i=1}^n a_{mi} x_i = 0
\end{cases}
\end{equation}

то любое решение системы~(\ref*{eq:nonhomogeneous_system}) можно найти по формуле
\begin{equation*}
\lambda_1 \overline e_1 + \ldots + \lambda_{n-m} \overline e_{n-m} + \overline e_0
\end{equation*}
где $\lambda_1, \ldots, \lambda_{n-m}$~--- произвольные числа.
\end{theorem}
\begin{proof}
Пусть $\overline v$~--- решение системы~(\ref*{eq:nonhomogeneous_system}).
Убедимся подстановкой, что $\overline v - \overline u_0$~--- решение системы~(\ref*{eq:homogeneous_system2}).
Тогда
\begin{equation*}
\overline v - \overline u_0 = \sum_{i=1}^{n-m} \lambda_i \overline u_i \Leftrightarrow
\overline v = \sum_{i=1}^{n-m} \lambda_i \overline u_i + \overline u_0
\end{equation*}
\end{proof}
\section{Квадратичные формы}
\index{Квадратичная форма} \textbf{Квадратичной формой} называется многочлен, все одночлены в~котором второй степени:
\begin{equation*}
f(x_1, \ldots, x_n) = \sum_{i=1}^n \sum_{j=1}^n a_{ij} x_i x_j
\end{equation*}
Для определённости полагают $a_{ij} = a_{ji}$.

Квадратичной форме можно сопоставить \textbf{матрицу квадратичной формы}, составленную из коэффициентов:
\begin{equation*}
\begin{Vmatrix}
a_{11} & a_{12} & \cdots & a_{1n} \\
a_{21} & a_{22} & \cdots & a_{2n} \\
\vdots & \vdots & \ddots & \vdots \\
a_{n1} & a_{n2} & \cdots & a_{nn}
\end{Vmatrix} =
\begin{Vmatrix}
a_{11} & a_{12} & \cdots & a_{1n} \\
a_{12} & a_{22} & \cdots & a_{2n} \\
\vdots & \vdots & \ddots & \vdots \\
a_{1n} & a_{2n} & \cdots & a_{nn}
\end{Vmatrix}
\end{equation*}

\textbf{Каноническим видом квадратичной формы} называется её представление в~виде суммы квадратов с некоторыми коэффициентами.

\begin{theorem}[метод Лагранжа]
Любая квадратичная форма может быть приведена к каноническому виду.
\end{theorem}
\begin{proof}
Пусть дана квадратичная форма~$\displaystyle f(x_1, \ldots, x_n) = \sum_{i=1}^n \sum_{j=1}^n a_{ij} x_i x_j$.
Возможны два случая:
\begin{enumerate}
	\item $\exists i \colon a_{ii} \neq 0$.
	Без ограничения общности будем считать, что $a_{11} \neq 0$, тогда
	\begin{equation*}
	f(x_1, \ldots, x_n) = a_{11}\left( x_1^2 + \frac{2 x_1}{a_{11}} \sum_{i=2}^n a_{1i} x_i + \frac1{a_{11}^2} \left( \sum_{i=2}^n a_{1i} x_i \right)^2 \right) + \sum_{i=2}^n \sum_{j=2}^n a_{ij} x_i x_j - \frac1{a_{11}} \left( \sum_{i=2}^n a_{1i} x_i \right)^2 =
	\end{equation*}
	\begin{equation*}
	= a_{11}\left( x_1 + \frac1{a_{11}} \sum_{i=2}^n a_{1i} x_i \right)^2 + f_1(x_2, \ldots, x_n)
	\end{equation*}
	
	\item $\forall i \ a_{ii} = 0$.
	Тогда $\exists i, j \colon a_{ij} \neq 0$.
	Без ограничения общности будем считать, что $a_{12} \neq 0$, тогда заменой переменных $x_1 = y_1 + y_2$, $x_2 = y_1 - y_2$, $x_i = y_i$, $i = 3, 4, \ldots, n$ этот случай сводится к первому.
\end{enumerate}

$f_1(x_2, \ldots, x_n)$~--- квадратичная форма от $n - 1$~переменных.
Применяя к ней описанные действия, получим квадратичную форму от $n - 2$~переменных.
Продолжая таким образом, получим канонический вид $f(x_1, \ldots, x_n)$.
\end{proof}

\textbf{Нормальным видом квадратичной формы} называется её канонический вид, коэффициенты в~котором равны $-1$ или $1$.

\textbf{Рангом квадратичной формы} называется количество переменных в~её каноническом виде.
Количество положительных коэффициентов в~каноническом виде квадратичной формы называется её \textbf{положительным индексом}, а отрицательных~--- \textbf{отрицательным индексом}.
\textbf{Сигнатурой квадратичной формы} называется модуль разности положительного и отрицательного индексов.

Ранг, положительный и отрицательный индексы и сигнатура одинаковы для всех канонических видов квадратичной формы.
\chapter{Математический анализ}
\section{Ограниченные подмножества множества \texorpdfstring{$\mathbb R$}{}}
\index{Множество!ограниченное}\index{Мажоранта} Множество~$X \subset \mathbb R$ называется \textbf{ограниченным сверху}, если $\exists a \in \mathbb R \colon \forall x \in X \ x \leqslant a$.
Число~$a$ называется \textbf{мажорантой множества~$X$}.

\index{Миноранта} Множество~$X \subset \mathbb R$ называется \textbf{ограниченным снизу}, если $\exists a \in \mathbb R \colon \forall x \in X \ a \leqslant x$.
Число~$a$ называется \textbf{минорантой множества~$X$}.

Множество, ограниченное и~сверху, и~снизу, называется \textbf{ограниченным}.

\index{max}\index{Максимум!множества} Мажоранта ограниченного сверху множества~$A$, принадлежащая ему, называется \textbf{его максимальным элементом} и обозначается $\max A$.
\index{min}\index{Минимум!множества} Миноранта ограниченного снизу множества~$A$, принадлежащая ему, называется \textbf{его минимальным элементом} и обозначается $\min A$.

Очевидно, что во~множестве может быть не более одного минимального элемента и не более одного максимального элемента.

\index{sup}\index{Супремум} Минимальный элемент множества мажорант ограниченного сверху множества~$A$ называется \textbf{супремумом} и обозначается $\sup A$.

\begin{statement}
\label{st:single_supremum}
Если множество~$A$ ограничено сверху, то $\exists! \sup A$.
\end{statement}
\begin{proof}
Пусть $B$~--- множество всех мажорант множества~$A$, тогда $\forall a \in A, \ b \in B \ a \leqslant b$.
По~\hyperlink{eq:continuity_axiom}{аксиоме непрерывности} $\exists c \in \mathbb R \colon \forall a \in A, \ b \in B \ a \leqslant c \leqslant b$, тогда $c$~--- минимальная мажоранта множества~$A$.

Единственность следует из единственности минимального элемента.
\end{proof}

\begin{statement}
\label{st:inequality_of_supremum}
Если $a = \sup A$, то
$\forall \varepsilon > 0 \ \exists x \in A \colon a - \varepsilon < x \leqslant a$.
\end{statement}
\begin{proofcontra}
Пусть $\exists \varepsilon_0 \colon \forall x \in A \ x \leqslant a - \varepsilon_0$.
Тогда $a - \varepsilon_0$~--- мажоранта множества~$A$, значит, $a \neq \sup A$.
Противоречие.
\end{proofcontra}

\index{inf}\index{Инфимум} Максимальный элемент множества минорант ограниченного снизу множества~$A$ называется \textbf{инфимумом} и обозначается $\inf A$.

\begin{statement}
Если множество~$A$ ограничено снизу, то $\exists! \inf A$.
\end{statement}%
Доказательство аналогично доказательству утверждения~\ref*{st:single_supremum}.

\begin{statement}
\label{st:inequality_of_infimum}
Если $a = \inf A$, то $\forall \varepsilon > 0 \ \exists x \in A \colon a \leqslant x < a + \varepsilon$.
\end{statement}%
Доказательство аналогично доказательству утверждения~\ref*{st:inequality_of_supremum}.

\begin{theorem}[принцип Архимеда]
Если $h > 0$, то
$\forall x \in \mathbb R \ \exists k \in \mathbb Z \colon (k - 1)h \leqslant x < kh$.
\end{theorem}
\begin{proof}
Рассмотрим множество~$A = \left\{ z \in \mathbb Z \mid z > \frac{x}h \right\}$, тогда $\exists a = \inf A$.
По утверждению~\ref*{st:inequality_of_infimum}
$\forall \varepsilon \in \nobreak (0; 1] \ \exists z_0 \colon a \leqslant z_0 < a + \varepsilon$.
Т.\,к. в~промежутке~$[a; a + 1)$ лежит только одно целое число, то $a = z_0$, тогда $a - 1 \leqslant \frac{x}h < a$.
Т.\,о., $a$~--- искомое значение~$k$.
\end{proof}

Из принципа Архимеда следует, что не существует бесконечно больших чисел.

\begin{consequent}
\label{conseq:small_rational_exists}
\begin{equation*}
\forall \varepsilon > 0 \ \exists n \in \mathbb N \colon \frac1n < \varepsilon
\end{equation*}
\end{consequent}
\begin{proof}
По принципу Архимеда для $h = \varepsilon$, $x = 1$ получим:
\begin{equation*}
\forall \varepsilon > 0 \ \exists n \in \mathbb N \colon
(n - 1)\varepsilon \leqslant 1 < n\varepsilon \Leftrightarrow
(1 - \frac1n)\varepsilon \leqslant \frac1n < \varepsilon \Rightarrow
\frac1n < \varepsilon
\end{equation*}
\end{proof}

Отсюда следует, что не существует бесконечно малых чисел.

\begin{consequent}
\begin{equation*}
\forall a, b \in \mathbb R \ \exists c \in \mathbb Q \colon a < c < b
\end{equation*}
\end{consequent}
\begin{proof}
Из следствия~\ref*{conseq:small_rational_exists} для $\varepsilon = b - a$ получим $\exists n \in \mathbb N \colon \frac1n < b - a$.
По принципу Архимеда для $h = \frac1n$, $x = a$ получим:
\begin{equation*}
\exists k \in \mathbb Z \colon \frac{k - 1}n \leqslant a < \frac{k}n \Rightarrow
a < \frac{k}n = \frac{k - 1}n + \frac1n < a + (b - a) = b
\end{equation*}

Т.\,о., $\frac{k}n$~--- искомое значение~$c$.
\end{proof}

\index{Точка!предельная} Точка~$a \in \mathbb R$ называется \textbf{предельной точкой множества~$A \subset \mathbb R$}, если
$\forall \varepsilon > 0 \ \breve U_\varepsilon(a) \cap A \neq \varnothing$.

\index{Точка!дискретная} Точка~$a \in A$ называется \textbf{дискретной точкой множества~$A \subset \mathbb R$}, если
$\exists \varepsilon > 0 \colon \breve U_\varepsilon(a) \cap A = \varnothing$.

\index{Точка!внутренняя} Точка~$a \in A$ называется \textbf{внутренней точкой множества~$A \subset \mathbb R$}, если
$\exists \varepsilon > 0 \colon U_\varepsilon(a) \subset A$.

Множество называется \textbf{открытым}, если состоит только из внутренних точек.

Множество называется \textbf{замкнутым}, если его дополнение~$\overline A$ до $\mathbb R$ является открытым.

\begin{statement}
Множество~$A$ замкнуто $\Leftrightarrow$ оно содержит все свои предельные точки.
\end{statement}
\begin{proof}
\begin{enumerate}
	\item $\Rightarrow$. Докажем методом от противного, что $A$ содержит все свои предельные точки.
	Пусть $\exists a_0 \notin A$~--- предельная точка~$A$, тогда
	\begin{equation*}
	a_0 \in \overline A \Rightarrow
	\exists \varepsilon > 0 \colon U_\varepsilon(a_0) \subset \overline A \Rightarrow
	U_\varepsilon(a_0) \cap A = \varnothing
	\end{equation*}
	
	Значит, $a_0$ не является предельной точкой $A$.
	Противоречие.
	
	\item $\Leftarrow$. Докажем методом от противного, что $\overline A$ открыто.
	Пусть $\exists a \in \overline A \colon \forall \varepsilon > 0 \ U_\varepsilon(a) \cap A \neq \varnothing$, тогда $a \notin A$~--- предельная точка~$A$.
	Противоречие.
\end{enumerate}
\end{proof}

\index{Теорема!Вейерштрасса}
\begin{theorem}[Вейерштрасса]
\label{th:Weierstrass}
Если $A$~--- бесконечное ограниченное множество, то $\exists a \in \mathbb R$~--- предельная точка $A$.
\end{theorem}
\begin{proof}
$A \subseteq [a; b]$, где $a = \inf A$, $b = \sup A$.
Пусть $a$ не является предельной точкой $A$, т.\,е. $\exists \varepsilon_0 > 0 \colon \allowbreak \breve U_{\varepsilon_0}(a) \cap A = \varnothing$, тогда $a \in A$, значит, $a$~--- дискретная точка $A$.

Рассмотрим множество~$B$ точек~$y$ таких, что интервал~$(-\infty; y)$ содержит конечное число точек $A$.
Интервал~$(-\infty; a + \varepsilon_0)$ содержит только одну точку множества~$A$~--- $a$, значит, $\forall k \in (0; 1] \ a + k\varepsilon_0 \in B$.

$A \subset (-\infty; b]$, тогда $b$~--- мажоранта $B$, значит, $\exists c = \sup B$.
\begin{enumerate}
	\item $\forall \varepsilon > 0 \ (-\infty; c - \varepsilon)$ содержит конечное число точек множества~$A$.
	\item $\forall \varepsilon > 0 \ (-\infty; c + \varepsilon)$ содержит бесконечное число точек множества~$A$, т.\,к. $c + \varepsilon \notin B$.
\end{enumerate}

Тогда $\forall \varepsilon > 0 \ \breve U_\varepsilon(c)$ содержит бесконечное число точек множества~$A$, значит, $c$~--- предельная точка множества~$A$.
\end{proof}
\section{Предел последовательности}
\index{lim}\index{Предел!последовательности} \index{Сходимость} Число~$a$ называется \textbf{пределом последовательности~$(x_n)$}, если
\begin{equation*}
\forall \varepsilon > 0 \ \exists n_0 \in \mathbb N \colon \forall n > n_0 \ |x_n - a| < \varepsilon
\end{equation*}
и обозначается $\displaystyle \lim_{n \to \infty} x_n$.
Говорят, что последовательность~$(x_n)$ \textbf{сходится}, если $\displaystyle \exists\lim_{n \to \infty} x_n$, иначе говорят, что $(x_n)$ \textbf{расходится}.

Последовательность~$(x_n)$ называется \textbf{ограниченной}, или \textbf{ограниченной величиной}, если
$\exists a > 0 \colon \forall n \in \mathbb N \allowbreak |x_n| < a$.

\index{Бесконечно малая величина} \textbf{Бесконечно малой величиной} называется последовательность~$(x_n)$, если
\begin{equation*}
\forall \varepsilon > 0 \ \exists n_0 \in \mathbb N \colon \forall n > n_0 \ |x_n| < \varepsilon
\end{equation*}

Можно определить предел последовательности, используя понятие бесконечно малой величины.

Число~$a$ называется \textbf{пределом последовательности~$(x_n)$}, если $x_n = a + \alpha_n$, где $\alpha_n$~--- бесконечно малая величина.

Докажем эквивалентность этих определений.
\begin{proof}
\begin{enumerate}
	\item Пусть дана последовательность~$(x_n)$ такая, что
	\begin{equation*}
	\forall \varepsilon > 0 \ \exists n_0 \in \mathbb N \colon \forall n > n_0 \ |x_n - a| < \varepsilon
	\end{equation*}
	
	Докажем, что $x_n = a + \alpha_n$.
	В~самом деле, $\alpha_n = x_n - a$~--- бесконечно малая величина.
	Т.\,о., $x_n = a + \alpha_n$.
	
	\item Проведя те~же самые рассуждения в~обратную сторону, докажем обратное утверждение.
\end{enumerate}
\end{proof}

Также определяется бесконечный предел:
\begin{itemize}
	\item $\displaystyle \lim_{n \to \infty} x_n = \infty \Leftrightarrow
	\forall M > 0 \ \exists n_0 \in \mathbb N \colon \forall n > n_0 \ |x_n| > M$
	\item $\displaystyle \lim_{n \to \infty} x_n = +\infty \Leftrightarrow
	\forall M > 0 \ \exists n_0 \in \mathbb N \colon \forall n > n_0 \ x_n > M$
	\item $\displaystyle \lim_{n \to \infty} x_n = -\infty \Leftrightarrow
	\forall M > 0 \ \exists n_0 \in \mathbb N \colon \forall n > n_0 \ x_n < -M$
\end{itemize}

\subsection{Элементарные свойства пределов}
\begin{enumerate}
	\item Последовательность может иметь не более одного предела.
	\begin{proof}
	Пусть $\displaystyle a = \lim_{n \to \infty} x_n$, $\displaystyle b = \lim_{n \to \infty} x_n$.
	Тогда
	\begin{equation*}
	\forall \varepsilon > 0 \
	\left( \exists n_1 \in \mathbb N \colon \forall n > n_1 \ |x_n - a| < \frac\varepsilon2 \right) \lAnd
	\left( \exists n_2 \in \mathbb N \colon \forall n > n_2 \ |x_n - b| < \frac\varepsilon2 \right) \Rightarrow
	\end{equation*}
	\begin{equation*}
	\Rightarrow \forall n > \max \{ n_1, n_2 \} \ |a - b| = |a - x_n + x_n - b| \leqslant |a - x_n| + |x_n - b| < \varepsilon
	\end{equation*}
	
	Значит, $a = b$.
	\end{proof}
	
	\item \begin{theorem}[о~двух милиционерах]
	\index{Теорема!о двух милиционерах}
	Пусть $\forall n \in \mathbb N \ x_n \leqslant y_n \leqslant z_n$.
	Если $\displaystyle \lim_{n \to \infty} x_n = \lim_{n \to \infty} z_n = a$, то $\displaystyle \lim_{n \to \infty} y_n = a$.
	\end{theorem}
	\begin{proof}
	\begin{equation*}
	\forall \varepsilon > 0 \
	\left( \exists n_1 \in \mathbb N \colon \forall n > n_1 \ |x_n - a| < \varepsilon \right) \lAnd
	\left( \exists n_2 \in \mathbb N \colon \forall n > n_2 \ |z_n - a| < \varepsilon \right) \Rightarrow
	\end{equation*}
	\begin{equation*}
	\Rightarrow \forall n > \max \{ n_1, n_2 \} \ a - \varepsilon < x_n \leqslant y_n \leqslant z_n < a + \varepsilon \Rightarrow |y_n - a| < \varepsilon
	\end{equation*}
	\end{proof}
	
	\item Если $\displaystyle \forall n \in \mathbb N \ x_n \geqslant 0, \ \lim_{n \to \infty} x_n = a$, то $a \geqslant 0$.
	\begin{proofcontra}
	Пусть $a < 0$.
	\begin{equation*}
	\exists n_0 \in \mathbb N \colon \forall n > n_0 \ |x_n - a| < -\frac{a}2 \Rightarrow
	\frac{a}2 < x_n - a < -\frac{a}2 \Rightarrow
	x_n < \frac{a}2 < 0
	\end{equation*}
	
	Противоречие.
	\end{proofcontra}
	
	\item Если $\displaystyle \lim_{n \to \infty} x_n = a$, то $\displaystyle \lim_{n \to \infty} |x_n| = |a|$.
	\begin{proof}
	\begin{equation*}
	\begin{cases}
	|a| - |x_n| \leqslant |a - x_n| \\
	|x_n| - |a| \leqslant |x_n - a|
	\end{cases}
	\Rightarrow ||x_n| - |a|| \leqslant |x_n - a|
	\end{equation*}
	
	Тогда
	\begin{equation*}
	\lim_{n \to \infty} x_n = a \Rightarrow
	\forall \varepsilon > 0 \ \exists n_0 \in \mathbb N \colon \forall n > n_0 \ ||x_n| - |a|| \leqslant |x_n - a| < \varepsilon \Rightarrow
	\lim_{n \to \infty} |x_n| = |a|
	\end{equation*}
	\end{proof}
	
	\item Если последовательность~$(x_n)$ сходится, то она ограничена.
	\begin{proof}
	Пусть $\displaystyle \lim_{n \to \infty} x_n = a \Rightarrow \lim_{n \to \infty} |x_n| = |a|$.
	Получим:
	\begin{equation*}
	\exists n_0 \in \mathbb N \colon \forall n > n_0 \ ||x_n| - |a|| < 1 \Rightarrow |x_n| < |a| + 1
	\end{equation*}
	
	Тогда $\forall n \in \mathbb N \ |x_n| < \max \{ |x_1| + 1, |x_2| + 1, \ldots, |x_{n_0}| + 1, |a| + 1 \}$.
	\end{proof}
	
	\item Если последовательности $(x_n)$ и $(y_n)$~--- ограниченная и бесконечно малая величины соответственно, то $z_n \opbr= x_n y_n$~--- бесконечно малая величина.
	\begin{proof}
	Пусть $a > 0 \colon \forall n \in \mathbb N \ |x_n| < a$, тогда
	\begin{equation*}
	\forall \varepsilon > 0 \ \exists n_0 \in \mathbb N \colon
	\forall n > n_0 \ |y_n| < \frac\varepsilon{a} \Rightarrow
	|x_n y_n| < \varepsilon \Rightarrow
	\lim_{n \to \infty} z_n = 0
	\end{equation*}
	\end{proof}
	
	\item Если последовательности $(x_n)$ и $(y_n)$~--- бесконечно малые величины, то $z_n = x_n + y_n$~--- тоже бесконечно малая величина.
	\begin{proof}
	\begin{equation*}
	\forall \varepsilon > 0 \
	\left( \exists n_1 \in \mathbb N \colon \forall n > n_1 \ |x_n| < \frac\varepsilon2 \right) \lAnd
	\left( \exists n_2 \in \mathbb N \colon \forall n > n_2 \ |y_n| < \frac\varepsilon2 \right) \Rightarrow
	\end{equation*}
	\begin{equation*}
	\Rightarrow \forall n > \max \{ n_1, n_2 \} \ |x_n + y_n| \leqslant |x_n| + |y_n| < \varepsilon \Rightarrow
	\lim_{n \to \infty} z_n = 0
	\end{equation*}
	\end{proof}
\end{enumerate}

\subsection{Арифметические свойства пределов}
Пусть даны сходящиеся последовательности~$x_n = a + \alpha_n$ и~$y_n = b + \beta_n$, где $(\alpha_n), (\beta_n)$~--- бесконечно малые величины.
\begin{enumerate}
	\item $\displaystyle \lim_{n \to \infty} (x_n + y_n) = a + b$
	\begin{proof}
	\begin{equation*}
	\lim_{n \to \infty} (x_n + y_n) = \lim_{n \to \infty} (a + b + \alpha_n + \beta_n) = a + b
	\end{equation*}
	\end{proof}
	
	\item $\displaystyle \lim_{n \to \infty} x_n y_n = ab$
	\begin{proof}
	\begin{equation*}
	\lim_{n \to \infty} x_n y_n = \lim_{n \to \infty} (ab + a\beta_n + b\alpha_n + \alpha_n \beta_n) = ab
	\end{equation*}
	\end{proof}
	
	\item Если $a \neq 0$, то $\displaystyle \lim_{n \to \infty} \frac1{x_n} = \frac1a$
	\begin{proof}
	Покажем, что $\displaystyle \left| \frac1{x_n} - \frac1a \right| = \frac{|\alpha_n|}{|a||\alpha_n + a|}$~--- бесконечно малая величина.
	\begin{equation*}
	\exists n_0 \in \mathbb N \colon \forall n > n_0 \ |x_n - a| < \frac{|a|}2 \Rightarrow
	\end{equation*}
	\begin{equation*}
	\Rightarrow \forall n > n_0 \ |a| = |a - x_n + x_n| \leqslant |a - x_n| + |x_n| < \frac{|a|}2 + |x_n| \Rightarrow |x_n| > \frac{|a|}2 \Rightarrow \frac1{|x_n|} < \frac2{|a|}
	\end{equation*}
	
	Тогда $\dfrac1{|x_n|} = \dfrac1{|\alpha_n + a|}$~--- ограниченная величина, значит,
	$\displaystyle \frac{|\alpha_n|}{|a||\alpha_n + a|} = \left| \frac1{x_n} - \frac1a \right|$~--- бесконечно малая величина.
	Отсюда $\displaystyle \lim_{n \to \infty} \left( \frac1{x_n} - \frac1a \right) = 0 \Leftrightarrow \lim_{n \to \infty} \frac1{x_n} = \frac1a$.
	\end{proof}
	
	\item Если $b \neq 0$, то $\displaystyle \lim_{n \to \infty} \frac{x_n}{y_n} = \frac{a}b$.
	\begin{proof}
	\begin{equation*}
	\lim_{n \to \infty} \frac{x_n}{y_n} =
	\lim_{n \to \infty} \left( x_n \cdot \frac1{y_n} \right) =
	a \cdot \frac1b = \frac{a}b
	\end{equation*}
	\end{proof}
\end{enumerate}

\subsection{Основные свойства пределов последовательностей}
\begin{enumerate}
	\item Из ограниченной последовательности можно выбрать сходящуюся подпоследовательность.
	\begin{proof}
	Пусть $A$~--- множество значений, принимаемых членами ограниченной последовательности~$(x_n)$.
	\begin{enumerate}
		\item Пусть $A$ конечно.
		Тогда бесконечное множество членов последовательности~$(x_n)$ принимает хотя~бы одно значение из~$A$, значит, подпоследовательность, состоящая из~них, сходится к этому значению.
		
		\item Пусть $A$ бесконечно, тогда оно ограничено, значит, по \hyperref[th:Weierstrass]{теореме Вейерштрасса} оно имеет предельную точку~$a$.
		В окрестности~$\breve U_1(a)$ содержится хотя~бы одна точка из множества~$A$, а соответствующее значение принимает член~$x_{n_1}$.
		
		Рассмотрим множество~$A_1$, полученное из~$A$ удалением значений, принимаемых членами $x_1, x_2, \ldots, x_{n_1}$.
		$A_1$ бесконечно и имеет предельную точку~$a$, поэтому в окрестности~$\breve U_\frac12(a)$ найдётся значение, принимаемое членом~$x_{n_2}$, причём $n_1 < n_2$.
		
		Рассмотрим множество~$A_2$, полученное из~$A_1$ удалением значений, принимаемых членами $x_{n_1 + 1}, x_{n_1 + 2}, \allowbreak \ldots, x_{n_2}$.
		$A_2$ бесконечно и имеет предельную точку~$a$, поэтому в окрестности~$\breve U_\frac13(a)$ найдётся значение, принимаемое членом~$x_{n_3}$, причём $n_2 < n_3$.
		
		Продолжая, получим последовательность~$(x_{n_k}) \colon |x_{n_k} - a| < \dfrac1k$. По следствию~\ref{conseq:small_rational_exists} $\displaystyle \lim_{k \to \infty} x_{n_k} \opbr= a$.
	\end{enumerate}
	\end{proof}
	
	\item Монотонная ограниченная последовательность~$(x_n)$ сходится.
	\begin{proof}
	Для опредёленности предположим, что $\forall n \in \mathbb N \ x_n \leqslant x_{n+1}$.
	Последовательность ограничена, поэтому множество~$A$ её значений имеет супремум~$a = \sup A$.
	По утверждению~\ref{st:inequality_of_supremum}
	\begin{equation*}
	\forall \varepsilon > 0 \ \exists k \in \mathbb N \colon a - \varepsilon < x_k \leqslant a \Rightarrow
	\forall n > k \ a - \varepsilon < x_k \leqslant x_n \leqslant a \Rightarrow
	|x_n - a| < \varepsilon \Rightarrow \lim_{n \to \infty} x_n = a
	\end{equation*}
	\end{proof}
	
	\item\index{Лемма!о~вложенных отрезках} \begin{lemma}[о~вложенных отрезках]
	\label{lemma:about_nested_intervals}
	Пусть $(a_n), (b_n)$~--- последовательности концов последовательно вложенных друг в~друга отрезков (т.\,е. $[a_n; b_n] \subset [a_{n-1}; b_{n-1}]$), причём $\displaystyle \lim_{n \to \infty} (b_n - a_n) = 0$.
	Тогда $\displaystyle \bigcap_{k=1}^\infty [a_k; b_k] = \{ a \}$.
	\end{lemma}
	\begin{proof}
	Очевидно, что $(a_n)$ монотонна и ограничена сверху, $(b_n)$ монотонна и ограничена снизу, тогда $\displaystyle \lim_{n \to \infty} a_n = a$, $\displaystyle \lim_{n \to \infty} b_n = b$.
	Имеем:
	\begin{equation*}
	b = \lim_{n \to \infty} b_n = \lim_{n \to \infty} (b_n - a_n + a_n) = \lim_{n \to \infty} (b_n - a_n) + \lim_{n \to \infty} a_n = a
	\end{equation*}
	
	Отрезки последовательно вложены друг в~друга, поэтому $\displaystyle \bigcap_{k=1}^n [a_k; b_k] = [a_n; b_n]$.
	\begin{equation*}
	\bigcap_{k=1}^\infty [a_k; b_k] = \lim_{n \to \infty} \bigcap_{k=1}^n [a_k; b_k] = \lim_{n \to \infty} [a_n; b_n] = \{ a \}
	\end{equation*}
	\end{proof}
\end{enumerate}

\subsection{Число Эйлера}
\begin{statement}
\begin{equation*}
\exists \lim_{n \to \infty} \left( 1 + \frac1n \right)^n
\end{equation*}
\end{statement}
\begin{proof}
Рассмотрим последовательность~$(x_n) \colon$
\begin{equation*}
x_n = \left( 1 + \frac1n \right)^n = 1 + \frac{n}n + \frac{n(n - 1)}{2n^2} + \frac{n(n - 1)(n - 2)}{2 \cdot 3n^3} + \ldots + \frac{n!}{n!n^n} =
\end{equation*}
\begin{equation}
\label{eq:Euler's_number_1}
= 2 + \frac1{2!} \left( 1 - \frac1n \right) + \frac1{3!} \left( 1 - \frac1n \right) \left( 1 - \frac2n \right) + \ldots + \frac1{n!} \left( 1 - \frac1n \right) \left( 1 - \frac2n \right) \cdot \ldots \cdot \left( 1 - \frac{n - 1}n \right) <
\end{equation}
\begin{equation*}
< 2 + \frac1{2!} + \frac1{3!} + \ldots + \frac1{n!} < 2 + \frac1{2^1} + \frac1{2^2} + \ldots + \frac1{2^{n-1}} = 2 + 1 - \frac1{2^{n-1}} < 3
\end{equation*}

Значит, $(x_n)$ ограничена.
Кроме того, из выражения~(\ref*{eq:Euler's_number_1}) ясно, что $(x_n)$ монотонна.
Тогда $(x_n)$ сходится.
\end{proof}

\index{e@$e$} \index{Число!Эйлера} Число~$\displaystyle e = \lim_{n \to \infty} \left( 1 + \frac1n \right)^n = 2{,}718281828\ldots$ называется \textbf{числом Эйлера} (иногда \textbf{числом Непера}, или \textbf{неперовым числом}).

\subsection{Критерий Коши}
\index{Последовательность!фундаментальная} Последовательность~$(x_n)$ называется \textbf{фундаментальной}, если
\begin{equation*}
\forall \varepsilon > 0 \ \exists n_0 \in \mathbb N \colon \forall m, n > n_0 \ |x_n - x_m| < \varepsilon
\end{equation*}

\begin{theorem}[критерий Коши]
Последовательность сходится $\Leftrightarrow$ она фундаментальна.
\end{theorem}
\begin{proof}
\begin{enumerate}
	\item $\Rightarrow$. Пусть $\displaystyle \lim_{n \to \infty} x_n = a$, тогда
	\begin{equation*}
	\forall \varepsilon > 0 \ \exists n_0 \in \mathbb N \colon \forall n > n_0 \ |x_n - a| < \frac\varepsilon2
	\end{equation*}
	
	Пусть $m, n > n_0$.
	\begin{equation*}
	|x_n - x_m| = |x_n - a + a - x_m| \leqslant |x_n - a| + |a - x_m| < \frac\varepsilon2 + \frac\varepsilon2 = \varepsilon
	\end{equation*}
	
	\item $\Leftarrow$.
	\begin{equation*}
	\exists n_0 \in \mathbb N \colon \forall m, n > n_0 \ |x_n - x_m| < 1 \Rightarrow
	\end{equation*}
	\begin{equation*}
	\Rightarrow |x_n - x_{n_0 + 1}| < 1 \Rightarrow
	||x_n| - |x_{n_0 + 1}|| < 1 \Rightarrow
	|x_n| < |x_{n_0 + 1}| + 1
	\end{equation*}
	
	Значит, $\forall x \in \mathbb N \ |x_n| < \max \{ |x_1| + 1, |x_2| + 1, \ldots, |x_{n_0}| + 1 \}$, т.\,е. $(x_n)$ ограничена.
	
	Выберем из неё сходящуюся подпоследовательность~$\displaystyle (x_{n_k}): \lim_{k \to \infty} x_{n_k} = a$.
	\begin{equation*}
	\forall \varepsilon > 0 \ \exists k_0, n_1 \in \mathbb N \colon \forall k > k_0, n > n_1 \ |x_n - x_{n_k}| < \varepsilon \Rightarrow
	\end{equation*}
	\begin{equation*}
	\left| \text{ При } k \to \infty \text{ получим } \right|
	\end{equation*}
	\begin{equation*}
	\Rightarrow |x_n - a| < \varepsilon \Rightarrow
	\lim_{n \to \infty} x_n = a
	\end{equation*}
\end{enumerate}
\end{proof}
\section{Предел функции}
\subsection{Предел функции в точке}
\index{lim} \index{Предел!функции}

Пусть $a$~--- предельная точка области определения функции~$f(x)$.
Следующие определения эквивалентны:
\begin{enumerate}
	\item \textbf{Определение по Гейне}
	
	Число~$b$ называется \textbf{пределом функции~$f(x)$ в точке~$a$}, если $\lim\limits_{n \to \infty} f(x_n) = b$ для любой последовательности~$(x_n) \colon \lim\limits_{n \to \infty} x_n = a$.
	
	\item \textbf{Определение по Коши}
	
	Число~$b$ называется \textbf{пределом функции~$f(x)$ в точке~$a$}, если
	\begin{equation*}
	\forall \varepsilon > 0 \ \exists \delta > 0 \colon \forall x \ (|x - a| < \delta \Rightarrow |f(x) - b| < \varepsilon)
	\end{equation*}
\end{enumerate}

Предел функции~$f(x)$ в точке~$a$ обозначается $\lim\limits_{x \to a} f(x)$.
\begin{proof}[эквивалентности]
\begin{enumerate}
	\item (2) $\Rightarrow$ (1).
	Пусть $\lim\limits_{n \to \infty} x_n = a$, тогда
	\begin{equation*}
	\forall \delta > 0 \ \exists n_0 \in \mathbb N \colon \forall n > n_0 \ |x_n - a| < \delta \Rightarrow
	\forall \varepsilon > 0 \ |f(x_n) - b| < \varepsilon \Rightarrow
	\lim_{n \to \infty} f(x_n) = b
	\end{equation*}
	
	\item (1) $\Rightarrow$ (2).
	Докажем методом от противного, что условия определения~(2) выполняются.
	Пусть
	\begin{equation*}
	\exists \varepsilon_0 > 0 \colon \forall \delta > 0 \ \exists x_0 \colon |x_0 - a| < \delta \lAnd |f(x_0) - b| \geqslant \varepsilon_0
	\end{equation*}
	
	Тогда
	\begin{equation*}
	\forall n \in \mathbb N \ \exists x_n \colon |x_n - a| < \frac1n \lAnd |f(x_n) - b| \geqslant \varepsilon_0
	\end{equation*}
	
	Получили последовательность $(x_n) \colon \lim\limits_{n \to \infty} x_n = a \lAnd \lim\limits_{n \to \infty} f(x_n) \neq b$.
	Противоречие.
\end{enumerate}
\end{proof}

Также определяются односторонние пределы.

Число~$b$ называется \textbf{левым пределом}, или \textbf{пределом слева}, \textbf{функции~$f(x)$ в точке~$a$}, если
\begin{equation*}
\forall \varepsilon > 0 \ \exists \delta > 0 \colon \forall x \ (0 < a - x < \delta \Rightarrow |f(x) - b| < \varepsilon)
\end{equation*}
и обозначается $\lim\limits_{x \to a-0} f(x)$.

Число~$b$ называется \textbf{правым пределом}, или \textbf{пределом справа}, \textbf{функции~$f(x)$ в точке~$a$}, если
\begin{equation*}
\forall \varepsilon > 0 \ \exists \delta > 0 \colon \forall x \ (0 < x - a < \delta \Rightarrow |f(x) - b| < \varepsilon)
\end{equation*}
и обозначается $\lim\limits_{x \to a+0} f(x)$.

При $a = 0$ также существуют обозначения $\lim\limits_{x \to -0} f(x)$ и $\lim\limits_{x \to +0} f(x)$ для левого и правого пределов соответственно.

Т.\,о., $\lim\limits_{x \to a} f(x) = b \Leftrightarrow \lim\limits_{x \to a-0} f(x) = \lim\limits_{x \to a+0} f(x) = b$.

С помощью определения по Гейне и свойств предела последовательности доказываются свойства предела функции в~точке.

Элементарные свойства:
\begin{enumerate}
	\item Функция может иметь не более одного предела в одной точке.
	\item \index{Теорема!о двух милиционерах}
	\begin{theorem}[о двух милиционерах]
	\label{th:about_two_policemen}
	Если в окрестности точки~$a$ $f(x) \leqslant g(x) \leqslant h(x)$, $\lim\limits_{x \to a} f(x) \opbr= \lim\limits_{x \to a} h(x) \opbr= b$, то $\lim\limits_{x \to a} g(x) = b$.
	\end{theorem}
	\item Если в окрестности точки~$a$ $f(x) \geqslant 0$, $\lim\limits_{x \to a} f(x) = b$, то $b \geqslant 0$.
	\item $\lim\limits_{x \to a} f(x) = b \Rightarrow \lim\limits_{x \to a} |f(x)| = |b|$.
\end{enumerate}

Арифметические свойства.
Пусть $\lim\limits_{x \to x_0} f(x) = a$, $\lim\limits_{x \to x_0} g(x) = b$.
\begin{enumerate}
	\item $\lim\limits_{x \to x_0} (f(x) + g(x)) = a + b$
	\item $\lim\limits_{x \to x_0} f(x)g(x) = ab$
	\item Если $a \neq 0$, то $\lim\limits_{x \to x_0} \frac1{f(x)} = \frac1a$
	\item Если $b \neq 0$, то $\lim\limits_{x \to x_0} \frac{f(x)}{g(x)} = \frac{a}b$
\end{enumerate}

\subsection{Предел функции на бесконечности}
Пусть $f(x)$~--- функция.
Следующие определения эквивалентны:
\begin{enumerate}
	\item \textbf{Определение по Гейне}
	
	Число~$a$ называется \textbf{пределом функции~$f(x)$ на бесконечности}, если $\lim\limits_{n \to \infty} f(x_n) = a$ для любой последовательности~$(x_n) \colon \lim\limits_{n \to \infty} x_n = \infty$.
	
	\item \textbf{Определение по Коши}
	
	Число~$a$ называется \textbf{пределом функции~$f(x)$ на бесконечности}, если
	\begin{equation*}
	\forall \varepsilon > 0 \ \exists M > 0 \colon \forall x \ (|x| > M \Rightarrow |f(x) - b| < \varepsilon)
	\end{equation*}
\end{enumerate}

Предел функции~$f(x)$ на бесконечности обозначается $\lim\limits_{x \to \infty} f(x)$.
\begin{proof}[эквивалентности]
\begin{enumerate}
	\item (2) $\Rightarrow$ (1).
	Пусть $\lim\limits_{n \to \infty} x_n = \infty$, тогда
	\begin{equation*}
	\forall M > 0 \ \exists n_0 \in \mathbb N \colon \forall n > n_0 \ |x_n| > M \Rightarrow
	\end{equation*}
	\begin{equation*}
	\Rightarrow \forall \varepsilon > 0 \ |f(x_n) - a| < \varepsilon \Rightarrow
	\lim_{n \to \infty} f(x_n) = a
	\end{equation*}
	
	\item (1) $\Rightarrow$ (2).
	Докажем методом от противного, что условия определения~(2) выполняются.
	Пусть
	\begin{equation*}
	\exists \varepsilon_0 > 0 \colon \forall M > 0 \ \exists x_0 \colon |x_0| > M \lAnd |f(x_0) - a| \geqslant \varepsilon_0
	\end{equation*}
	
	Тогда
	\begin{equation*}
	\forall n \in \mathbb N \ \exists x_n \colon |x_n| > n \lAnd |f(x_n) - a| \geqslant \varepsilon_0
	\end{equation*}
	
	Получили последовательность $(x_n) \colon \lim\limits_{n \to \infty} x_n = \infty \lAnd \lim\limits_{n \to \infty} f(x_n) \neq a$.
	Противоречие.
\end{enumerate}
\end{proof}

Аналогично доказывается эквивалентность следующих определений:
\begin{enumerate}
	\item \textbf{Определение по Гейне}
	
	Число~$a$ называется \textbf{пределом функции~$f(x)$ на плюс (минус) бесконечности}, если $\lim\limits_{n \to \infty} f(x_n) = a$ для любой последовательности~$(x_n) \colon \lim\limits_{n \to \infty} x_n = +\infty \ (\lim\limits_{n \to \infty} x_n = -\infty)$.
	
	\item \textbf{Определение по Коши}
	
	Число~$a$ называется \textbf{пределом функции~$f(x)$ на плюс (минус) бесконечности}, если
	\begin{equation*}
	\forall \varepsilon > 0 \ \exists M > 0 \colon \forall x \ (x > M \Rightarrow |f(x) - b| < \varepsilon)
	\end{equation*}
	\begin{equation*}
	(\forall \varepsilon > 0 \ \exists M > 0 \colon \forall x \ (x < -M \Rightarrow |f(x) - b| < \varepsilon))
	\end{equation*}
\end{enumerate}

Предел функции на бесконечности обладает теми же свойствами, что и предел функции в~точке.

\subsection{Замечательные пределы}
Замечательными пределами называют два тождества, часто используемых при нахождении других пределов.

\subsubsection{Первый замечательный предел}
\begin{statement}
\begin{equation*}
\lim_{x \to 0} \frac{\sin x}x = 1
\end{equation*}
\end{statement}
\begin{wrapfigure}{r}{0pt}
\noindent
\shorthandoff{"}
\begin{tikzpicture}[scale=3]
\clip (-0.15, -0.2) rectangle (1.6, 1.1);
\draw circle (1);
\draw (0:1) coordinate["$A$" {below right}] (A) node {$\bullet$}
	-- (0:0) coordinate["$O$" {left}] (O) node {$\bullet$}
	-- (45:1) coordinate["$B$" {above right}] (B) node {$\bullet$}
	-- cycle;
\draw pic[draw, "$x$", angle eccentricity=1.7, angle radius=3mm] {angle = A--O--B};

\draw[name path=tangent] (B) +(135:1) -- +(-45:2);
\path[name path=OA] (O) -- (0:2);
\draw[name intersections={of=OA and tangent, by=C}]
	(C) node {$\bullet$} node[above right] {$C$} -- (A);
\end{tikzpicture}
\shorthandon{"}
\end{wrapfigure}
\begin{proof}
Пусть $x > 0$.
Рассмотрим сектор~$AOB$ единичного круга ($OA = OB = 1$) с углом~$x$ и касательную~$BC$ к нему.

\begin{equation*}
S_{AOB} < S_\text{сект} < S_{BOC} \Leftrightarrow
\sin x < x < \tg x \Leftrightarrow
1 < \frac{x}{\sin x} < \frac1{\cos x}
\end{equation*}

Применяя \hyperref[th:about_two_policemen]{теорему о двух милиционерах}, получим:
\begin{equation*}
\lim_{x \to 0} 1 = \lim_{x \to 0} \frac1{\cos x} = 1 \Rightarrow
\lim_{x \to 0} \frac{\sin x}x = \lim_{x \to 0} \frac{x}{\sin x} = 1
\end{equation*}

Для $x < 0$ $\displaystyle \lim_{x \to 0} \frac{\sin x}x = \lim_{x \to 0} \frac{\sin (-x)}{-x} = 1$.
\end{proof}

Следствия:
\begin{itemize}
	\item $\displaystyle \lim_{x \to 0} \frac{\sin ax}x = a$, $a \neq 0$
	\begin{proof}
	\begin{equation*}
	\lim_{x \to 0} \frac{\sin ax}x = a\lim_{x \to 0} \frac{\sin ax}{ax} = a
	\end{equation*}
	\end{proof}

	\item $\displaystyle \lim_{x \to 0} \frac{\tg ax}x = a$, $a \neq 0$
	\begin{proof}
	\begin{equation*}
	\lim_{x \to 0} \frac{\tg ax}x = a\lim_{x \to 0} \frac{\sin ax}{ax \cos ax} = a
	\end{equation*}
	\end{proof}
	
	\item $\displaystyle \lim_{x \to 0} \frac{1 - \cos ax}{x^2} = \frac{a^2}2$, $a \neq 0$
	\begin{proof}
	\begin{equation*}
	\lim_{x \to 0} \frac{1 - \cos ax}{x^2} =
	\lim_{x \to 0} \frac{2\sin^2 \dfrac{ax}2}{x^2} =
	\frac{a^2}2 \lim_{x \to 0} \frac{\sin^2 \dfrac{ax}2}{\left( \dfrac{ax}2 \right)^2} =
	\frac{a^2}2
	\end{equation*}
	\end{proof}
	
	\item $\displaystyle \lim_{x \to 0} \frac{\arcsin ax}x = a$, $a \neq 0$
	\begin{proof}
	\begin{equation*}
	\lim_{x \to 0} \frac{\arcsin ax}x \;
	\left| \text{Пусть } ax = \sin y \right| =
	a\lim_{y \to 0} \frac{y}{\sin y} = a
	\end{equation*}
	\end{proof}
	
	\item $\displaystyle \lim_{x \to 0} \frac{\arctg ax}x = a$, $a \neq 0$
	\begin{proof}
	\begin{equation*}
	\lim_{x \to 0} \frac{\arctg ax}x \;
	\left| \text{Пусть } x = \tg y \right| =
	a\lim_{y \to 0} \frac{y}{\tg y} = a
	\end{equation*}
	\end{proof}
\end{itemize}

\subsubsection{Второй замечательный предел}
\begin{statement}
\begin{equation*}
\lim_{x \to \infty} \left( 1 + \frac1x \right)^x = e
\end{equation*}
\end{statement}
\begin{proof}
\begin{enumerate}
	\item Пусть $x > 0$.
	По определению числа Эйлера
	$\displaystyle \lim_{x \to +\infty} \left( 1 + \frac1{[x]} \right)^{[x]} =
	\lim_{x \to +\infty} \left( 1 + \frac1{[x] + 1} \right)^{[x] + 1} =
	e$.
	\begin{equation*}
	\left( 1 + \frac1{[x] + 1} \right)^{[x] + 1} \left( 1 + \frac1{[x] + 1} \right)^{-1} =
	\left( 1 + \frac1{[x] + 1} \right)^{[x]} <
	\end{equation*}
	\begin{equation*}
	< \left( 1 + \frac1x \right)^x <
	\left( 1 + \frac1{[x]} \right)^{[x] + 1} =
	\left( 1 + \frac1{[x]} \right)^{[x]} \left( 1 + \frac1{[x]} \right)
	\end{equation*}
	
	Применяя теорему о двух милиционерах, получим:
	\begin{equation*}
	\lim_{x \to +\infty} \left( 1 + \frac1{[x] + 1} \right)^{[x]} =
	\lim_{x \to +\infty} \left( 1 + \frac1{[x] + 1} \right)^{[x] + 1} \left( 1 + \frac1{[x] + 1} \right)^{-1} = e,
	\end{equation*}
	\begin{equation*}
	\lim_{x \to +\infty} \left( 1 + \frac1{[x]} \right)^{[x] + 1} =
	\lim_{x \to +\infty} \left( 1 + \frac1{[x]} \right)^{[x]} \left( 1 + \frac1{[x]} \right) = e \Rightarrow
	\end{equation*}
	\begin{equation*}
	\Rightarrow \lim_{x \to +\infty} \left( 1 + \frac1x \right)^x = e
	\end{equation*}
	
	\item Пусть $x < 0$, $y = -x$, тогда
	\begin{equation*}
	\lim_{x \to -\infty} \left( 1 + \frac1x \right)^x =
	\lim_{y \to +\infty} \left( 1 - \frac1y \right)^{-y} =
	\lim_{y \to +\infty} \left( \frac{y}{y - 1} \right)^y =
	\lim_{y \to +\infty} \left( 1 + \frac1{y - 1} \right)^y =
	\end{equation*}
	\begin{equation*}
	= \lim_{y \to +\infty} \left( 1 + \frac1{y - 1} \right)^{y-1} \left( 1 + \frac1{y - 1} \right) = e
	\end{equation*}
\end{enumerate}
\end{proof}

Следствия:
\begin{itemize}
	\item $\displaystyle \lim_{x \to 0} (1 + ax)^{\tfrac1x} = e^a$, $a \neq 0$
	\begin{proof}
	\begin{equation*}
	\lim_{x \to 0} (1 + ax)^{\tfrac1x} \;
	\left| \text{Пусть } y = \frac1{ax} \right| =
	\lim_{y \to \infty} \left( \left( 1 + \frac1y \right)^y \right)^a =
	e^a
	\end{equation*}
	\end{proof}
	
	\item $\displaystyle \lim_{x \to \infty} \left( 1 + \frac{a}x \right)^x = e^a$, $a \neq 0$
	\begin{proof}
	\begin{equation*}
	\lim_{x \to \infty} \left( 1 + \frac{a}x \right)^x =
	\lim_{x \to \infty} \left( \left( 1 + \frac{a}x \right)^{\tfrac{x}a} \right)^a =
	e^a
	\end{equation*}
	\end{proof}
	
	\item $\displaystyle \lim_{x \to 0} \frac{\ln (1 + ax)}{bx} = \frac{a}b$, $a, b \neq 0$
	\begin{proof}
	\begin{equation*}
	\lim_{x \to 0} \frac{\ln (1 + ax)}{bx} =
	\lim_{x \to 0} \ln (1 + ax)^{\tfrac1{bx}} =
	\ln \lim_{x \to 0} \left( (1 + ax)^{\tfrac1{ax}} \right)^{\tfrac{a}b} =
	\ln e^{\tfrac{a}b} =
	\frac{a}b
	\end{equation*}
	\end{proof}
	
	\item $\displaystyle \lim_{x \to 0} \frac{c^{ax} - 1}{bx} = \frac{a}b \ln c$, $a, b \neq 0$, $c > 0$
	\begin{proof}
	\begin{equation*}
	\lim_{x \to 0} \frac{c^{ax} - 1}{bx} =
	\end{equation*}
	\begin{equation*}
	\left| \text{Пусть } c^{ax} - 1 = y \Leftrightarrow ax\ln c = \ln (y + 1) \Leftrightarrow x = \frac{\ln (y + 1)}{a\ln c} \right|
	\end{equation*}
	\begin{equation*}
	= \frac{a\ln c}b \lim_{y \to 0} \frac{y}{\ln (y + 1)} =
	\frac{a}b \ln c
	\end{equation*}
	\end{proof}
	
	\item $\displaystyle \lim_{x \to 0} \frac{(1 + ax)^n - 1}{bx} = \frac{an}b$, $a, b, n \neq 0$
	\begin{proof}
	\begin{equation*}
	\lim_{x \to 0} \frac{(1 + ax)^n - 1}{bx} \;
	\left| \text{Пусть } 1 + ax = e^y \right| =
	\frac{a}b \lim_{y \to 0} \frac{e^{ny} - 1}{e^y - 1} =
	\frac{an}b \lim_{y \to 0} \frac{e^{ny} - 1}{ny} \cdot \frac{y}{e^y - 1} =
	\frac{an}b
	\end{equation*}
	\end{proof}
\end{itemize}
\section{Бесконечно малые и бесконечно большие функции}
\index{Функция!бесконечно малая} Функция~$\alpha(x)$ называется \textbf{бесконечно малой} при~$x \to x_0$, если $\displaystyle \lim_{x \to x_0} \alpha(x) = 0$.

\index{Функция!бесконечно большая} Функция~$A(x)$ называется \textbf{бесконечно большой} при~$x \to x_0$, если $\displaystyle \lim_{x \to x_0} A(x) = \infty$.

Очевидны следующие утверждения.
\begin{statement}
Если $\alpha(x)$~--- бесконечно малая функция, то $\dfrac1{\alpha(x)}$~--- бесконечно большая функция.
\end{statement}

\begin{statement}
Если $A(x)$~--- бесконечно большая функция, то $\dfrac1{A(x)}$~--- бесконечно малая функция.
\end{statement}

Функции $\alpha(x)$ и $\beta(x)$ называются \textbf{бесконечно малыми одного порядка малости} при~$x \to x_0$, если
$\displaystyle 0 < \left| \lim_{x \to x_0} \frac{\alpha(x)}{\beta(x)} \right| < \infty$.

Функции $\alpha(x)$ и $\beta(x)$ называются \textbf{эквивалентными бесконечно малыми} при~$x \to x_0$, если
$\displaystyle \lim_{x \to x_0} \frac{\alpha(x)}{\beta(x)} = 1$.
При этом пишут $f(x) \sim g(x)$.

Функция $\alpha(x)$ называется \textbf{бесконечно малой более высокого порядка малости}, чем $\beta(x)$, при~$x \to x_0$, если
$\displaystyle \lim_{x \to x_0} \frac{\alpha(x)}{\beta(x)} = 0$, и обозначается $\alpha(x) = o(\beta(x))$.
Следует помнить, что это не равенство в~обычном смысле, т.\,е. запись $o(\beta(x)) = \alpha(x)$ бессмысленна.
\section{Непрерывность функции}
\index{Функция!одной переменной!непрерывная} Пусть функция~$f$ задана на множестве~$D \subseteq \mathbb R$ и $a \in D$.
$f$ называется \textbf{непрерывной в точке~$a$}, если $\displaystyle \lim_{x \to a} f(x) \opbr= f(\lim_{x \to a} x)$, что эквивалентно $\displaystyle \lim_{x \to a} f(x) = f(a)$.

\index{Точка!разрыва} \textbf{Точкой разрыва первого рода функции~$f(x)$} называется точка~$a$, в~которой и~левый, и~правый пределы функции~$f(x)$ конечны, причём $f(x)$ не является непрерывной в точке~$a$.

\textbf{Точкой разрыва второго рода функции~$f(x)$} называется предельная точка~$a$ множества~$D(f)$, в~которой левый или правый предел функции~$f(x)$ не~существует или бесконечен.

Функция называется \textbf{непрерывной на} некотором \textbf{множестве}, если она непрерывна в~каждой точке этого множества.
\section{Производная функции}
Условие непрерывности функции~$f(x)$ в точке~$a$ можно сформулировать так:
\begin{equation*}
\lim_{\Delta x \to 0} \Delta f = \lim_{\Delta x \to 0} (f(a + \Delta x) - f(a)) = 0
\end{equation*}
где $\Delta x = x - a$ называется \textbf{приращением аргумента}, $\Delta f = f(x) - f(a)$~--- \textbf{приращением функции}.

\index{Функция!одной переменной!дифференцируемая}\index{Производная} Функция~$f(x)$ называется \textbf{дифференцируемой в точке~$a$}, если
$\Delta f \opbr= k \Delta x + o(\Delta x)$ при~$\Delta x \to 0$, где
$\Delta f \opbr= f(x) - f(a)$,
$\Delta x \opbr= x - a$,
$k$~--- константа, называемая \textbf{производной функции~$f(x)$ в точке~$a$} и обозначаемая $f'(a)$.

Из определения следует, что функция, дифференцируемая в точке~$a$, непрерывна в~ней.

Функция называется \textbf{дифференцируемой на} некотором \textbf{множестве}, если она дифференцируема в~каждой точке этого множества.

\index{Функция!гладкая} Точки, в~которых функция дифференцируема, называются \textbf{точками гладкости}.
Функция называется \textbf{гладкой}, если она дифференцируема на всей области определения.

Найдём производную функции~$f(x)$ в точке~$a$.
\begin{equation*}
f'(a) = \lim_{\Delta x \to 0} \frac{\Delta f}{\Delta x} = \lim_{x \to a} \frac{f(x) - f(a)}{x - a}
\end{equation*}

Т.\,о., производная функции~$f(x)$ является функцией~$f'(x)$.

Также можно определить односторонние производные путём рассмотрения соответствующих односторонних пределов.

\textbf{Левой производной функции~$f(x)$ в точке~$a$} называется предел~$\displaystyle \lim_{x \to a-0} \frac{f(x) - f(a)}{x - a}$ и обозначается $f_-'(a)$.

\textbf{Правой производной функции~$f(x)$ в точке~$a$} называется предел~$\displaystyle \lim_{x \to a+0} \frac{f(x) - f(a)}{x - a}$ и обозначается $f_+'(a)$.

\subsection{Геометрический смысл производной}
\begin{wrapfigure}{r}{0pt}
\noindent
\shorthandoff{"}
\begin{tikzpicture}
\drawaxis{-0.5}{4}{-0.5}{3};
\draw (-0.6, 0.5) to[out=80, in=180] (1, 1.35)
	to[out=0, in=-120] (2, 2) coordinate["$c$" {above left}] (c) node {$\bullet$}
	to[out=60, in=135] (4, 1.5)
	node[below left] {$y = f(x)$};
	
\draw[name path=tangent] (c) +(60:1.5) -- +(-120:4);
\coordinate (x) at (\right_x, 0);
\draw[name intersections={of=tangent and x_axis, by=z}]
	pic[draw, "$f'(c)$", angle eccentricity=2.75, angle radius=3mm] {angle = x--z--c};
\end{tikzpicture}
\shorthandon{"}
\end{wrapfigure}

Пусть дана кривая, заданная уравнением~$y = f(x)$, $f(x)$ непрерывна на~$[a; b]$.
Проведём касательную к этой кривой в точке~$c \in (a; b)$.
Заметим, что касательная~--- это прямая, получающаяся в~пределе из хорд, проходящих через точки $(c, f(c))$ и $(c + \Delta x, f(c + \Delta x))$.
Уравнение такой хорды имеет вид
\begin{equation*}
\frac{x - c}{(c + \Delta x) - c} = \frac{y - f(c)}{f(c + \Delta x) - f(c)} \Leftrightarrow
y = f(c) + \frac{f(c + \Delta x) - f(c)}{\Delta x} (x - c)
\end{equation*}

Переходя к пределу при~$\Delta x \to 0$, получим значение углового коэффициента~$k$ касательной:
\begin{equation*}
k = \lim_{\Delta x \to 0} \frac{f(c + \Delta x) - f(c)}{\Delta x} = f'(c)
\end{equation*}

Т.\,о., $y = f(c) + f'(c)(x - c)$~--- уравнение касательной в точке~$c$.

Существование касательной означает, что
\begin{equation*}
\displaystyle \exists \lim_{\Delta x \to 0} \frac{f(c + \Delta x) - f(c)}{\Delta x} = f'(c) \Rightarrow
f(c + \Delta x) - f(c) = f'(c) \Delta x + a\Delta x
\end{equation*}
где $\displaystyle \lim_{\Delta x \to 0} a = 0 \Rightarrow a\Delta x = o(\Delta x)$.
Т.\,о., существование касательной к графику функции~$f(x)$ в точке~$c$ равносильно дифференцируемости функции~$f(x)$ в точке~$c$.

\subsection{Физический смысл производной}
Пусть зависимость пути, пройденного некоторой точкой, от времени выражается функцией~$S(t)$.
Чтобы найти среднюю скорость движения в~промежутке времени $[t_0; t_0 + \Delta t]$, достаточно вычислить $\dfrac{S(t_0 + \Delta t) - S(t_0)}{\Delta t}$.
Перейдём к пределу при~$\Delta t \to 0$, тогда $[t_0; t_0 + \Delta t]$ выродится в~точку, а средняя скорость движения превратится в~мгновенную скорость в точке~$t_0$.
Т.\,о., производная функции~$S(t)$ представляет зависимость мгновенной скорости от времени.

\subsection{Дифференциал функции}
\begin{wrapfigure}{r}{0pt}
\noindent
\begin{tikzpicture}[scale=1.5]
\drawaxis{-2}{2.5}{-0.5}{3};
\draw[name path=curve] (-2, 0.15) to[out=10, in=-135] (0.7, 1.15) coordinate (M)
	to[out=45, in=-100] (1.5, 3)
	node[below left] {$y = f(x)$};
\draw[name path=tangent] (M) +(-135:3) -- +(45:3);
\printcoordsonaxis{M}{x_0}{f(x_0)};

\path[name path=vertical] (1.35, \bottom_y) -- (1.35, \top_y);
% рисуем выступы для указания значения df(x_0)
\draw[dashed,
	name intersections={of=curve and vertical, by=x},
	name intersections={of=tangent and vertical, by=delta_x}]
	(delta_x) +(3mm, 0) -- (delta_x -| 0,0)
	node[left] {$f(x_0) + f'(x_0) \Delta x$}
	(delta_x |- M) +(3mm, 0) -- (M);
% указываем значение df(x_0)
\draw[<->] (delta_x |- M) +(2mm, 0) coordinate (temp_point)
	(delta_x) +(2mm, 0) -- (temp_point)
	node[pos=0.5, right] {$df(x_0)$};

\printcoordsonaxis{x}{x}{f(x)};

% надпись \Delta x
\path (M |- 0,0) -- (x |- 0,0) node[pos=0.5, above] {$\Delta x$};
\end{tikzpicture}
\end{wrapfigure}

\index{d@$d$} \index{Дифференциал} Пусть $f(x)$~--- функция, дифференцируемая в точке~$x_0$, тогда по определению
\begin{equation*}
f(x) - f(x_0) = f'(x_0)(x - x_0) + \alpha
\end{equation*}
где $\alpha = o(x - x_0)$ при~$x \to x_0$.
Слагаемое~$f'(x_0)(x - x_0)$ представляет линейную часть приращения функции.
Его называют \textbf{дифференциалом функции~$f(x)$ в точке~$x_0$} и обозначают $df(x_0) = f'(x_0)\,dx$, где $dx = \Delta x$~--- приращение аргумента.

Можно записать производную, используя дифференциал:
\begin{equation*}
f'(x) = \frac{df}{dx}
\end{equation*}

\subsection{Правила дифференцирования}
Пусть $f(x)$, $g(x)$~--- функции, дифференцируемые в точке~$a$, $C$~--- константа.
\begin{enumerate}
	\item $h(x) = f(x) + g(x)$ дифференцируема в точке~$a$, причём $h'(x) = f'(x) + g'(x)$.
	\begin{proof}
	\begin{equation*}
	h'(a) =
	\lim_{x \to a} \frac{(f(x) + g(x)) - (f(a) + g(a))}{x - a} =
	\lim_{x \to a} \frac{f(x) - f(a)}{x - a} + \lim_{x \to a} \frac{g(x) - g(a)}{x - a} =
	f'(a) + g'(a)
	\end{equation*}
	\end{proof}
	
	\item $h(x) = Cf(x)$ дифференцируема в точке~$a$, причём $h'(x) = Cf'(x)$.
	\begin{proof}
	\begin{equation*}
	h'(a) =
	\lim_{x \to a} \frac{Cf(x) - Cf(a)}{x - a} =
	Cf'(x)
	\end{equation*}
	\end{proof}
	
	\item $h(x) = f(x)g(x)$ дифференцируема в точке~$a$, причём $h'(x) = f'(x)g(x) + f(x)g'(x)$.
	\begin{proof}
	\begin{equation*}
	h'(a) =
	\lim_{x \to a} \frac{f(x)g(x) - f(a)g(a)}{x - a} =
	\lim_{x \to a} \frac{f(x)g(x) - f(a)g(x) + f(a)g(x) - f(a)g(a)}{x - a} =
	\end{equation*}
	\begin{equation*}
	= \lim_{x \to a} \frac{f(x) - f(a)}{x - a} g(x) + \lim_{x \to a} f(a) \frac{g(x) - g(a)}{x - a} =
	f'(a)g(a) + f(a)g'(a)
	\end{equation*}
	\end{proof}
	
	\item Если $f(a) \neq 0$, то $h(x) = \dfrac{C}{f(x)}$ дифференцируема в точке~$a$, причём $h'(x) = -\dfrac{Cf'(x)}{f^2(x)}$.
	\begin{proof}
	\begin{equation*}
	h'(a) =
	\lim_{x \to a} \frac{\frac{C}{f(x)} - \frac{C}{f(a)}}{x - a} =
	-C\lim_{x \to a} \frac{f(x) - f(a)}{f(x)f(a)(x - a)} =
	-\frac{Cf'(a)}{f^2(a)}
	\end{equation*}
	\end{proof}
	
	\item Если $g(a) \neq 0$, то $h(x) = \dfrac{f(x)}{g(x)}$ дифференцируема в точке~$a$, причём $h'(x) = \dfrac{f'(x)g(x) - f(x)g'(x)}{g^2(x)}$.
	\begin{proof}
	\begin{equation*}
	h'(x) =
	\left( f(x) \cdot \frac1{g(x)} \right)' =
	\frac{f'(x)}{g(x)} - \frac{f(x)g'(x)}{g^2(x)} =
	\frac{f'(x)g(x) - f(x)g'(x)}{g^2(x)}
	\end{equation*}
	\end{proof}
\end{enumerate}

\subsubsection{Сложная функция}
Если $h(x) = g(f(x))$, то $h'(x) = g'(f(x)) \cdot f'(x)$.
\begin{proof}
\begin{equation*}
h'(a) =
\lim_{x \to a} \frac{g(f(x)) - g(f(a))}{x - a} =
\lim_{x \to a} \frac{g(f(x)) - g(f(a))}{f(x) - f(a)} \cdot \lim_{x \to a} \frac{f(x) - f(a)}{x - a} =
g'(f(a)) \cdot f'(a)
\end{equation*}
\end{proof}

\subsubsection{Обратная функция}
Если $f'(x) \neq 0$, $g(f(x)) = x$, то $g'(x) = \dfrac1{f'(g(x))}$.
\begin{proof}
\begin{equation*}
g'(f(a)) =
\lim_{f(x) \to f(a)} \frac{g(f(x)) - g(f(a))}{f(x) - f(a)} =
\lim_{x \to a} \frac{x - a}{f(x) - f(a)} =
\frac1{f'(a)} = \frac1{f'(g(f(a)))} \Rightarrow g'(x) = \frac1{f'(g(x))}
\end{equation*}
\end{proof}

\subsubsection{Метод логарифмического дифференцирования}
Если $f(x) > 0$, то $(\ln f(x))' = \dfrac{f'(x)}{f(x)} \Leftrightarrow f'(x) = f(x) \cdot (\ln f(x))'$.
\begin{equation*}
(g(x)^{h(x)})' =
g(x)^{h(x)} \cdot (h(x) \ln g(x))' =
g(x)^{h(x)} \cdot \left( h'(x) \ln g(x) + h(x) \frac{g'(x)}{g(x)} \right)
\end{equation*}

\subsubsection{Параметрически заданная функция}
Если $x = \varphi(t)$, $y = \psi(t)$, $\varphi'(t) \neq 0$, то $y'(x) = \dfrac{y'(t)}{x'(t)}$.
\begin{proof}
\begin{equation*}
y'(x) = \frac{dy}{dx} = \frac{\frac{dy}{dt}}{\frac{dx}{dt}} = \frac{y'(t)}{x'(t)}
\end{equation*}
\end{proof}

\subsection{Таблица производных}
Здесь производная берётся по переменной~$x$.
\begin{itemize}
	\item $(C)' = 0$
	
	\item $(x^n)' = nx^{n-1}, \ n \neq 0$
	\begin{proof}
	Пусть $h = x - a$.
	Пользуясь \hyperref[eq:binomial_expansion]{формулой бинома Ньютона}, получим
	\begin{equation*}
	(a^n)' =
	\lim_{h \to 0} \frac{(a + h)^n - a^n}h =
	\lim_{h \to 0} \frac{a^n - a^n + n a^{n-1} h + \frac12 n(n - 1) a^{n-2} h^2 + \ldots}h =
	\end{equation*}
	\begin{equation*}
	= n a^{n-1} + \lim_{h \to 0} \left( \frac12 n(n - 1) a^{n-2} h + \ldots \right) =
	n a^{n-1}
	\end{equation*}
	\end{proof}
	
	\item $(f^n(x))' = nf'(x)f^{n-1}(x), \ n \neq 0$
	
	\item $(|x|)' = \sgn x$
		
	\item $(\ln x)' = \dfrac1x$
	
	\item $(a^x)' = a^x \cdot \ln a, \ a > 0$
	
	\item $(a^{f(x)})' = a^x \cdot \ln a \cdot f'(x), \ a > 0$
	
	\item $(\sin x)' = \cos x$
	\begin{proof}
	\begin{equation*}
	\sin a =
	\lim_{x \to a} \frac{\sin x - \sin a}{x - a} =
	\lim_{x \to a} \frac{2 \cos \frac{x + a}2 \sin \frac{x - a}2}{x - a} =
	\lim_{x \to a} \frac{2 \sin \frac{x - a}2}{x - a} \cdot \lim_{x \to a} \cos \frac{x + a}2 =
	\cos a
	\end{equation*}
	\end{proof}
	
	\item $(\cos x)' = -\sin x$
	\begin{proof}
	\begin{equation*}
	(\cos x)' =
	(\sin \left( \frac\pi{2} - x \right))' =
	-\cos \left( \frac\pi{2} - x \right) =
	-\sin x
	\end{equation*}
	\end{proof}
	
	\item $(\tg x)' = \dfrac1{\cos^2 x}$
	\begin{proof}
	\begin{equation*}
	(\tg x)' =
	\left( \frac{\sin x}{\cos x} \right)' =
	\frac{\cos^2 x + \sin^2 x}{\cos^2 x} =
	\frac1{\cos^2 x}
	\end{equation*}
	\end{proof}
	
	\item $(\ctg x)' = -\dfrac1{\sin^2 x}$
	\begin{proof}
	\begin{equation*}
	(\ctg x)' =
	\left( \frac{\cos x}{\sin x} \right)' =
	\frac{-\sin^2 x - \cos^2 x}{\sin^2 x} =
	-\frac1{\sin^2 x}
	\end{equation*}
	\end{proof}
	
	\item $(\arcsin x)' = \dfrac1{\sqrt{1 - x^2}}$
	\begin{proof}
	\begin{equation*}
	(\arcsin x)' =
	\frac1{\cos \arcsin x} =
	\frac1{\sqrt{1 - \sin^2 \arcsin x}} =
	\frac1{\sqrt{1 - x^2}}
	\end{equation*}
	\end{proof}
	
	\item $(\arccos x)' = -\dfrac1{\sqrt{1 - x^2}}$
	\begin{proof}
	\begin{equation*}
	(\arccos x)' =
	\left( \frac\pi{2} - \arcsin x \right)' =
	-\frac1{\sqrt{1 - x^2}}
	\end{equation*}
	\end{proof}
	
	\item $(\arctg x)' = \dfrac1{1 + x^2}$
	\begin{proof}
	\begin{equation*}
	(\arctg x)' =
	\cos^2 \arctg x =
	\frac1{1 + \tg^2 \arctg x} =
	\frac1{1 + x^2}
	\end{equation*}
	\end{proof}
	
	\item $(\arcctg x)' = -\dfrac1{1 + x^2}$
	\begin{proof}
	\begin{equation*}
	(\arcctg x)' =
	\left( \frac\pi{2} - \arctg x \right)' =
	-\frac1{1 + x^2}
	\end{equation*}
	\end{proof}
\end{itemize}

\subsection{Теоремы о дифференцируемых функциях}
\begin{theorem}[Ролля]
\index{Теорема!Ролля}
Если функция~$f(x)$ непрерывна на~$[a; b]$, дифференцируема на~$(a; b)$, причём $f(a) \opbr= f(b)$, то $\exists c \in (a; b) \colon f'(c) = 0$.
\end{theorem}

\begin{wrapfigure}{r}{0pt}
\noindent
\begin{tikzpicture}
\drawaxis{-2}{4}{-0.5}{2};
\coordinate (c) at (1.25, 1.5);
\printcoordsonaxis{c}{c};

% рисуем кривую
\draw (0.5, 0.5) coordinate (a) parabola bend (c) (3, 0.5) coordinate (b);
\printcoordsonaxis{a}{a};
\printcoordsonaxis{b}{b}{f(a) = f(b)};

% рисуем касательную
\draw (c -| a) -- (c -| b);
\end{tikzpicture}
\end{wrapfigure}

\begin{proof}
Если $f(x) = f(a)$, то в~качестве точки~$c$ можно взять любую точку из~$(a; b)$.

Пусть $f(x)$ не является константой на~$[a; b]$, тогда по свойству~\ref{st:continuous_function_takes_inf_and_sup} непрерывной функции $\displaystyle \exists c \in (a; b) \colon \allowbreak f(c) \opbr= \inf_{x \in [a; b]} f(x) \lOr f(c) = \sup_{x \in [a; b]} f(x)$.
Для определённости предположим, что
\begin{equation*}
f(c) = \inf_{x \in [a; b]} f(x) \Leftrightarrow
\forall x \in [a; b] \ f(x) - f(c) > 0 \Rightarrow
\begin{cases}
\dfrac{f(x) - f(c)}{x - c} < 0, \ x < c \\
\dfrac{f(x) - f(c)}{x - c} > 0, \ x > c
\end{cases}
\end{equation*}

$f(x)$ дифференцируема в точке~$c$, тогда $\displaystyle \exists f'(c) = \lim_{x \to c} \frac{f(x) - f(c)}{x - c}$.
\begin{equation*}
\lim_{x \to c-0} \frac{f(x) - f(c)}{x - c} \leqslant 0 \lAnd
\lim_{x \to c+0} \frac{f(x) - f(c)}{x - c} \geqslant 0 \Rightarrow
f'(c) = 0
\end{equation*}

Доказательство в~случае $\displaystyle f(c) = \sup_{x \in [a; b]} f(x)$ аналогично.
\end{proof}

\begin{theorem}[Коши о среднем значении]
\label{th:Cauchy's_mean_value}
\index{Теорема!Коши о среднем значении}
Если функции $f(x)$ и $g(x)$ непрерывны на~$[a; b]$, дифференцируемы на~$(a; b)$, $g(a) \neq g(b)$, то
\begin{equation*}
\exists c \in (a; b) \colon \frac{f(b) - f(a)}{g(b) - g(a)} = \frac{f'(c)}{g'(c)}
\end{equation*}
\end{theorem}
\begin{proof}
Пусть
\begin{equation*}
F(x) = f(x) - \frac{f(b) - f(a)}{g(b) - g(a)}(g(x) - g(a))
\end{equation*}

$F(x)$ дифференцируема на~$(a; b)$, $F(a) = F(b) = f(a)$, тогда по теореме Ролля
\begin{equation*}
\exists c \in (a; b) \colon 0 = F'(c) = f'(c) - \frac{f(b) - f(a)}{g(b) - g(a)} g'(c) \Rightarrow
\frac{f(b) - f(a)}{g(b) - g(a)} = \frac{f'(c)}{g'(c)}
\end{equation*}
\end{proof}

\index{Формула!конечных приращений} Полагая $g(x) = x$, получим \textbf{формулу конечных приращений}:
\begin{theorem}[Лагранжа о среднем значении]
\label{th:mean_value}
\index{Теорема!Лагранжа о среднем значении}
Если функция~$f(x)$ непрерывна на~$[a; b]$, дифференцируема на~$(a; b)$, то
\begin{equation*}
\exists c \in (a; b) \colon f(b) - f(a) = f'(c)(b - a)
\end{equation*}
\end{theorem}

\begin{center}
\begin{tikzpicture}
\drawaxis{-0.5}{6}{-0.5}{3};
\draw[name path=curve] (0.1, -0.5) to[out=75, in=-110] (0.4, 0.65) coordinate (a)
	to[out=70, in=-150] (1.3, 2) coordinate (c)
	to[out=30, in=135] (2.5, 1)
	to[out=-45, in=-120] (3.5, 1.5)
	to[out=60, in=-110] (4.2, 3)
	node[below right] {$y = f(x)$};
\printcoordsonaxis{a}{a};
\printcoordsonaxis{c}{c};
\draw (c) +(-150:2) -- +(30:3);

\path[name path=AB] (a) -- +(30:7);
\draw[name intersections={of=curve and AB, total=\t}]
	(a) -- (intersection-\t) coordinate (b);
\printcoordsonaxis{b}{b};
\end{tikzpicture}
\end{center}

\subsection{Производные и дифференциалы высших порядков}
Производная произвольного порядка определяется рекуррентно:
\begin{equation*}
\forall n \in \mathbb N \ f^{(0)}(x) = f(x), \ f^{(1)}(x) = f'(x), f^{(2)}(x) = f''(x) \ f^{(n + 1)}(x) = (f^{(n)}(x))'
\end{equation*}

Также определяется дифференциал произвольного порядка:
\begin{equation*}
\forall n \in \mathbb N \ d^0 f(x) = f(x), \ df(x) = f'(x)dx, \ d^2 f(x) = f''(x) dx^2, \ d^n f(x) = f^{(n)}(x) dx^n
\end{equation*}

\subsection{Формула Тейлора}
\index{Формула!Тейлора}
\begin{theorem}[формула Тейлора]
\label{eq:Taylor_series}
Если функция~$f(x)$ в~некоторой окрестности~$U(a)$ имеет все производные порядка $n + 1$ и ниже, то
\begin{equation*}
\forall x \in U(a) \
f(x) = \sum_{k=0}^n \frac{f^{(k)} (x - a)^k}{k!} + R(x), \
R(x) = \frac{f^{(n + 1)}(a + \Theta(x - a))}{(n + 1)!}(x - a)^{n + 1}, \
\Theta \in (0; 1)
\end{equation*}
\end{theorem}

$R(x)$ называется \textbf{остаточным членом} в~форме Лагранжа и используется для оценки ошибки. Также его можно представить в~форме Пеано~--- $R(x) = o((x - a)^n)$~--- которая используется при вычислении пределов.

\index{Формула!Маклорена}
Подставив $a = 0$ в~формулу Тейлора, получим \textbf{формулу Маклорена}:
\begin{equation}
\label{eq:Maclaurin_series}
f(x) = \sum_{k=0}^n \frac{f^{(k)} x^k}{k!} + R(x), \
R(x) = \frac{f^{(n + 1)}(\Theta x)}{(n + 1)!} x^{n + 1}, \
\Theta \in (0; 1)
\end{equation}

\subsubsection{Разложения некоторых функций в ряд Маклорена}
\begin{itemize}
	\item $f(x) = e^x, \
	f^{(n)}(x) = e^x, \
	f^{(n)}(0) = 1$
	\begin{equation*}
	\forall n \in \mathbb N \ f(x) = 1 + x + \frac{x^2}{2!} + \ldots + \frac{x^n}{n!} + R(x), \
	R(x) = \frac{e^{\Theta x}}{(n + 1)!} x^{n + 1}
	\end{equation*}
	\begin{equation*}
	|R(x)| \leqslant e^{\max \{ 0, x \}} \cdot \frac{|x|^{n + 1}}{(n + 1)!} \Rightarrow
	\begin{cases}
	\displaystyle |R(x)| \leqslant \frac{|x|^{n + 1}}{(n + 1)!}, x < 0 \\
	\displaystyle |R(x)| \leqslant 3^x \cdot \frac{|x|^{n + 1}}{(n + 1)!}, x > 0
	\end{cases}
	\end{equation*}
	
	\item $f(x) = \sin x, \
	f^{(n)}(x) = \sin \left( x + \dfrac\pi{2} n \right), \
	f^{(n)}(0) = \sin \dfrac\pi{2} n =
	\begin{cases}
	0, \ n \mult 2 \\
	1, \ \exists k \in \mathbb Z \colon n = 4k + 1 \\
	-1, \ \exists k \in \mathbb Z \colon n = 4k + 3
	\end{cases}$
	\begin{equation*}
	\forall n \in \mathbb N \ f(x) = x - \frac{x^3}{3!} + \frac{x^5}{5!} - \ldots + \frac{(-1)^{n-1} \cdot x^{2n-1}}{(2n - 1)!} + R(x), \
	R(x) = \frac{\sin \left( \Theta x + \frac\pi{2} (2n + 1) \right)}{(2n + 1)!} x^{2n+1}
	\end{equation*}
	\begin{equation*}
	|R(x)| \leqslant \frac{|x|^{2n+1}}{(2n + 1)!}
	\end{equation*}
	
	\item $f(x) = \cos x,
	f^{(n)}(x) = \cos \left( x + \dfrac\pi{2} n \right), \
	f^{(n)}(0) = \cos \dfrac\pi{2} n =
	\begin{cases}
	0, \ n \notmult 2 \\
	1, \ \exists k \in \mathbb Z \colon n = 4k \\
	-1, \ \exists k \in \mathbb Z \colon n = 4k + 2
	\end{cases}$
	\begin{equation*}
	\forall n \in \mathbb N \ f(x) = 1 - \frac{x^2}{2!} + \frac{x^4}{4!} - \ldots + \frac{(-1)^{n-1} x^{2n-2}}{(2n - 2)!} + R(x), \
	R(x) = \frac{\cos (\Theta x + \pi n)}{(2n)!} x^{2n}
	\end{equation*}
	\begin{equation*}
	|R(x)| \leqslant \frac{x^{2n}}{(2n)!}
	\end{equation*}
	
	\item $f(x) = \ln (1 + x), \
	f^{(n)}(x) = \dfrac{(-1)^{n-1} (n - 1)!}{(1 + x)^n}, \
	f^{(n)}(0) = (-1)^{n-1} (n - 1)!$
	\begin{equation*}
	\forall n \in \mathbb N \ f(x) = x - \frac{x^2}2 + \frac{x^3}3 - \ldots + \frac{(-1)^{n-1} x^n}n + R(x), x \in (-1; 1], \
	R(x) = \frac{(-1)^n x^{n+1}}{(n + 1)(1 + \Theta x)^{n+1}}
	\end{equation*}
	
	Для вычисления $\ln a$, $a \neq -1$, можно воспользоваться формулой
	\begin{equation*}
	\forall n \in \mathbb N \ \ln \frac{1 + x_0}{1 - x_0} = \ln (1 + x_0) - \ln (1 - x_0) =
	2 \left( x_0 + \frac{x_0^3}3 + \frac{x_0^5}5 + \ldots \right)
	\end{equation*}
	\begin{equation*}
	a = \frac{1 + x_0}{1 - x_0} \Leftrightarrow
	a - a x_0 = 1 + x_0 \Leftrightarrow
	x_0 = \frac{a - 1}{a + 1}
	\end{equation*}
	
	\item $f(x) = (1 + x)^\alpha, \
	f^{(n)}(x) = \alpha (\alpha - 1) (\alpha - 2) \cdot \ldots \cdot (\alpha - n + 1) (1 + x)^{\alpha - n}, \
	f^{(n)}(0) = \alpha (\alpha - 1) (\alpha - 2) \cdot \ldots \cdot (\alpha - n + 1)$
	\begin{equation*}
	\forall n \in \mathbb N \ f(x) = 1 + \alpha x + \frac{\alpha (\alpha - 1) x^2}{2!} + \ldots + \frac{\alpha (\alpha - 1) \cdot \ldots \cdot (\alpha - n + 1) x^n}{n!} + R(x), |x| < 1,
	\end{equation*}
	\begin{equation*}
	R(x) = \frac{\alpha (\alpha - 1) \cdot \ldots \cdot (\alpha - n) (1 + \Theta x)^{\alpha-n-1}}{(n + 1)!} x^{n+1}
	\end{equation*}
\end{itemize}

\subsection{Правило Лопиталя}
\begin{theorem}[правило Лопиталя]
Если
\begin{enumerate}
	\item $\displaystyle \lim_{x \to a} f(x) = \lim_{x \to a} g(x) = 0 \lOr
	\lim_{x \to a} f(x) = \lim_{x \to a} g(x) = \infty$
	\item $g'(x) \neq 0$
	\item $\displaystyle \exists \lim_{x \to a} \dfrac{f'(x)}{g'(x)}$	
\end{enumerate}
то $\displaystyle \lim_{x \to a} \dfrac{f(x)}{g(x)} = \lim_{x \to a} \dfrac{f'(x)}{g'(x)}$.
\end{theorem}

Если $f(x)$ и $g(x)$ непрерывны в~окрестности точки~$a$, то для случая $\displaystyle \lim_{x \to a} f(x) = \lim_{x \to a} g(x) = 0$ можно провести следующее доказательство:
\begin{equation*}
\lim_{x \to a} \frac{f(x)}{g(x)} =
\lim_{x \to a} \frac{f(x) - 0}{g(x) - 0} =
\lim_{x \to a} \frac{f(x) - f(a)}{g(x) - g(a)} \;
\left| \text{По \hyperref[th:Cauchy's_mean_value]{теореме Коши}} \right| =
\lim_{x \to a} \frac{f'(c)}{g'(c)} =
\end{equation*}
\begin{equation*}
\left| c \in (x; a) \lOr c \in (a; x) \right| =
\lim_{x \to a} \frac{f'(x)}{g'(x)}
\end{equation*}

Полное доказательство правила Лопиталя слишком сложно, поэтому здесь не приводится.
\section{Исследование функции}
\subsection{Локальный экстремум функции}
\begin{theorem}
\label{th:criterion_of_monotonic_function}
Если функция~$f(x)$ дифференцируема на~$(a; b)$, то она не убывает (не возрастает) на~$(a; b) \opbr\Leftrightarrow \forall x \in (a; b) \ f'(x) \geqslant 0 \ (f'(x) \leqslant 0)$.
\end{theorem}
\begin{proof}
\begin{enumerate}
	\item $\Rightarrow$. Пусть $f(x)$ не убывает на~$(a; b)$, $x_1, x_2 \in (a; b)$.
	\begin{equation*}
	\frac{f(x_2) - f(x_1)}{x_2 - x_1} \geqslant 0 \Rightarrow
	\lim_{x_2 \to x_1} \frac{f(x_2) - f(x_1)}{x_2 - x_1} \geqslant 0 \Leftrightarrow
	f'(x_1) \geqslant 0
	\end{equation*}
	
	Доказательство в~случае невозрастания $f(x)$ аналогично.
	
	\item $\Leftarrow$. Пусть $\forall x \in (a; b) \ f'(x) \geqslant 0$, $a < x_1 < x_2 < b$.
	По \hyperref[th:mean_value]{теореме Лагранжа}
	\begin{equation*}
	\exists x_3 \in (x_1; x_2) \colon f(x_2) - f(x_1) = f'(x_3)(x_2 - x_1)
	\end{equation*}
	
	$f'(x_3)(x_2 - x_1) \geqslant 0 \Leftrightarrow
	f(x_2) - f(x_1) \geqslant 0 \Leftrightarrow
	f(x)$ не убывает на~$(a; b)$.
	
	Доказательство в~случае $f'(x) \leqslant 0$ аналогично.
\end{enumerate}
\end{proof}

\index{Максимум!функции} Точка~$x_0$ называется \textbf{точкой локального минимума функции}~$f(x)$, если существует проколотая окрестность~$\breve U(x_0) \colon \forall x \in \breve U(x_0) \ f(x) > f(x_0)$.

\index{Минимум!функции} Точка~$x_0$ называется \textbf{точкой локального максимума функции}~$f(x)$, если существует проколотая окрестность~$\breve U(x_0) \colon \forall x \in \breve U(x_0) \ f(x) < f(x_0)$.

\index{Экстремум} Точки локального минимума и максимума называются \textbf{точками локального экстремума}.

\begin{theorem}
Если $x_0$~--- точка локального экстремума функции~$f(x)$, то $\nexists f'(x_0) \lOr f'(x_0) = 0$.
\end{theorem}
\begin{proof}
Пусть $x_0$~--- точка локального минимума, $\exists f'(x_0)$, тогда
\begin{equation*}
\begin{cases}
\displaystyle \frac{f(x) - f(x_0)}{x - x_0} < 0, \ x < x_0 \\
\displaystyle \frac{f(x) - f(x_0)}{x - x_0} > 0, \ x > x_0
\end{cases} \Rightarrow
\begin{cases}
\displaystyle \lim_{x \to x_0-0} \frac{f(x) - f(x_0)}{x - x_0} \leqslant 0 \\
\displaystyle \lim_{x \to x_0+0} \frac{f(x) - f(x_0)}{x - x_0} \geqslant 0
\end{cases} \Rightarrow
\lim_{x \to x_0} \frac{f(x) - f(x_0)}{x - x_0} = 0 \Leftrightarrow
f'(x_0) = 0
\end{equation*}

Доказательство для локального максимума аналогично.
\end{proof}

\index{Точка!критическая} Точка, в~которой производная функции не существует или равна нулю, называется \textbf{критической}.

Существуют следующие признаки локального экстремума:
\begin{enumerate}
	\item Если $f_-'(x_0) \leqslant 0 \ (\geqslant 0) \lAnd f_+'(x_0) \geqslant 0 \ (\leqslant 0)$, то $x_0$~--- точка локального минимума (максимума).
	\begin{proof}
	Пусть
	\begin{equation*}
	f_-'(x_0) \leqslant 0 \lAnd f_+'(x_0) \geqslant 0 \Leftrightarrow
	\lim_{x \to x_0-0} \frac{f(x) - f(x_0)}{x - x_0} \leqslant 0 \lAnd
	\lim_{x \to x_0+0} \frac{f(x) - f(x_0)}{x - x_0} \geqslant 0
	\end{equation*}
	
	Значит, в некоторой окрестности точки~$x_0$ $f(x) \geqslant f(x_0)$, тогда $x_0$~--- точка локального минимума.
	
	Аналогичное доказательство для максимума.
	\end{proof}
	
	\item Если $f'(x_0) = f''(x_0) = \ldots = f^{(2n-1)}(x_0) = 0$, то
	\begin{itemize}
		\item $x_0$~--- точка локального максимума при~$f^{(2n)}(x_0) < 0$;
		\item $x_0$~--- точка локального минимума при~$f^{(2n)}(x_0) > 0$;
		\item $x_0$ не является точкой локального экстремума при~$f^{(2n)}(x_0) = 0$, $f^{(2n+1)}(x_0) \neq 0$.
	\end{itemize}
	\begin{proof}
	\begin{itemize}
		\item Пусть $f^{(2n)}(x_0) < 0$.
		По \hyperref[eq:Taylor_series]{формуле Тейлора}
		\begin{equation*}
		f(x) = f(x_0) + f'(x_0)(x - x_0) + \frac{f''(x_0)}{2!} (x - x_0)^2 + \ldots + \frac{f^{(2n)}(x_0)}{(2n)!} (x - x_0)^{2n} + o((x - x_0)^{2n}) \Leftrightarrow
		\end{equation*}
		\begin{equation*}
		\Leftrightarrow f(x) - f(x_0) = \frac{f^{(2n)}(x_0)}{(2n)!} (x - x_0)^{2n} + o((x - x_0)^{2n}) < 0
		\end{equation*}
		
		Тогда $x_0$~--- точка локального максимума.
		
		\item Случай при~$f^{(2n)}(x_0) > 0$ доказывается аналогично.
		
		\item Пусть $f^{(2n)}(x_0) = 0$, $f^{(2n+1)}(x_0) \neq 0$.
		По \hyperref[eq:Taylor_series]{формуле Тейлора}
		\begin{equation*}
		f(x) = f(x_0) + f'(x_0)(x - x_0) + \frac{f''(x_0)}{2!} (x - x_0)^2 + \ldots +
		\end{equation*}
		\begin{equation*}
		\vphantom1 + \frac{f^{(2n)}(x_0)}{(2n)!} (x - x_0)^{2n} + \frac{f^{(2n+1)}(x_0)}{(2n+1)!} (x - x_0)^{2n+1} + o((x - x_0)^{2n+1}) \Leftrightarrow
		\end{equation*}
		\begin{equation*}
		\Leftrightarrow f(x) - f(x_0) = \frac{f^{(2n+1)}(x_0)}{(2n+1)!} (x - x_0)^{2n+1} + o((x - x_0)^{2n+1})
		\end{equation*}
		
		Знак $f(x) - f(x_0)$ зависит от знака $x - x_0$, поэтому в точке~$x_0$ не может быть локального экстремума.
	\end{itemize}
	\end{proof}
\end{enumerate}

\subsection{Наименьшее и наибольшее значения функции}
Минимальное и максимальное значения функции на некотором отрезке не всегда находятся в~точках экстремума.
Для того, чтобы найти эти значения, необходимо вычислить значения функции в~критических и граничных точках и выбрать среди них наименьшее и наибольшее.

\subsection{Выпуклость функции}
Кривая называется \textbf{выпуклой}, или \textbf{выпуклой вверх}, \textbf{в точке}, если в некоторой окрестности данной точки касательная к кривой в этой точке находится выше этой кривой.

Кривая называется \textbf{вогнутой}, или \textbf{выпуклой вниз}, \textbf{в точке}, если в некоторой окрестности данной точки касательная к кривой в этой точке находится ниже этой кривой.

\begin{theorem}
Пусть дана функция~$f(x)$.
Если $f''(x_0) < 0$, то кривая, задаваемая уравнением~$y \opbr= f(x)$, выпукла в точке~$x_0$.
Если же $f''(x_0) > 0$, то эта кривая вогнута в точке~$x_0$.
\end{theorem}
\begin{proof}
Касательная к кривой в точке~$(x_0, f(x_0))$ задаётся уравнением~$y = g(x)$, где
\begin{equation*}
g(x) = f(x_0) + f'(x_0)(x - x_0)
\end{equation*}

По \hyperref[eq:Taylor_series]{формуле Тейлора}
\begin{equation*}
f(x) = f(x_0) + f'(x_0)(x - x_0) + \frac12 f''(x_0)(x - x_0)^2 + o((x - x_0)^2)
\end{equation*}

Тогда
\begin{equation*}
f(x) - g(x) = \frac12 f''(x_0)(x - x_0)^2 + o((x - x_0)^2)
\end{equation*}

Т.\,о., знак разности~$f(x) - g(x)$ совпадает со знаком~$f''(x_0)$.
\begin{itemize}
	\item При $f''(x_0) < 0$ получим $f(x) < g(x)$ в некоторой окрестности точки~$x_0$, значит, кривая выпукла.
	\item При $f''(x_0) > 0$ получим $f(x) > g(x)$ в некоторой окрестности точки~$x_0$, значит, кривая вогнута.
\end{itemize}
\end{proof}

Пусть $f(x)$~--- функция.
\index{Точка!перегиба} Если $f_-''(x_0) \cdot f_+''(x_0) \leqslant 0$, то точка~$x_0$ называется \textbf{точкой перегиба}.

\subsection{Асимптоты}
\index{Асимптота} Прямая называется \textbf{асимптотой кривой}, если расстояние от переменной точки кривой до данной прямой при удалении этой точки в бесконечность стремится к нулю.
Если указанное расстояние стремится к нулю при~$x \to \infty$, то такая асимптота называется \textbf{наклонной}, а если при~$y \to \infty$, то \textbf{вертикальной}.
Если наклонная асимптота задаётся уравнением $y = b$, то она называется \textbf{горизонтальной}.

\begin{theorem}
Кривая, задаваемая уравнением~$y = f(x)$, имеет наклонную асимптоту, задаваемую уравнением~$y = kx + b$, если $\lim\limits_{x \to \infty} \frac{f(x)}x = k$ и $\lim\limits_{x \to \infty} (f(x) - kx) = b$.
\end{theorem}
\begin{proof}
Из определения наклонной асимптоты $f(x) - (kx + b) = \alpha(x)$, где $\alpha(x)$~--- бесконечно малая при~$x \to \infty$.
Тогда
\begin{equation*}
\lim_{x \to \infty} \frac{f(x)}x =
\lim_{x \to \infty} \left( k + \frac{b}x + \frac{\alpha(x)}x \right) =
k, \
\lim_{x \to \infty} (f(x) - kx) =
\lim_{x \to \infty} (b + \alpha(x)) =
b
\end{equation*}
\end{proof}
\section{Функции нескольких переменных}
\index{Функция!нескольких переменных} \textbf{Функцией от $n$~переменных} называется функция~$f \colon D \to E$, где $D \subseteq \mathbb R^n$, $E \subseteq \mathbb R$, и обозначается $f(\overline x)$, или $f(x_1, x_2, \ldots, x_n)$.

\index{Расстояние между точками} \textbf{Расстоянием между точками} $\overline a = (a_1, \ldots, a_n)$ и $\overline b = (b_1, \ldots, b_n)$ называется величина
\begin{equation*}
\rho(\overline a, \overline b) = \sqrt{\sum_{i=0}^n (a_i - b_i)^2}
\end{equation*}
\section{Функции двух и трёх переменных}
\subsection{Геометрическая интерпретация частных производных функции двух переменных}
Пусть функция~$f(x, y)$ имеет частные производные в точке~$(x_0, y_0)$.
Пересечением плоскости~$x = x_0$ с поверхностью~$z = f(x, y)$ является кривая~$z = f(x_0, y)$.
Т.\,о., значение~$f_y'(x_0, y_0)$ равно тангенсу угла между касательной к кривой~$z = f(x_0, y)$ в точке~$(x_0, y_0)$ и положительным направлением оси~$Oy$, а направляющий вектор этой касательной имеет координаты~$(0, 1, f_y'(x_0, y_0))$.

Аналогичный геометрический смысл имеет частная производная~$f_x'$.

\subsection{Уравнение касательной плоскости к поверхности}
\subsubsection{Поверхность, заданная явно}
Пусть поверхность задана уравнением~$z = f(x, y)$.
Проведём через точку~$(x_0, y_0, f(x_0, y_0))$ такую плоскость, что векторы $(0, 1, f_y'(x_0, y_0))$ и $(1, 0, f_x'(x_0, y_0))$ лежат в~ней.
Эта плоскость называется \textbf{касательной}.
Найдём вектор~$(A, B, C)$, перпендикулярный этим векторам, а значит, и проведённой плоскости:
\begin{equation*}
\begin{cases}
B + C f_y'(x_0, y_0) = 0 \\
A + C f_x'(x_0, y_0) = 0
\end{cases} \Leftrightarrow
\begin{cases}
A = -C f_x'(x_0, y_0) \\
B = -C f_y'(x_0, y_0)
\end{cases}
\end{equation*}

Вектор~$(f_x'(x_0, y_0), f_y'(x_0, y_0), -1)$ перпендикулярен проведённой плоскости, тогда её уравнение
\begin{equation*}
f_x'(x_0, y_0)(x - x_0) + f_y'(x_0, y_0)(y - y_0) - (z - f(x_0, y_0)) = 0 \Leftrightarrow
\end{equation*}
\begin{equation}
\label{eq:tangent_plane}
\Leftrightarrow z = f(x_0, y_0) + f_x'(x_0, y_0)(x - x_0) + f_y'(x_0, y_0)(y - y_0)
\end{equation}

\subsubsection{Поверхность, заданная параметрически}
Пусть поверхность задана функцией~$f(u, v) = (x(u, v), y(u, v), z(u, v))$, а в точке~$(x_0, y_0, z_0)$ к ней проведена касательная плоскость, причём $\dfrac{\partial(x, y)}{\partial(u, v)} \neq 0$.
Тогда $\exists u_0, v_0 \colon x(u_0, v_0) = x_0 \lAnd y(u_0, v_0) = y_0$.

Имеем явное и параметрическое задание одной и той~же поверхности: $z(u, v) = z(x(u, v), y(u, v))$.
Рассматривая производные матрицы этих функций, получим:
\begin{equation*}
\begin{Vmatrix}
z_u' & z_v'
\end{Vmatrix} =
\begin{Vmatrix}
z_x' & z_y'
\end{Vmatrix} \cdot
\begin{Vmatrix}
x_u' & x_v' \\
y_u' & y_v'
\end{Vmatrix} \Rightarrow
\begin{cases}
z_u' = z_x' x_u' + z_y' y_u' \\
z_v' = z_x' x_v' + z_y' y_v'
\end{cases}
\end{equation*}

Решая систему относительно $z_x'$ и $z_y'$, получим
\begin{equation*}
z_x'(x_0, y_0) = \frac
{\left. \frac{\partial(z, y)}{\partial(u, v)} \right|_{(u_0, v_0)}}
{\left. \frac{\partial(x, y)}{\partial(u, v)} \right|_{(u_0, v_0)}} =
-\frac
{\left. \frac{\partial(y, z)}{\partial(u, v)} \right|_{(u_0, v_0)}}
{\left. \frac{\partial(x, y)}{\partial(u, v)} \right|_{(u_0, v_0)}}, \
z_y'(x_0, y_0) = \frac
{\left. \frac{\partial(x, z)}{\partial(u, v)} \right|_{(u_0, v_0)}}
{\left. \frac{\partial(x, y)}{\partial(u, v)} \right|_{(u_0, v_0)}} =
-\frac
{\left. \frac{\partial(z, x)}{\partial(u, v)} \right|_{(u_0, v_0)}}
{\left. \frac{\partial(x, y)}{\partial(u, v)} \right|_{(u_0, v_0)}}
\end{equation*}

Подставим полученные значения в уравнение~(\ref*{eq:tangent_plane}) и получим уравнение касательной плоскости:
\begin{equation*}
z = z_0 - \frac
{\left. \frac{\partial(y, z)}{\partial(u, v)} \right|_{(u_0, v_0)}}
{\left. \frac{\partial(x, y)}{\partial(u, v)} \right|_{(u_0, v_0)}} (x - x_0) -
\frac
{\left. \frac{\partial(z, x)}{\partial(u, v)} \right|_{(u_0, v_0)}}
{\left. \frac{\partial(x, y)}{\partial(u, v)} \right|_{(u_0, v_0)}} (y - y_0)  \Leftrightarrow
\end{equation*}
\begin{equation*}
\Leftrightarrow \left. \frac{\partial(y, z)}{\partial(u, v)} \right|_{(u_0, v_0)} (x - x_0) +
\left. \frac{\partial(z, x)}{\partial(u, v)} \right|_{(u_0, v_0)} (y - y_0) +
\left. \frac{\partial(x, y)}{\partial(u, v)} \right|_{(u_0, v_0)} (z - z_0)
\end{equation*}
\section{Вектор-функции}
\index{Вектор-функция} \textbf{Вектор-функцией} размерности~$m$ называется функция~$f \colon D \to E$, где $D \subseteq \mathbb R^n$, $E \subseteq \mathbb R^m$, и обозначается
$f(\overline x) = (f_1(\overline x), f_2(\overline x), \ldots, f_m(\overline x))$.
$f_1, \ldots, f_m$ называются \textbf{координатными функциями}.

\subsection{Дифференцируемость вектор-функции}
\index{Вектор-функция!дифференцируемая} Вектор-функция~$f(x_1, \ldots, x_n) = (f_1(\overline x), f_2(\overline x), \ldots, f_m(\overline x))$ называется \textbf{дифференцируемой в точке~$\overline x_0 \opbr= (x_{10}, \ldots, x_{n0})$}, если
\begin{equation*}
\exists A =
\begin{Vmatrix}
a_{11} & a_{12} & \cdots & a_{1n} \\
a_{21} & a_{22} & \cdots & a_{2n} \\
\vdots & \vdots & \ddots & \vdots \\
a_{m1} & a_{m2} & \cdots & a_{mn}
\end{Vmatrix} \colon
\forall \Delta \overline x = (\Delta x_1, \ldots, \Delta x_n) \
\begin{Vmatrix}
f_1(\overline x_0 + \Delta \overline x) - f_1(\overline x_0) \\
f_2(\overline x_0 + \Delta \overline x) - f_2(\overline x_0) \\
\vdots \\
f_m(\overline x_0 + \Delta \overline x) - f_m(\overline x_0)
\end{Vmatrix} =
A \cdot
\begin{Vmatrix}
\Delta x_1 \\
\Delta x_2 \\
\vdots \\
\Delta x_n
\end{Vmatrix} +
\begin{Vmatrix}
\alpha_1 \\
\alpha_2 \\
\vdots \\
\alpha_m
\end{Vmatrix} \lAnd
\end{equation*}
\begin{equation*}
\lAnd \lim_{\rho(\overline x_0 + \Delta \overline x, \overline x_0) \to 0}
\frac{\sqrt{\alpha_1^2 + \alpha_2^2 + \ldots + \alpha_m^2}}
{\rho(\overline x_0 + \Delta \overline x, \overline x_0)} = 0
\end{equation*}

\index{Матрица!производная} \index{Матрица!Якоби} Матрица~$A$ называется \textbf{производной матрицей}, или \textbf{матрицей Як\'{о}би}, и состоит из значений всех частных производных всех координатных функций в~данной точке:
\begin{equation*}
A =
\begin{Vmatrix}
f_{1\, x_1}'(\overline x_0) & f_{1\, x_2}'(\overline x_0) & \cdots & f_{1\, x_n}'(\overline x_0) \\
f_{2\, x_1}'(\overline x_0) & f_{2\, x_2}'(\overline x_0) & \cdots & f_{2\, x_n}'(\overline x_0) \\
\vdots & \vdots & \ddots & \vdots \\
f_{m\, x_1}'(\overline x_0) & f_{m\, x_2}'(\overline x_0) & \cdots & f_{m\, x_n}'(\overline x_0) \\
\end{Vmatrix}
\end{equation*}

\index{Якобиан} Если $f(x_1, \ldots, x_n)$~--- вектор-функция размерности~$n$, дифференцируемая в точке~$\overline x_0$, то \textbf{якобианом} называется определитель её производной матрицы и обозначается $\left. \dfrac{\partial f}{\partial \overline x} \right|_{\overline x_0}$.

\subsection{Суперпозиция вектор-функций}
Пусть $f(x_1, \ldots, x_n)$ и $g(x_1, \ldots, x_k)$~--- $m$-мерная и~$n$-мерная вектор-функции соответственно, тогда \textbf{их суперпозицией} называется вектор-функция~$h(x_1, \ldots, x_k) = f(g_1(\overline x), \ldots, g_n(\overline x))$.

\begin{theorem}
Если $f(\overline x)$ и $g(\overline x)$ дифференцируемы в точках~$g(\overline x_0)$ и $\overline x_0$ соответственно и имеют в~этих точках производные матрицы $A$ и $B$ соответственно, то $h(\overline x)$ дифференцируема в точке~$\overline x_0$ и имеет в~ней производную матрицу~$A \cdot B$.
\end{theorem}
\section{Неопределённый интеграл}
\index{Первообразная} Первообразной функции~$f(x)$ называется функция~$F(x) \colon F'(x) = f(x)$.

\begin{theorem}
Если $F'(x) = G'(x) = f(x)$, то $F(x) - G(x) = C$, где $C$~--- некоторая константа.
\end{theorem}
\begin{proof}
Пусть $H(x) = F(x) - G(x)$, тогда по \hyperref[th:mean_value]{теореме Лагранжа}
\begin{equation*}
H(b) - H(a) = H'(c)(b - a) = 0 \Rightarrow
H(b) - H(a) = (F'(c) - G'(c))(b - a) = 0 \Rightarrow
H(x) = C
\end{equation*}
\end{proof}

\index{Интеграл!неопределённый} Множество всех первообразных функции~$f(x)$ называется \textbf{неопределённым интегралом} и обозначается $\int f(x)\,dx = F(x) + C$, где $F(x)$~--- первообразная~$f(x)$, $C$~--- произвольная константа.
$f(x)$ называется \textbf{подынтегральной функцией}, а $f(x)\,dx$~--- \textbf{подынтегральным выражением}.

\subsection{Свойства неопределённого интеграла}
Пусть $C$~--- произвольная константа.
\begin{enumerate}
	\item Пусть $F(x)$~--- первообразная функции~$f(x)$, тогда $d \left( \int f(x)\,dx \right) = f(x)\,dx$.
	
	\item $\int dF(x) = F(x) + C$.
	
	\item $\int (f(x) + g(x))\,dx = \int f(x)\,dx + \int g(x)\,dx$.
	\begin{proof}
	Пусть $F'(x) = f(x)$, $G'(x) = g(x)$, тогда $(F(x) + G(x))' = f(x) + g(x)$.
	Получим:
	\begin{equation*}
	\int f(x)\,dx + \int g(x)\,dx = F(x) + C_1 + G(x) + C_2 = (F(x) + G(x)) + C = \int (f(x) + g(x))\,dx
	\end{equation*}
	\end{proof}
	
	\item $\int a f(x)\,dx = a \int f(x)\,dx$.
	\begin{proof}
	Пусть $F'(x) = f(x)$, тогда $(a F(x))' = a f(x)$.
	Получим:
	\begin{equation*}
	a \int f(x)\,dx = a(F(x) + C_1) = a F(x) + C = \int a f(x)\,dx
	\end{equation*}
	\end{proof}
	
	\item Если $\int f(x)\,dx = F(x) + C$, $u(x)$~--- дифференцируемая функция, то $\int f(u)\,du = F(u) + C$.
\end{enumerate}

Пусть $u(x)$ и $v(x)$~--- дифференцируемые функции.
Существует \textbf{метод интегрирования по частям}, использующий следующее свойство:
$\int uv'\,dx = uv - \int u'v\,dx$.
\begin{proof}
\begin{equation*}
d(uv) = du\,v + u\,dv \Rightarrow
\int d(uv) = \int du\,v + u\,dv \Rightarrow
uv + C = \int du\,v + \int u\,dv \Rightarrow
\int uv'\,dx = u\,v - \int u'v\,dx
\end{equation*}
\end{proof}

\subsection{Таблица первообразных}
\begin{itemize}
	\item $\displaystyle \int x^n\,dx = \frac{x^{n+1}}{n + 1} + C$, $n \neq -1$
	
	\item $\displaystyle \int \frac{dx}{x + a} = \ln |x + a| + C$
	\begin{proof}
	\begin{equation*}
	(\ln |x + a| + C)' = (\ln \sqrt{(x + a)^2})' = \frac1{|x + a|} \cdot \frac1{2\sqrt{(x + a)^2}} \cdot 2(x + a) = \frac1{x + a}
	\end{equation*}
	\end{proof}
	
	\item $\displaystyle \int \frac{dx}{x^2 + a^2} = \frac1a \arctg \frac{x}a + C$, $a > 0$
	\begin{proof}
	\begin{equation*}
	\int \frac{dx}{x^2 + a^2} =
	\frac1{a^2} \int \frac{dx}{1 + \left( \frac{x}a \right)^2} =
	\frac1a \int \frac{d \left( \frac{x}a \right)}{1 + \left( \frac{x}a \right)^2} =
	\frac1a \arctg \frac{x}a + C
	\end{equation*}
	\end{proof}
	
	\item Т.\,н. <<высокий логарифм>>: $\displaystyle \int \frac{dx}{x^2 - a^2} = \frac1{2a} \ln \left| \frac{x - a}{x + a} \right| + C$, $a > 0$
	\begin{proof}
	\begin{equation*}
	\int \frac{dx}{x^2 - a^2} =
	\int \frac{dx}{2a(x - a)} - \int \frac{dx}{2a(x + a)} =
	\frac1{2a} ((\ln |x - a| + C_1) - (\ln |x + a| + C_2)) =
	\frac1{2a} \ln \left| \frac{x - a}{x + a} \right| + C
	\end{equation*}
	\end{proof}
	
	\item $\displaystyle \int \frac{x}{x^2 + a}\,dx = \frac12 \ln |x^2 + a| + C$, $a \neq 0$
	\begin{proof}
	\begin{equation*}
	\int \frac{x}{x^2 + a}\,dx =
	\frac12 \int \frac{d(x^2 + a)}{x^2 + a} =
	\frac12 \ln |x^2 + a| + C
	\end{equation*}
	\end{proof}
	
	\item Т.\,н. <<длинный логарифм>>: $\displaystyle \int \frac{dx}{\sqrt{x^2 + a}} = \ln \left| x + \sqrt{x^2 + a} \right| + C$, $a \neq 0$
	\begin{proof}
	Пусть $k = \sqrt{|a|}$.
	\begin{enumerate}
		\item Пусть $a < 0$, $x = \dfrac{k}{\sin t}$.
		\begin{equation*}
		\int \frac{dx}{\sqrt{x^2 - k^2}} =
		-\int \frac{\cos t}{\sin^2 t \cdot \sqrt{\frac1{\sin^2 t} - 1}}\,dt =
		-\int \frac{dt}{\sin t} =
		\int \frac{d(\cos t)}{1 - \cos^2 t} =
		-\frac12 \ln \left| \frac{\cos t - 1}{\cos t + 1} \right| + C_1 =
		\end{equation*}
		\begin{equation*}
		= -\frac12 \ln \left| \frac{\cos^2 t - 1}{\cos^2 t + 1 + 2\cos t} \right| + C_1 =
		\left| x = \frac{k}{\sin t} \Rightarrow
		\sqrt{1 - \cos^2 t} = \frac{k}x \Rightarrow
		\cos^2 t = 1 - \frac{k^2}{x^2} \right|
		\end{equation*}
		\begin{equation*}
		= -\frac12 \ln \left| \frac{-\frac{k^2}{x^2}}{-\frac{k^2}{x^2} + 2 + 2\sqrt{1 - \frac{k^2}{x^2}}} \right| + C_1 =
		\frac12 \ln \left| 1 - \frac{2x^2}{k^2} - \frac{2x}{k^2}\sqrt{x^2 - k^2} \right| + C_1 =
		\end{equation*}
		\begin{equation*}
		= \frac12 \ln \frac1{k^2} \left| 2x^2 + 2x\sqrt{x^2 + a} + a \right| + C_1 =
		\frac12 \ln \left| x^2 + 2x\sqrt{x^2 + a} + (x^2 + a) \right| + C =
		\end{equation*}
		\begin{equation*}
		= \ln \left| x + \sqrt{x^2 + a} \right| + C
		\end{equation*}
		
		\item Пусть $a > 0$, $x = k\tg t$.
		\begin{equation*}
		\int \frac{dx}{\sqrt{x^2 + k^2}} =
		\int \frac{dt}{\cos^2 t \cdot \sqrt{\tg^2 t + 1}} =
		\int \frac{dt}{\cos t} =
		\int \frac{d(\sin t)}{1 - \sin^2 t} =
		-\frac12 \ln \left| \frac{\sin t - 1}{\sin t + 1} \right| + C_1 =
		\end{equation*}
		\begin{equation*}
		= -\frac12 \ln \left| \frac{\sin^2 t - 1}{\sin^2 t + 1 + 2\sin t} \right| + C_1 =
		\end{equation*}
		\begin{equation*}
		\left| x = k\tg t \Rightarrow
		\sqrt{\frac1{\cos^2 t} - 1} = \frac{x}k \Rightarrow
		\cos^2 t = \frac{k^2}{x^2 + k^2} \Leftrightarrow
		\sin^2 t = \frac{x^2}{x^2 + k^2} \right|
		\end{equation*}
		\begin{equation*}
		= -\frac12 \ln \left| \frac{-\frac{k^2}{x^2 + k^2}}{\frac{2x^2 + k^2}{x^2 + k^2} + 2\sqrt{\frac{x^2}{x^2 + k^2}}} \right| + C_1 =
		\frac12 \ln \frac1{k^2} \left| 2x^2 + k^2 + 2x\sqrt{x^2 + k^2} \right| + C_1 =
		\end{equation*}
		\begin{equation*}
		= \frac12 \ln \left| x^2 + 2x\sqrt{x^2 + a} + (x^2 + a) \right| + C =
		\ln \left| x + \sqrt{x^2 + a} \right| + C
		\end{equation*}
	\end{enumerate}
	\end{proof}
	
	\item $\displaystyle \int \frac{dx}{\sqrt{a^2 - x^2}} = \arcsin \frac{x}a + C$, $a > 0$
	\begin{proof}
	Пусть $x = a \sin t$, тогда
	\begin{equation*}
	\int \frac{dx}{\sqrt{a^2 - x^2}} =
	a \int \frac{\cos t}{a \sqrt{1 - \sin^2 t}}\,dt =
	\int dt =
	t + C =
	\arcsin \frac{x}a + C
	\end{equation*}
	\end{proof}
	
	\item $\displaystyle \int \frac{x}{\sqrt{a^2 \pm x^2}}\,dx = \pm \sqrt{a^2 \pm x^2} + C$, $a \neq 0$
	\begin{proof}
	\begin{equation*}
	\int \frac{x}{\sqrt{a^2 \pm x^2}}\,dx =
	\pm \frac12 \int (a^2 \pm x^2)^{-\tfrac12}\,d(a^2 \pm x^2) =
	\pm \sqrt{a^2 \pm x^2} + C
	\end{equation*}
	\end{proof}
	
	\item $\displaystyle \int a^x\,dx = \frac{a^x}{\ln a} + C$
	
	\item $\displaystyle \int \ln x\,dx = x \ln x - x + C$
	\begin{proof}
	\begin{equation*}
	\int \ln x\,dx =
	\int \ln x \cdot 1 \cdot dx =
	x \ln x - \int \frac{x}x\,dx =
	x \ln x - x + C
	\end{equation*}
	\end{proof}
	
	\item $\displaystyle \int \sin x\,dx = -\cos x + C$
	
	\item $\displaystyle \int \cos x\,dx = \sin x + C$
	
	\item $\displaystyle \int \frac{dx}{\cos^2 x} = \tg x + C$
	
	\item $\displaystyle \int \frac{dx}{\sin^2 x} = -\ctg x + C$
	
	\item $\displaystyle \int \tg x\,dx = -\ln |\cos x| + C$
	\begin{proof}
	\begin{equation*}
	\int \tg x\,dx =
	\int \frac{\sin x}{\cos x}\,dx =
	-\int \frac{d(\cos x)}{\cos x} =
	-\ln |\cos x| + C
	\end{equation*}
	\end{proof}
	
	\item $\displaystyle \int \ctg x\,dx = \ln |\sin x| + C$
	\begin{proof}
	\begin{equation*}
	\int \ctg x\,dx =
	\int \frac{\cos x}{\sin x}\,dx =
	\int \frac{d(\sin x)}{\sin x} =
	\ln |\sin x| + C
	\end{equation*}
	\end{proof}
	
	\item $\displaystyle \int \arcsin x\,dx = x \arcsin x + \sqrt{1 - x^2} + C$
	\begin{proof}
	\begin{equation*}
	\int \arcsin x =
	\int \arcsin x \cdot 1 \cdot dx =
	x \arcsin x - \int \frac{x}{\sqrt{1 - x^2}}\,dx =
	\end{equation*}
	\begin{equation*}
	= x \arcsin x + \frac12 \int (1 - x^2)^{-\tfrac12}\,d(1 - x^2) =
	x \arcsin x + \sqrt{1 - x^2} + C
	\end{equation*}
	\end{proof}
	
	\item $\displaystyle \int \arccos x\,dx = x \arccos x - \sqrt{1 - x^2} + C$
	\begin{proof}
	\begin{equation*}
	\int \arccos x =
	\int \arccos x \cdot 1 \cdot dx =
	x \arccos x + \int \frac{x}{\sqrt{1 - x^2}}\,dx =
	\end{equation*}
	\begin{equation*}
	= x \arccos x - \frac12 \int (1 - x^2)^{-\tfrac12}\,d(1 - x^2) =
	x \arccos x - \sqrt{1 - x^2} + C
	\end{equation*}
	\end{proof}
	
	\item $\displaystyle \int \arctg x\,dx = x \arctg x - \frac12 \ln |1 + x^2| + C$
	\begin{proof}
	\begin{equation*}
	\int \arctg x =
	\int \arctg x \cdot 1 \cdot dx =
	x \arctg x - \int \frac{x}{1 + x^2}\,dx =
	\end{equation*}
	\begin{equation*}
	= x \arctg x - \frac12 \int \frac{d(1 + x^2)}{1 + x^2} =
	x \arctg x - \frac12 \ln |1 + x^2| + C
	\end{equation*}
	\end{proof}
	
	\item $\displaystyle \int \arcctg x\,dx = x \arcctg x + \frac12 \ln |1 + x^2| + C$
	\begin{proof}
	\begin{equation*}
	\int \arcctg x =
	\int \arcctg x \cdot 1 \cdot dx =
	x \arcctg x + \int \frac{x}{1 + x^2}\,dx =
	\end{equation*}
	\begin{equation*}
	= x \arcctg x + \frac12 \int \frac{d(1 + x^2)}{1 + x^2} =
	x \arcctg x + \frac12 \ln |1 + x^2| + C
	\end{equation*}
	\end{proof}
\end{itemize}

\subsection{Интегрирование простейших дробей}
\begin{itemize}
	\item $\displaystyle \int \frac{dx}{(x - a)^n} =
	\begin{cases}
	\frac{(x - a)^{1-n}}{1 - n} + C, \ n \neq 1 \\
	\ln |x - a| + C, \ n = 1
	\end{cases}$
	\begin{proof}
	\begin{equation*}
	\int \frac{dx}{(x - a)^n} =
	\int (x - a)^{-n}\,d(x - a) =
	\frac{(x - a)^{1-n}}{1 - n} + C, \ n \neq 1
	\end{equation*}
	\end{proof}
	
	\item $\displaystyle \int \frac{x + a}{(x - b)^2 + c^2}\,dx =
	\frac12 \ln ((x - b)^2 + c^2) + \frac{a + b}c \arctg \frac{x - b}c + C$
	\begin{proof}
	\begin{equation*}
	\int \frac{x + a}{(x - b)^2 + c^2}\,dx \;
	\left| \text{Пусть } t = x - b \Rightarrow dt = dx \right| =
	\int \frac{t + b + a}{t^2 + c^2}\,dt =
	\end{equation*}
	\begin{equation*}
	= \int \frac{t}{t^2 + c^2}\,dt + (a + b) \int \frac{dt}{t^2 + c^2} =
	\frac12 \ln ((x - b)^2 + c^2) + \frac{a + b}c \arctg \frac{x - b}c + C
	\end{equation*}
	\end{proof}
	
	\item Интеграл~$\displaystyle \int \frac{x + a}{((x - b)^2 + c^2)^n}\,dx$ при~$n \neq 1$ нельзя взять непосредственно.
	
	Пусть $\displaystyle I_n = \int \frac{dx}{((x - b)^2 + c^2)^n}$, тогда
	\begin{equation*}
	\int \frac{x + a}{((x - b)^2 + c^2)^n}\,dx =
	(a + b)I_n - \frac1{2(n - 1)((x - b)^2 + c^2)^{n-1}}
	\end{equation*}
	\begin{equation*}
	I_n = \left( 1 + \frac1{2(n - 1)} \right) \frac{I_{n-1}}{c^2} + \frac{x - b}{2(n - 1)c^2 ((x - b)^2 + c^2)^{n-1}}
	\end{equation*}
	\begin{proof}
	\begin{equation}
	\label{eq:int_of_simple_frac_proof1}
	\begin{gathered}
	\int \frac{x + a}{((x - b)^2 + c^2)^n}\,dx \;
	\left| \text{Пусть } t = x - b \Rightarrow dt = dx \right| =
	\int \frac{t + b + a}{(t^2 + c^2)^n}\,dt = \\
	\int \frac{t}{(t^2 + c^2)^n}\,dt + (a + b) \int \frac{dt}{(t^2 + c^2)^n} =
	\frac12 \int (t^2 + c^2)^{-n}\,d(t^2 + c^2) + (a + b) \int \frac{dt}{(t^2 + c^2)^n} = \\
	\left| \text{Пусть } I_n = \int \frac{dt}{(t^2 + c^2)^n} \right| =
	\frac{(t^2 + c^2)^{1-n}}{2(1 - n)} + (a + b)I_n
	\end{gathered}
	\end{equation}
	
	Найдём $I_n$:
	\begin{equation*}
	I_n = \frac1{c^2} \int \frac{(t^2 + c^2) - t^2}{(t^2 + c^2)^n}\,dt =
	\frac{I_{n-1}}{c^2} - \frac1{c^2} \int \frac{t^2}{(t^2 + c^2)^n}\,dt
	\end{equation*}
	
	Найдём $\displaystyle \int \frac{t^2}{(t^2 + c^2)^n}\,dt$:
	\begin{equation*}
	\int \frac{t^2}{(t^2 + c^2)^n}\,dt =
	\left| \text{Пусть } u = t, v' = \frac{t}{(t^2 + c^2)^n} \right|
	\end{equation*}
	\begin{equation*}
	= \frac{t(t^2 + c^2)^{1-n}}{2(1 - n)} - \int \frac{(t^2 + c^2)^{1-n}}{2(1 - n)}\,dt =
	\frac{t}{2(1 - n)(t^2 + c^2)^{n-1}} - \frac{I_{n-1}}{2(1 - n)}
	\end{equation*}
	
	Тогда
	\begin{equation*}
	I_n = \frac{I_{n-1}}{c^2} + \frac1{2(n - 1)c^2} \left( \frac{t}{(t^2 + c^2)^{n-1}} - I_{n-1} \right) =
	\left( 1 + \frac1{2(n - 1)} \right) \frac{I_{n-1}}{c^2} + \frac{t}{2(n - 1)c^2 (t^2 + c^2)^{n-1}}
	\end{equation*}
	
	Получим рекуррентную формулу:
	\begin{equation*}
	I_n = \left( 1 + \frac1{2(n - 1)} \right) \frac{I_{n-1}}{c^2} + \frac{x - b}{2(n - 1)c^2 ((x - b)^2 + c^2)^{n-1}}
	\end{equation*}
	
	Используя (\ref{eq:int_of_simple_frac_proof1}), получим конечную формулу:
	\begin{equation*}
	\int \frac{x + a}{((x - b)^2 + c^2)^n}\,dx =
	\frac{(t^2 + c^2)^{1-n}}{2(1 - n)} + (a + b)I_n =
	(a + b)I_n - \frac1{2(n - 1)((x - b)^2 + c^2)^{n-1}}
	\end{equation*}
	\end{proof}
\end{itemize}

\subsection{Интегрирование дробно-рациональных выражений}
Пусть $P_n(x)$ и $Q_m(x)$~--- многочлены $n$-й и $m$-й степеней соответственно, $n < m$.
По теореме~\ref{th:polynomial_factorization} $Q_m(x)$ можно разложить на множители
\begin{equation*}
Q_m(x) = \prod_{i=1}^{p} (x - a_i)^{\alpha_i} \cdot \prod_{i=1}^{q} ((x - b_i)^2 + c_i^2)^{\beta_i}
\end{equation*}

Тогда дробь~$\dfrac{P_n(x)}{Q_m(x)}$ может быть представлена в~виде суммы простейших дробей
\begin{equation*}
\frac{P_n(x)}{Q_m(x)} = \sum_{i=1}^p \sum_{j=1}^{\alpha_i} \frac{A_{ij}}{(x - a_i)^j} + \sum_{i=1}^q \sum_{j=1}^{\beta_q} \frac{B_{ij}x + C_{ij}}{((x - b_i)^2 + c_i^2)^j}
\end{equation*}

Т.\,о., интегрирование дробно-рациональных выражений сводится к интегрированию простейших дробей и, в~случае $n \geqslant m$, многочленов от переменной~$x$.

\subsection{Интегрирование тригонометрических выражений}
Пусть $R(x_1, \ldots, x_n) = \dfrac{P(x_1, \ldots, x_n)}{Q(x_1, \ldots, x_n)}$, где $P(\overline x)$ и $Q(\overline x)$~--- многочлены.
\begin{itemize}
	\item $\displaystyle \int R(\sin x) \cos^{2k+1} x\,dx = \int R(\sin x) (1 - \sin^2 x)^k\,d(\sin x)$, $k \in \mathbb Z$
	
	\item $\displaystyle \int R(\cos x) \sin^{2k+1} x\,dx = -\int R(\cos x) (1 - \cos^2 x)^k\,d(\cos x)$, $k \in \mathbb Z$
	
	\item $\displaystyle \int R(\sin^2 x, \cos^2 x)\,dx = \int \frac
	{R\left( \frac{\tg^2 x}{1 + \tg^2 x}, \frac1{1 + \tg^2 x} \right)}
	{1 + \tg^2 x}\,d(\tg x)$
	
	\item $\displaystyle \int R(\tg x)\,dx = \int \frac{R(\tg x)}{1 + \tg^2 x}\,d(\tg x)$
	
	\item $\displaystyle \int R(\sin x, \cos x)\,dx = \int \frac
	{2R\left( \frac{2\tg \frac{x}2}{1 + \tg^2 \frac{x}2}, \frac{1 - \tg^2 \frac{x}2}{1 + \tg^2 \frac{x}2} \right)}
	{1 + \tg^2 \frac{x}2}\,d(\tg \frac{x}2)$
\end{itemize}

Т.\,о., интегрирование тригонометрических выражений сводится к интегрированию рациональных дробей.

\subsection{Интегрирование квадратичных иррациональностей}
Пусть $R(x_1, \ldots, x_n) = \dfrac{P(x_1, \ldots, x_n)}{Q(x_1, \ldots, x_n)}$, где $P(\overline x)$ и $Q(\overline x)$~--- многочлены.
\begin{equation*}
\int R(x, \sqrt{ax^2 + bx + c})\,dx =
\int R\left( x, \sqrt{a \left( x + \frac{b}{2a} \right)^2 + c - \frac{b^2}{4a}} \right) dx =
\end{equation*}
\begin{equation*}
\left| \text{Пусть } y = x + \frac{b}{2a}, \ z = c - \frac{b^2}{4a} \right| =
\int R\left( y - \frac{b}{2a}, \sqrt{ay^2 + z} \right) dy
\end{equation*}

Пусть $\alpha = \sqrt{|a|}$, $\beta = \sqrt{|z|}$.
Возможны три случая:
\begin{itemize}
	\item Если $a > 0$, $z > 0$
	\begin{equation*}
	\int R\left( y - \frac{b}{2a}, \sqrt{\alpha^2 y^2 + \beta^2} \right) dy =
	\left| \text{Пусть } y = \frac\beta\alpha \tg t \right|
	\end{equation*}
	\begin{equation*}
	= \frac\beta\alpha \int \frac
	{R\left( \frac\beta\alpha \tg t - \frac{b}{2a}, \beta \sqrt{\tg^2 t + 1} \right)}
	{\cos^2 t}\,dt =
	\frac\beta\alpha \int \frac
	{R\left( \frac{\beta \sin t}{\alpha \cos t} - \frac{b}{2a}, \frac\beta{\cos t} \right)}
	{\cos^2 t}\,dt
	\end{equation*}
	
	\item Если $a > 0$, $z < 0$
	\begin{equation*}
	\int R\left( y - \frac{b}{2a}, \sqrt{\alpha^2 y^2 - \beta^2} \right)\,dy =
	\left| \text{Пусть } y = \frac\beta{\alpha \sin t} \right|
	\end{equation*}
	\begin{equation*}
	= -\frac\beta\alpha \int \frac
	{R\left( \frac\beta{\alpha \sin t} - \frac{b}{2a}, \beta \sqrt{\frac1{\sin^2 t} - 1} \right) \cos t}
	{\sin^2 t}\,dt =
	-\frac\beta\alpha \int \frac
	{R\left( \frac\beta{\alpha \sin t} - \frac{b}{2a}, \frac{\beta \cos t}{\sin t} \right) \cos t}
	{\sin^2 t}\,dt
	\end{equation*}
	
	\item Если $a < 0$, $z > 0$
	\begin{equation*}
	\int R\left( y - \frac{b}{2a}, \sqrt{-\alpha^2 y^2 + \beta^2} \right) dy =
	\left| \text{Пусть } y = \frac\beta\alpha \sin t \right|
	\end{equation*}
	\begin{equation*}
	= \frac\beta\alpha \int R\left( \frac\beta\alpha \sin t - \frac{b}{2a}, \beta \sqrt{1 - \sin^2 t} \right) \cos t\,dt =
	\frac\beta\alpha \int R\left( \frac\beta\alpha \sin t - \frac{b}{2a}, \beta \cos t \right) \cos t\,dt
	\end{equation*}
\end{itemize}

Т.\,о., интегрирование квадратичных иррациональностей сводится к интегрированию тригонометрических выражений.
\section{Интеграл Римана}
Пусть на отрезке~$[a; b]$ определена функция~$f(x)$.
Рассмотрим разбиение~$R = (x_0, x_1, \ldots, x_n)$ этого отрезка, где $a = x_0 < x_1 < \ldots < x_{n-1} < x_n = b$.
Среди длин подынтервалов $\Delta x_i = x_i - x_{i-1}$, $i = 1, \ldots, n$, найдём диаметр $d = \max \{ \Delta x_1, \Delta x_2, \ldots, \Delta x_n \}$ разбиения.
Выберём множество точек $\xi = \{ \xi_1, \ldots, \xi_n \}$, где $\xi_i \in [x_{i-1}; x_i]$, $i = 1, \ldots, n$.

\begin{wrapfigure}[10]{r}{0pt}
\noindent
\begin{tikzpicture}
\drawaxis{-0.5}{7}{-0.5}{4};
\draw[name path=curve] (0.5, 1.5) coordinate (a) to[out=60, in=180] (1.25, 2)
	to[out=0, in=135] (2, 1.6)
	to[out=-45, in=180] (2.75, 1)
	to[out=0, in=-120] (4.5, 3)
	to[out=60, in=135] (5.5, 3) coordinate (b) node[above right] {$y = f(x)$};
\printcoordsonaxis[solid]{a}{a};
\printcoordsonaxis[solid]{b}{b};

\coordinate (x0) at (a);
\coordinate (x6) at (b);

% разбиваем фигуру на части
\def\getpoint#1#2{% {имя точки}{значение x}
	\path[name path=vertical] (#2, \bottom_y) -- (#2, \top_y);
	\path[name intersections={of=curve and vertical, by=i}]
		(i) coordinate (#1);
}
\foreach \i/\x in {1/1, 2/1.5, 3/2.5, 4/3.75, 5/4.6}
	\getpoint{x\i}{\x};
\printcoordsonaxis[solid]{x1}{x_1};
\printcoordsonaxis[solid]{x2}{x_2};
\foreach \i in {3, 4}
	\printcoordsonaxis[solid]{x\i}{};
\printcoordsonaxis[solid]{x5}{x_{n-1}};

% приближаем её прямоугольниками
\coordinate (prev) at (x0);
\foreach \i/\x in {1/0.75, 2/1.25, 3/1.75, 4/3.25, 5/4.3, 6/5}
{
	\getpoint{e\i}{\x};
	\printcoordsonaxis{e\i}{};
	
	\draw[dashed] (prev) -- (prev |- e\i) -- (e\i -| x\i) -- (x\i);
	
	\coordinate (prev) at (x\i);
}
\end{tikzpicture}
\end{wrapfigure}

\textbf{Интегральной суммой Римана} называется сумма $\sigma(f, R, \xi) \opbr= \sum\limits_{i=1}^n f(\xi_i) \Delta x_i$.

Рассмотрим криволинейную трапецию~$M$, ограниченную прямыми $y = 0$, $x = a$, $x = b$ и кривой $y = f(x)$. Пусть $S$~--- площадь $M$, $S_1, \ldots, S_n$~--- площади трапеций, на которые разбивается $M$ прямыми $x = x_1$, \ldots, $x = x_{n-1}$, тогда
\begin{equation*}
S = \sum_{i=1}^n S_i \approx \sum_{i=1}^n f(\xi_i) \Delta x_i = \sigma(f, R, \xi)
\end{equation*}

Т.\,о., интегральная сумма Римана приближённо равна площади соответствующей криволинейной трапеции.

\index{Интеграл!Римана} \index{Интеграл!определённый} Если независимо от выбора $\xi$ $\exists \lim\limits_{d \to 0} \sigma(f, R, \xi)$, то значение предела~$\lim\limits_{d \to 0} \sigma(f, R, \xi)$ называется \textbf{интегралом Римана}, или \textbf{определённым интегралом}, и обозначается $\int\limits_a^b f(x)\,dx$.
$f(x)$ называется \textbf{интегрируемой (по Риману) на~$[a; b]$}.

Интеграл Римана $\int\limits_a^b f(x)\,dx$ численно равен площади трапеции, ограниченной прямыми $y = 0$, $x = a$, $x = b$ и кривой $y = f(x)$.

\begin{theorem}
Любая непрерывная функция интегрируема по Риману.
\end{theorem}

\subsection{Свойства интеграла Римана}
\begin{enumerate}
	\item \label{st:b_a_integral} $\int\limits_a^b f(x)\,dx = -\int\limits_b^a f(x)\,dx$
	\begin{proof}
	Пусть $R' = (x_0' = b, x_1' = x_{n-1}, \ldots, x_{n-1}' = x_1, x_n' = a)$.
	\begin{equation*}
	\sigma(f, R, \xi) =
	\sum_{i=1}^n f(\xi_i) (x_i - x_{i-1}) =
	-\sum_{i=1}^n f(\xi_i) (x_i' - x_{i-1}') =
	-\sigma(f, R', \xi)
	\end{equation*}
	
	Тогда
	\begin{equation*}
	\int_a^b f(x)\,dx =
	\lim_{d \to 0} \sigma(f, R, \xi) =
	\lim_{d \to 0} -\sigma(f, R', \xi) =
	-\lim_{d \to 0} \sigma(f, R, \xi) =
	-\int_b^a f(x)\,dx
	\end{equation*}
	\end{proof}
	
	\item Линейность: $\int\limits_a^b (\alpha f(x) + \beta g(x))\,dx = 
	\alpha \int\limits_a^b f(x)\,dx + \beta \int\limits_a^b g(x)\,dx$
	\begin{proof}
	\begin{equation*}
	\sigma(\alpha f(x) + \beta g(x), R, \xi) =
	\sum_{i=1}^n (\alpha f(\xi_i) + \beta g(\xi_i)) \Delta x_i =
	\end{equation*}
	\begin{equation*}
	= \alpha \sum_{i=1}^n f(\xi_i) \Delta x_i + \beta \sum_{i=1}^n g(\xi_i) \Delta x_i =
	\alpha \sigma(f(x), R, \xi) + \beta \sigma(g(x), R, \xi)
	\end{equation*}
	
	Тогда
	\begin{equation*}
	\int_a^b (\alpha f(x) + \beta g(x))\,dx =
	\lim_{d \to 0} \sigma(\alpha f(x) + \beta g(x), R, \xi) =
	\end{equation*}
	\begin{equation*}
	= \lim_{d \to 0} \alpha \sigma(f(x), R, \xi) + \beta \sigma(g(x), R, \xi) =
	\end{equation*}
	\begin{equation*}
	= \alpha \lim_{d \to 0} \sigma(f(x), R, \xi) + \beta \lim_{d \to 0} \sigma(g(x), R, \xi) =
	\alpha \int_a^b f(x)\,dx + \beta \int_a^b g(x)\,dx
	\end{equation*}
	\end{proof}
	
	\item Аддитивность: $\int\limits_a^b f(x)\,dx = \int\limits_a^c f(x)\,dx + \int\limits_c^b f(x)\,dx$
	\begin{proof}
	\begin{enumerate}
		\item Пусть $c \in [a; b]$, $\int\limits_a^b f(x)\,dx = \lim\limits_{d \to 0} \sigma(f(x), R, \xi)$, причём $c = x_k \in R$, а также\\
		$R_1 = (a = x_0, \ldots, x_k = c)$, $R_2 = (c = x_k, \ldots, x_n = b)$,\\
		$\lambda_1 = \{ \xi_1, \ldots, \xi_k \}$, $\lambda_2 = \{ \xi_{k+1}, \ldots, \xi_n \}$.
		\begin{equation*}
		\sigma(f(x), R, \xi) =
		\sum_{i=1}^n f(\xi_i) \Delta x_i =
		\sum_{i=1}^k f(\xi_i) \Delta x_i + \sum_{i=k+1}^n f(\xi_i) \Delta x_i =
		\sigma(f(x), R_1, \lambda_1) + \sigma(f(x), R_2, \lambda_2)
		\end{equation*}
		
		Тогда
		\begin{equation*}
		\int_a^b f(x)\,dx =
		\lim_{d \to 0} \sigma(f(x), R, \xi) =
		\lim_{d \to 0} (\sigma(f(x), R_1, \lambda_1) + \sigma(f(x), R_2, \lambda_2)) =
		\end{equation*}
		\begin{equation*}
		= \lim_{d \to 0} \sigma(f(x), R_1, \lambda_1) + \lim_{d \to 0} \sigma(f(x), R_2, \lambda_2) =
		\int_a^c f(x)\,dx + \int_c^b f(x)\,dx
		\end{equation*}
		
		\item Для остальных случаев свойство легко доказывается с использованием свойства~\ref*{st:b_a_integral} и уже рассмотренного случая.
		Например, пусть $f(x)$ интегрируема на~$[c; b]$, $a \in [c; b]$, тогда
		\begin{equation*}
		\int_c^b f(x)\,dx = \int_c^a f(x)\,dx + \int_a^b f(x)\,dx \Leftrightarrow
		\int_a^b f(x)\,dx = -\int_c^a f(x)\,dx + \int_c^b f(x)\,dx \Leftrightarrow
		\end{equation*}
		\begin{equation*}
		\Leftrightarrow \int_a^b f(x)\,dx = \int_a^c f(x)\,dx + \int_c^b f(x)\,dx
		\end{equation*}
	\end{enumerate}
	\end{proof}
	
	\item Невырожденность: $\int\limits_a^b dx = b - a$
	\begin{proof}
	Пусть $f(x) = 1$.
	\begin{equation*}
	\sigma(f, R, \xi) =
	\sum_{i=1}^n f(\xi_i) \Delta x_i =
	\sum_{i=1}^n \Delta x_i =
	-x_0 + x_1 - x_1 + x_2 - \ldots - x_{n-2} + x_{n-1} - x_{n-1} + x_n =
	x_n - x_0 =
	b - a
	\end{equation*}
	
	Тогда
	\begin{equation*}
	\int_a^b dx =
	\lim_{d \to 0} \sigma(f, R, \xi) =
	\lim_{d \to 0} (b - a) =
	b - a
	\end{equation*}
	\end{proof}
	
	\item Если $f(x) \geqslant 0$ при $x \in [a; b]$, то $\int\limits_a^b f(x)\,dx \geqslant 0$
	
	\item Если $f(x) \leqslant g(x)$ при $x \in [a; b]$, то $\int\limits_a^b f(x)\,dx \leqslant \int\limits_a^b g(x)\,dx$
	\begin{proof}
	\begin{equation*}
	f(x) \leqslant g(x) \Leftrightarrow
	g(x) - f(x) \geqslant 0 \Rightarrow
	\int_a^b (g(x) - f(x))\,dx \geqslant 0 \Leftrightarrow
	\int_a^b g(x)\,dx - \int_a^b f(x)\,dx \geqslant 0 \Leftrightarrow
	\int_a^b f(x)\,dx \leqslant \int_a^b g(x)\,dx
	\end{equation*}
	\end{proof}
	
	\item Если $m = \min\limits_{x \in [a; b]} f(x)$, $M = \max\limits_{x \in [a; b]} f(x)$, то $m(b - a) \leqslant \int\limits_a^b f(x)\,dx \leqslant M(b - a)$
	
	\item \begin{theorem}[о среднем]
	Если $f(x)$ непрерывна на~$[a; b]$, то $\exists x_0 \in [a; b] \colon \int\limits_a^b f(x)\,dx = f(x_0)(b - a)$.
	\end{theorem}
	\begin{proof}
	$m(b - a) \leqslant \int\limits_a^b f(x)\,dx \leqslant M(b - a) \Leftrightarrow
	m \leqslant \frac1{b - a} \int\limits_a^b f(x)\,dx \leqslant M$, тогда по \hyperref[th:intermediate_value]{теореме о промежуточном значении}
	$\exists x_0 \colon f(x_0) = \frac1{b - a} \int\limits_a^b f(x)\,dx \Leftrightarrow
	\int\limits_a^b f(x)\,dx = f(x_0)(b - a)$.
	\end{proof}
\end{enumerate}

Методы вычисления определённого интеграла:
\begin{enumerate}
	\item \textbf{Замена переменной}
	
	Если $\varphi(t) \colon [\alpha; \beta] \to [a; b]$~--- монотонная функция, то $\int\limits_a^b f(x)\,dx =
	\int\limits_\alpha^\beta f(\varphi(t)) \varphi'(t)\,dt$
	
	\item \textbf{Интегрирование по частям}
	
	$\int\limits_a^b u(x) v'(x)\,dx = \left. (u(x) v(x)) \right|_a^b - \int\limits_a^b u'(x) v(x)\,dx$
\end{enumerate}

\subsection{Формула Ньютона"--~Лейбница}
\begin{theorem}[Ньютона"--~Лейбница]
\index{Теорема!Ньютона"--~Лейбница}
Если функция~$f(x)$ непрерывна на~$[a; b]$, $\Phi(x)$~--- её первообразная, то
\begin{equation}
\label{eq:Newton-Leibniz_formula}
\int_a^b f(x)\,dx = \left. \Phi(x) \right|_a^b = \Phi(b) - \Phi(a)
\end{equation}
\end{theorem}
\index{Формула!Ньютона"--~Лейбница} Равенство~(\ref{eq:Newton-Leibniz_formula}) называется \textbf{формулой Ньютона"--~Лейбница}.
\begin{proof}
Пусть $I(t) = \int\limits_a^t f(x)\,dx$, тогда $I(a) = 0$, $I(b) = \int\limits_a^b f(x)\,dx$.
\begin{equation*}
I(t + \Delta t) - I(t) =
\int_a^{t+\Delta t} f(x)\,dx - \int_a^t f(x)\,dx =
\int_t^{t + \Delta t} f(x)\,dx \Rightarrow
\end{equation*}
\begin{equation*}
\Rightarrow \exists \Theta \colon 0 \leqslant \Theta \leqslant 1 \lAnd I(t + \Delta t) - I(t) = f(t + \Theta \Delta t) \Delta t \Rightarrow
\lim_{\Delta t \to 0} \frac{I(t + \Delta t) - I(t)}{\Delta t} = \lim_{\Delta t \to 0} f(t + \Theta \Delta t) \Rightarrow
I'(t) = f(t)
\end{equation*}

Тогда $I(x)$~--- первообразная функции~$f(x)$, значит, $\Phi(a) = I(a) + C$, $\Phi(b) = I(b) + C$.
\begin{equation*}
\Phi(b) - \Phi(a) = I(b) - I(a) = \int_a^b f(x)\,dx
\end{equation*}
\end{proof}

\subsection{Приложения интеграла Римана}
\subsubsection{Вычисление площади плоской фигуры}
Пусть фигура, расположенная над отрезком~$[a; b]$, заключена между кривыми $y = f_1(x)$ и $y = f_2(x)$, причём $f_1(a) = f_2(a) \lAnd f_1(b) = f_2(b) \lAnd \forall x \in (a; b) \ f_1(x) < f_2(x)$.
Тогда площадь этой фигуры равна
\begin{equation*}
S = \int_a^b (f_2(x) - f_1(x))\,dx
\end{equation*}

\subsubsection{Вычисление площади криволинейного отрезка}
Пусть криволинейный сектор ограничен лучами $\varphi = \alpha$ и $\varphi = \beta$ и кривой $r = f(\varphi)$, где $(r, \varphi)$~--- точка, заданная полярными координатами.
Рассмотрим его разбиение лучами~$\alpha = \varphi_0 < \varphi_1 < \ldots < \varphi_n = \beta$ и обозначим $\Delta \opbr= \max \{ \varphi_1 - \varphi_0, \varphi_2 - \varphi_1, \ldots, \varphi_n - \varphi_{n-1} \}$.
Выберем точки $\xi_i \in [\varphi_{i-1}; \varphi_i]$, где $i = 1, \ldots, n$, тогда площадь сектора равна
\begin{equation*}
S = \lim_{\Delta \to 0} \sum_{i=1}^n \frac{f^2(\xi_i)}2\,(\varphi_i - \varphi_{i-1}) =
\frac12 \int_\alpha^\beta f^2(\varphi)\,d\varphi
\end{equation*}

\subsubsection{Длина кривой}
Пусть кривая задана уравнением $y = f(x)$ на отрезке~$[a; b]$.
Рассмотрим разбиение~$R = (x_0, x_1, \ldots, x_n)$ этого отрезка, где $a = x_0 < x_1 < \ldots < x_{n-1} < x_n = b$ и обозначим $d = \max \{ x_1 - x_0, x_2 - x_1, \ldots, x_n - x_{n-1} \}$.
Выберём точки $\xi_i \in [x_{i-1}; x_i]$, где $i = 1, \ldots, n$, тогда длина кривой равна
\begin{equation*}
L = \lim_{d \to 0} \sum_{i=1}^n \sqrt{(x_i - x_{i-1})^2 + (f(x_i) - f(x_{i-1}))^2} =
\left| \text{По \hyperref[th:mean_value]{формуле конечных приращений}} \right|
\end{equation*}
\begin{equation*}
= \lim_{d \to 0} \sum_{i=1}^n \sqrt{(x_i - x_{i-1})^2 + (x_i - x_{i-1})^2 (f'(\xi_i))^2} =
\lim_{d \to 0} \sum_{i=1}^n \sqrt{1 + (f'(\xi_i))^2} (x_i - x_{i-1}) =
\int_a^b \sqrt{1 + (f'(x))^2}\,dx
\end{equation*}

Если кривая задана параметрически: $\begin{cases}
x = x(t) \\
y = y(t) \\
z = z(t)
\end{cases}$\\
где $t \in [a; b]$, то её длина равна $L = \int\limits_a^b \sqrt{(x'(t))^2 + (y'(t))^2 + (z'(t))^2}\,dt$.

\subsubsection{Вычисление объёма тела}
Пусть тело расположено над отрезком~$[a; b]$, а площадь его сечения плоскостью~$x = x_0$ выражается функцией $S(x)$.
Рассмотрим разбиение~$R = (x_0, x_1, \ldots, x_n)$ отрезка~$[a; b]$, где $a = x_0 < x_1 < \ldots < x_{n-1} < x_n = b$.
Среди длин подынтервалов $\Delta x_i = x_i - x_{i-1}$, $i = 1, \ldots, n$, найдём диаметр $d = \max \{ \Delta x_1, \Delta x_2, \ldots, \Delta x_n \}$ разбиения.
Выберём множество точек $\xi = \{ \xi_1, \ldots, \xi_n \}$, где $\xi_i \in [x_{i-1}; x_i]$, $i = 1, \ldots, n$, тогда объём тела равен
\begin{equation*}
V = \lim_{d \to 0} \sum_{i=1}^n S(\xi_i) \Delta x_i =
\int_a^b S(x)\,dx
\end{equation*}

Пусть тело получено вращением кривой $y = f(x)$, заданной на отрезке~$[a; b]$, вокруг оси $Ox$, тогда $S(x) = \pi f^2(x)$ и $V = \pi \int\limits_a^b f^2(x)\,dx$.

\subsection{Приближённые методы вычисления интеграла}
Пусть непрерывная функция~$f(x)$ задана на~$[a; b]$.
Рассмотрим приближённые методы нахождения интеграла~$\int\limits_a^b f(x)\,dx$.

\subsubsection{Формула прямоугольников}
\index{Формула!прямоугольников} Разобьём отрезок~$[a; b]$ на $n$~частей точками $a = x_0 < x_1 < \ldots < x_n = b$ и заменим каждую криволинейную трапецию с основанием~$[x_{i-1}; x_i]$ на прямоугольник шириной $x_i - x_{i-1}$ и высотой либо $f(x_{i-1})$, либо $f(x_i)$, где $i = 1, \ldots, n$.
Тогда получим \textbf{формулу левых прямоугольников}
\begin{equation*}
\int_a^b f(x)\,dx \approx \sum_{i=1}^n f(x_{i-1}) (x_i - x_{i-1})
\end{equation*}
а также \textbf{формулу правых прямоугольников}
\begin{equation*}
\int_a^b f(x)\,dx \approx \sum_{i=1}^n f(x_i) (x_i - x_{i-1})
\end{equation*}

\subsubsection{Формула трапеций}
\begin{wrapfigure}[6]{r}{0pt}
\noindent
\begin{tikzpicture}
\drawaxis{-0.5}{4}{-0.5}{3};
\draw (0.5, 0.75) coordinate (x0) to[out=80, in=-135] (0.85, 1.5) coordinate (x1)
	to[out=45, in=120] (1.5, 1) coordinate (x2)
	to[out=-60, in=-130] (2.5, 2) coordinate (x3)
	to[out=50, in=135] (3.5, 2.5) coordinate (x4);

\printcoordsonaxis[solid]{x0}{a};
\foreach \i in {1, 2}
	\printcoordsonaxis{x\i}{x_\i};
\printcoordsonaxis{x3}{x_{n-1}};
\printcoordsonaxis[solid]{x4}{b};

\coordinate (prev) at (x0);
\foreach \i in {1, ..., 4}
	\draw[dashed] (prev) -- (x\i) coordinate (prev);
\end{tikzpicture}
\end{wrapfigure}
\index{Формула!трапеций} Разобьём отрезок~$[a; b]$ на $n$~частей точками $a = x_0 < x_1 < \ldots < x_n = b$ и построим отрезки, соединяющие точки $(x_{i-1}, f(x_{i-1}))$ и $(x_i, f(x_i))$, где $i = 1, \ldots, n$.
Площадь получившейся фигуры приближённо равна площади всей криволинейной трапеции.
Т.\,о., получим \textbf{формулу трапеций}, или \textbf{формулу средних прямоугольников}:
\begin{equation*}
\int_a^b f(x)\,dx \approx \sum_{i=1}^n \frac{f(x_{i-1}) + f(x_i)}2\,(x_i - x_{i-1})
\end{equation*}
\section{Несобственный интеграл}
\index{Интеграл!несобственный}

\subsection{Несобственный интеграл I рода}
Пусть функция~$f(x)$ непрерывна на~$[a; +\infty)$ и $\forall b > a \ \exists \int\limits_a^b f(x)\,dx$.

\textbf{Несобственным интегралом Римана I рода} называется интеграл $\int\limits_a^{+\infty} f(x)\,dx = \lim\limits_{b \to +\infty} \int\limits_a^b f(x)\,dx$.

Пусть функция~$f(x)$ непрерывна на~$(-\infty, b]$ и $\forall a < b \ \exists \int\limits_a^b f(x)\,dx$.

\textbf{Несобственным интегралом Римана I рода} называется интеграл $\int\limits_{-\infty}^b f(x)\,dx = \lim\limits_{a \to -\infty} \int\limits_a^b f(x)\,dx$.

Несобственный интеграл называется \textbf{сходящимся}, если он существует и конечен, иначе~--- \textbf{расходящимся}.

Если функция~$f(x)$ непрерывна на~$\mathbb R$, то
$\int\limits_{-\infty}^{+\infty} f(x)\,dx =
\lim\limits_{a \to -\infty} \int\limits_a^c f(x)\,dx + \lim\limits_{b \to +\infty} \int\limits_c^b f(x)\,dx$

\subsection{Несобственный интеграл II рода}
Пусть функция~$f(x)$ непрерывна на~$[a; b)$, терпит разрыв II рода в точке~$b$ и $\forall \varepsilon > 0 \ \exists \int\limits_a^{b-\varepsilon} f(x)\,dx$.

\textbf{Несобственным интегралом Римана II рода} называется интеграл $\int\limits_a^b f(x)\,dx = \lim\limits_{\varepsilon \to +0} \int\limits_a^{b-\varepsilon} f(x)\,dx$.

Пусть функция~$f(x)$ непрерывна на~$(a; b]$, терпит разрыв II рода в точке~$a$ и $\forall \varepsilon > 0 \ \exists \int\limits_{a+\varepsilon}^b f(x)\,dx$.

\textbf{Несобственным интегралом Римана II рода} называется интеграл $\int\limits_a^b f(x)\,dx = \lim\limits_{\varepsilon \to +0} \int\limits_{a+\varepsilon}^b f(x)\,dx$.

Если функция~$f(x)$ терпит разрыв II рода в точке $c \in (a; b)$, то $\int\limits_a^b f(x)\,dx = \int\limits_a^c f(x)\,dx + \int\limits_c^b f(x)\,dx$.
\section{Числовые ряды}
Пусть дана бесконечная последовательность~$(a_n)$, $S_n = \sum\limits_{k=1}^n a_k$.

\index{Ряд!числовой} \textbf{Числовым рядом} называется совокупность последовательностей $(a_n)$ и $(S_n)$ и обозначается $\series a_k$.
$S_n$ называется \textbf{частичной суммой числового ряда}.

Если существует конечный предел~$\lim\limits_{n \to \infty} S_n = S$, то $S$ называется \textbf{суммой числового ряда}, а ряд называется \textbf{сходящимся}, иначе~--- \textbf{расходящимся}.

Свойства рядов:
\begin{enumerate}
	\item Ряд~$\series a_k$ сходится $\Rightarrow$ $\lim\limits_{n \to \infty} a_n = 0$.
	\begin{proof}
	\begin{equation*}
	a_n = S_n - S_{n-1}
	\left| \text{По \hyperref[th:Cauchy_criterion]{критерию Коши}} \right| \Rightarrow
	\lim_{n \to \infty} a_n = \lim_{n \to \infty} (S_n - S_{n-1}) = 0
	\end{equation*}
	\end{proof}
	
	\item Если ряды $\series a_k = S$ и $\series b_k = \sigma$, то $\series (\alpha a_k + \beta b_k) = \alpha S + \beta \sigma$.
	\begin{proof}
	\begin{equation*}
	\lim_{n \to \infty} \sum_{k=1}^n (\alpha a_k + \beta b_k) =
	\lim_{n \to \infty} (\alpha \sum_{k=1}^n a_k + \beta \sum_{k=1}^n b_k) =
	\alpha S + \beta \sigma
	\end{equation*}
	\end{proof}
	
	\item $\forall N \in \mathbb N$ $\series a_{k+N}$ сходится $\Leftrightarrow$ $\series a_k$ сходится.
	\begin{proof}
	$\sigma_n = S_{n+N} - S_N$, тогда
	\begin{equation*}
	\exists \lim_{n \to \infty} \sigma_n = \lim_{n \to \infty} S_n - S_N \Leftrightarrow
	\exists \lim_{n \to \infty} S_n
	\end{equation*}
	\end{proof}
\end{enumerate}

\subsection{Знакоположительные ряды}
\index{Ряд!знакоположительный} Числовой ряд называется \textbf{знакоположительным}, если все его члены положительны.

\begin{lemma}
\label{lemma:convergence_of_sign-positive_series}
Знакоположительный ряд сходится $\Leftrightarrow$ последовательность его частных сумм ограничена.
\end{lemma}
\begin{proof}
\begin{enumerate}
	\item $\Rightarrow$. Пусть $\exists \lim\limits_{n \to \infty} S_n = S$, тогда $\forall n \in N \ S_n \leqslant S$.
	\item $\Leftarrow$. Пусть $\exists T > 0 \colon \forall n \in \mathbb N \ S_n \leqslant T$, тогда по свойству~\ref{st:monotonic_bounded_sequence} предела последовательности ряд сходится.
\end{enumerate}
\end{proof}

\begin{theorem}[сравнения]
\label{th:direct_comparison_test1}
Пусть даны знакоположительные ряды $\series a_k$ и $\series b_k$, $\forall n \in \mathbb N \ a_n \leqslant b_n$.
\begin{itemize}
	\item Если $\series b_k$ сходится, то $\series a_k$ тоже сходится.
	\item Если $\series a_k$ расходится, то $\series b_k$ тоже расходится.
\end{itemize}
\end{theorem}
\begin{proof}
Пусть $(\sigma_n)$ и $(S_n)$~--- частичные суммы рядов $\series a_k$ и $\series b_k$ соответственно.
\begin{itemize}
	\item Пусть $\series b_k$ сходится.
	\begin{equation*}
	\exists T > 0 \colon \forall n \in N \ S_n \leqslant T \Rightarrow
	\sigma_n \leqslant S_n < T
	\end{equation*}
	
	По лемме~\ref*{lemma:convergence_of_sign-positive_series} $\series a_k$ сходится.
	
	\item Если $\series a_k$ расходится, то $\series b_k$ тоже расходится, что легко доказывается методом от противного.
\end{itemize}
\end{proof}

\begin{theorem}[сравнения]
Пусть даны знакоположительные ряды $\series a_k$ и $\series b_k$, $\lim\limits_{n \to \infty} \frac{a_n}{b_n} = c$.
Если $0 < c < \infty$, то $\series a_k$ сходится $\Leftrightarrow$ $\series b_k$ сходится.
\end{theorem}
\begin{proof}
\begin{equation*}
\forall \varepsilon > 0 \exists N \in \mathbb N \colon \forall k > N \ \left| \frac{a_k}{b_k} - L \right| < \varepsilon
\end{equation*}

Пусть $\varepsilon = \frac{c}2$, тогда
\begin{equation*}
\exists N_0 \in \mathbb N \colon \forall k > N_0 \ c - \frac{c}2 < \frac{a_k}{b_k} < \frac{c}2 + c \Rightarrow
a_k < \frac32 b_k c \lAnd b_k < \frac{2 a_k}c
\end{equation*}

По теореме~\ref*{th:direct_comparison_test1} получим:
\begin{itemize}
	\item Если $\series a_k$ сходится, то $\series \frac{2 a_k}c$ сходится $\Rightarrow$ $\series b_k$ сходится.
	\item Если $\series b_k$ сходится, то $\series \frac32 b_k c$ сходится $\Rightarrow$ $\series a_k$ сходится.
\end{itemize}
\end{proof}

\begin{theorem}[д'Аламбера]
Пусть $\series a_k$~--- знакоположительный ряд, $\lim\limits_{n \to \infty} \frac{a_{n+1}}{a_n} = p$.
Если $p < 1$, то $\series a_k$ сходится, а если $p > 1$, то расходится.
\end{theorem}
\begin{proof}
\begin{enumerate}
	\item Пусть $p < 1$.
	Выберём $\varepsilon > 0 \colon p + \varepsilon < 1$, тогда
	\begin{equation*}
	\exists N \in \mathbb N \colon \forall n > N \ \left| \frac{a_{n+1}}{a_n} - p \right| < \varepsilon \Rightarrow
	\frac{a_{n+1}}{a_n} < p + \varepsilon \Rightarrow
	\end{equation*}
	\begin{equation*}
	\Rightarrow \frac{a_{N+2}}{a_{N+1}} < p + \varepsilon \lAnd \frac{a_{N+3}}{a_{N+2}} < p + \varepsilon \Rightarrow
	a_{N+3} < (p + \varepsilon) a_{N+2} < (p + \varepsilon)^2 a_{N+1} \Rightarrow
	a_{N+k} < (p + \varepsilon)^{k-1} a_{N+1}
	\end{equation*}
	
	$\series (p + \varepsilon)^{k-1} a_{N+1}$ сходится $\Rightarrow$ $\series a_{N+k}$ сходится $\Rightarrow$ $\series a_k$ сходится.
	
	\item Пусть $p > 1$.
	Выберём $\varepsilon > 0 \colon p - \varepsilon > 1$, тогда
	\begin{equation*}
	\exists N \in \mathbb N \colon \forall n > N \ \left| \frac{a_{n+1}}{a_n} - p \right| < \varepsilon \Rightarrow
	\frac{a_{n+1}}{a_n} > p - \varepsilon \Rightarrow
	\end{equation*}
	\begin{equation*}
	\Rightarrow \frac{a_{N+2}}{a_{N+1}} > p - \varepsilon \lAnd \frac{a_{N+3}}{a_{N+2}} > p - \varepsilon \Rightarrow
	a_{N+3} > (p - \varepsilon) a_{N+2} > (p - \varepsilon)^2 a_{N+1} \Rightarrow
	a_{N+k} > (p - \varepsilon)^{k-1} a_{N+1}
	\end{equation*}
	
	$\series (p - \varepsilon)^{k-1} a_{N+1}$ расходится $\Rightarrow$ $\series a_{N+k}$ расходится $\Rightarrow$ $\series a_k$ расходится.
\end{enumerate}
\end{proof}

\begin{theorem}[радикальный признак Коши]
Пусть $\series a_k$~--- знакоположительный ряд и $\exists \lim\limits_{n \to \infty} \sqrt[n]{a_n} = p$.
Если $p < 1$, то $\series a_k$ сходится, а если $p > 1$, то расходится.
\end{theorem}
\begin{proof}
\begin{enumerate}
	\item Пусть $p < 1$.
	Выберём $\varepsilon > 0 \colon p + \varepsilon < 1$, тогда
	\begin{equation*}
	\exists N \in \mathbb N \colon \forall n > N \ \left| \sqrt[n]{a_n} - p \right| < \varepsilon \Rightarrow
	\sqrt[n]{a_n} < p + \varepsilon \Rightarrow
	a_n < (p + \varepsilon)^n \Rightarrow
	\frac{a_{n+1}}{a_n} < p + \varepsilon \Rightarrow
	\lim_{n \to \infty} \frac{a_{n+1}}{a_n} < 1
	\end{equation*}
	
	Тогда по признаку д'Аламбера $\series a_k$ сходится.
	
	\item Пусть $p > 1$.
	Выберём $\varepsilon > 0 \colon p - \varepsilon > 1$, тогда
	\begin{equation*}
	\exists N \in \mathbb N \colon \forall n > N \ \left| \sqrt[n]{a_n} - p \right| < \varepsilon \Rightarrow
	\sqrt[n]{a_n} > p - \varepsilon \Rightarrow
	a_n > (p - \varepsilon)^n \Rightarrow
	\frac{a_{n+1}}{a_n} > p - \varepsilon \Rightarrow
	\lim_{n \to \infty} \frac{a_{n+1}}{a_n} > 1
	\end{equation*}
	
	Тогда по признаку д'Аламбера $\series a_k$ расходится.
\end{enumerate}
\end{proof}
\chapter{Теория булевых функций}
\index{Множество!булево} \textbf{Булевым} называется множество~$B = \{ 0, 1 \}$.

\index{Функция!булева} \textbf{Булевой} называется функция~$f \colon B^n \to B$.
Булеву функцию можно задать таблицей, называемой \textbf{таблицей истинности}.

\begin{statement}
Количество булевых функций от $n$~переменных равно $2^{2^n}$.
\end{statement}
\begin{proof}
Число всех возможных наборов аргументов булевой функции от $n$~переменных равно $2^n$, тогда число всех возможных таких функций равно $2^{2^n}$.
\end{proof}

Пусть $f(x_1, \ldots, x_n)$~--- булева функция.
\index{Переменная!существенная} \index{Переменная!фиктивная} Переменная $x_i$ называется \textbf{существенной}, если
\begin{equation*}
\exists a_1, \ldots, a_{i-1}, a_{i+1}, \ldots, a_n \colon
f(a_1, \ldots, a_{i-1}, 0, a_{i+1}, \ldots, a_n) \neq f(a_1, \ldots, a_{i-1}, 1, a_{i+1}, \ldots, a_n)
\end{equation*}
иначе~--- \textbf{несущественной}, или \textbf{фиктивной}.

Две булевы функции называются \textbf{равными}, если одну из них можно получить из другой последовательным удалением или введением несущественных переменных.
\section{Логические операции}
Пусть $x, y \in B$.
\index{Операции!логические} Основные логические операции (в порядке убывания приоритета выполнения):
\begin{enumerate}
	\item \index{Инверсия} \index{Отрицание} \textbf{Отрицание} (\textbf{инверсия}): $\lnot x$, $\overline x$.
	\item \index{Конъюнкция} \textbf{Конъюнкция} (\textbf{логическое И}): $x \land y$, $x \wedge y$, $x \cdot y$, $xy$.
	\item \index{Дизъюнкция} \textbf{Дизъюнкция} (\textbf{логическое ИЛИ}): $x \lor y$
\end{enumerate}

Следующие операции не имеют общепринятого приоритета выполнения, но обычно выполняются после вышеуказанных.
\begin{enumerate}
	\item \textbf{Импликация}: $x \rightarrow y$.
	\item \textbf{Эквиваленция}: $x \leftrightarrow y$, $x \sim y$, $x \equiv y$.
	\item \textbf{Сложение по модулю~$2$} (\textbf{исключающее ИЛИ}): $x \oplus y$, $x + y$.
	\item \textbf{Штрих Шеффера}: $x \mid y$.
	\item \textbf{Стрелка Пирса}: $x \downarrow y$.
\end{enumerate}

Приведём таблицу истинности рассмотренных логических операций:
\begin{equation*}
\begin{array}{|c|c|c|c|c|c|c|c|c|c|}
\hline
x & \overline x & y & x \land y & x \lor y & x \rightarrow y & x \sim y & x + y & x \mid y & x \downarrow y \\
\hline
0 & 1 & 0 & 0 & 0 & 1 & 1 & 0 & 1 & 1 \\
\hline
0 & 1 & 1 & 0 & 1 & 1 & 0 & 1 & 1 & 0 \\
\hline
1 & 0 & 0 & 0 & 1 & 0 & 0 & 1 & 1 & 0 \\
\hline
1 & 0 & 1 & 1 & 1 & 1 & 1 & 0 & 0 & 0 \\
\hline
\end{array}
\end{equation*}

\index{Степень!булева} Также определяется \textbf{булева степень}:
\begin{equation*}
x^\sigma =
\begin{cases}
\overline x, \ \sigma = 0 \\
x, \ \sigma = 1
\end{cases}
\end{equation*}
где $\sigma \in B$~--- параметр.
\section{Формулы}
\index{Формула} \textbf{Формулой над множеством~$F$ булевых функций} называется $f(x_1, \ldots, x_n) \in F$ или $\Phi(A_1, \ldots, A_n)$, где $\Phi(x_1, \ldots, x_n)$~--- формула, $A_1, \ldots, A_n$~--- переменные или функции из~$F$, называемые \textbf{подформулами}.

\begin{statement}
Каждой формуле~$\Phi$ над множеством~$F$ булевых функций соответствует булева функция.
\end{statement}
\begin{proof}
Возможны два случая:
\begin{enumerate}
	\item Если $\Phi = f(x_1, \ldots, x_n) \in F$, то $\Phi \to f(x_1, \ldots, x_n)$.
	\item Если $\Phi = f(A_1, \ldots, A_n)$, где $A_1 \to f_1$, $A_2 \to f_2$, \ldots, $A_n \to f_n$, $f_1, \ldots, f_n \in F$, то $\Phi \to f(f_1, \ldots, f_n)$.
\end{enumerate}
\end{proof}

Две формулы называются \textbf{эквивалентными}, если им соответствуют равные функции.

\begin{statement}
Если формула~$\Phi'$ получается из формулы~$\Phi$ заменой подформулы~$A$ на эквивалентную ей $A'$, то $\Phi'$ эквивалентна $\Phi$.
\end{statement}
\section{Разложение булевых функций по переменным}
\begin{theorem}
Булева функция~$f(x_1, \ldots, x_n)$ может быть записана в~виде
\begin{equation*}
f(x_1, \ldots, x_n) = \bigvee_{(\sigma_1, \ldots, \sigma_m)}
x_1^{\sigma_1} \land \ldots \land x_m^{\sigma_m} \land f(\sigma_1, \ldots, \sigma_m, x_{m+1}, \ldots, x_n)
\end{equation*}
\end{theorem}
\begin{proof}
Рассмотрим произвольный набор $(\alpha_1, \ldots, \alpha_n)$ и покажем, что левая и правая части данного соотношения принимают на нём одно и то~же значение:
\begin{enumerate}
	\item Для левой части получим $f(\alpha_1, \ldots, \alpha_n)$.
	\item Для правой части получим
	\begin{equation*}
	\bigvee_{(\sigma_1, \ldots, \sigma_m)}
	\alpha_1^{\sigma_1} \land \ldots \land \alpha_m^{\sigma_m} \land f(\sigma_1, \ldots, \sigma_m, \alpha_{m+1}, \ldots, \alpha_n) =
	\end{equation*}
	\begin{equation*}
	\left| \alpha_1^{\sigma_1} \land \ldots \land \alpha_m^{\sigma_m} = 1 \Leftrightarrow
	\sigma_1 = \alpha_1, \ \ldots, \ \sigma_m = \alpha_m \right|
	\end{equation*}
	\begin{equation*}
	= \alpha_1^{\alpha_1} \land \ldots \land \alpha_m^{\alpha_m} \land f(\alpha_1, \ldots, \alpha_m, \alpha_{m+1}, \ldots, \alpha_n)
	= f(\alpha_1, \ldots, \alpha_n)
	\end{equation*}
\end{enumerate}
\end{proof}

\begin{consequent}
Любая булева функция может быть реализована формулой над~$\{ \neg, \land, \lor \}$.
\end{consequent}
\section{Дизъюнктивная нормальная форма}
\index{Литерал} \textbf{Литералом} называется переменная или её отрицание.

\index{Конъюнкт} \textbf{Элементарным конъюнктом} называется конъюнкция литералов, в которую каждая переменная входит не более одного раза.
Элементарный конъюнкт называется \textbf{полным}, если он содержит все рассматриваемые переменные.

\index{Импликант} Элементарный конъюнкт~$K$ называется \textbf{импликантом} булевой функции~$f$, если $K \lor f = f$.
Импликант называется \textbf{простым}, если вычёркиванием литералов из него нельзя получить импликант $f$.

\begin{statement}
Элементарный конъюнкт~$K$~--- импликант булевой функции~$f$ $\Leftrightarrow K \rightarrow f = 1$.
\end{statement}
\begin{proof}
$K \lor f = f \Leftrightarrow
\overline K \lor K \lor f = f \lor \overline K \Leftrightarrow
1 = K \rightarrow f$.
\end{proof}

\index{Сокращения!ДНФ} \textbf{Дизъюнктивной нормальной формой} (\textbf{ДНФ}) называется дизъюнкция элементарных конъюнктов.

\index{Сокращения!СДНФ} \textbf{Совершенной дизъюнктивной нормальной формой} (\textbf{СДНФ}) называется дизъюнкция полных элементарных конъюнктов.

\begin{statement}
Булева функция~$f$ от $n$~переменных, не равная тождественно~$0$, представима в~виде СДНФ.
\end{statement}%
Для доказательства достаточно разложить её по всем переменным.

ДНФ булевой функции~$f$ называется \textbf{сокращённой}, если все её конъюнкты~--- простые импликанты $f$.

\begin{statement}
Булева функция представима в~виде сокращённой ДНФ, причём единственным образом.
\end{statement}
\begin{proof}
Пусть $f(x_1, \ldots, x_n)$~--- булева функция, $D = K_1 \lor K_2 \lor \ldots \lor K_m$~--- дизъюнкция всех простых импликантов $f$.
Возможны два случая:
\begin{enumerate}
	\item Пусть $f = 0 \Leftrightarrow
	K_i = 0 \Leftrightarrow
	D = 0 \Rightarrow
	D = f$, где $i = 1, \ldots, m$.
	
	\item Пусть $f = 1$.
	Запишем $f$ в~виде СДНФ:
	\begin{equation*}
	1 = f = \bigvee_{\begin{smallmatrix}
	(\sigma_1, \ldots, \sigma_n) \\
	f(\sigma_1, \ldots, \sigma_n) = 1
	\end{smallmatrix}} x_1^{\sigma_1} \land x_2^{\sigma_2} \land \ldots \land x_n^{\sigma_n} \Rightarrow
	\exists \alpha_1, \ldots, \alpha_n \colon x_1^{\alpha_1} \land x_2^{\alpha_2} \land \ldots \land x_n^{\alpha_n} = 1
	\end{equation*}
	
	Получили импликант.
	Из него можно получить простой импликант~$K$ вычёркиванием литералов, причём $K = 1$.
	$K$ входит в~$D$, тогда $D = 1 = f$.
\end{enumerate}
\end{proof}
\section{Принцип двойственности}
\index{Функция!булева!двойственная} Если $f(x_1, \ldots, x_n)$~--- булева функция, то \textbf{двойственной к~$f$ функцией} называется функция $\overline f(\overline{x_1}, \overline{x_2}, \ldots, \overline{x_n})$, обозначаемая $f^*(x_1, x_2, \ldots, x_n)$.

\begin{statement}
$(f^*)^* = f$.
\end{statement}
\begin{proof}
$\bigl( f^*(x_1, x_2, \ldots, x_n) \bigr)^* =
\bigl( \overline f(\overline{x_1}, \overline{x_2}, \ldots, \overline{x_n}) \bigr)^* =
f(x_1, x_2, \ldots, x_n)$.
\end{proof}

\begin{theorem}[принцип двойственности]
Если формула~$\Phi_1 = f_0(f_1, f_2, \ldots, f_n)$ задаёт некоторую функцию~$f$, то формула~$\Phi_2 = f_0^*(f_1^*, f_2^*, \ldots, f_n^*)$ задаёт $f^*$.
\end{theorem}
\begin{proof}
Без ограничения общности можно считать, что $f_1, \ldots, f_n$~--- функции от $m$~переменных, т.\,к. можно добавить несущественные переменные.
\begin{equation*}
f^*(x_1, \ldots, x_m) =
\overline f(\overline{x_1}, \ldots, \overline{x_m}) =
\overline{f_0}(\overline{\overline{f_1}}(\overline{x_1}, \ldots, \overline{x_m}), \ldots, \overline{\overline{f_n}}(\overline{x_1}, \ldots, \overline{x_m})) =
\end{equation*}
\begin{equation*}
= \overline{f_0}(\overline{f_1^*}(x_1, \ldots, x_m), \ldots, \overline{f_n^*}(x_1, \ldots, x_m)) =
f_0^*(f_1^*(x_1, \ldots, x_m), \ldots, f_n^*(x_1, \ldots, x_m)) =
\Phi_2
\end{equation*}
\end{proof}
\section{Конъюнктивная нормальная форма}
\index{Дизъюнкт} \textbf{Элементарным дизъюнктом} называется дизъюнкция литералов, в которую каждая переменная входит не более одного раза.
Элементарный дизъюнкт называется \textbf{полным}, если он содержит все рассматриваемые переменные.

\textbf{Конъюнктивной нормальной формой}, или \textbf{КНФ}, называется конъюнкция элементарных дизъюнктов.

\textbf{Совершенной конъюнктивной нормальной формой}, или \textbf{СКНФ}, называется конъюнкция полных элементарных дизъюнктов.

\begin{statement}
Булева функция~$f(x_1, \ldots, x_n)$, не равная тождественно~$1$, представима в~виде СКНФ.
\end{statement}
\begin{proof}
\begin{equation*}
f^*(x_1, \ldots, x_n) = \bigvee_{\begin{smallmatrix}
(\sigma_1, \ldots, \sigma_n) \\
f^*(\sigma_1, \ldots, \sigma_n) = 1
\end{smallmatrix}} x_1^{\sigma_1} \And \ldots \And x_n^{\sigma_n} \Leftrightarrow
\end{equation*}
\begin{equation*}
\Leftrightarrow f^{**}(x_1, \ldots, x_n) =
\bigwedge_{\begin{smallmatrix}
(\sigma_1, \ldots, \sigma_n) \\
f^*(\sigma_1, \ldots, \sigma_n) = 1
\end{smallmatrix}} x_1^{\sigma_1} \lor \ldots \lor x_n^{\sigma_n} =
\end{equation*}
\begin{equation*}
= \bigwedge_{\begin{smallmatrix}
(\sigma_1, \ldots, \sigma_n) \\
f(\overline{\sigma_1}, \ldots, \overline{\sigma_n}) = 0
\end{smallmatrix}} x_1^{\sigma_1} \lor \ldots \lor x_n^{\sigma_n} =
\bigwedge_{\begin{smallmatrix}
(\sigma_1, \ldots, \sigma_n) \\
f(\sigma_1, \ldots, \sigma_n) = 0
\end{smallmatrix}} x_1^{\overline{\sigma_1}} \lor \ldots \lor x_n^{\overline{\sigma_n}} \Leftrightarrow
\end{equation*}
\begin{equation*}
\Leftrightarrow f(x_1, \ldots, x_n)
= \bigwedge_{\begin{smallmatrix}
(\sigma_1, \ldots, \sigma_n) \\
f(\sigma_1, \ldots, \sigma_n) = 0
\end{smallmatrix}} x_1^{\overline{\sigma_1}} \lor \ldots \lor x_n^{\overline{\sigma_n}}
\end{equation*}
\end{proof}
\section{Методы нахождения сокращённой ДНФ}
\begin{statement}
\label{st:about_prime_implicants_1}
Пусть булева функция~$f = D(K_1, \ldots, K_m) = K_1 \lor K_2 \lor \ldots \lor K_m$, где $K_1, \ldots, K_m$~--- элементарные конъюнкты и $K$~--- простой импликант~$f$.
Тогда
\begin{itemize}
	\item $K$ содержит только переменные, входящие в~$D$;
	\item $\exists i, K' \colon K_i = K \lor K'$.
\end{itemize}
\end{statement}
\begin{proof}
\begin{equation*}
K \rightarrow f = 1 \Rightarrow (K = 1 \Rightarrow f = 1)
\end{equation*}
\begin{itemize}
	\item Докажем методом от противного, что $K$ содержит только переменные, входящие в~$D$.
	Пусть $K$ содержит переменную~$y$, не входящую в~$D$, тогда
	\begin{equation*}
	K = x_1^{\sigma_1} \cdot \ldots \cdot x_n^{\sigma_n} y^\sigma \Rightarrow
	K(\sigma_1, \ldots, \sigma_n, \sigma) = 1 \Rightarrow
	f(\sigma_1, \ldots, \sigma_n, \sigma) = 1
	\end{equation*}
	
	$y$ не входит в~$D$ $\Rightarrow$ $f(\sigma_1, \ldots, \sigma_n, \overline\sigma) = 1 \Rightarrow x_1^{\sigma_1} \cdot \ldots \cdot x_n^{\sigma_n}$~--- импликант $\Rightarrow$ $K$ не является простым импликантом.
	Противоречие.
	
	\item Пусть
	\begin{equation*}
	K = x_1^{\sigma_1} \cdot \ldots \cdot x_n^{\sigma_n} \Rightarrow
	K(\sigma_1, \ldots, \sigma_n) = 1 \Rightarrow
	f(\sigma_1, \ldots, \sigma_n) = 1 \Rightarrow
	\exists i \colon K_i(\sigma_1, \ldots, \sigma_n) = 1
	\end{equation*}
	
	Покажем, что все литералы, содержащиеся в~$K$, содержатся и в~$K_i$.
	\begin{itemize}
		\item Пусть $x_j$ входит в~$K$, тогда $K(\sigma_1, \ldots, \sigma_{j-1}, 1, \sigma_{j+1}, \ldots, \sigma_n) = 1$.
		Если $\overline{x_j}$ входит в~$K_i$, то $K_i(\sigma_1, \ldots, \sigma_{j-1}, 1, \sigma_{j+1}, \ldots, \sigma_n) = 0$.
		Противоречие, значит, $\overline{x_j}$ не входит в~$K_i$.
		
		\item Пусть $\overline{x_j}$ входит в~$K$.
		Аналогично доказывается, что $x_j$ не входит в~$K_i$.
		
		\item Пусть $x_j^{\sigma_j}$ входит в~$K$.
		Если ни $x_j$, ни $\overline{x_j}$ не входят в~$K_i$, то $K_i(\sigma_1, \ldots, \sigma_j, \ldots, \sigma_n) \opbr=
		K_i(\sigma_1, \ldots, \overline{\sigma_j}, \ldots, \sigma_n) = 1 \Rightarrow
		x_1^{\sigma_1} \cdot \ldots \cdot x_{i-1}^{\sigma_{i-1}} x_{i+1}^{\sigma_{i+1}} \cdot \ldots \cdot x_n^{\sigma_n}$~--- импликант $\Rightarrow$ $K$ не является простым импликантом.
		Противоречие, значит, $x_j$ или $\overline{x_j}$ входит в~$K_i$.
	\end{itemize}
	
	Остаётся единственный возможный случай: если $x_j^{\sigma_j}$ входит в~$K$, то $x_j^{\sigma_j}$ входит и в~$K_i$.
	Тогда $\exists K' \colon K_i = K \land K'$.
\end{itemize}
\end{proof}

\begin{statement}
\label{st:prime_implicants_of_conjuction}
Если $K_1, \ldots, K_m$ и $M_1, \ldots, M_r$~--- простые импликанты булевых функций $f$ и $g$ соответственно, то любой простой импликант функции~$f \land g$ равен $K_i M_j$ для некоторых $i, j$.
\end{statement}
\begin{proof}
Пусть $K$~--- простой импликант $f \land g$.
По утверждению~\ref*{st:about_prime_implicants_1} $\exists i, j, K' \colon K_i M_j \opbr= K K'$.
\begin{enumerate}
	\item Докажем методом от противного, что $K_i M_j = K$.
	Пусть $K' \neq 1$, $y$~--- переменная, входящая в~$K'$, тогда $y$ входит в $K_i$ или в $M_j$.
	Без ограничения общности можно считать, что $y$ входит в~$K_i$.
	Рассмотрим элементарный конъюнкт~$\widetilde{K_i}$, полученный из~$K_i$ удалением $y$.
	Тогда $\widetilde{K_i}$~--- импликант~$f$, т.\,к. если изменить значение $y$, то значение $\widetilde{K_i}$ и $K$ не изменятся.
	Если $\widetilde{K_i} = 1$, то можно подобрать такой набор значений переменных, что $K_i = 1 \Rightarrow f \land g = 1 \Rightarrow f = 1$, тогда $K_i$ не является простым импликантом.
	Противоречие, значит, $K_i M_j = K$.
	
	\item Докажем методом от противного, что $K_i M_j$~--- простой импликант.
	Пусть найдётся переменная~$y$ такая, что её удаление из~$K_i M_j$ даёт импликант~$\widetilde{K_i M_j}$.
	$\widetilde{K_i M_j} = 1 \Rightarrow f \land g = 1 \Rightarrow f = 1 \lAnd g = 1$, тогда $K_i$ или $M_j$ не является простым импликантом.
	Противоречие, значит, $K_i M_j$~--- простой импликант.
\end{enumerate}
\end{proof}

\subsubsection{Метод Блейка}
\index{Метод!Блейка}
Применяется к произвольной ДНФ.
\begin{enumerate}
	\item Применяем формулу обобщённого склеивания, пока возможно: $x K_1 \lor \overline x K_2 =
	x K_1 \lor \overline x K_2 \lor K_1 K_2$.
	
	\item Применяем формулу поглощения: $K \lor K K_1 = K$.
\end{enumerate}
\begin{proof}[корректности]
Пусть после первого этапа метода Блейка получена ДНФ~$D'$ булевой функции~$f$.
\begin{enumerate}
	\item Докажем методом математической индукции, что для любого импликанта~$K$ функции~$f$ найдётся такой конъюнкт~$K'$ в~$D'$, что $K$~--- импликант~$K'$.
		\indbase Пусть $K$ содержит все переменные $f$.
		$K$~--- импликант~$f$, значит, $K$ входит в~$D'$.
		\indstep Пусть для любого импликанта~$K$ функции~$f$, содержащего более $n$~переменных, найдётся такой конъюнкт~$K'$ в~$D'$, что $K$~--- импликант~$K'$.
		Докажем то~же для импликанта~$K$, содержащего $n$~переменных.
		Пусть $y$~--- переменная, не входящая в~$K$, тогда $K_1 = y \land K$, $K_2 = \overline y \land K$~--- импликанты $f$.
		По предположению индукции для них найдутся конъюнкты~$K_1', K_2'$ в~$D'$ такие, что $K_1, K_2$~--- импликанты $K_1', K_2'$ соответственно.
		Возможны два случая:
		\begin{enumerate}
			\item Пусть $K_1'$ или $K_2'$ не содержит $y$, тогда $K$~--- его импликант.
			\item Пусть и $K_1'$, и $K_2'$ содержат $y$, тогда $K_1' = y \land K_1''$, $K_2' = \overline y \land K_2''$.
			$D'$ содержит $K_1'$ и $K_2'$, значит, $D'$ содержит $K_1'' \land K_2''$.
			$K$~--- импликант $K_1'' \land K_2''$.
		\end{enumerate}
		\indend
	
	Если $K$~--- простой импликант $f$, то существует такой конъюнкт~$K'$ в~$D'$, что $K$~--- импликант~$K'$.
	Легко показать методом от противного, что $K = K'$, значит, в~$D'$ входят все простые импликанты.
	
	\item Пусть $K$~--- не простой импликант $f$, содержащийся в~$D'$, тогда из него вычёркиванием литералов можно получить простой импликант~$K'$.
	$D'$ содержит $K'$, тогда на втором этапе метода Блейка имеем $K \lor K' = K'$.
	Т.\,о., после второго этапа метода Блейка получим ДНФ, содержащую только простые импликанты.
\end{enumerate}
\end{proof}

\subsubsection{Метод Квайна}
\index{Метод!Квайна}
Применяется к СДНФ.
Пусть дана булева функция от $n$~переменных.
Начинаем с~$k = n$.
\begin{enumerate}
	\item Применяем формулу $x K \lor \overline x K	= x K \lor \overline x K \lor K$ ко всем конъюнктам, содержащим $k$~литералов.
	\item Применяем формулу поглощения: $K \lor K K_1 = K$.
	\item Уменьшаем значение~$k$ на~$1$ и повторяем с начала.
\end{enumerate}
\begin{proof}[корректности]
Пусть в сокращённой ДНФ $K$~--- элементарный конъюнкт, не содержащий переменной~$y$, тогда $K = y K \lor \overline y K$.
Добавляя таким образом переменные, получим СДНФ.
Тогда, если проделаем обратные операции (что и является методом Квайна), получим сокращённую ДНФ.
\end{proof}

\subsubsection{Метод Нельсона}
\index{Метод!Нельсона}
Применяется к произвольной КНФ.
\begin{enumerate}
	\item Раскрываем скобки: $(a \lor c)(b \lor c) = ab \lor c$.
	\item Упрощаем, используя формулы $xxK = xK$, $x \overline x K = 0$, $K \lor K K_1 = K$.
\end{enumerate}
\begin{proof}[корректности]
КНФ~--- конъюнкция сокращённых ДНФ, поэтому раскрытием скобок и упрощением по утверждению~\ref*{st:prime_implicants_of_conjuction} получим сокращённую ДНФ.
\end{proof}
\section{Геометрическая интерпретация булевых функций}
Пусть дана булева функция~$f \colon B^n \to B$.
$B^n$ можно отождествить с вершинами единичного куба в~$\mathbb R^n$.

С каждой булевой функцией~$f$ можно связать множество~$N_f$, состоящее из вершин, в которых~$f = 1$.

Пусть дан $n$-мерный куб.
\index{Грань} Множество его вершин, для которых значения~$x_{i_1}, \ldots, x_{i_k}$ совпадают, называется \textbf{гранью ранга $n - k$}.
Очевидно, что каждая грань однозначно соответствует некоторому элементарному конъюнкту, который принимает значение~$1$ в точности на вершинах грани.

Грань~$N_K$ называется \textbf{максимальной относительно булевой функции~$f$}, если $N_K \subseteq N_f \lAnd \forall N_{K'} \ N_K \opbr\subseteq N_{K'} \opbr\subseteq N_f \Rightarrow N_K = N_K'$.

\begin{statement}
Каждой максимальной относительно булевой функции~$f$ грани соответствует простой импликант $f$.
\end{statement}
\begin{proof}
Пусть $N_K$~--- максимальная грань.
\begin{enumerate}
	\item $K(\alpha_1, \ldots, \alpha_n) = 1 \Rightarrow
	(\alpha_1, \ldots, \alpha_n) \in N_K \subseteq N_f \Rightarrow
	f(\alpha_1, \ldots, \alpha_n) = 1 \Rightarrow$
	$K$~--- импликант $f$.
	
	\item Докажем методом от противного простоту $K$.
	Пусть $K$ не простой, тогда можно получить простой импликант~$K'$ из $K$ вычёркиванием переменных, поэтому $N_K \subset N_{K'} \subseteq N_f$, значит, $N_K$ не является максимальной гранью.
	Противоречие.
\end{enumerate}
\end{proof}

\begin{statement}
Если $K$~--- простой импликант, то $N_K$~--- максимальная грань.
\end{statement}
\begin{proofcontra}
Пусть $N_K$ не является максимальной гранью, тогда $N_K \subset N_K'$, где $N_K'$~--- максимальная грань, значит, можно получить простой импликант~$K'$ из $K$ вычёркиванием переменных, поэтому $K$ не является простым импликантом.
Противоречие.
\end{proofcontra}

\index{Покрытие} Набор граней $N_{K_1}, \ldots, N_{K_m}$ называется \textbf{покрытием булевой функции~$f$}, если $N_f = N_{K_1} \cup \ldots \cup N_{K_m}$.

\begin{statement}
Каждое покрытие булевой функции~$f$ однозначно соответствует ДНФ для~$f$.
\end{statement}
\begin{proof}
Каждому конъюнкту в ДНФ однозначно соответствует некоторая грань в покрытии, тогда покрытие из этих граней однозначно соответствует данной ДНФ.
\end{proof}

\begin{consequent}
Сокращённой ДНФ соответствует покрытие из всех максимальных граней.
\end{consequent}

ДНФ булевой функции~$f$, содержащая наименьшее число литералов, считая повторяющиеся, называется \textbf{минимальной}.

ДНФ булевой функции~$f$, содержащая наименьшее число элементарных конъюнктов, называется \textbf{кратчайшей}.

Покрытие булевой функции~$f$ называется \textbf{неприводимым}, если оно состоит только из максимальных граней и при удалении любой грани из него оно перестаёт быть покрытием, а ДНФ, соответствующая ему, называется \textbf{тупиковой}.

\begin{statement}
Кратчайшая или минимальная ДНФ является тупиковой.
\end{statement}
\begin{proofcontra}
Пусть кратчайшая/минимальная ДНФ не является тупиковой, тогда из соответствующего ей покрытия можно удалить грань.
Но в таком случае исходная ДНФ не является кратчайшей/минимальной.
Противоречие.
\end{proofcontra}
\section{Операции типа I и II}
Пусть даны ДНФ~$D$ и эквивалентная ей $D'$.
Говорят, что:
\begin{itemize}
	\item ДНФ~$D'$ получается из~$D$ \textbf{операцией типа I}, если $D'$ получена из~$D$ вычёркиванием элементарного конъюнкта;
	\item ДНФ~$D'$ получается из~$D$ \textbf{операцией типа II}, если $D'$ получена из~$D$ вычёркиванием одного или нескольких литералов в каком-либо элементарном конъюнкте.
\end{itemize}

ДНФ~$D$ называется \textbf{тупиковой относительно операций типа I и II}, если они к ней неприменимы.

\begin{statement}
ДНФ~$D$ тупиковая относительно операций типа I и II $\Leftrightarrow$ $D$ тупиковая в геометрическом смысле.
\end{statement}
\begin{proof}
\begin{enumerate}
	\item $\Rightarrow$. Если к~$D$ неприменимы операции типа I и II, то ей соответствует неприводимое покрытие, поэтому $D$ тупиковая в геометрическом смысле.
	\item $\Leftarrow$.
	\begin{enumerate}
		\item Операции типа I соответствует удаление грани из покрытия, соответствующего $D$.
		Удалить грань нельзя, значит, операция типа I неприменима к~$D$.
		\item Пусть $D = K_1 \lor \ldots \lor K_m$, $K_i = x_j^{\sigma_j} K_i'$, тогда $N_{K_i'} \subset N_{K_i}$.
		Если $N_f = N_{K_1} \opbr\cup \ldots \opbr\cup N_{K_i'} \opbr\cup \ldots \opbr\cup N_{K_m}$, то $N_{K_i}$ не является максимальной гранью, что неверно, значит, операция типа II неприменима к~$D$.
	\end{enumerate}
	
	Тогда $D$ тупиковая относительно операций типа I и II.
\end{enumerate}
\end{proof}
\section{Построение тупиковых ДНФ}
Пусть дана булева функция~$f$.
\begin{enumerate}
	\item Находим сокращённую ДНФ $D = K_1 \lor \ldots \lor K_m$ для~$f$.
	
	\item Пусть $N_f = \{ P_1, \ldots, P_r \}$.
	Составляем следующую таблицу, называемую \textbf{таблицей Квайна}:
	\begin{equation*}
	\begin{array}{|c|c|c|c|c|}
	\hline
	    & P_1 & P_2 & \cdots & P_r \\
    \hline
	K_1 & \sigma_{11} & \sigma_{12} & \cdots & \sigma_{1r} \\
	\hline
	K_2 & \sigma_{21} & \sigma_{22} & \cdots & \sigma_{2r} \\
	\hline
	\vdots & \vdots & \vdots & \ddots & \vdots \\
	\hline
	K_m & \sigma_{m1} & \sigma_{m2} & \cdots & \sigma_{mr} \\
	\hline
	\end{array}\,, \
	\sigma_{ij} =
	\begin{cases}
	0, \ P_j \notin N_{K_i} \\
	1, \ P_j \in N_{K_i}
	\end{cases}
	\end{equation*}
	
	\item Составляем выражение $\bigand\limits_{j=1}^r (\sigma_{1j} K_1 \lor \sigma_{2j} K_2 \lor \ldots \lor \sigma_{mj} K_m)$ и раскрываем в нём скобки по формулам $(A \lor B)C \opbr= AC \lor BC$, $A \lor BA = A$.
	В полученной ДНФ относительно переменных $K_1, \ldots, K_m$ каждому конъюнкту однозначно соответствует тупиковая ДНФ для~$f$: $K_{i_1} \land \ldots \land K_{i_p} \to K_{i_1} \lor \ldots \lor K_{i_p}$.
\end{enumerate}
\begin{proof}
$\sigma_{1j} K_1 \lor \ldots \lor \sigma_{mj} K_m = 1 \Rightarrow
P_j \in N_{\sigma_{1j} K_1} \cup \ldots \cup N_{\sigma_{mj} K_m}$, тогда если
$\bigand\limits_{j=1}^r (\sigma_{1j} K_1 \lor \ldots \lor \sigma_{mj} K_m) = 1$, то набор полученных граней покрывает все вершины из~$N_f$.
\end{proof}
\section{Полнота и замкнутость классов булевых функций}
Множество~$F$ булевых функций (также называемое \textbf{классом}) называется \textbf{полным}, если любая булева функция реализуется формулой над~$F$.

Множество, состоящее из всех булевых функций, обозначается $P_2$.

\begin{theorem}
Если $F = \{ f_1, f_2, \ldots \}$~--- полный набор булевых функций, $G = \{ g_1, g_2, \ldots \}$~--- набор булевых функций, причём каждая функция из~$F$ реализуется формулой над~$G$, то $G$~--- полный набор.
\end{theorem}
\begin{proof}
\begin{equation*}
f_1 = \Phi_1(g_1, g_2, \ldots), f_2 = \Phi_2(g_1, g_2, \ldots), \ldots \Rightarrow
\end{equation*}
\begin{equation*}
\Rightarrow \forall f = \Phi(f_1, f_2, \ldots) =
\Phi(\Phi_1(g_1, g_2, \ldots), \Phi_2(g_1, g_2, \ldots), \ldots) =
\Phi(g_1, g_2, \ldots)
\end{equation*}

Значит, $G$~--- полный набор.
\end{proof}

\index{Замыкание} \textbf{Замыканием} множества~$F$ булевых функций называется множество всех булевых функций, реализуемых формулами над~$F$, и обозначается $[F]$.
Свойства замыкания:
\begin{enumerate}
	\item $F \subseteq [F]$;
	\item $[[F]] = [F]$, т.\,к. $[[F]]$~--- множество функций, реализуемых формулами над~$[F]$, которые реализуются формулами над~$F$;
	\item $F \subseteq K \Rightarrow [F] \subseteq [K]$, т.\,к. формула над~$F$ является формулой над~$K$.
\end{enumerate}

\begin{statement}
Множество~$F$ булевых функций полно $\Leftrightarrow [F] = P_2$.
\end{statement}

\subsubsection{Класс \texorpdfstring{$T_0$}{} функций, сохраняющих константу 0}
\begin{equation*}
T_0 = \{ f(x_1, \ldots, x_n) \mid f(0, \ldots, 0) = 0 \}
\end{equation*}
\begin{proof}[замкнутости]
Пусть $f, f_1, \ldots, f_n \in T_0$, тогда
\begin{equation*}
\Phi =
f(f_1(0, \ldots, 0), \ldots, f_n(0, \ldots, 0)) =
f(0, \ldots, 0) = 0
\end{equation*}

Значит, $\Phi$ реализует функцию из~$T_0$.
\end{proof}

$T_0$ содержит $\dfrac{2^{2^n}}2 = 2^{2^n - 1}$~функций от $n$~переменных.

\subsubsection{Класс \texorpdfstring{$T_1$}{} функций, сохраняющих константу 1}
\begin{equation*}
T_1 = \{ f(x_1, \ldots, x_n) \mid f(1, \ldots, 1) = 1 \}
\end{equation*}
\begin{proof}[замкнутости]
Пусть $f, f_1, \ldots, f_n \in T_1$, тогда
\begin{equation*}
\Phi =
f(f_1(1, \ldots, 1), \ldots, f_n(1, \ldots, 1)) =
f(1, \ldots, 1) = 1
\end{equation*}

Значит, $\Phi$ реализует функцию из~$T_1$.
\end{proof}

$T_1$ содержит $\dfrac{2^{2^n}}2 = 2^{2^n - 1}$~функций от $n$~переменных.

\subsubsection{Класс \texorpdfstring{$S$}{} самодвойственных функций}
\index{Функция!булева!самодвойственная} Булева функция~$f$ называется \textbf{самодвойственной}, если $f^* = f$.
\begin{equation*}
S = \{ f(x_1, \ldots, x_n) \mid
f(x_1, \ldots, x_n) = f^*(x_1, \ldots, x_n) \} \Leftrightarrow
\end{equation*}
\begin{equation*}
\Leftrightarrow S = \{ f(x_1, \ldots, x_n) \mid
f(x_1, \ldots, x_n) = \overline f(\overline{x_1}, \ldots, \overline{x_n}) \}
\end{equation*}
\begin{proof}[замкнутости]
Пусть $f, f_1, \ldots, f_n \in S$, тогда $\Phi = f(f_1, \ldots, f_n) \opbr= f^*(f_1^*, \ldots, f_n^*) \opbr= \Phi^*$, значит, $\Phi$ реализует функцию из~$S$.
\end{proof}

$S$ содержит $2^{2^{n-1}} = \sqrt{2^{2^n}}$~функций от $n$~переменных.

\begin{lemma}[о несамодвойственной функции]
Если булева функция $f \notin S$, то подстановкой $x, \overline x$ вместо переменных можно получить константу.
\end{lemma}
\begin{proof}
\begin{equation*}
f(x_1, \ldots, x_n) \notin S \Leftrightarrow
\exists \alpha_1, \ldots, \alpha_n \colon
f(\alpha_1, \ldots, \alpha_n) \neq \overline f(\overline{\alpha_1}, \ldots, \overline{\alpha_n})
\end{equation*}

Подставим в~$f$ $x^{\alpha_i}$ вместо $x_i$.
\begin{enumerate}
	\item При $x = 0$ получим $f(0^{\alpha_1}, \ldots, 0^{\alpha_n}) =
	f(\overline{\alpha_1}, \ldots, \overline{\alpha_n}) =
	f(\alpha_1, \ldots, \alpha_n)$.
	
	\item При $x = 1$ получим $f(1^{\alpha_1}, \ldots, 1^{\alpha_n}) =
	f(\alpha_1, \ldots, \alpha_n)$.
\end{enumerate}

$f(0^{\alpha_1}, \ldots, 0^{\alpha_n}) = f(1^{\alpha_1}, \ldots, 1^{\alpha_n})$, значит, $f(x^{\alpha_1}, \ldots, x^{\alpha_n})$~--- константа.
\end{proof}

\subsubsection{Класс \texorpdfstring{$M$}{} монотонных функций}
Введём отношение~$\preccurlyeq$: $\overline\alpha = (\alpha_1, \ldots, \alpha_n) \preccurlyeq \overline\beta = (\beta_1, \ldots, \beta_n) \Leftrightarrow
\alpha_1 \leqslant \beta_1, \ldots, \alpha_n \leqslant \beta_n$.

\index{Функция!булева!монотонная} Булева функция~$f$ называется \textbf{монотонной}, если
$\forall \overline\alpha, \overline\beta \
\overline\alpha \preccurlyeq \overline\beta \Rightarrow f(\overline\alpha) \leqslant f(\overline\beta)$.
\begin{equation*}
M = \{ f(x_1, \ldots, x_n) \mid
\forall \overline\alpha, \overline\beta \
\overline\alpha \preccurlyeq \overline\beta \Rightarrow f(\overline\alpha) \leqslant f(\overline\beta) \} \Leftrightarrow
\end{equation*}
\begin{equation*}
\Leftrightarrow M = \{ f(x_1, \ldots, x_n) \mid
(\forall i \ \alpha_i \leqslant \beta_i) \Rightarrow f(\alpha_1, \ldots, \alpha_n) \leqslant f(\beta_1, \ldots, \beta_n) \}
\end{equation*}
\begin{proof}[замкнутости]
Пусть $f, f_1, \ldots, f_n \in S$, $\Phi = f(f_1, \ldots, f_n)$.
\begin{equation*}
\overline\alpha \preccurlyeq \overline\beta \Rightarrow
f(\overline\alpha) \leqslant f(\overline\beta), \
f_1(\overline\alpha) \leqslant f_1(\overline\beta), \ \ldots, \
f_n(\overline\alpha) \leqslant f_n(\overline\beta) \Rightarrow
\end{equation*}
\begin{equation*}
\Rightarrow (f_1(\overline\alpha), \ldots, f_n(\overline\alpha)) \preccurlyeq (f_1(\overline\beta), \ldots, f_n(\overline\beta)) \Rightarrow
f(f_1(\overline\alpha), \ldots, f_n(\overline\alpha)) \leqslant f(f_1(\overline\beta), \ldots, f_n(\overline\beta))
\end{equation*}

Значит, $\Phi$ реализует функцию из~$M$.
\end{proof}

\begin{consequent}
Если некоторая ДНФ для булевой функции~$f$ не содержит отрицаний, то $f$ монотонна.
\end{consequent}
\begin{proof}
ДНФ без отрицаний — это формула над~$\{ \And, \lor \}$.
$\And, \lor$~--- монотонные функции, значит, ДНФ без отрицаний тоже монотонна, тогда и $f$ монотонна.
\end{proof}

\begin{lemma}[о немонотонной функции]
Если булева функция~$f \notin M$, то подстановкой $0, 1, x$ вместо переменных можно получить $\overline x$.
\end{lemma}
\begin{proof}
\begin{equation*}
f(x_1, \ldots, x_n) \notin M \Leftrightarrow
\exists \overline\alpha, \overline\beta \colon
\overline\alpha \preccurlyeq \overline\beta, \ f(\overline\alpha) \nleqslant f(\overline\beta) \Leftrightarrow
f(\overline\alpha) = 1, \ f(\overline\beta) = 0
\end{equation*}
\begin{enumerate}
	\item Пусть $\overline\alpha$ и $\overline\beta$ отличаются в нескольких позициях:
	$\overline\alpha = (\ldots, \gamma_1, \ldots, \gamma_2, \ldots, \gamma_k, \ldots)$, где
	$\gamma_1, \ldots, \gamma_k$~--- позиции, в которых $\overline\alpha$ и $\overline\beta$ отличаются, тогда $\gamma_1 = \gamma_2 = \ldots = \gamma_k = 0$.
	
	Рассматривая эти позиции, введём наборы значений
	\begin{equation*}
	\overline\alpha_0 = \overline\alpha = (\ldots, 0, \ldots, 0, \ldots, 0, \ldots), \
	\overline\alpha_1 = (\ldots, 1, \ldots, 0, \ldots, 0, \ldots),
	\end{equation*}
	\begin{equation*}
	\overline\alpha_2 = (\ldots, 1, \ldots, 1, \ldots, 0, \ldots), \ \ldots, \
	\overline\alpha_k = (\ldots, 1, \ldots, 1, \ldots, 1, \ldots)
	\end{equation*}
	
	Легко показать методом от противного, что $f(\overline\alpha) > f(\overline\beta) \Rightarrow
	\exists 0 \leqslant i < k \colon f(\overline\alpha_i) > f(\overline\alpha_{i+1})$.
	Т.\,о., этот случай сведён к следующему случаю.
	
	\item Пусть $\overline\alpha$ и $\overline\beta$ отличаются только в одной позиции.
	\begin{equation*}
	f(\alpha_1, \ldots, \alpha_{i-1}, 0, \alpha_{i+1}, \ldots, \alpha_n) = 1, \
	f(\alpha_1, \ldots, \alpha_{i-1}, 1, \alpha_{i+1}, \ldots, \alpha_n) = 0 \Rightarrow
	\end{equation*}
	\begin{equation*}
	\Rightarrow f(\alpha_1, \ldots, \alpha_{i-1}, x, \alpha_{i+1}, \ldots, \alpha_n) = \overline x
	\end{equation*}
\end{enumerate}
\end{proof}

\begin{theorem}
Если булева функция~$f \in M$ и $K$~--- простой импликант $f$, то $K$ не содержит отрицаний.
\end{theorem}
\begin{proofcontra}
Пусть дана $f(x_1, \ldots, x_n, y)$, $K = x_1^{\sigma_1} \cdot \ldots \cdot x_k^{\sigma_k} \overline y$.
\begin{equation*}
K(\sigma_1, \ldots, \sigma_k, 0) = 1 \Rightarrow
f(\sigma_1, \ldots, \sigma_k, x_{k+1}, \ldots, x_n, 0) = 1 \Rightarrow
f(\sigma_1, \ldots, \sigma_k, x_{k+1}, \ldots, x_n, 1) = 1
\end{equation*}

Пусть $K' = x_1^{\sigma_1} \cdot \ldots \cdot x_k^{\sigma_k} \Rightarrow
K(\sigma_1, \ldots, \sigma_k) = 1$, $f(\sigma_1, \ldots, \sigma_k, x_{k+1}, \ldots, x_n, y) = 1 \Rightarrow$ $K'$~--- импликант $f \Rightarrow K$ не является простым импликантом $f$.
Противоречие.
\end{proofcontra}

\begin{theorem}
Булева функция $f \in M \Leftrightarrow$ сокращённая ДНФ для неё не содержит отрицаний.
\end{theorem}
\begin{proof}
\begin{enumerate}
	\item $\Rightarrow$. Простые импликанты $f$ не содержат отрицаний.
	Сокращённая ДНФ~--- дизъюнкция простых импликантов, значит, она не содержит отрицаний.
	\item $\Leftarrow$. $f \in M$ как функция, реализуемая формулой над монотонными функциями.
\end{enumerate}
\end{proof}

\begin{theorem}
Сокращённая ДНФ для монотонной булевой функции является тупиковой.
\end{theorem}
\begin{proof}
Пусть $K = x_1 \And \ldots \And x_k$~--- простой импликант монотонной булевой функции~$f(x_1, \ldots, x_n)$, $f = K \lor f'$.
Каждый импликант $f'$ должен содержать один из литералов $x_{k+1}, \ldots, x_n$, т.\,к. $K$~--- простой импликант.
Тогда при $x_1 = \ldots = x_k = 1$, $x_{k+1} = \ldots = x_n = 0$ имеем $f(x_1, \ldots, x_n) = 1$, $f'(x_1, \ldots, x_n) = 0 \Rightarrow f \neq f' \Rightarrow$ из сокращённой ДНФ нельзя вычеркнуть ни один из конъюнктов, поэтому она тупиковая.
\end{proof}

\subsubsection{Класс \texorpdfstring{$L$}{} линейных функций}
\index{Функция!булева!линейная} Булева функция~$f$ называется \textbf{линейной}, если она реализуется формулой над~$\{ +, 1 \}$.
\begin{equation*}
L = \{ f(x_1, \ldots, x_n) \mid
f(x_1, \ldots, x_n) = a_0 + a_1 x_1 + \ldots + a_n x_n, \ a_i \in B \}
\end{equation*}
\begin{proof}[замкнутости]
Пусть
\begin{equation*}
f(x_1, \ldots, x_n) = c_0 + c_1 x_1 + \ldots + c_n x_n,
\end{equation*}
\begin{equation*}
f_1(x_1, \ldots, x_n) = c_{10} + c_{11} x_1 + \ldots + c_{1n} x_n, \ \ldots, \
f_n(x_1, \ldots, x_n) = c_{n0} + c_{n1} x_1 + \ldots + c_{nn} x_n
\end{equation*}
\begin{equation*}
\Phi
= f(f_1, \ldots, f_n) =
c_0 + c_1(c_{10} + c_{11} x_1 + \ldots + c_{1n} x_n) + \ldots + c_n(c_{n0} + c_{n1} x_1 + \ldots + c_{nn} x_n) =
\end{equation*}
\begin{equation*}
= c_0 + c_1 c_{10} + \ldots + c_n c_{n0} + (c_1 c_{11} + \ldots + c_n c_{n1}) x_1 + \ldots + (c_1 c_{1n} + \ldots + c_n c_{nn}) x_n
\end{equation*}

Значит, $\Phi$ реализует функцию из~$L$.
\end{proof}

$L$ содержит $2^{n+1}$~функций от $n$~переменных.

\begin{lemma}[о нелинейной функции]
Если булева функция~$f \notin L$, то подстановкой $0, 1, x, y, \overline x, \overline y$ вместо переменных и, возможно, инверсией значения $f$ можно получить $x \And y$.
\end{lemma}
\begin{proof}
$f \notin L$, тогда без ограничения общности можно считать, что $f$ содержит $x_1 x_2$.
Пусть
\begin{equation*}
f(x_1, \ldots, x_n) = x_1 x_2 f_1(x_3, \ldots, x_n) + x_1 f_2(x_2, x_3, \ldots, x_n) + x_2 f_3(x_1, x_3, \ldots, x_n) + f_4(x_3, \ldots, x_n) \Rightarrow
\end{equation*}
\begin{equation*}
\Rightarrow \exists \alpha_3, \ldots, \alpha_n \colon f_1(\alpha_3, \ldots, \alpha_n) = 1
\end{equation*}

Рассмотрим
\begin{equation*}
\varphi(x_1, x_2) = f(x_1, x_2, \alpha_3, \ldots, \alpha_n) = x_1 x_2 + \alpha x_1 + \beta x_2 + \gamma,
\end{equation*}
\begin{equation*}
\psi(x_1, x_2) = \varphi(x_1 + \beta, x_2 + \alpha) + \alpha\beta + \gamma =
\end{equation*}
\begin{equation*}
= x_1 x_2 + \alpha x_1 + \beta x_2 + \alpha\beta + \alpha x_1 + \alpha\beta + \beta x_2 + \alpha\beta + \gamma + \alpha\beta + \gamma =
x_1 x_2
\end{equation*}

$\psi$ получена из~$\varphi$ подстановкой $x, y, \overline x, \overline y$ и, возможно, инверсией значения $\varphi$, которая, в свою очередь, получена из~$f$ подстановкой $0, 1$.
\end{proof}
\section{Многочлен Жегалкина}
\index{Многочлен!Жегалкина} \textbf{Многочленом Жегалкина} называется многочлен вида
$\displaystyle \sum_{(i_1, \ldots, i_k)} a_{i_1, \ldots, i_k} x_{i_1} \cdot \ldots \cdot x_{i_k}$, где $a_{i_1, \ldots, i_k} \in B$.

\begin{theorem}[Жегалкина]
\index{Теорема!Жегалкина}
Каждая булева функция единственным образом представима в виде многочлена Жегалкина.
\end{theorem}
\begin{proof}
\begin{enumerate}
	\item Докажем представимость.
	Любую булеву функцию можно реализовать формулой над~$\{ \neg, \land, \lor \}$.
	Тогда
	\begin{enumerate}
		\item Заменим $\overline x = x + 1$, $x \land y = xy$, $x \lor y = xy + x + y$.
		\item Раскроем скобки по дистрибутивности: $x(y + z) = xy + xz$.
		\item Упростим по формулам $x + x = 0$, $x + \overline x = 1$, $x + 0 = x$, $x \cdot 0 = 0$, $xx = x$.
	\end{enumerate}
	
	Получим многочлен Жегалкина.
	
	\item Докажем единственность.
	Многочлен Жегалкина для булевой функции от $n$~переменных можно представить в виде $\displaystyle \sum_{i=0}^{2^n-1} c_i K_i$, где $c_i$~--- некоторые коэффициенты, $K_i$~--- элементарные конъюнкты.
	Тогда всего различных многочленов $2^{2^n}$, т.\,е. столько же, сколько и булевых функций.
	Каждая булева функция представима хотя~бы одним многочленом, значит, каждой функции однозначно соответствует многочлен Жегалкина.
\end{enumerate}
\end{proof}

Методы построения многочленов Жегалкина по заданной функции:
\begin{enumerate}
	\item \textbf{Эквивалентные преобразования}
	
	Описаны в доказательстве теоремы Жегалкина.
	
	\item \textbf{Эквивалентные преобразования СДНФ}
	
	Заменим в СДНФ $K_1 \lor K_2 = K_1 + K_2$, $\overline x = x + 1$ и упростим.
	\begin{proof}[корректности]
	Пусть $K_1$ и $K_2$~--- элементарные конъюнкты СДНФ.
	Тогда
	\begin{equation*}
	\exists x \colon K_1 = x K_1' \lAnd K_2 = \overline x K_2' \Rightarrow
	K_1 \lor K_2 =
	x \overline x K_1' K_2' + x K_1' + \overline x K_2' =
	x K_1' + \overline x K_2' =
	K_1 + K_2
	\end{equation*}
	\end{proof}
	
	\item \index{Метод!неопределённых коэффициентов} \textbf{Метод неопределённых коэффициентов}
	
	Пусть $\displaystyle  f(x_1, \ldots, x_n) = \sum_{i=0}^{2^n-1} c_i K_i$.
	Составим систему уравнений:
	\begin{equation*}
	\begin{cases}
	f(0, \ldots, 0, 0) = c_0 \\
	f(0, \ldots, 0, 1) = c_0 + c_{2^n-1} \\
	\ldots \\
	f(1, \ldots, 1, 1) = c_0 + \ldots + c_{2^n-1}
	\end{cases}
	\end{equation*}
	
	Решив её, найдём коэффициенты многочлена Жегалкина.
	
	\item \index{Метод!Паскаля} \textbf{Метод Паскаля}
	
	Введём по индукции операцию~$T$ над векторами размерности $2^n$:
		\indbase $n = 1$.
		Пусть $\overline\alpha = (\alpha_0, \alpha_1)$, тогда $T(\overline\alpha) = (\alpha_0, \alpha_0 + \alpha_1)$.
		\indstep $n > 1$.
		Пусть $\overline\alpha = (\overline\alpha_0, \overline\alpha_1)$, где
		$\overline\alpha_0, \overline\alpha_1$~--- векторы размерности $2^{n-1}$, тогда
		$T(\overline\alpha) \opbr= (T(\overline\alpha_0), T(\overline\alpha_0) + T(\overline\alpha_1))$.
		\indend
		
	Исследуем свойства операции~$T$:
	\begin{enumerate}
		\item $T(\overline\alpha + \overline\beta) = T(\overline\alpha) + T(\overline\beta)$
		\begin{proofmathind}
			\indbase $n = 1$.
			Пусть $\overline\alpha = (\alpha_0, \alpha_1)$, $\overline\beta = (\beta_0, \beta_1)$, тогда
			\begin{equation*}
			T(\overline\alpha + \overline\beta) =
			T((\alpha_0, \alpha_1) + (\beta_0, \beta_1)) =
			T((\alpha_0 + \beta_0, \alpha_1 + \beta_1)) =
			(\alpha_0 + \beta_0,  \alpha_0 + \beta_0 + \alpha_1 + \beta_1) =
			\end{equation*}
			\begin{equation*}
			= (\alpha_0 + \beta_0,  \alpha_0 + \alpha_1 + \beta_0 + \beta_1) =
			T(\alpha_0, \alpha_1) + T(\beta_0, \beta_1) =
			T(\overline\alpha) + T(\overline\beta)
			\end{equation*}
			
			\indstep Предположим, что утверждение верно для векторов размерности $2^n$.
			Докажем его для $\overline\alpha = (\overline\alpha_0, \overline\alpha_1)$,
			$\overline\beta = (\overline\beta_0, \overline\beta_1)$, где
			$\overline\alpha_0, \overline\alpha_1, \overline\beta_0, \overline\beta_1$~--- векторы размерности $2^n$, тогда
			\begin{equation*}
			T(\overline\alpha + \overline\beta) =
			T((\overline\alpha_0, \overline\alpha_1) + (\overline\beta_0, \overline\beta_1)) =
			T((\overline\alpha_0 + \overline\beta_0, \overline\alpha_1 + \overline\beta_1)) =
			\end{equation*}
			\begin{equation*}
			= (T(\overline\alpha_0 + \overline\beta_0), T(\overline\alpha_0 + \overline\beta_0) + T(\overline\alpha_1 + \overline\beta_1)) =
			\end{equation*}
			\begin{equation*}
			= (T(\overline\alpha_0) + T(\overline\beta_0),  T(\overline\alpha_0) + T(\overline\beta_0) + T(\overline\alpha_1) + T(\overline\beta_1)) =
			\end{equation*}
			\begin{equation*}
			= (T(\overline\alpha_0), T(\overline\alpha_0) + T(\overline\alpha_1)) + (T(\overline\beta_0), T(\overline\beta_0) + T(\overline\beta_1)) =
			\end{equation*}
			\begin{equation*}
			= T(\overline\alpha_0, \overline\alpha_1) + T(\overline\beta_0, \overline\beta_1) =
			T(\overline\alpha) + T(\overline\beta)
			\end{equation*}
			\indend
		\end{proofmathind}
		
		\item $T(T(\overline\alpha)) = \overline\alpha$
		\begin{proofmathind}
			\indbase $n = 1$.
			Пусть $\overline\alpha = (\alpha_0, \alpha_1)$, тогда
			\begin{equation*}
			T(T(\overline\alpha)) =
			T(T((\alpha_0, \alpha_1))) =
			T((\alpha_0, \alpha_0 + \alpha_1)) =
			(\alpha_0, \alpha_0 + \alpha_0 + \alpha_1) =
			(\alpha_0, \alpha_1) =
			\overline\alpha
			\end{equation*}
			
			\indstep Предположим, что утверждение верно для векторов размерности $2^n$.
			Докажем его для $\overline\alpha = (\overline\alpha_0, \overline\alpha_1)$, где
			$\overline\alpha_0, \overline\alpha_1$~--- векторы размерности $2^n$, тогда
			\begin{equation*}
			T(T(\overline\alpha)) =
			T(T((\overline\alpha_0, \overline\alpha_1))) =
			T((T(\overline\alpha_0), T(\overline\alpha_0) + T(\overline\alpha_1))) =
			\end{equation*}
			\begin{equation*}
			= (T(T(\overline\alpha_0)), T(T(\overline\alpha_0)) + T(T(\overline\alpha_0) + T(\overline\alpha_1))) =
			\end{equation*}
			\begin{equation*}
			= (T(T(\overline\alpha_0)), T(T(\overline\alpha_0)) + T(T(\overline\alpha_0)) + T(T(\overline\alpha_1))) =
			\end{equation*}
			\begin{equation*}
			= (T(T(\overline\alpha_0)), T(T(\overline\alpha_1))) =
			(\overline\alpha_0, \overline\alpha_1) =
			\overline\alpha
			\end{equation*}
			\indend
		\end{proofmathind}
	\end{enumerate}
	
	Если $\overline c_f$~--- вектор коэффициентов многочлена Жегалкина, соответствующего булевой функции~$f$, $\overline\alpha_f$~--- вектор значений $f$, то $T(\overline\alpha_f) = \overline c_f$, $T(\overline c_f) = \overline\alpha_f$.
	\begin{proof}
	Докажем методом математической индукции, что $T(\overline\alpha_f) = \overline c_f$.
		\indbase $n = 1$.
		Пусть $\overline\alpha_f = (a, b)$.
		Найдём $\overline c_f$ методом неопределённых коэффициентов:
		\begin{equation*}
		\begin{cases}
		f(0) = c_0 \\
		f(1) = c_0 + c_1
		\end{cases}
		\Leftrightarrow
		\begin{cases}
		c_0 = a \\
		c_0 + c_1 = b
		\end{cases}
		\Leftrightarrow
		\begin{cases}
		c_0 = a \\
		c_1 = a + b
		\end{cases}
		\Rightarrow
		\overline c_f = (a, a + b) = T(\overline\alpha_f)
		\end{equation*}
		
		\indstep Предположим, что утверждение верно для вектора значений размерности $2^n$.
		Докажем его для размерности $2^{n+1}$.
		\begin{equation*}
		f(x_1, \ldots, x_{n+1}) =
		\sum_{i=0}^{2^{n+1} - 1} c_i K_i =
		\sum_{i=0}^{2^n - 1} c_i K_i + \sum_{i=2^n}^{2^{n+1} - 1} c_i K_i
		\end{equation*}
		
		$x_1$ не входит ни в один из конъюнктов $K_0, K_1, \ldots, K_{2^n - 1}$ и входит в каждый из $K_{2^n}, K_{2^n + 1}, \ldots, K_{2^{n+1} - 1}$.
		\begin{enumerate}
			\item Пусть $x_1 = 0$.
			\begin{equation*}
			f_0(x_2, \ldots, x_{n+1}) = \sum_{i=0}^{2^n - 1} c_i K_i \Rightarrow
			T(\overline\alpha_{f_0}) = \overline c_{f_0} = (c_0, c_1, \ldots, c_{2^n - 1})
			\end{equation*}
			
			\item Пусть $x_1 = 1$.
			\begin{equation*}
			f_1(x_2, \ldots, x_{n+1}) =
			f(1, x_2, \ldots, x_{n+1}) =			
			\sum_{i=0}^{2^n - 1} c_i K_i + x_1 \sum_{i=2^n}^{2^{n+1} - 1} c_i K_i' =
			\sum_{i=0}^{2^n - 1} (c_i + c_{2^n + i}) K_i
			\end{equation*}
			т.\,к. $K_0 = 1$ и $K_{2^n} = x_1$, $K_1 = x_n$ и $K_{2^n + 1} = x_1 x_n$, \ldots, т.\,е. $K_{2^n + i}' = K_i$, тогда $T(\overline\alpha_{f_1}) \opbr= \overline c_{f_1} \opbr= (c_0 + c_{2^n}, c_1 + c_{2^n + 1}, \ldots, c_{2^n - 1} + c_{2^{n+1} - 1})$.
		\end{enumerate}
		
		Т.\,о., получим
		\begin{equation*}
		T(\overline\alpha_f) =
		T((\overline\alpha_{f_0}, \overline\alpha_{f_1})) =
		(T(\overline\alpha_{f_0}), T(\overline\alpha_{f_0}) + T(\overline\alpha_{f_1})) =
		\end{equation*}
		\begin{equation*}
		= ((c_0, \ldots, c_{2^n - 1}), (c_0 + c_0 + c_{2^n}, c_1 + c_1 + c_{2^n + 1}, \ldots, c_{2^n - 1} + c_{2^n - 1} + c_{2^{n+1} - 1})) =
		\end{equation*}
		\begin{equation*}
		= ((c_0, \ldots, c_{2^n - 1}), (c_{2^n}, \ldots, c_{2^{n+1} - 1})) =
		\overline c_f
		\end{equation*}
		\indend
		
	Тогда $T(\overline c_f) = T(T(\overline\alpha_f)) = \overline\alpha_f$.
	\end{proof}
\end{enumerate}
\section{Замкнутые классы булевых функций}
\index{Теорема!Поста}
\begin{theorem}[Поста о функциональной полноте]
Класс~$F$ булевых функций полон $\Leftrightarrow$ он не лежит целиком ни в одном из классов $T_0, T_1, S, M, L$.
\end{theorem}
\begin{proof}
\begin{enumerate}
	\item $\Rightarrow$. Докажем методом от противного.
	Пусть среди классов $T_0, T_1, S, M, L$ найдётся класс~$K \colon F \subseteq K$, тогда $[F] \subseteq [K] = K \neq P_2$, значит, $F$ не является полным.
	Противоречие.
	
	\item $\Leftarrow$.
	\begin{equation*}
	\exists f_0, f_1, f_S, f_M, f_L \in F \colon
	f_0 \notin T_0 \lAnd f_1 \notin T_1 \lAnd f_S \notin S \lAnd f_M \notin M \lAnd f_L \notin L
	\end{equation*}
	\begin{enumerate}
		\item Покажем, что $0$ и $1$ реализуются формулой над~$\{ f_0, f_1, f_S \}$.
		Пусть $\varphi(x) = f_0(x, \ldots, x) \Rightarrow \varphi(0) = 1$.
		\begin{enumerate}
			\item Пусть $\varphi(1) = 0 \Rightarrow \varphi(x) = \overline x$.
			Подставляя $x, \overline x$ в~$f_S$, получим одну из констант.
			Другую константу можно выразить через полученную константу и~$\overline x$.
			\item Пусть $\varphi(1) = 1 \Rightarrow \varphi(x) = 1 \Rightarrow
			f_1(\varphi(x), \ldots, \varphi(x)) = f(1, \ldots, 1) = 0$.
		\end{enumerate}
		
		\item Подставляя $0, 1, x$ в~$f_M$, получим $\overline x$.
		
		\item Подставляя $0, 1, x, \overline x, y, \overline y$ в~$f_L$ и, возможно, изменяя её значение на противоположное, получим~$x \land y$.
	\end{enumerate}
	
	Т.\,о., функции из полного набора~$\{ \neg, \land \}$ реализуются формулами над~$\{ f_0, f_1, f_S, f_M, f_L \} \subseteq F$, значит, $F$~--- полный набор.
\end{enumerate}
\end{proof}

\begin{statement}
$T_0, T_1, S, M, L$ попарно различны.
\end{statement}
\begin{proof}
Составим таблицу, в которой $+$ означает принадлежность функции классу, а $-$ означает её отсутствие в классе.
\begin{equation*}
\begin{array}{|c|c|c|c|c|c|}
\hline
  & T_0 & T_1 & S & M & L \\
  \hline
0 & + & - & - & + & + \\
\hline
1 & - & + & - & + & + \\
\hline
\overline x & - & - & + & - & + \\
\hline
\end{array}
\end{equation*}
\end{proof}

Класс~$F$ булевых функций называется \textbf{предполным}, если $[F] \neq P_2$ и $\forall f \notin F \colon [F \cup \{ f \}] = P_2$.

\begin{statement}
Существует ровно 5 предполных классов булевых функций: $T_0, T_1, S, M, L$.
\end{statement}
\begin{proof}
\begin{enumerate}
	\item $\Rightarrow$. Пусть $K \in \{ T_0, T_1, S, M, L \}$, $f \notin K$.
	$K \cup \{ f \}$ не лежит целиком ни в одном из классов $T_0, T_1, S, M, L$, значит, $[K \cup \{ f \}] = P_2$, т.\,е. $K$~--- предполный класс.
	\item $\Leftarrow$. Пусть $K$~--- предполный класс $\Rightarrow$ $K$ не является полным $\Rightarrow$ $K$ лежит в одном из классов $T_0, T_1, S, M, L$.
	
	Докажем методом от противного, что $K$ равен одному из них.
	Пусть $K_1 \in \{ T_0, T_1, S, M, L \}$, $K \subset K_1$.
	$f \notin K \opbr\lAnd f \in K_1 \opbr\Rightarrow K \cup \{ f \} \subseteq K_1 \neq P_2$, значит, $K$ не является предполным классом.
	Противоречие.
\end{enumerate}
\end{proof}

\begin{consequent}
Любой замкнутый класс булевых функций $F \neq P_2$ целиком лежит в одном из классов $T_0, T_1, S, M, L$.
\end{consequent}
\begin{proofcontra}
Пусть $F$ не лежит ни в одном из классов $T_0, T_1, S, M, L$, тогда $F = [F] = P_2$.
Противоречие.
\end{proofcontra}

Пусть $F$~--- замкнутый набор булевых функций. Набор~$M \subseteq F$ называется \textbf{полным в~$F$}, если $[M] = F$.

\index{Базис!класса булевых функций} Набор~$K$ булевых функций называется \textbf{базисом замкнутого класса~$F$ булевых функций}, если $K$ полон в~$F$ и при удалении из него любой булевой функции он перестаёт быть полным.
\chapter{Теория графов}
\section{Неориентированные графы}
\index{Графы!неориентированные} \textbf{Неориентированным графом} называется пара $(V, E)$, где $V$~--- непустое конечное множество вершин графа, $E$~--- совокупность множеств $\{ u, v \}$, где $u, v \in V$.
Элементы~$V$ называются \textbf{вершинами графа}.
Элементы~$E$ называются \textbf{рёбрами графа}.

В~этом разделе будем рассматривать только неориентированные графы.

На рисунках вершины графа изображают точками, а рёбра $e = \{ u, v \}$~--- кривыми, соединяющими точки, которые изображают вершины $u$ и $v$.

Если $e = \{ u, v \} \in E$, то говорят, что:
\begin{itemize}
	\item ребро~$e$ соединяет вершины $u$ и $v$;
	\item $u$ и $v$~--- концы ребра~$e$;
	\item ребро~$e$ инцидентно вершинам $u$ и $v$;
	\item вершины $u$ и $v$ инцидентны ребру~$e$.
\end{itemize}

Вершины называются \textbf{соседними}, или \textbf{смежными}, если их соединяет ребро, иначе~--- \textbf{несоседними}, или \textbf{несмежными}.

\index{deg} Число рёбер, инцидентных вершине~$u$, называется \textbf{степенью вершины} и обозначается $\deg u$.

Если степень вершины равна $0$, то она называется \textbf{изолированной}, а если $1$~--- то \textbf{висячей}.

\begin{lemma}[о~рукопожатиях]
\index{Лемма!о~рукопожатиях}
\begin{equation*}
\sum_{u \in V} \deg u = 2|E|
\end{equation*}
где $(V, E)$~--- граф.
\end{lemma}
\begin{proof}
Достаточно заметить, что каждое ребро увеличивает степени двух некоторых вершин на~$1$.
\end{proof}

\begin{wrapfigure}{r}{0pt}
\noindent
\begin{tikzpicture}[scale=1.5]
\foreach \i in {0, ..., 4}
	\draw (72*\i + 90:1) coordinate (x\i) node {$\bullet$};
\foreach \i in {0, ..., 3}
	\foreach \j in {\i, ..., 4}
		\draw (x\i) -- (x\j);
\end{tikzpicture}
\caption{Граф $K_5$}
\end{wrapfigure}

\index{Петля} Ребро вида $e = \{ u, u \}$ называется \textbf{петлёй}.

Рёбра, инцидентные одним и тем~же вершинам, называются \textbf{кратными}.

Граф называется \textbf{простым}, если он не содержит петель и кратных рёбер.

Граф, в~котором любые две вершины соединены ребром, называется \textbf{полным} и обозначается $K_n$, где $n$~--- число вершин в~нём.

Графы $G_1 = (V_1, E_1)$ и $G_2 = (V_2, E_2)$ называются \textbf{изоморфными}, если существует биекция~$\varphi \colon V_1 \to V_2$ такая, что
$\{ u, v \} \in E_1 \Leftrightarrow \{ \varphi(u), \varphi(v) \} \in E_2$, иначе~--- \textbf{неизоморфными}.
$\varphi$ называется \textbf{изоморфизмом}.

\index{Маршрут} \textbf{Маршрутом} в графе называется последовательность вершин и рёбер вида
$(v_1, e_1, v_2, \ldots, \allowbreak e_k, v_{k+1})$, где $e_i \opbr= \{ v_i, v_{i+1} \}$.

\index{Цепь} Маршрут, в~котором все рёбра различны, называется \textbf{цепью}.

Цепь, в~которой все вершины, за исключением, может быть, первой и последней, различны, называется \textbf{простой}.

Маршрут, в~котором первая и последняя вершины совпадают, называется \textbf{замкнутым}.

\index{Цикл} Замкнутая цепь называется \textbf{циклом}.

Маршрут, соединяющий вершины $u$ и $v$, называется \textbf{$(u, v)$-маршрутом}.

\begin{lemma}
\label{lemma:walk_contains_simple_chain}
$(u, v)$-маршрут содержит $(u, v)$-простую цепь.
\end{lemma}
\begin{proof}
Пусть $(u = v_1, e_1, v_2, \ldots, e_k, v_{k+1} = v)$~--- не простая цепь, тогда $\exists i < j \colon v_i = v_j$.
Уберём из маршрута подпоследовательность $(e_i, v_{i+1}, \ldots, e_{j-1}, v_j)$ и получим маршрут, в~котором совпадающих вершин на одну меньше.
Повторяя, получим простую цепь, являющуюся частью данного маршрута.
\end{proof}

\begin{lemma}
\label{lemma:cycle_contains_simple_one}
Любой цикл содержит простой цикл, причём каждая вершина и ребро цикла принадлежат некоторому простому циклу.
\end{lemma}
\begin{proof}
Пусть $(u = v_1, e_1, v_2, \ldots, e_k, v_{k+1} = u)$~--- не простой цикл, тогда $\exists i < j \colon v_i = v_j$.
Уберём из цикла подпоследовательность $(e_i, v_{i+1}, \ldots, e_{j-1}, v_j)$ и получим цикл, в~котором совпадающих вершин на одну меньше.
Повторяя, получим простой цикл, являющийся частью данного цикла.

Заметим, что подпоследовательность $(v_i, e_i, v_{i+1}, \ldots, e_{j-1}, v_j = v_i)$ также является циклом, из которого можно получить простой цикл.
Значит, некоторые вершины и рёбра этой подпоследовательности принадлежат простому циклу, остальные же снова образуют некоторые циклы, из которых можно получить простые.
Продолжая рассуждать таким образом, приходим к выводу, что любая вершина и ребро исходного цикла принадлежат некоторому простому циклу.
\end{proof}
	
\begin{lemma}
\label{lemma:existence_of_simple_cycle}
Если в~графе есть две различные простые цепи, соединяющие одни и те~же вершины, то в~этом графе есть простой цикл.
\end{lemma}
\begin{proof}
Пусть $(u = v_1, e_1, v_2, \ldots, e_n, v_{n+1} = v)$, $(u = v_1', e_1', v_2', \ldots, e_m', v_{m+1}' = v)$~--- простые цепи.
Найдём наименьшее~$i \colon e_i \neq e_i'$, тогда $(v_i, e_i, v_{i+1}, \ldots, e_n, v_{n+1} = v_{m+1}', e_m', \ldots, e_i', v_i' = v_i)$~--- цикл, значит, можно получить простой цикл.
\end{proof}

\subsection{Связность неориентированных графов}
Вершины $u$ и $v$ называются \textbf{связанными}, если существует $(u, v)$-маршрут, иначе~--- \textbf{несвязанными}.

\index{Графы!связные} Граф называется \textbf{связным}, если в~нём любые две вершины связаны, иначе~--- \textbf{несвязным}.

Граф~$G' = (V', E')$ называется \textbf{подграфом графа~$G = (V, E)$}, если $V' \subseteq V \lAnd E' \subseteq E$.

\index{Компонента связности} \textbf{Компонентой связности графа} называется его максимальный относительно включения связный подграф.

\subsection{Эйлеровы графы}
Цикл, содержащий все рёбра графа, называется \textbf{эйлеровым}.

\index{Графы!эйлеровы} Граф, содержащий эйлеров цикл, называется \textbf{эйлеровым}.

\begin{theorem}
Связный граф эйлеров $\Leftrightarrow$ степени всех вершин чётны.
\end{theorem}
\begin{proof}
\begin{enumerate}
	\item $\Rightarrow$. Пусть в~графе есть эйлеров цикл.
	Выберем вершину~$v_0$ в~этом цикле и начнём обходить его.
	При каждом посещении вершины~$v \neq v_0$ её степень увеличивается на~$2$.
	Т.\,о., если посетить её $k$~раз, то $\deg v = 2k \mult 2$.
	
	Для $v_0$ степень увеличивается на~$1$ в~начале обхода, на~$1$ в~конце обхода и на~$2$ при промежуточных посещениях.
	Т.\,о., её степень чётна.
	
	\item $\Leftarrow$. Пусть степени всех вершин чётны.
	Выберём цепь~$C = (v_0, e_0, v_1, e_1, \ldots, e_{k-1}, v_k)$ наибольшей длины.
	Все рёбра, инцидентные~$v_k$, присутствуют в~этой цепи, иначе её можно было~бы удлинить.
	
	Докажем методом от противного, что $v_0 = v_k$.
	Пусть $v_0 \neq v_k$.
	При прохождении вершины~$v_i = v_k$, $i = 1, 2, \ldots, k - 1$, степень~$v_k$ увеличивается на~$2$.
	Также проходим по ребру~$e_{k-1}$, тогда степень~$v_k$ нечётна.
	Противоречие.
	
	Докажем методом от противного, что $C$ содержит все рёбра.	
	Пусть найдётся ребро~$e = \{ u, v \}$, не входящее в~$C$.
	Возьмём первое ребро~$e' = \{ v_i, v' \}$ из $(v_0, u)$-маршрута, не входящее в~$C$.
	Тогда цепь~$(v', e', v_i, e_i, \ldots, e_{k-1}, \allowbreak v_k = v_0, \allowbreak e_0, v_1, e_1, \ldots, v_{i-1})$ длиннее, чем~$C$.
	Противоречие.
\end{enumerate}
\end{proof}

\subsubsection{Алгоритмы нахождения эйлерова цикла}
\begin{enumerate}
	\item \index{Алгоритм!Флёри} \textbf{Алгоритм Флёри}
	
	В качестве текущей вершины выберем произвольную.
	\begin{enumerate}
		\item Выбираем ребро, инцидентное текущей вершине.
		Оно не должно быть мостом, если есть другие рёбра, не являющиеся мостами.
		\item Проходим по выбранному ребру и вычёркиваем его.
		Вершина, в~которой теперь находимся,~--- текущая.
		\item Повторяем с шага~(a), пока есть рёбра.
	\end{enumerate}
	
	\item \index{Алгоритм!объединения циклов} \textbf{Алгоритм объединения циклов}
	\begin{enumerate}
		\item Выбираем произвольную вершину.
		\item Выбираем любое непосещённое ребро и идём по нему.
		\item Повторяем шаг~(b), пока не вернёмся в~начальную вершину.
		\item Получили цикл~$C$.
		Если он не эйлеров, то $\exists u \in C, \ e = \{ u, u' \} \colon u' \notin C$.
		Повторяем шаги~(b)--(c), начиная с вершины~$u$.
		Получили цикл~$C'$, рёбра которого не совпадают с рёбрами~$C$.
		Объединим эти циклы и получим новый.
		Повторяем шаг~(d).
	\end{enumerate}
\end{enumerate}

Цепь называется \textbf{эйлеровым путём}, если она не является циклом и содержит все рёбра графа.

\index{Графы!полуэйлеровы} Граф называется \textbf{полуэйлеровым}, если в~нём есть эйлеров путь.

\begin{theorem}
Связный граф полуэйлеров $\Leftrightarrow$ степени двух вершин нечётны, а остальных~--- чётны.
\end{theorem}
\begin{proof}
\begin{enumerate}
	\item $\Rightarrow$. Пусть в~графе есть эйлеров путь.
	Соединив его концы ребром, получим эйлеров цикл.
	Степени соединённых вершин увеличились каждая на~$1$, значит, они были нечётными, а степени остальных вершин~--- чётными.
	\item $\Leftarrow$. Пусть степени двух вершин нечётны, а остальных~--- чётны.
	Соединим нечётные вершины ребром, тогда можно получить эйлеров цикл.
	Убрав из него добавленное ребро, получим эйлеров путь.
\end{enumerate}
\end{proof}

\subsection{Гамильтоновы графы}
Простой цикл, содержащий все вершины графа, называется \textbf{гамильтоновым}.

\index{Графы!гамильтоновы} Граф называется \textbf{гамильтоновым}, если в~нём есть гамильтонов цикл.

\index{Теорема!Оре}
\begin{theorem}[Оре]
Если в~графе с $n \geqslant 3$~вершинами для любых двух несмежных вершин $u$ и $v$ $\deg u \opbr+ \deg v \opbr\geqslant n$, то граф гамильтонов.
\end{theorem}
\begin{proof}
\begin{enumerate}
	\item Докажем методом от противного, что граф связный.
	Пусть он несвязный, тогда в~нём найдутся хотя~бы две компоненты связности $G_1(V_1, E_1)$ и $G_2(V_2, E_2)$.
	Пусть $u \in V_1$, $v \in V_2$.
	$u$ и $v$ несмежные, тогда
	\begin{equation*}
	\deg u \leqslant |V_1| - 1, \ \deg v \leqslant |V_2| - 1 \Rightarrow \deg u + \deg v \leqslant |V_1| + |V_2| - 2 \leqslant n - 2
	\end{equation*}
	
	Противоречие с условием.
	
	\item Докажем, что граф гамильтонов.
	Выберем цепь~$W = (v_0, e_0, v_1, \ldots, e_{k-1}, v_k)$ наибольшей длины.
	В~ней содержатся все вершины, соседние с~$v_0$ или с~$v_k$.
	Т.\,о., среди вершин $v_1, \ldots, v_k$ находится ровно $\deg v_0$ соседних с~$v_0$ вершин.
	Аналогично для $v_k$.
	
	$\deg v_0 + \deg v_k \geqslant n$, тогда найдутся $v_i$ и $v_{i+1}$ такие, что $v_i$ соседняя с~$v_k$, а $v_{i+1}$~--- с~$v_0$.
	
	Докажем, что $(v_{i+1}, e_{i+1}, \ldots, v_k, e, v_i, e_{i-1}, v_{i-1}, \ldots, e_0, v_0, e', v_{i+1})$~--- гамильтонов цикл, методом от противного.
	Предположим обратное, тогда есть вершина~$u$, не входящая в~цикл, и существует $(v_0, u)$-маршрут.
	Значит, существует ребро, инцидентное одной из вершин цикла, но не входящее в~него, и можно получить более длинную цепь.
	Противоречие, значит, $G$~--- гамильтонов граф.
\end{enumerate}
\end{proof}

\index{Теорема!Дирака}
\begin{theorem}[Дирака]
\label{th:Dirac}
Если в графе~$G = (V, E)$ с $n \geqslant 3$~вершинами $\forall u \in V \ \deg u \geqslant \frac{n}2$, то граф гамильтонов.
\end{theorem}
\begin{proof}
Пусть $u$, $v$~--- несвязные вершины в~$G$, тогда $\deg u \geqslant \frac{n}2 \lAnd \deg v \geqslant \frac{n}2 \Rightarrow \deg u + \deg v \geqslant n$ $\Rightarrow$ по теореме Оре $G$ гамильтонов.
\end{proof}

Цепь называется \textbf{гамильтоновым путём}, если она не является циклом и содержит все вершины графа.

\index{Графы!полугамильтоновы} Граф называется \textbf{полугамильтоновым}, если в нём есть гамильтонов путь.

\subsection{Планарность графов}
\index{Графы!плоские} \textbf{Плоским} называется граф~$G = (V, E)$ такой, что:
\begin{itemize}
	\item $V \subset \mathbb R^2$;
	\item рёбра~--- кривые, концами которых являются вершины;
	\item различные рёбра не имеют общих точек, за исключением концов.
\end{itemize}

\index{Графы!планарные} \textbf{Планарным} называется граф, изоморфный плоскому.

Если $G$~--- граф и $G'$~--- плоский граф, изоморфный $G$, то $G'$ называется \textbf{укладкой $G$} в~$\mathbb R^2$.

Аналогично можно определить плоский граф в~$\mathbb R^3$, на~сфере и~т.\,д.

\begin{theorem}
Любой граф можно уложить в~$\mathbb R^3$.
\end{theorem}
\begin{proof}
Пусть $G = (V, E)$~--- граф, $V = \{ (1, 0, 0), (2, 0, 0), \ldots, (n, 0, 0) \}$.
Рассмотрим плоскости, проходящие через~$Ox$ и образующие с плоскостью~$Oxy$ углы
$\dfrac\pi2, \dfrac\pi{2\cdot2}, \ldots, \dfrac\pi{2m}$, где $m = |E|$.
В каждой такой плоскости можно провести ровно одно ребро, тогда получим плоский граф, т.\,к. плоскости пересекаются только по прямой~$Ox$.
\end{proof}

\begin{theorem}
Граф укладывается на~плоскость $\Leftrightarrow$ он укладывается на~сферу.
\end{theorem}
\begin{proof}
Пусть плоскость~$z = 0$ касается сферы в точке~$O(0, 0, 0)$, $N$~--- точка на~сфере, диаметрально противоположная точке~$O$.
Для каждой точки сферы, не совпадающей с~$N$, проведём прямую через неё и точку~$N$, которая пересечёт сферу и плоскость, причём любые две из таких прямых имеют единственную общую точку~$N$.
Получим биекцию между точками сферы и точками плоскости, тогда можно построить биекцию между укладками на сфере и укладками на плоскости.
\end{proof}

\begin{center}
\noindent
\shorthandoff{"}
\begin{tikzpicture}[>=stealth]
% рисуем оси
\def\tilt_angle{60}
\draw[->] (0, 0) coordinate["$O$" {below right}] (O) node {$\bullet$}
	+(\tilt_angle:3.5) -- +(\tilt_angle-180:2.5) node[below right] {$x$};
\draw[->] (-4, 0) -- (4.5, 0) node[below] {$y$};
\draw[->] (0, -1.3) -- (0, 4) node[left] {$z$};

% рисуем сферу
\def\radius{1.5}
\draw[name path=sphere] (0, \radius) circle (\radius);

% рисуем плоскость
\def\lenAB{4}
\draw (-4, -1.5) coordinate (A) -- ++(\tilt_angle:\lenAB) coordinate (B);
\draw (A) -- (2, -1.5) coordinate (D) -- ++(\tilt_angle:\lenAB) coordinate (C);
\path[name path=BC] (B) -- (C);
\draw[name intersections={of=BC and sphere, name=i}]
	(B) -- (i-2) (i-1) -- (C);
\draw[dashed] (i-1) -- (i-2);

% рисуем прямую
\draw[dashed] (0, 2*\radius) node[above right] {$N$} node {$\bullet$} -- (-0.6, 0.75) coordinate (point) node {$\bullet$};
\draw (point) -- (-1.15, -1) node {$\bullet$};
\end{tikzpicture}
\shorthandon{"}
\end{center}

Множество на плоскости называется \textbf{линейно связным}, если любые две точки этого множества можно соединить кривой, целиком лежащей в~этом множестве.

\index{Грань} \textbf{Гранью плоского графа~$G = (V, E)$} называется часть множества~$\mathbb R^2 \setminus G$, которая линейно связна и не является подмножеством другого линейно связного множества.

\begin{theorem}[формула Эйлера]
\index{Формула!Эйлера!в~теории графов}
В~плоском связном графе $n - m + f = 2$, где $n, m, f$~--- число вершин, рёбер и граней соответственно.
\end{theorem}
\begin{proof}
Рассмотрим остов данного графа.
В~нём $n$~вершин, $n - 1$~рёбер и $1$~грань.
Формула Эйлера верна для него: $n - (n - 1) + 1 = 2$.

Добавим $1$~ребро данного графа, тогда оно разобьёт одну грань на две, т.\,е. число граней увеличится на~$1$.
Формула Эйлера верна для полученного графа.
Повторяя $m - (n - 1)$~раз, получим исходный граф, для которого формула Эйлера верна.
\end{proof}

\begin{theorem}
\label{th:property_of_planarity_of_graph}
Пусть $G$~--- планарный граф с $n \geqslant 3$~вершинами и $m$~рёбрами. Тогда $m \leqslant 3n - 6$.
\end{theorem}
\begin{proof}
При $m = 2$ неравенство выполняется.

Пусть в~графе $f$~граней, $m_i$~--- число рёбер в~границе $i$-й грани.
Тогда $m_i \geqslant 3$, $\displaystyle \sum_{i=1}^f m_i \geqslant 3f$.
С~другой стороны, $\displaystyle \sum_{i=1}^f m_i \leqslant 2m$, т.\,к. каждое ребро является границей для не более чем $2$ граней.
По формуле Эйлера $n - m + f = 2 \Leftrightarrow f = m + 2 - n$.
Получим:
\begin{equation*}
2m \geqslant 3f \Leftrightarrow 2m \geqslant 3m + 6 - 3n \Leftrightarrow m \leqslant 3n - 6
\end{equation*}
\end{proof}

\begin{consequent}
Планарный граф~$G = (V, E)$ содержит хотя~бы одну вершину со~степенью, не большей~$5$.
\end{consequent}
\begin{proofcontra}
Пусть $\forall v \in V \ \deg v \geqslant 6$, $|V| = n$, $|E| = m$, тогда
$\displaystyle m \opbr= \frac12 \sum_{v \in V} \deg v \opbr\geqslant 3n$.
Имеем:
\begin{equation*}
3n \leqslant m \leqslant 3n - 6 \Rightarrow 0 \leqslant -6
\end{equation*}

Противоречие.
\end{proofcontra}

\begin{theorem}
Графы $K_5$ и $K_{3,3}$ не планарные.
\end{theorem}
\begin{proof}
\begin{itemize}
	\item Рассмотрим $K_5$.
	Для него $n = 5$, $m = 10$.
	Тогда $m \leqslant 3n - 6 \Leftrightarrow 10 \leqslant 9$.	
	Неверно, значит, $K_5$ не планарен.
	\item Рассмотрим $K_{3,3}$.
	Пусть он планарный.
	В~нём самый короткий цикл имеет длину~$4$.
	Тогда рассуждениями, аналогичными рассуждениям при доказательстве теоремы~\ref*{th:property_of_planarity_of_graph}, получим
	\begin{equation*}
	2m \geqslant 4f \Leftrightarrow 2m \geqslant 4m + 8 - 4n \Leftrightarrow m \leqslant 2n - 4
	\end{equation*}
	
	Для $K_{3,3}$ $n = 6$, $m = 9$, тогда $9 \leqslant 8$.
	Неверно, значит, $K_{3,3}$ не планарен.
\end{itemize}
\end{proof}

Граф~$G' = (V', E')$ получается \textbf{подразбиением ребра~$e = \{ u, v \}$} графа~$G = (V, E)$, если:
\begin{itemize}
	\item $V' = V \cup \{ u' \}$;
	\item $E' = (E \setminus \{ e \}) \cup \{ \{ u, u' \}, \{ v, u' \} \}$.
\end{itemize}

\index{Графы!гомеоморфные} Графы $G$ и $G'$ \textbf{гомеоморфны}, если они изоморфны графам, получающимся подразбиениями рёбер одного и того~же графа.

\index{Теорема!Понтрягина---Куратовского}
\begin{theorem}[Понтрягина---Куратовского]
Граф~$G$ планарен $\Leftrightarrow$ он не содержит подграфов, гомеоморфных $K_5$ или $K_{3,3}$.
\end{theorem}
\begin{proof}
\begin{enumerate}
	\item $\Rightarrow$. Очевидно, что подграф планарного графа планарен.
	Если $G$~--- планарный граф, содержащий подграф $G'$, гомеоморфный $K_5$ или $K_{3,3}$, то $G'$ тоже планарный, значит, $K_5$ или $K_{3,3}$ планарен, т.\,к. подразбиение ребёр не влияет на планарность.
	Противоречие, значит, $G$ не планарен.
	\item $\Leftarrow$. Доказательство слишком сложно, поэтому здесь не приводится.
\end{enumerate}
\end{proof}

\subsection{Деревья}
\index{Лес} Граф без циклов называется \textbf{лесом}.

\index{Дерево} Связный лес называется \textbf{деревом}.

\index{Мост} Ребро называется \textbf{мостом}, если при его удалении увеличивается число компонент связности.

\begin{statement}
\label{st:criterion_of_bridge_in_graph}
Ребро~--- мост $\Leftrightarrow$ оно не содержится в~цикле.
\end{statement}
\begin{proof}
\begin{enumerate}
	\item $\Leftarrow$.
	Пусть ребро $e$ содержится в цикле $W = (v_0, e_0, \ldots, u, e, v, \ldots, v_k)$, $u'$ и $v'$~--- связные вершины.
	\begin{enumerate}
		\item Если в~$(u', v')$-маршруте нет ребра~$e$, то при его удалении из графа $u'$ и $v'$ останутся связными.
		\item Пусть $(u' = v_0', e_0', \ldots, u, e, v, \ldots, e_m', v_m' = v')$~--- маршрут, соединяющий $u'$ и $v'$, тогда при удалении $e$ из графа $u'$ и $v'$ соединяет маршрут
		$(u' = v_0', e_0', \ldots, u, \ldots, e_0, v_0 = v_k, e_{k-1}, \ldots, v, \ldots, e_m', v_m' = v')$.
	\end{enumerate}
	
	\item $\Rightarrow$.
	Пусть $e = \{ u, v \}$ не является мостом, тогда $u$, $v$ лежат в~одной компоненте связности.
	Удалим $e$ из графа.
	Число компонент связности не изменится, значит, $u$ и $v$ также лежат в~одной компоненте связности, т.\,е. существует цепь, соединяющая $u$ и $v$: $(u = v_0, e_0, \ldots, e_{k-1}, v_k = v)$.
	Тогда в~исходном графе существует цикл $(u = v_0, e_0, \ldots, e_{k-1}, v_k = v, e, u)$.
\end{enumerate}
\end{proof}

\begin{theorem}
Следующие утверждения о графе~$G = (V, E)$ с~$n$ вершинами эквивалентны:
\begin{enumerate}[(1)]
	\item $G$~--- дерево.
	\item $G$ связный и каждое его ребро~--- мост.
	\item $G$ связный и имеет $n - 1$~ребро.
	\item $G$ не содержит циклов и имеет $n - 1$~ребро.
	\item Любые две вершины графа~$G$ соединены ровно одной простой цепью.
	\item $G$ не содержит циклов и добавление ребра приводит к появлению ровно одного цикла.
\end{enumerate}
\end{theorem}
\begin{proof}
\begin{itemize}
	\item (1) $\Rightarrow$ (2).
	Связность следует из определения дерева.
	
	В~силу утверждения~\ref*{st:criterion_of_bridge_in_graph} каждое ребро~--- мост.
	
	\item (2) $\Rightarrow$ (3).
	Связность следует из предположения.
	
	Докажем методом математической индукции, что в~графе $n - 1$~ребро.
	\indbase Для $n = 1, 2$ очевидно.
	\indstep Пусть утверждение верно для чисел, меньших $n$.
	Возьмём мост~$e$ и удалим его.
	Получим две компоненты связности $G_1 = (V_1, E_1)$, $G_2 = (V_2, E_2)$.
	По предположению индукции $|E_1| = |V_1| - 1$, $|E_2| = |V_2| - 1$.
	Тогда в~исходном графе рёбер $|E_1| + |E_2| + 1 = |V_1| + |V_2| - 1 = n - 1$. \indend
	
	\item (3) $\Rightarrow$ (4).
	$G$ имеет $n - 1$~ребро по предположению.
	
	Докажем методом математической индукции, что $G$ не содержит циклов.
	\indbase Для $n = 1, 2$ очевидно.
	\indstep Пусть утверждение верно для чисел, меньших $n$.
	Докажем методом от противного, что в~графе есть вершина степени~$1$.
	Пусть
	\begin{equation*}
	\forall u \in V \ \deg u \geqslant 2 \Rightarrow 2|E| = \sum_{u \in V} \deg u \geqslant 2n \Rightarrow n - 1 = |E| \geqslant n \Rightarrow -1 \geqslant 0
	\end{equation*}
	Противоречие, значит, в~графе найдётся вершина степени~$1$.
	
	Удалим её и инцидентное ей ребро.
	Полученный граф содержит $n - 1$~вершину и удовлетворяет утверждению~(3).
	По предположению индукции он не содержит циклов, тогда и исходный граф не содержит циклов. \indend
	
	\item (4) $\Rightarrow$ (5).
	
	Пусть в~графе $k$~компонент связности: $G_1 = (V_1, E_1)$, $G_2 = (V_2, E_2)$, \ldots, $G_k = (V_k, E_k)$.
	Они не содержат циклов по предположению, тогда они являются деревьями.
	\begin{equation*}
	|E_1| = |V_1| - 1 \lAnd |E_2| = |V_2| - 1 \lAnd \ldots \lAnd |E_k| = |V_k| - 1 \lAnd 
	n - 1 = |E_1| + \ldots + |E_k| = n - k \Rightarrow k = 1
	\end{equation*}
	Значит, граф связный.
	
	Пусть существуют вершины $u$ и $v$ такие, что их соединяют две простые цепи, тогда по лемме~\ref{lemma:existence_of_simple_cycle} в~графе есть цикл, что противоречит предположению.
	Значит, эти вершины соединены ровно одной простой цепью.
	
	\item (5) $\Rightarrow$ (6).
	
	Докажем методом от противного, что в~графе нет циклов.
	Предположим, что есть цикл $(v_0, e_0, v_1, \ldots, v_k = v_0)$, тогда есть две простые цепи $(v_0, e_0, \ldots, v_{k-1})$ и $(v_{k-1}, e_k, v_k = v_0)$, соединяющие $v_0$ и $v_{k-1}$, что противоречит предположению.
	
	Докажем, что добавление ребра приводит к появлению ровно одного цикла.
	Рассмотрим несоседние вершины $u$ и $v$.
	По предположению есть цепь $(u = v_0, e_0, \ldots, v_k = v)$, соединяющая их.
	Тогда, добавив $e = \{ u, v \}$, получим цикл $(u = v_0, e_0, \ldots, v_k = v, e, u)$.
	
	Пусть есть $2$~цикла, соединяющих $u$ и $v$.
	Удалим $e$, тогда один цикл останется.
	Получим исходный граф, в~котором не должно быть циклов.
	Противоречие.
	
	\item (6) $\Rightarrow$ (1).
	
	Докажем связность методом от противного.
	Рассмотрим несвязные вершины $u$ и~$v$.
	Соединим их и по~предположению получим цикл $(v_0, e_0, \ldots, u, e, v, \ldots, e_{k-1}, v_k = v_0)$.
	Тогда в~исходном графе $(u, \ldots, e_0, v_0 = v_k, \allowbreak e_{k-1}, \ldots, v)$~--- $(u, v)$-маршрут.
	Противоречие.
\end{itemize}
\end{proof}

В~ходе доказательства было получено, что в~связном графе с $n$~вершинами и $n - 1$~рёбрами существует висячая вершина.
Т.\,к. доказано, что такой граф является деревом, то верно следующее утверждение.
\begin{statement}
В~дереве существует висячая вершина.
\end{statement}

\begin{statement}
Если в лесу $n$~вершин, $m$~рёбер и $k$~компонент связности, то $m = n - k$.
\end{statement}
\begin{proof}
Пусть $n_1, \ldots, n_k$~--- число вершин в каждой компоненте связности, тогда
\begin{equation*}
m = (n_1 - 1) + (n_2 - 1) + \ldots + (n_k - 1) = n - k
\end{equation*}
\end{proof}

\subsection{Остовы}
\index{Остов} \textbf{Остовом графа $G = (V, E)$} называется его подграф~$G' = (V', E') \colon V = V' \lAnd G'$~--- дерево.

\begin{statement}
Любой связный граф содержит остов.
\end{statement}

\begin{statement}
Если граф не является деревом, то в~нём несколько остовов.
\end{statement}

Пусть $G = (V, E)$~--- граф.

\index{Вес} \textbf{Весом} называется функция~$\alpha \colon E \to \mathbb R^+$.

\textbf{Весом ребра}~$e \in E$ называется $\alpha(e)$.

\textbf{Весом графа} называется $\displaystyle \sum_{e \in E} \alpha(e)$.

Пусть дан граф~$G = (V, E)$, $n = |V|$ и весовая функция $\alpha \colon E \to R^+$, и необходимо найти остов наименьшего веса $T = (V, P)$.

\subsubsection{Алгоритм Краскала}
\index{Алгоритм!Краскала}
\begin{enumerate}
	\item[1.] Выбираем ребро~$e \in E$ с наименьшим весом: $P_1 = \{ e \}$, $T_1 = (V, P_1)$.
	\item[i.] Выбираем ребро~$e \in E$ с наименьшим весом такое, что $e \notin P_i$ и добавление этого ребра не приводит к образованию цикла в~$T$: $T_{i+1} = (V, P_i \cup \{ e \})$.
\end{enumerate}

$T_n$~--- искомый остов.
\begin{proof}[корректности]
	Пусть $T = (V, P)$~--- построенный остов, где
	$P = \{ e_1, e_2, \ldots, e_{n-1} \}$, $e_1, e_2, \ldots, \allowbreak e_{n-1}$~--- рёбра в~порядке их добавления в~остов, а также $D = (V, M)$~--- другой остов, где
	$M = \{ e_1', e_2', \ldots, e_{n-1}' \}$, $e_1', e_2', \ldots, e_{n-1}'$~--- рёбра в~порядке неубывания их весов.
	
	Если $T \neq D$, то пусть $i$~--- наименьшее число такое, что $e_i \neq e_i'$.
	$e_i'$ не входит в~$T$, значит, оно образует цикл с рёбрами в~$T$, выбранными ранее, тогда вес этих рёбер не больше $\alpha(e_i')$.
	Выберем из них ребро~$e$ такое, что при добавлении его в~$D$ образуется цикл.
	Пусть $D_1 = (V, M \cup \{ e \} \setminus \{ e_i' \})$.
	Этот граф~--- остов, причём вес~$D_1$ не больше веса~$D$ и у~$T$ и $D_1$ на~$1$ общее ребро больше, чем у~$T$ и $D$.
	Повторяя, получим $D_k = T$.
	Значит, вес построенного остова не превосходит веса любого другого остова.
\end{proof}

\subsubsection{Алгоритм Прима}
\index{Алгоритм!Прима}
Строится последовательность деревьев $S_1 \subset S_2 \subset \ldots \subset S_n = T$.
\begin{enumerate}
	\item[1.] Выбираем произвольную вершину~$v$.
	$S_1 = (\{ v \}, \varnothing)$.
	\item[i.] Пусть построено $S_{i-1} = (V_{i-1}, E_{i-1})$.
	Находим ребро~$e = \{ u, v_{i-1} \} \in E$, где $u \in V_{i-1}$, $v_{i-1} \notin V_{i-1}$, наименьшего веса, добавление которого не приводит к образованию цикла: $S_i = (V_{i-1} \cup \{ v_{i-1} \}, E_{i-1} \cup \{ e \})$.
\end{enumerate}

$S_n$~--- искомый остов.

\subsection{Помеченные деревья}
\index{Дерево!помеченное} Дерево с $n$~вершинами, которым сопоставлены числа~$1, \ldots, n$, называется \textbf{помеченным}.

\index{Код Прюфера} Каждому помеченному дереву можно взаимнооднозначно сопоставить последовательность из $n - 2$~чисел от $1$ до $n$, называемую \textbf{кодом Прюфера}.
Алгоритм построения кода Прюфера для помеченного дерева~$G = (V, E)$:
\begin{enumerate}
	\item Выбираем висячую вершину~$v$ с наименьшим номером.
	\item Добавляем номер вершины, смежной с~$v$, в~код.
	\item Удаляем~$v$ и ребро, инцидентное $v$, из дерева.
	\item Повторить, начиная с шага~1, $n - 2$~раза.
\end{enumerate}

\begin{statement}
Различным помеченным деревьям соответствуют различные коды Прюфера.
\end{statement}
\begin{proofmathind}
	\indbase При $n = 3$ легко проверить.
	\indstep Пусть утверждение верно при $n$, $G = (V, E)$ и $G' = (V', E')$~--- различные помеченные деревья с $n + 1$~вершинами в~каждом.
	Если в $G$ и $G'$ вершины с наименьшим номером смежны с вершинами с одинаковыми номерами, то выполняем шаг построения кода, тогда оставшиеся деревья различны, значит, по предположению индукции у них различные коды. \indend
\end{proofmathind}

Алгоритм построения дерева по коду $A_0 = (a_0, \ldots, a_{n-3})$:

Пусть $B_0 = \{ 1, \ldots, n \}$.
\begin{enumerate}
	\item Находим наименьшее $b \in B_i \colon b \notin A_i$.
	Тогда в~дереве есть ребро $\{ b, a_i \}$: $A_{i+1} = A_i \setminus \{ a_i \} \lAnd B_{i+1} = B_i \setminus \{ b \}$.
	\item Повторяем шаг~1 $n - 2$~раз.
	Получим $B_{n-2} = \{ b', b'' \}$, значит, в~дереве есть ребро~$\{ b', b'' \}$.
\end{enumerate}

Докажем, что указанный алгоритм по коду из $n$~чисел строит дерево.
\begin{proofmathind}
	\indbase При $n = 1$ легко проверить.
	\indstep Рассмотрим графы $T_1, \ldots, T_{n-1}$, полученные в~процессе построения дерева.
	$T_1$ не содержит циклов.
	$T_2$ получается из $T_1$ либо добавлением новой вершины, либо добавлением моста, что не приводит к появлению цикла.	
	По индукции получим, что $T_{n-1}$ не содержит циклов и содержит $n$~вершин и $n - 1$~ребёр, значит, $T_{n-1}$~--- дерево.
	\indend
\end{proofmathind}

\index{Теорема!Кэли}
\begin{theorem}[Кэли]
Количество неизоморфных помеченных деревьев с $n$~вершинами равно~$n^{n-2}$.
\end{theorem}
\section{Ориентированные графы}
\index{Графы!ориентированные} \textbf{Ориентированным графом} называется пара~$(V, E)$, где $V$~--- непустое конечное множество, $E$~--- совокупность элементов множества~$V^2$.
Элементы~$V$ называются \textbf{вершинами графа}.
Элементы~$E$ называются \textbf{дугами графа}.

На рисунках ориентированные графы изображаются так же, как неориентированные, за тем исключением, что на дуги дополнительно наносятся стрелки, направленные от начальной вершины к конечной.

Если $e = (u, v) \in E$, то говорят, что:
\begin{itemize}
	\item дуга~$e$ выходит из вершины~$u$ и входит в вершину~$v$;
	\item $u$~--- начало дуги~$e$, а $v$~--- её конец;
	\item дуга~$e$ инцидентна вершинам $u$ и $v$;
	\item вершины $u$ и $v$ инцидентны дуге~$e$.
\end{itemize}

\index{deg} Количество выходящих из вершины~$v$ дуг называется \textbf{полустепенью исхода вершины} и обозначается $\deg_+ v$.

Количество входящих в вершину~$v$ дуг называется \textbf{полустепенью захода вершины} и обозначается $\deg_- v$.

Количество инцидентных вершине~$v$ дуг называется \textbf{степенью вершины} и обозначается $\deg v$.
Очевидно, что $\deg v = \deg_+ v + \deg_- v$.

\begin{statement}
\begin{equation*}
\sum_{v \in V} \deg_+ v = \sum_{v \in V} \deg_- v = |E|
\end{equation*}
где $(V, E)$~--- граф.
\end{statement}
\begin{proof}
Достаточно заметить, что каждая дуга увеличивает полустепень исхода некоторой вершины на $1$ и полустепень захода некоторой вершины на $1$.
\end{proof}

\index{Петля} Дуга~$e = (u, u)$ называется \textbf{петлёй}.

Если в графе есть несколько дуг~$(u, v)$, то они называются \textbf{кратными}.

Дуги $(u, v)$ и $(v, u)$ называются \textbf{противоположно направленными}.

Граф называется \textbf{простым}, если в нём нет пётель и кратных дуг.

Граф~$(V_1, E_1)$ называется \textbf{подграфом графа~$(V, E)$}, если $V_1 \subseteq V \lAnd E_1 \subseteq E$.

Неориентированный граф, полученный из ориентированного графа~$G$ заменой дуг на рёбра, называется \textbf{основанием графа~$G$}. 

Графы $G_1 = (V_1, E_1)$ и $G_2 = (V_2, E_2)$ называются \textbf{изоморфными}, если существует биекция~$\varphi \colon V_1 \to V_2$ такая, что
$(u, v) \in E_1 \Leftrightarrow (\varphi(u), \varphi(v)) \in E_2$, иначе~--- \textbf{неизоморфными}.

\index{Путь} \textbf{Путём} в графе называется последовательность $(v_1, e_1, v_2, e_2, \ldots, v_{k-1}, e_k, v_k)$ его вершин и дуг такая, что $e_i \opbr= (v_i, v_{i+1})$.

Путь называется \textbf{простым}, если в нём все вершины, кроме, возможно, первой и последней, различны.

Путь называется \textbf{замкнутым}, если в нём первая и последняя вершины совпадают.

\index{Контур} Замкнутый путь называется \textbf{контуром}.

Путь, соединяющий вершины $u$ и $v$, называется \textbf{$(u, v)$"=путём}.

Если в графе существует $(u, v)$"=путь, то говорят, что вершина~$v$ \textbf{достижима} из вершины~$u$.
Если также существует $(v, u)$"=путь, то говорят, что вершины $u$ и $v$ \textbf{взаимодостижимы}.

\begin{lemma}
Любой путь содержит простой путь.
\end{lemma}%
Доказательство аналогично доказательству леммы~\ref{lemma:walk_contains_simple_chain}.

\begin{lemma}
Любой контур содержит простой контур, причём каждая вершина и дуга контура принадлежат некоторому простому контуру.
\end{lemma}%
Доказательство аналогично доказательству леммы~\ref{lemma:cycle_contains_simple_one}.

\subsection{Связность ориентированных графов}
Ориентированный граф называется \textbf{сильно связным}, если для любых его вершин $u$ и $v$ существуют $(u, v)$"=путь и $(v, u)$"=путь.

Ориентированный граф называется \textbf{слабо связным}, если связно его основание.

\index{Теорема!Роббинса}
\begin{theorem}[Роббинса]
Связный неориентированный граф обладает сильно связной ориентацией $\Leftrightarrow$ он не содержит мостов.
\end{theorem}
\begin{proof}
\begin{enumerate}
	\item $\Rightarrow$. Если граф содержит мост~$\{ u, v \}$, то при его ориентации можно получить либо дугу~$(u, v)$, либо дугу~$(v, u)$.
	В таком случае одна из компонент связности, соединяемых мостом, будет недостижима из другой.
	
	\item $\Leftarrow$. Любое ребро принадлежит некоторому циклу~$C$, так как оно не является мостом.
	Ориентируем все рёбра цикла в одну сторону.
	Если остались другие рёбра, то в силу связности графа можно получить ещё один цикл, часть которого является частью цикла~$C$.
	Ориентируем все рёбра полученного цикла в одну сторону, не изменяя уже ориентированные рёбра.
	Повторяя, ориентируем все рёбра графа.
	
	Для любых двух вершин $u$ и $v$ исходного графа существует маршрут $(u = v_1, e_1, \ldots, e_{k-1}, v_k = v)$.
	Чтобы получить $(u, v)$"=путь, идём по дугам $(v_1, v_2)$, $(v_2, v_3)$ и т.\,д.
	Если дуги $(v_i, v_{i+1})$ нет, то есть дуга $(v_{i+1}, v_i)$, состоящая в некотором цикле, пройдя по которому, можно из вершины~$v_i$ попасть в вершину~$v_{i+1}$.
	
	Аналогично получим $(v, u)$"=путь.
\end{enumerate}
\end{proof}

\index{Компонента связности} \textbf{Компонентой сильной связности} ориентированного графа называется максимальный относительно включения сильно связный подграф.
Аналогично определяется \textbf{компонента слабой связности}.

\textbf{Конденсацией ориентированного графа} называется ориентированный граф, вершинами которого являются компоненты сильной связности исходного графа, а дуга между вершинами показывает наличие пути между вершинами компонент.

\begin{statement}
Конденсация не содержит контуров.
\end{statement}
\begin{proofcontra}
Предположим, что в конденсации существует контур.
Тогда очевидно, что вершины различных компонент, входящих в него, взаимодостижимы, а значит, лежат в одной компоненте сильной связности.
Противоречие.
\end{proofcontra}

\subsection{Способы задания ориентированного графа}
Пронумеруем вершины графа~$G = (V, E)$, т.\,е. зададим биекцию $\varphi \colon V \to \{ 1, 2, \ldots, n \}$, где $n = |V|$, и будем их обозначать $1, 2, \ldots, n$.

\subsubsection{Матрица смежности}
\index{Матрица!смежности} \textbf{Матрицей смежности графа~$G$} называется матрица $A = \|a_{ij}\|_{\begin{smallmatrix}
i = \overline{1,n} \\
j = \overline{1,n}
\end{smallmatrix}}$, где $a_{ij}$ равно числу рёбер~$(i, j)$ в графе.

\begin{theorem}
Если $A = \|a_{ij}\|_{\begin{smallmatrix}
i = \overline{1,n} \\
j = \overline{1,n}
\end{smallmatrix}}$~--- матрица смежности графа~$G$,
$A^k = \|b_{ij}\|_{\begin{smallmatrix}
i = \overline{1,n} \\
j = \overline{1,n}
\end{smallmatrix}}$, то $b_{ij}$ равно числу $(i, j)$-путей длины~$k$.
\end{theorem}
\begin{proofmathind}
	\indbase Для $k = 1$ истинность следует из определения.
	\indstep Пусть теорема верна для $k$, $A^{k+1} = \|c_{ij}\|_{\begin{smallmatrix}
	i = \overline{1,n} \\
	j = \overline{1,n}
	\end{smallmatrix}}$.
	\begin{equation*}
	A^{k+1} = A A^k \Rightarrow c_{ij} = \sum_{l=1}^n a_{il} b_{lj}
	\end{equation*}
	
	Значение выражения $a_{il} b_{lj}$, очевидно, равно числу $(i, j)$"=путей длины~$k + 1$, проходящих по дуге~$(i, l)$ (если таких дуг нет в графе, то выражение равно $0$).
	Тогда, суммируя эти выражения по всем~$l$, получим число всех $(i, j)$"=путей длины $k + 1$. \indend
\end{proofmathind}

\subsubsection{Матрица инцидентности}
Так же, как и вершины, пронумеруем дуги графа~$G$ и будем их обозначать $1, 2, \ldots, m$.

\index{Матрица!инцидентности} \textbf{Матрицей инцидентности графа~$G$} называется матрица $A = \|a_{ij}\|_{\begin{smallmatrix}
i = \overline{1,n} \\
j = \overline{1,m}
\end{smallmatrix}}$, где
\begin{equation*}
a_{ij} =
\begin{dcases*}
-1, & вершина~$i$~--- конец дуги~$j$ \\
0, & вершина~$i$ не инцидентна дуге~$j$ \\
1, & вершина~$i$~--- начало дуги~$j$
\end{dcases*}
\end{equation*}

\subsection{Взвешенные графы}
Граф~$(V, E)$ называется \textbf{взвешенным}, если задана функция~$w \colon E \to R^+$, называемая \textbf{весом}.

\textbf{Весом дуги~$e \in E$} называется $w(e)$.

\textbf{Весом} (\textbf{длиной}) \textbf{пути} называется сумма весов входящих в него дуг.

$(u, v)$"=путь называется \textbf{кратчайшим}, если он имеет наименьший вес среди всех $(u, v)$"=путей.

Наименьший вес среди всех $(u, v)$"=путей называется \textbf{расстоянием между вершинами $u$ и $v$} и обозначается $d(u, v)$.

\begin{statement}
$d(u, v) \leqslant d(u, k) + d(k, v)$, причём $d(u, v) = d(u, k) + d(k, v)$ $\Leftrightarrow$ $k$ лежит на одном из кратчайших $(u, v)$"=путей.
\end{statement}
\begin{proof}
Пусть $p_1$ и $p_2$~--- кратчайшие $(u, k)$"=путь и $(k, v)$"=путь соответственно.
Тогда $p_1 \cup p_2$~--- $(u, v)$"=путь.
Значит, либо этот путь кратчайший и его вес минимален (в таком случае $d(u, v) = d(u, k) + d(k, v)$), либо его вес больше, чем $d(u, v)$ (в этом случае ни один из кратчайших $(u, v)$"=путей не проходит через вершину~$k$).
\end{proof}

Пусть дан граф~$(V, E)$ и вес~$w \colon E \to R^+$.

\subsubsection{Алгоритм Дейкстры}
\index{Алгоритм!Дейкстры} Алгоритм Дейкстры ищет длины кратчайших путей между некоторой вершиной~$u$ и всеми остальными.
Каждой вершине~$v$ на шаге~$i$ сопоставим метки~$l_i(v) \geqslant d(u, v)$.
\begin{enumerate}
	\item[0.] $l_0(u) = 0$, $\forall v \in V \ v \neq u \Rightarrow l_0(v) = \infty$
	\item[k.] Пусть $m$~--- непосещённая вершина с минимальным~$l_{k-1}(m)$.
	Отметим $m$ как посещённую.
	\begin{equation*}
	\forall v \in V \ l_k(v) = \min \{ l_{k-1}(v), l_{k-1}(m) + w(m, v) \}
	\end{equation*}
\end{enumerate}

\begin{theorem}
Если вершина~$v$ становится посещённой на $k$"~м шаге, то $d(u, v) = l_k(v)$.
\end{theorem}
\begin{proofmathind}
	\indbase $n = 0$: $v = u$, $0 = l_0(v) = d(u, u) = 0$.
	\indstep Пусть для $k \leqslant n$ $l_k(v) = d(u, v)$, $(u, \ldots, x, y, \ldots, z)$~--- кратчайший $(u, z)$"=путь, причём $y$~--- первая непосещённая вершина, $z$~--- вершина, посещённая на шаге~$k + 1$, тогда
	\begin{equation*}
	l_{k+1}(z) \leqslant l_{k+1}(y) = \min \{ l_k(y), l_k(x) + w(x, y) \} \leqslant l_k(x) + w(x, y) = d(u, x) + w(x, y) = d(u, y) \leqslant d(u, z)
	\end{equation*}
	\begin{equation*}
	l_{k+1}(z) \leqslant d(u, z) \lAnd d(u, z) \leqslant l_{k+1}(z) \Rightarrow l_{k+1}(z) = d(u, z)
	\end{equation*}
	\indend
\end{proofmathind}

\subsubsection{Алгоритм Флойда"--~Уоршелла}
\index{Алгоритм!Флойда---Уоршелла} Обозначим через~$P_k(i, j)$ кратчайший $(i, j)$"=путь с промежуточными вершинами из множества~$\{ 1, 2, \ldots, k \}$, а через $d_k(i, j)$~--- его длину.
Очевидно, что $P_0(i, j)$ либо состоит из дуги~$(i, j)$, либо не существует.
Рассмотрим $P_k(i, j)$.
$P_{k+1}(i, j)$ либо совпадает с $P_k(i, j)$, либо проходит через вершину~$k + 1$ и включает ребро~$(k + 1, j)$.
Тогда
\begin{equation*}
d_0(i, j) =
\begin{cases}
w(i, j), \ (i, j) \in E \\
\infty, \ (i, j) \notin E
\end{cases}, \
d_{k+1}(i, j) = \min(d_k(i, j), d_k(i, k + 1) + d_k(k + 1, j))
\end{equation*}
\section{Транспортные сети}
\index{Транспортная сеть} \index{Источник} \index{Сток} \textbf{Транспортной сетью} называется ориентированный граф, в котором выделены две вершины, одна из которых называется \textbf{источником} и обозначается~$s$, а другая~--- \textbf{стоком} и обозначается~$t$.
Источник имеет нулевую полустепень захода, а сток~--- нулевую полустепень исхода.
\index{Пропускная способность} Кроме того, каждой дуге графа сопоставлено положительное число, называемое \textbf{пропускной способностью}, т.\,е. задана функция~$q \colon E \to R^+$.

\index{Поток} \textbf{Потоком в сети~$(V, E)$} называется функция~$p \colon E \to R^+$ такая, что
\begin{itemize}
	\item $\forall (i, j) \in E \ p(i, j) \leqslant q(i, j)$;
	\item $\forall k \in V \setminus \{ s, t \} \ \sum\limits_{(i, k) \in E} p(i, k) = \sum\limits_{(k, j) \in E} p(k, j)$.
\end{itemize}

\index{Разрез} \textbf{Разрезом в сети~$(V, E)$} называется разбиение множества вершин на два подмножества $V$ и $\overline V$ таких, что $s \in X$, $t \in \overline X$.

\textbf{Пропускной способностью разреза} называется сумма $\sum\limits_{i \in X, j \in \overline X} q(i, j)$.

\textbf{Величиной потока~$p$ через разрез~$(X, \overline X)$} называется сумма $\sum\limits_{i \in X, j \in \overline X} p(i, j) -
\sum\limits_{i \in X, j \in \overline X} p(j, i)$ и обозначается $p(X, \overline X)$.

\begin{lemma}
Величина потока через любой разрез одна и та же.
\end{lemma}
\begin{proof}
Пусть $(X, \overline X)$~--- разрез сети~$(V, E)$, причём $s \neq j \in X$.
Рассмотрим разрез~$(X \setminus \{ j \}, \overline X \cup \{ j \})$.
\begin{equation*}
p(X \setminus \{ j \}, \overline X \cup \{ j \}) = p(X, \overline X) +
\sum_{i \in X} p(i, j) + \sum_{i \in \overline X} p(i, j) -
\sum_{i \in X} p(j, i) - \sum_{i \in \overline X} (j, i) =
\end{equation*}
\begin{equation*}
= p(X, \overline X) + \sum_{i \in V} p(i, j) - \sum_{i \in V} p(j, i) =
p(X, \overline X)
\end{equation*}
\end{proof}

Т.\,о., \textbf{величиной потока~$p$} называется его величина через любой разрез и обозначается $|p|$.

\begin{consequent}
$\sum\limits_{i \in V} p(s, i) = \sum\limits_{i \in V} p(i, t)$.
\end{consequent}

\begin{lemma}
Для любого потока~$p$ верно $\forall (X, \overline X) \ p(X, \overline X) \leqslant q(X, \overline X)$.
\end{lemma}
\begin{proof}
\begin{equation*}
p(X, \overline X) =
\sum_{i \in X, j \in \overline X} p(i, j) - \sum_{i \in X, j \in \overline X} p(j, i) \leqslant
\sum_{i \in X, j \in \overline X} p(i, j) \leqslant
\sum_{i \in X, j \in \overline X} q(i, j) =
q(X, \overline X)
\end{equation*}
\end{proof}

Разрез с минимальной пропускной способностью называется \textbf{минимальным разрезом}.

Поток максимальной величины называется \textbf{максимальным потоком}.

\begin{lemma}
Если $p_0(X_0, \overline X_0) = q(X_0, \overline X_0)$, то $(X_0, \overline X_0)$~--- минимальный разрез, а $p_0$~--- максимальный поток.
\end{lemma}
\begin{proof}
Пусть $p$~--- произвольный поток, $(X, \overline X)$~--- произвольный разрез.
\begin{itemize}
	\item $|p_0| = q(X_0, \overline X_0) \geqslant |p|$
	\item $q(X_0, \overline X_0) = |p_0| \leqslant q(X, \overline X)$
\end{itemize}
\end{proof}

\subsection{Алгоритмы нахождения максимального потока}
\subsubsection{Алгоритм Форда"--~Фалкерсона}
\index{Алгоритм!Форда---Фалкерсона}
Изначально считаем, что $\forall i, j \in V \ p(i, j) = 0$.
Также пометим источник~$s$ меткой~$(-, \infty)$.
\begin{enumerate}
	\item Пусть вершина~$i$ имеет метку~$(x^\pm, \varepsilon)$.
	\begin{itemize}
		\item Если $(i, j) \in E$ и $q(i, j) - p(i, j) > 0$, то присваиваем $j$ метку~$(i^+, \min \{\varepsilon, q(i, j) - p(i, j)\})$.
		\item Если $(j, i) \in E$ и $p(i, j) > 0$, то присваиваем $j$ метку~$(i^-, \min \{\varepsilon, p(i, j)\})$.
	\end{itemize}
	
	Если сток не достигнут, то поток максимален, иначе сток имеет метку$(x^+, \delta)$.
	
	\item Идём в обратном направлении.
	Если вершина~$u$ имеет метку~$(x^+, \varepsilon)$, то увеличиваем поток~$p(i, u)$ на~$\delta$, а если $(x^-, \varepsilon)$, то уменьшаем $p(i, u)$ на~$\delta$.
	Т.\,о., величина потока увеличилась на~$\delta > 0$.
\end{enumerate}

Повторяя, увеличим поток до максимума, т.\,е. в какой-то момент не сможем пометить следующую вершину.
Получим некоторый разрез~$(X, \overline X)$.

\begin{lemma}
Найденный поток максимален, а разрез минимален.
\end{lemma}
\begin{proof}
\begin{equation*}
|p| = \sum_{i \in X, j \in \overline X} p(i, j) - \sum_{i \in X, j \in \overline X} p(j, i)
\end{equation*}

Т.\,к. из $i$ нельзя попасть в~$j$ и из~$j$ нельзя попасть в~$i$, где $i \in X$, $j \in \overline X$, то $p(i, j) = q(i, j)$, $p(j, i) = 0$, тогда
\begin{equation*}
|p| = \sum_{i \in X, j \in \overline X} q(i, j)
\end{equation*}
\end{proof}

\subsubsection{Алгоритм Эдмондса"--~Карпа}
\index{Алгоритм!Эдмондса---Карпа}
Путь~$(s = v_1, \ldots, v_{k+1} = t)$ в сети~$(V, E)$ называется \textbf{увеличивающим}, если $\forall i \in \{ 1, \ldots, k \} \ \varepsilon(v_i, v_{i+1}) > 0$, где $\varepsilon(v_i, v_{i+1}) = \begin{cases}
q(v_i, v_{i+1}) - p(v_i, v_{i+1}), \ (v_i, v_{i+1}) \in E \\
p(v_{i+1}, v_i), \ (v_{i+1}, v_i) \in E
\end{cases}$

Введём $\delta = \min\limits_{0 \leqslant i \leqslant k} \varepsilon(v_i, v_{i+1})$, тогда новое значение потока в сети равно
$\begin{cases}
p(v_i, v_{i+1}) + \delta, \ (v_i, v_{i+1}) \in E \\
p(v_{i+1}, v_i) - \delta, \ (v_{i+1}, v_i) \in E
\end{cases}$

Рассмотрим случай, когда до~$t$ не существует увеличивающего пути.
Пусть $X$~--- множество вершин, до которых существует увеличивающий путь, $u \in X$, $v \notin X$.
Если $(u, v) \in E$, то $q(u, v) = p(u, v)$, а если $(v, u) \in E$, то $p(v, u) = 0$, тогда
\begin{equation*}
p(X, \overline X) =
\sum_{u \in V, v \in \overline X} p(u, v) - \sum_{u \in V, v \in \overline X} p(v, u) =
\sum_{u \in V, v \in \overline X} q(u, v)
\end{equation*}

\begin{lemma}
В ходе работы алгоритма Эдмондса"--~Карпа кратчайший $(s, t)$"=путь не уменьшается.
\end{lemma}
\begin{proofcontra}
Рассмотрим самую близкую к~$s$ вершину~$v$, для которой кратчайший путь~$(s, \ldots, u, v)$ уменьшается, тогда для вершины~$u$ кратчайший $(s, u)$"=путь не уменьшается.
Пусть $d_u$ и $d_v$~--- длины кратчайших $(s, u)$- и $(s, v)$"=путей соответственно на предыдущем шаге, а $d_u'$ и $d_v'$~--- на текущем.
\begin{equation*}
d_v > d_v' = d_u' + 1 \geqslant d_u + 1 \Rightarrow d_v \geqslant d_u + 2
\end{equation*}

Значит, на предыдущем шаге не было дуги~$(u, v)$, тогда не было и кратчайшего $(s, v)$"=пути.
Противоречие.
\end{proofcontra}

Назовём дугу~$(v_i, v_{i+1})$ \textbf{критической}, если $\varepsilon(v_i, v_{i+1}) = \delta$.

\begin{lemma}
Каждая дуга может быть критической на увеличивающем пути порядка $\frac{|V|}2$~раз.
\end{lemma}
\begin{proof}
Пусть дуга~$(u, v)$ критическая на шагах $t_1$ и $t_2$.
Если она была использована как прямая два раза, то между этими использованиями она должна была быть использована как обратная (на шаге~$t_3$), тогда
\begin{equation*}
d_v(t_2) = d_u(t_2) + 1 \geqslant
d_u(t_3) + 1 = d_v(t_3) + 2 \geqslant
d_v(t_1) + 2
\end{equation*}
\end{proof}
\chapter{Теория матриц}
\index{Матрица} \textbf{Матрицей} называется прямоугольная таблица из чисел, содержащая $m$~строк и $n$~столбцов, и обозначается
\begin{equation*}
A = (a_{ij})_{\begin{smallmatrix}
i = \overline{1, m} \\
j = \overline{1, n}
\end{smallmatrix}} =
\begin{pmatrix}
a_{11} & a_{12} & \cdots & a_{1n} \\
a_{21} & a_{22} & \cdots & a_{2n} \\
\vdots & \vdots & \ddots & \vdots \\
a_{m1} & a_{m2} & \cdots & a_{mn}
\end{pmatrix} =
\begin{Vmatrix}
a_{11} & a_{12} & \cdots & a_{1n} \\
a_{21} & a_{22} & \cdots & a_{2n} \\
\vdots & \vdots & \ddots & \vdots \\
a_{m1} & a_{m2} & \cdots & a_{mn}
\end{Vmatrix} =
\|a_{ij}\|_{\begin{smallmatrix}
i = \overline{1, m} \\
j = \overline{1, n}
\end{smallmatrix}}
\end{equation*}

Числа $m$ и $n$ называются \textbf{порядками} матрицы.

Если $m = n$, то матрица называется \textbf{квадратной}, а число~$m = n$~--- её \textbf{порядком}.
\textbf{Главной} называется диагональ квадратной матрицы, состоящая из элементов $a_{11}, a_{22}, \ldots, a_{nn}$, а \textbf{побочной}~--- состоящая из элементов $a_{n1}, a_{n-1\, 2}, \ldots, a_{1n}$.

$i$"~я строка матрицы обозначается $A_i$, $j$"~й столбец~--- $A^j$.

Две матрицы называются \textbf{равными}, если их порядки и соответствующие элементы совпадают, иначе~--- \textbf{неравными}.
\section{Операции над матрицами}
Матрица, все элементы которой равны~$0$, называется \textbf{нулевой} и обозначается $O$.

Квадратная матрица, в~которой элементы главной диагонали равны~$1$, а остальные~---~$0$, называется \textbf{единичной} и обозначается $E$.

Над матрицами определены следующие операции:
\begin{itemize}
	\item\textbf{Сложение.}
	Определено только над матрицами одинакового размера.
	\begin{equation*}
	\begin{Vmatrix}
	a_{11} & a_{12} & \cdots & a_{1n} \\ 
	a_{21} & a_{22} & \cdots & a_{2n} \\ 
	\vdots & \vdots & \ddots & \vdots \\ 
	a_{m1} & a_{m2} & \cdots & a_{mn}
	\end{Vmatrix} +
	\begin{Vmatrix}
	b_{11} & b_{12} & \cdots & b_{1n} \\ 
	b_{21} & b_{22} & \cdots & b_{2n} \\ 
	\vdots & \vdots & \ddots & \vdots \\ 
	b_{m1} & b_{m2} & \cdots & b_{mn}
	\end{Vmatrix} =
	\begin{Vmatrix}
	a_{11} + b_{11} & a_{12} + b_{12} & \cdots & a_{1n} + b_{1n} \\ 
	a_{21} + b_{21} & a_{22} + b_{22} & \cdots & a_{2n} + b_{2n} \\ 
	\vdots & \vdots & \ddots & \vdots \\ 
	a_{m1} + b_{m1} & a_{m2} + b_{m2} & \cdots & a_{mn} + b_{mn}
	\end{Vmatrix}
	\end{equation*}
	
	Пусть $A, B, C$~--- матрицы. Свойства сложения:
	\begin{itemize}
		\item коммутативность:
		$A + B = B + A$
		\item ассоциативность:
		$(A + B) + C = A + (B + C)$
	\end{itemize}
	
	\item\textbf{Умножение на число.}
	\begin{equation*}
	\lambda \cdot
	\begin{Vmatrix}
	a_{11} & a_{12} & \cdots & a_{1n} \\ 
	a_{21} & a_{22} & \cdots & a_{2n} \\ 
	\vdots & \vdots & \ddots & \vdots \\ 
	a_{m1} & a_{m2} & \cdots & a_{mn}
	\end{Vmatrix} =
	\begin{Vmatrix}
	\lambda a_{11} & \lambda a_{12} & \cdots & \lambda a_{1n} \\ 
	\lambda a_{21} & \lambda a_{22} & \cdots & \lambda a_{2n} \\ 
	\vdots & \vdots & \ddots & \vdots \\ 
	\lambda a_{m1} & \lambda a_{m2} & \cdots & \lambda a_{mn}
	\end{Vmatrix}
	\end{equation*}
	
	Пусть $\alpha, \beta$~--- числа, $A, B$~--- матрицы. Свойства умножения на число:
	\begin{itemize}
		\item ассоциативность:
		$(\alpha \cdot \beta) \cdot A = \alpha \cdot (\beta \cdot A)$
		\item дистрибутивность относительно сложения чисел:
		$(\alpha + \beta) \cdot A = \alpha \cdot A + \beta \cdot A$
		\item дистрибутивность относительно сложения матриц:
		$\alpha \cdot (A + B) = \alpha \cdot A + \alpha \cdot B$
	\end{itemize}
	
	\item\textbf{Умножение.} $A \cdot B$ определено, только если количество столбцов в матрице~$A$ совпадает с количеством строк в матрице~$B$.
	\begin{equation*}
	\begin{Vmatrix}
	a_{11} & a_{12} & \cdots & a_{1k} \\ 
	a_{21} & a_{22} & \cdots & a_{2k} \\ 
	\vdots & \vdots & \ddots & \vdots \\ 
	a_{m1} & a_{m2} & \cdots & a_{mk}
	\end{Vmatrix} \cdot
	\begin{Vmatrix}
	b_{11} & b_{12} & \cdots & b_{1n} \\ 
	b_{21} & b_{22} & \cdots & b_{2n} \\ 
	\vdots & \vdots & \ddots & \vdots \\ 
	b_{k1} & b_{k2} & \cdots & b_{kn}
	\end{Vmatrix} =
	\begin{Vmatrix}
	\sum a_{1i}b_{i1} & \sum a_{1i}b_{i2} & \cdots & \sum a_{1i}b_{in} \\
	\sum a_{2i}b_{i1} & \sum a_{2i}b_{i2} & \cdots & \sum a_{2i}b_{in} \\
	\vdots & \vdots & \ddots & \vdots \\
	\sum a_{mi}b_{i1} & \sum a_{mi}b_{i2} & \cdots & \sum a_{mi}b_{in}
	\end{Vmatrix}
	\end{equation*}
	где суммирование производится по $i$ от $1$ до $k$.
	
	Пусть $\lambda$~--- число, $A, B, C$~--- матрицы. Свойства умножения:
	\begin{itemize}
		\item ассоциативность:
		$(A \cdot B) \cdot C = A \cdot (B \cdot C)$
		\item дистрибутивность:
		$(A + B) \cdot C \opbr= A \cdot C + B \cdot C$,
		$A \cdot (B + C) \opbr= A \cdot B + A \cdot C$
		\item ассоциативность и коммутативность относительно умножения на число:
		$(\lambda \cdot A) \cdot B \opbr= \lambda \cdot (A \cdot B) \opbr= A \cdot (\lambda \cdot B)$
	\end{itemize}
	
	\item\textbf{Транспонирование.}
	\begin{equation*}
	\begin{Vmatrix}
	a_{11} & a_{12} & \cdots & a_{1n} \\
	a_{21} & a_{22} & \cdots & a_{2n} \\
	\vdots & \vdots & \ddots & \vdots \\
	a_{n1} & a_{n2} & \cdots & a_{nn}
	\end{Vmatrix}^T =
	\begin{Vmatrix}
	a_{11} & a_{21} & \cdots & a_{n1} \\
	a_{12} & a_{22} & \cdots & a_{n2} \\
	\vdots & \vdots & \ddots & \vdots \\
	a_{1n} & a_{2n} & \cdots & a_{nn}
	\end{Vmatrix}
	\end{equation*}
\end{itemize}
\section{Блочные матрицы}
\index{Матрица!блочная} Если матрицу при помощи горизонтальных и вертикальных прямых разделить на прямоугольные клетки, называемые \textbf{блоками}, то получится \textbf{блочная матрица}, состоящая из блоков, которые, в~свою очередь, также являются матрицами.
Легко проверить непосредственно, что операции над блочными матрицами осуществляются так~же, как и над обычными.
\section{Определитель матрицы}
\index{det}\index{Определитель} \textbf{Определителем порядка~$n$} квадратной {матрицы~$A$} порядка~$n$, называется число, равное
\begin{equation}
\label{eq:determinant}
\Delta = \det A = |A| =
\begin{vmatrix}
a_{11} & a_{12} & \cdots & a_{1n} \\
a_{21} & a_{22} & \cdots & a_{2n} \\
\vdots & \vdots & \ddots & \vdots \\
a_{n1} & a_{n2} & \cdots & a_{nn}
\end{vmatrix} =
\sum_{\sigma = (i_1, \ldots, i_n) \in S_n} (-1)^{|\sigma|} a_{1\, i_1} a_{2\, i_2} \cdot \ldots \cdot a_{n\, i_n}, \ 
|\sigma| =
\begin{cases}
0, \sigma \text{ чётная} \\
1, \sigma \text{ нечётная}
\end{cases}
\end{equation}
где $S_n$~--- множество всех перестановок $n$-элементного множества.

Матрица называется \textbf{вырожденной}, если её определитель равен~$0$, иначе~--- \textbf{невырожденной}.

Свойства определителя:
\begin{itemize}
	\item Если элементы какой-либо строки или столбца определителя имеют общий множитель~$\lambda$, то его можно вынести за знак определителя.
	\begin{proof}
	\begin{equation*}
	\Delta = \sum (-1)^{|\sigma|} a_{1\, i_1} a_{2\, i_2} \cdot \ldots \cdot a_{n\, i_n}
	\end{equation*}
	Каждое слагаемое имеет множитель из каждой строки, а также из каждого столбца, т.\,к. $\sigma$ является перестановкой и содержит все номера столбцов от $1$ до $n$ включительно.
	Тогда все слагаемые имеют общий множитель~$\lambda$, поэтому его можно вынести за скобки.
	\end{proof}
	
	\item Если какая-либо строка или столбец определителя состоит из нулей, то он равен~$0$.
	
	\item \begin{equation*}
	\begin{vmatrix}
	a_{11} & a_{12} & \cdots & a_{1n} \\
	\vdots & \vdots & \ddots & \vdots \\
	a_{i1} + b_{i1} & a_{i2} + b_{i2} & \cdots & a_{in} + b_{in} \\
	\vdots & \vdots & \ddots & \vdots \\
	a_{n1} & a_{n2} & \cdots & a_{nn}
	\end{vmatrix} =
	\begin{vmatrix}
	a_{11} & a_{12} & \cdots & a_{1n} \\
	\vdots & \vdots & \ddots & \vdots \\
	a_{i1} & a_{i2} & \cdots & a_{in} \\
	\vdots & \vdots & \ddots & \vdots \\
	a_{n1} & a_{n2} & \cdots & a_{nn}
	\end{vmatrix} +
	\begin{vmatrix}
	a_{11} & a_{12} & \cdots & a_{1n} \\
	\vdots & \vdots & \ddots & \vdots \\
	b_{i1} & b_{i2} & \cdots & b_{in} \\
	\vdots & \vdots & \ddots & \vdots \\
	a_{n1} & a_{n2} & \cdots & a_{nn}
	\end{vmatrix}
	\end{equation*}
	Свойство для столбцов аналогично.
	\begin{proof}
	\begin{equation*}
	\Delta = \begin{vmatrix}
	a_{11} & a_{12} & \cdots & a_{1n} \\
	\vdots & \vdots & \ddots & \vdots \\
	a_{i1} + b_{i1} & a_{i2} + b_{i2} & \cdots & a_{in} + b_{in} \\
	\vdots & \vdots & \ddots & \vdots \\
	a_{n1} & a_{n2} & \cdots & a_{nn}
	\end{vmatrix} =
	\sum (-1)^{|\sigma|} a_{1\, i_1} \cdot \ldots \cdot a_{n\, i_n} =
	\end{equation*}
	\begin{equation*}
	\left| \text{ Каждое слагаемое содержит ровно~$1$ элемент из $i$\nobreakdash-й строки и поэтому имеет вид } \right|
	\end{equation*}
	\begin{equation*}
	= \sum (-1)^{|\sigma|} a_{1\, i_1} \cdot \ldots \cdot a_{k-1\, i_{k-1}} (a_{k\, i_k} + b_{k\, i_k}) a_{k+1\, i_{k+1}} \cdot \ldots \cdot a_{n\, i_n} =
	\end{equation*}
	\begin{equation*}
	= \sum (-1)^{|\sigma|} a_{1\, i_1} \cdot \ldots \cdot a_{k\, i_k} \cdot \ldots \cdot a_{n\, i_n} +
	\sum (-1)^{|\sigma|} a_{1\, i_1} \cdot \ldots \cdot b_{k\, i_k} \cdot \ldots \cdot a_{n\, i_n} =
	\end{equation*}
	\begin{equation*}
	= \begin{vmatrix}
	a_{11} & a_{12} & \cdots & a_{1n} \\
	\vdots & \vdots & \ddots & \vdots \\
	a_{i1} & a_{i2} & \cdots & a_{in} \\
	\vdots & \vdots & \ddots & \vdots \\
	a_{n1} & a_{n2} & \cdots & a_{nn}
	\end{vmatrix} +
	\begin{vmatrix}
	a_{11} & a_{12} & \cdots & a_{1n} \\
	\vdots & \vdots & \ddots & \vdots \\
	b_{i1} & b_{i2} & \cdots & b_{in} \\
	\vdots & \vdots & \ddots & \vdots \\
	a_{n1} & a_{n2} & \cdots & a_{nn}
	\end{vmatrix}
	\end{equation*}
	Свойство для столбцов доказывается аналогично.
	\end{proof}
	
	\item Если в~определителе поменять две строки или два столбца местами, то он изменит знак.
	\begin{proof}
	При перестановке строк или столбцов местами по утверждению~\ref{st:parity_of_permutation} все перестановки в формуле~(\ref{eq:determinant}) меняют чётность, значит, каждое слагаемое меняет знак, тогда и определитель меняет знак.
	\end{proof}
	
	\item Если в~определителе две строки или два столбца совпадают, то он равен~$0$.
	\begin{proof}
	Если поменять местами совпадающие строки или столбцы, то он, с~одной стороны, не изменится, а с~другой, поменяет знак. Значит, определитель равен~$0$.
	\end{proof}
	
	\item \begin{equation*}
	\begin{vmatrix}
	a_{11} & a_{12} & \cdots & a_{1n} \\
	\vdots & \vdots & \ddots & \vdots \\
	a_{i1} & a_{i2} & \cdots & a_{in} \\
	\vdots & \vdots & \ddots & \vdots \\
	a_{j1} & a_{j2} & \cdots & a_{jn} \\
	\vdots & \vdots & \ddots & \vdots \\
	a_{n1} & a_{n2} & \cdots & a_{nn}
	\end{vmatrix} =
	\begin{vmatrix}
	a_{11} & a_{12} & \cdots & a_{1n} \\
	\vdots & \vdots & \ddots & \vdots \\
	a_{i1} & a_{i2} & \cdots & a_{in} \\
	\vdots & \vdots & \ddots & \vdots \\
	\lambda a_{i1} + a_{j1} & \lambda a_{i2} + a_{j2} & \cdots & \lambda a_{in} + a_{jn} \\
	\vdots & \vdots & \ddots & \vdots \\
	a_{n1} & a_{n2} & \cdots & a_{nn}
	\end{vmatrix}
	\end{equation*}
	Свойство для столбцов аналогично.
	\begin{proof}
	\begin{equation*}
	\begin{vmatrix}
	a_{11} & a_{12} & \cdots & a_{1n} \\
	\vdots & \vdots & \ddots & \vdots \\
	a_{i1} & a_{i2} & \cdots & a_{in} \\
	\vdots & \vdots & \ddots & \vdots \\
	a_{j1} & a_{j2} & \cdots & a_{jn} \\
	\vdots & \vdots & \ddots & \vdots \\
	a_{n1} & a_{n2} & \cdots & a_{nn}
	\end{vmatrix} =
	\begin{vmatrix}
	a_{11} & a_{12} & \cdots & a_{1n} \\
	\vdots & \vdots & \ddots & \vdots \\
	a_{i1} & a_{i2} & \cdots & a_{in} \\
	\vdots & \vdots & \ddots & \vdots \\
	\lambda a_{i1} & \lambda a_{i2} & \cdots & \lambda a_{in} \\
	\vdots & \vdots & \ddots & \vdots \\
	a_{n1} & a_{n2} & \cdots & a_{nn}
	\end{vmatrix} +
	\begin{vmatrix}
	a_{11} & a_{12} & \cdots & a_{1n} \\
	\vdots & \vdots & \ddots & \vdots \\
	a_{i1} & a_{i2} & \cdots & a_{in} \\
	\vdots & \vdots & \ddots & \vdots \\
	a_{j1} & a_{j2} & \cdots & a_{jn} \\
	\vdots & \vdots & \ddots & \vdots \\
	a_{n1} & a_{n2} & \cdots & a_{nn}
	\end{vmatrix} =
	\end{equation*}
	\begin{equation*}
	= \begin{vmatrix}
	a_{11} & a_{12} & \cdots & a_{1n} \\
	\vdots & \vdots & \ddots & \vdots \\
	a_{i1} & a_{i2} & \cdots & a_{in} \\
	\vdots & \vdots & \ddots & \vdots \\
	\lambda a_{i1} + a_{j1} & \lambda a_{i2} + a_{j2} & \cdots & \lambda a_{in} + a_{jn} \\
	\vdots & \vdots & \ddots & \vdots \\
	a_{n1} & a_{n2} & \cdots & a_{nn}
	\end{vmatrix}
	\end{equation*}
	Свойство для столбцов доказывается аналогично.
	\end{proof}
\end{itemize}

Рассмотрим квадратную матрицу~$A$ $n$-го порядка.
Пусть $1 \leqslant i_1 < i_2 < \ldots < i_k \leqslant n$, $1 \leqslant j_1 < j_2 < \ldots < j_k \leqslant n$.

\index{Минор} \textbf{Минором $k$"~го порядка матрицы~$A$} называется определитель, образованный элементами матрицы, стоящими на пересечении строк с номерами~$i_1, i_2, \ldots, i_k$ и столбцов с номерами~$j_1, j_2, \ldots, j_k$, и обозначается~$M_{j_1 j_2 \ldots j_k}^{i_1 i_2 \ldots i_k}$.

\textbf{Дополнительным минором порядка $n - k$ к минору $M_{j_1 j_2 \ldots j_k}^{i_1 i_2 \ldots i_k}$} называется определитель, полученный вычеркиванием строк с номерами~$i_1, i_2, \ldots, i_k$ и столбцов с номерами~$j_1, j_2, \ldots, j_k$ из определителя матрицы~$A$, и обозначается $\overline M_{j_1 j_2 \ldots j_k}^{i_1 i_2 \ldots i_k}$.

\index{Алгебраическое дополнение} \textbf{Алгебраическим дополнением элемента~$a_{ij}$} матрицы~$A$ называется величина, равная $(-1)^{i+j} \overline M_j^i$, и обозначается $A_{ij}$.

\begin{theorem}
\label{th:determinant_expansion}
Любой определитель можно \textbf{разложить по элементам} произвольной строки или столбца:
\begin{equation*}
\begin{vmatrix}
a_{11} & a_{12} & \cdots & a_{1n} \\
a_{21} & a_{22} & \cdots & a_{2n} \\
\vdots & \vdots & \ddots & \vdots \\
a_{n1} & a_{n2} & \cdots & a_{nn}
\end{vmatrix} =
\sum_{j=1}^n a_{ij} A_{ij} =
\sum_{i=1}^n a_{ij} A_{ij}
\end{equation*}
\end{theorem}
\begin{proof}
\begin{equation*}
\begin{vmatrix}
a_{11} & a_{12} & \cdots & a_{1n} \\
\vdots & \vdots & \ddots & \vdots \\
a_{i1} & a_{i2} & \cdots & a_{in} \\
\vdots & \vdots & \ddots & \vdots \\
a_{n1} & a_{n2} & \cdots & a_{nn}
\end{vmatrix} =
(-1)^{i-1} \cdot
\begin{vmatrix}
a_{i1} & a_{i2} & \cdots & a_{in} \\
a_{11} & a_{12} & \cdots & a_{1n} \\
\vdots & \vdots & \ddots & \vdots \\
a_{i-1\, 1} & a_{i-1\, 2} & \cdots & a_{i-1\, n} \\
a_{i+1\, 1} & a_{i+1\, 2} & \cdots & a_{i+1\, n} \\
\vdots & \vdots & \ddots & \vdots \\
a_{n1} & a_{n2} & \cdots & a_{nn}
\end{vmatrix} =
(-1)^{i+1} \cdot
\end{equation*}
\begin{equation*}
\cdot \left(
\begin{vmatrix}
a_{i1} & 0 & \cdots & 0 \\
a_{11} & a_{12} & \cdots & a_{1n} \\
\vdots & \vdots & \ddots & \vdots \\
a_{i-1\, 1} & a_{i-1\, 2} & \cdots & a_{i-1\, n} \\
a_{i+1\, 1} & a_{i+1\, 2} & \cdots & a_{i+1\, n} \\
\vdots & \vdots & \ddots & \vdots \\
a_{n1} & a_{n2} & \cdots & a_{nn}
\end{vmatrix} +
\begin{vmatrix}
0 & a_{i2} & \cdots & 0 \\
a_{11} & a_{12} & \cdots & a_{1n} \\
\vdots & \vdots & \ddots & \vdots \\
a_{i-1\, 1} & a_{i-1\, 2} & \cdots & a_{i-1\, n} \\
a_{i+1\, 1} & a_{i+1\, 2} & \cdots & a_{i+1\, n} \\
\vdots & \vdots & \ddots & \vdots \\
a_{n1} & a_{n2} & \cdots & a_{nn}
\end{vmatrix} + \ldots + 
\begin{vmatrix}
0 & 0 & \cdots & a_{in} \\
a_{11} & a_{12} & \cdots & a_{1n} \\
\vdots & \vdots & \ddots & \vdots \\
a_{i-1\, 1} & a_{i-1\, 2} & \cdots & a_{i-1\, n} \\
a_{i+1\, 1} & a_{i+1\, 2} & \cdots & a_{i+1\, n} \\
\vdots & \vdots & \ddots & \vdots \\
a_{n1} & a_{n2} & \cdots & a_{nn}
\end{vmatrix}
\right) =
\end{equation*}
\begin{equation*}
= (-1)^{i+1} \cdot \left(
\begin{vmatrix}
a_{i1} & 0 & \cdots & 0 \\
a_{11} & a_{12} & \cdots & a_{1n} \\
\vdots & \vdots & \ddots & \vdots \\
a_{i-1\, 1} & a_{i-1\, 2} & \cdots & a_{i-1\, n} \\
a_{i+1\, 1} & a_{i+1\, 2} & \cdots & a_{i+1\, n} \\
\vdots & \vdots & \ddots & \vdots \\
a_{n1} & a_{n2} & \cdots & a_{nn}
\end{vmatrix} -
\begin{vmatrix}
a_{i2} & 0 & 0 & \cdots & 0 \\
a_{12} & a_{11} & a_{13} & \cdots & a_{1n} \\
\vdots & \vdots & \vdots & \ddots & \vdots \\
a_{i-1\, 2} & a_{i-1\, 1} & a_{i-1\, 3} & \cdots & a_{i-1\, n} \\
a_{i+1\, 2} & a_{i+1\, 1} & a_{i+1\, 3} & \cdots & a_{i+1\, n} \\
\vdots & \vdots & \vdots & \ddots & \vdots \\
a_{n2} & a_{n1} & a_{n3} & \cdots & a_{nn}
\end{vmatrix} + \ldots + \right.
\end{equation*}
\begin{equation*}
\left. \vphantom1 + (-1)^{n-1} \cdot
\begin{vmatrix}
a_{in} & 0 & \cdots & 0 \\
a_{1n} & a_{11} & \cdots & a_{1\, n-1} \\
\vdots & \vdots & \ddots & \vdots \\
a_{i-1\, n} & a_{i-1\, 1}  & \cdots & a_{i-1\, n-1} \\
a_{i+1\, n} & a_{i+1\, 1} & \cdots & a_{i+1\, n-1} \\
\vdots & \vdots & \ddots & \vdots \\
a_{nn} & a_{n1} & \cdots & a_{n\, n-1}
\end{vmatrix}
\right) =
\end{equation*}
\begin{equation*}
\left| \begin{gathered}
\text{Пользуясь формулой~(\ref*{eq:determinant}), получим} \\
\begin{vmatrix}
a & 0 & \cdots & 0 \\
a_{21} & a_{22} & \cdots & a_{2n} \\
\vdots & \vdots & \ddots & \vdots \\
a_{n1} & a_{n2} & \cdots & a_{nn}
\end{vmatrix} = \\
= \sum a \cdot a_{2\, i_2} \cdot \ldots \cdot a_{n\, i_n} +
\sum 0 \cdot a_{2\, i_2} \cdot \ldots \cdot a_{n\, i_n} + \ldots +
\sum 0 \cdot a_{2\, i_2} \cdot \ldots \cdot a_{n\, i_n} = \\
= a \sum a_{2\, i_2} \cdot \ldots \cdot a_{n\, i_n} =
a \cdot
\begin{vmatrix}
a_{22} & \cdots & a_{2n} \\
\vdots & \ddots & \vdots \\
a_{n2} & \cdots & a_{nn}
\end{vmatrix}
\end{gathered} \right|
\end{equation*}
\begin{equation*}
= (-1)^{i+1} a_{i1} \cdot
\begin{vmatrix}
a_{12} & \cdots & a_{1n} \\
\vdots & \ddots & \vdots \\
a_{i-1\, 2} & \cdots & a_{i-1\, n} \\
a_{i+1\, 2} & \cdots & a_{i+1\, n} \\
\vdots & \ddots & \vdots \\
a_{n2} & \cdots & a_{nn}
\end{vmatrix} +
(-1)^{i+2} a_{i2} \cdot
\begin{vmatrix}
a_{11} & a_{13} & \cdots & a_{1n} \\
\vdots & \vdots & \ddots & \vdots \\
a_{i-1\, 1} & a_{i-1\, 3} & \cdots & a_{i-1\, n} \\
a_{i+1\, 1} & a_{i+1\, 3} & \cdots & a_{i+1\, n} \\
\vdots & \vdots & \ddots & \vdots \\
a_{n1} & a_{n3} & \cdots & a_{nn}
\end{vmatrix} + \ldots + (-1)^{i+n} a_{in} \cdot
\begin{vmatrix}
a_{11} & \cdots & a_{1\, n-1} \\
\vdots & \ddots & \vdots \\
a_{i-1\, 1}  & \cdots & a_{i-1\, n-1} \\
a_{i+1\, 1} & \cdots & a_{i+1\, n-1} \\
\vdots & \ddots & \vdots \\
a_{n1} & \cdots & a_{n\, n-1}
\end{vmatrix} =
\end{equation*}
\begin{equation*}
= \sum_{j=1}^n a_{ij} A_{ij}
\end{equation*}

Аналогично доказывается
\begin{equation*}
\begin{vmatrix}
a_{11} & a_{12} & \cdots & a_{1n} \\
a_{21} & a_{22} & \cdots & a_{2n} \\
\vdots & \vdots & \ddots & \vdots \\
a_{n1} & a_{n2} & \cdots & a_{nn}
\end{vmatrix} =
\sum_{i=1}^n a_{ij} A_{ij}
\end{equation*}
\end{proof}

\begin{consequent}[фальшивое разложение определителя]
Пусть дана квадратная матрица~$A = \|a_{ij}\|$ $n$"~го порядка, тогда
\begin{equation*}
\sum_{k=1}^n a_{ik} A_{jk} = \sum_{k=1}^n a_{ki} A_{kj} = 0, \ i \neq j
\end{equation*}
\end{consequent}
\begin{proof}
\begin{equation*}
\sum_{k=1}^n a_{ik} A_{jk} =
\begin{vmatrix}
a_{11} & a_{12} & \cdots & a_{1n} \\
\vdots & \vdots & \ddots & \vdots \\
a_{i1} & a_{i2} & \cdots & a_{in} \\
\vdots & \vdots & \ddots & \vdots \\
a_{i1} & a_{i2} & \cdots & a_{in} \\
\vdots & \vdots & \ddots & \vdots \\
a_{n1} & a_{n2} & \cdots & a_{nn}
\end{vmatrix} = 0 =
\begin{vmatrix}
a_{11} & \cdots & a_{1i} & \cdots & a_{1i} & \cdots & a_{1n} \\
a_{21} & \cdots & a_{2i} & \cdots & a_{2i} & \cdots & a_{2n} \\
\vdots & \ddots & \vdots & \ddots & \vdots & \ddots & \vdots \\
a_{n1} & \cdots & a_{ni} & \cdots & a_{ni} & \cdots & a_{nn}
\end{vmatrix} =
\sum_{k=1}^n a_{ki} A_{kj}
\end{equation*}
\end{proof}

\begin{statement}
Определитель транспонированной матрицы равен определителю исходной.
\end{statement}
\begin{proof}
\begin{equation*}
\begin{vmatrix}
a_{11} & a_{12} & \cdots & a_{1n} \\
a_{21} & a_{22} & \cdots & a_{2n} \\
\vdots & \vdots & \ddots & \vdots \\
a_{n1} & a_{n2} & \cdots & a_{nn}
\end{vmatrix}^T =
\begin{vmatrix}
a_{11} & a_{21} & \cdots & a_{n1} \\
a_{12} & a_{22} & \cdots & a_{n2} \\
\vdots & \vdots & \ddots & \vdots \\
a_{1n} & a_{2n} & \cdots & a_{nn}
\end{vmatrix} =
\sum_{j=1}^n a_{1j} A_{1j} =
\begin{vmatrix}
a_{11} & a_{12} & \cdots & a_{1n} \\
a_{21} & a_{22} & \cdots & a_{2n} \\
\vdots & \vdots & \ddots & \vdots \\
a_{n1} & a_{n2} & \cdots & a_{nn}
\end{vmatrix}
\end{equation*}
\end{proof}

\begin{theorem}[Лапл\'{а}са]
\index{Теорема!Лапласа}
Пусть дана квадратная матрица~$A$ $n$"~го порядка.
\begin{equation*}
\forall 0 < k < n, \ 1 \leqslant i_1 < i_2 < \ldots < i_k \leqslant n \
\det A = \sum_{1 \leqslant j_1 < \ldots < j_k \leqslant n}
(-1)^{i_1 + \ldots + i_k + j_1 + \ldots + j_k}
M_{j_1 \ldots j_k}^{i_1 \ldots i_k}
\overline M_{j_1 \ldots j_k}^{i_1 \ldots i_k}
\end{equation*}
\end{theorem}
\begin{proofmathind}
	\indbase При~$k = 1$ данная теорема эквивалентна теореме~\ref*{th:determinant_expansion}.
	\indstep Пусть теорема верна при~$k - 1$. Докажем её для~$k$.
	\begin{equation*}
	\det A = \sum_{1 \leqslant j_1 < \ldots < j_{k-1} \leqslant n}
	(-1)^{i_1 + \ldots + i_{k-1} + j_1 + \ldots + j_{k-1}}
	M_{j_1 \ldots j_{k-1}}^{i_1 \ldots i_{k-1}}
	\overline M_{j_1 \ldots j_{k-1}}^{i_1 \ldots i_{k-1}} =
	\end{equation*}
	\begin{equation*}
	\left| \text{Разложим каждый минор~$\overline M_{j_1 \ldots j_{k-1}}^{i_1 \ldots i_{k-1}}$ по строке~$A_{i_k}$, полагая, что $\Theta_{j_1 \ldots j_k}$~--- некоторое число} \right|
	\end{equation*}
	\begin{equation*}
	= \sum_{1 \leqslant j_1 < \ldots < j_k \leqslant n}
	\Theta_{j_1 \ldots j_k} \overline M_{j_1 \ldots j_k}^{i_1 \ldots i_k}
	\end{equation*}
	
	Найдём значение~$\Theta_{j_1 \ldots j_k}$.
	Заметим, что минор~$\overline M_{j_1 \ldots j_k}^{i_1 \ldots i_k}$ получается при разложении только миноров
	\begin{equation*}
	\overline M_{j_1 \ldots j_{s-1} j_{s+1} \ldots j_k}^{i_1 \ldots i_{k-1}}, \ s = 1, 2, \ldots, k
	\end{equation*}
	причём
	\begin{equation*}
	\overline M_{j_1 \ldots j_{s-1} j_{s+1} \ldots j_k}^{i_1 \ldots i_{k-1}} =
	(-1)^{i_k - (k - 1) + j_s - (s - 1)} a_{i_k j_s}
	\overline M_{j_1 \ldots j_k}^{i_1 \ldots i_k} + \ldots
	\end{equation*}
	где многоточием обозначены остальные слагаемые.
	
	Тогда
	\begin{equation*}
	\Theta_{j_1 \ldots j_k} =
	(-1)^{i_1 + \ldots + i_k + j_1 + \ldots + j_k}
	\sum_{s=1}^k (-1)^{k+s} M_{j_1 \ldots j_{s-1} j_{s+1} \ldots j_k}^{i_1 \ldots i_{k-1}} =
	(-1)^{i_1 + \ldots + i_k + j_1 + \ldots + j_k}
	M_{j_1 \ldots j_k}^{i_1 \ldots i_k}
	\end{equation*}
	\indend
\end{proofmathind}

\begin{theorem}
Если $A = \|a_{ij}\|, B = \|b_{ij}\|$~--- квадратные матрицы $n$"~го порядка, то $\det AB \opbr= \det A \cdot \det B$.
\end{theorem}
\begin{proof}
Пусть $O, E$~--- нулевая и единичная соответственно квадратные матрицы $n$-го порядка, $C \opbr= AB$.
Рассмотрим блочные матрицы
\begin{equation*}
\begin{Vmatrix}
A & O \\
-E & B
\end{Vmatrix}, \
\begin{Vmatrix}
A & C \\
-E & O
\end{Vmatrix}
\end{equation*}

Раскладывая первую матрицу по первым $n$~строкам, а вторую~--- по последним $n$~строкам, получим
\begin{equation*}
\begin{vmatrix}
A & O \\
-E & B
\end{vmatrix} =
|A| |B|, \
\begin{vmatrix}
A & C \\
-E & O
\end{vmatrix} =
(-1)^{1 + \ldots + 2n}|-E| |C| =
|C|
\end{equation*}

Тогда
\begin{equation*}
\det A \cdot \det B =
\begin{vmatrix}
A & O \\
-E & B
\end{vmatrix} =
\begin{vmatrix}
a_{11} & \cdots & a_{1n} & 0 & \cdots & 0 \\
\vdots & \ddots & \vdots & \vdots & \ddots & \vdots \\
a_{n1} & \cdots & a_{nn} & 0 & \cdots & 0 \\
-1 & \cdots & 0 & b_{11} & \cdots & b_{1n} \\
\vdots & \ddots & \vdots & \vdots & \ddots & \vdots \\
0 & \cdots & -1 & b_{n1} & \cdots & b_{nn}
\end{vmatrix} =
\end{equation*}
\begin{equation*}
= \begin{vmatrix}
a_{11} & \cdots & a_{1n} & \sum\limits_{i=1}^n a_{1i} b_{i1} & \cdots & \sum\limits_{i=1}^n a_{1i} b_{in} \\
\vdots & \ddots & \vdots & \vdots & \ddots & \vdots \\
a_{n1} & \cdots & a_{nn} & \sum\limits_{i=1}^n a_{ni} b_{i1} & \cdots & \sum\limits_{i=1}^n a_{ni} b_{in} \\
-1 & \cdots & 0 & 0 & \cdots & 0 \\
\vdots & \ddots & \vdots & \vdots & \ddots & \vdots \\
0 & \cdots & -1 & 0 & \cdots & 0
\end{vmatrix} =
\begin{vmatrix}
A & C \\
-E & O
\end{vmatrix} =
\det C
\end{equation*}
\end{proof}
\section{Ранг матрицы}
Строка (столбец) матрицы называется \textbf{линейно зависимой}, если она является линейной комбинацией остальных строк (столбцов), иначе~--- \textbf{линейно независимой}.

\index{Ранг} \textbf{Рангом матрицы} называется максимальное количество её линейно независимых строк.

\index{Минор!базисный} Минор наибольшего порядка, отличный от нуля, называется \textbf{базисным}.

\begin{theorem}
Ранг матрицы равен порядку базисного минора.
\end{theorem}
\begin{proof}
Пусть $A = \|a_{ij}\|$~--- квадратная матрица $n$-го порядка, $M_k$~--- базисный минор $k$\nobreakdash-го порядка.
При перестановке строк и столбцов минора равенство с нулём сохраняется, значит, без ограничения общности можно считать, что
\begin{equation*}
M_k =
\begin{vmatrix}
a_{11} & a_{12} & \cdots & a_{1k} \\
a_{21} & a_{22} & \cdots & a_{2k} \\
\vdots & \vdots & \ddots & \vdots \\
a_{k1} & a_{k2} & \cdots & a_{kk}
\end{vmatrix}
\end{equation*}

$M_k \neq 0$, значит, строки~$A_1, \ldots, A_k$ линейно независимы. Пусть $M_{k+1}$~--- минор $k + 1$\nobreakdash-го порядка:
\begin{equation*}
M_{k+1} =
\begin{vmatrix}
a_{11} & a_{12} & \cdots & a_{1k} & a_{1j} \\
a_{21} & a_{22} & \cdots & a_{2k} & a_{2j} \\
\vdots & \vdots & \ddots & \vdots & \vdots \\
a_{k1} & a_{k2} & \cdots & a_{kk} & a_{kj} \\
a_{i1} & a_{i2} & \cdots & a_{ik} & a_{ij}
\end{vmatrix} = 0
\end{equation*}
т.\,к. $M_k$~--- базисный минор.
Тогда
\begin{equation*}
\forall j \ a_{1j} A_{1j} + a_{2j} A_{2j} + \ldots + a_{kj} A_{kj} + a_{ij} A_{ij} = 0 \lAnd A_{ij} = M_k \neq 0 \Rightarrow
\end{equation*}
\begin{equation*}
\Rightarrow a_{ij} = -\frac{A_{1j}}{A_{ij}} a_{1j} - \frac{A_{2j}}{A_{ij}} a_{2j} - \ldots - \frac{A_{kj}}{A_{ij}} a_{kj}
\end{equation*}
где $A_{1j}, \ldots, A_{kj}, A_{ij}$~--- алгебраические дополнения $a_{1j}, \ldots, a_{kj}, a_{ij}$ в миноре~$M_{k+1}$.
$A_{1j}, \ldots, A_{kj}, A_{ij}$ не зависят от~$j$, тогда $A_i$~--- линейная комбинация $A_1, \ldots, A_k$, значит, $k$~--- ранг матрицы $A$.
\end{proof}

Рангом матрицы по строкам (столбцам) называется максимальное количество её линейно независимых строк (столбцов).

\begin{consequent}
Ранг матрицы по строкам равен рангу матрицы по столбцам.
\end{consequent}%
Для доказательства достаточно заметить, что определитель транспонированной матрицы равен определителю исходной.
\section{Элементарные преобразования матриц}
\textbf{Элементарными преобразованиями} называются следующие операции над матрицей:
\begin{itemize}
	\item Перестановка строк матрицы~--- преобразование I типа
	\item Умножение строки на~$\lambda \neq 0$~--- преобразование II типа
	\item Прибавление к строке матрицы другой строки, умноженной на~$\lambda$.
\end{itemize}

Аналогично определяются элементарные преобразования над столбцами.

\begin{theorem}
Элементарные преобразования матрицы не изменяют её ранг.
\end{theorem}
\begin{proof}
Для доказательства достаточно показать, что в результате элементарных преобразований равенство определителя с нулём сохраняется.
\begin{itemize}
	\item Перестановка строк матрицы изменяет только знак определителя.
	\item Умножение строки матрицы на ненулевое число приводит к умножению определителя на это~же число.
	\item Прибавление к строке матрицы другой строки, умноженной на некоторое число, не изменяет определитель.
\end{itemize}
\end{proof}

Матрица~$A$ имеет \textbf{ступенчатый вид}, если:
\begin{itemize}
	\item все нулевые строки стоят последними;
	\item для любой ненулевой строки~$A_p$ верно, что $\forall i > p, \ j \leqslant q \ a_{ij} = 0$, где $a_{pq}$~--- первый ненулевой элемент строки~$A_p$.
\end{itemize}

\begin{theorem}
Любую матрицу путём элементарных преобразований только над строками можно привести к ступенчатому виду.
\end{theorem}
\begin{proof}
Приведём алгоритм, преобразующий любую матрицу~$\|a_{ij}\|_{\begin{smallmatrix}
i = \overline{1,m} \\
j = \overline{1,n}
\end{smallmatrix}}$ к ступенчатому виду путём элементарных преобразований только над строками.
В~качестве текущего элемента возьмём $a_{11}$.
\begin{enumerate}
	\item Если текущий элемент~$a_{ij} = 0$, то переходим к шагу~2, иначе к каждой строке~$A_k$, где $k \opbr= i + 1, i + 2, \ldots, n$, добавляем строку~$-\frac{a_{kj}}{a_{ij}} A_i$.
	Если $i = m$ или $j = n$, то матрица приведена к ступенчатому виду, иначе выбираем новый текущий элемент~$a_{i+1\, j+1}$ и повторяем шаг~1.
	\item Просматриваем элементы матрицы, расположенные под текущим элементом~$a_{ij}$.
	Если $a_{kj} \neq 0$, то меняем местами строки $A_i$ и $A_k$ и переходим к шагу~1, иначе переходим к шагу~3.
	\item Пусть $a_{ij}$~--- текущий элемент.
	Если $j = n$, то матрица приведена к ступенчатому виду, иначе выбираем новый текущий элемент~$a_{i\, j + 1}$ и переходим к шагу~1.
\end{enumerate}

Матрица имеет конечные размеры, а положение текущего элемента смещается как минимум на~$1$ столбец вправо за не более, чем $3$~шага, поэтому алгоритм закончит работу за не более, чем $3n$~шагов.
\end{proof}
\section{Обратные матрицы}
Матрица~$B$ называется \textbf{левой обратной} к квадратной матрице~$A$, если $BA = E$.

Матрица~$C$ называется \textbf{правой обратной} к квадратной матрице~$A$, если $AC = E$.

Заметим, что обе матрицы $B$ и $C$~--- квадратные того же порядка, что и~$A$.

\begin{statement}
Если существуют левая и правая обратные к $A$ матрицы $B$ и $C$, то они совпадают.
\end{statement}
\begin{proof}
$B = BE = BAC = EC = C$.
\end{proof}

\index{Матрица!обратная} Т.\,о., матрица~$A^{-1}$ называется \textbf{обратной} к матрице~$A$, если $A^{-1} A = A A^{-1} = E$.

Приведём методы вычисления обратной матрицы.

\begin{theorem}[метод присоединённой матрицы]
\label{th:inverse_matrix_by_matrix_of_cofactors}
Пусть даны матрицы $A = \|a_{ij}\|_{\begin{smallmatrix}
i = \overline{1, n} \\
j = \overline{1, n}
\end{smallmatrix}}$, $\hat A = \|A_{ij}\|_{\begin{smallmatrix}
i = \overline{1, n} \\
j = \overline{1, n}
\end{smallmatrix}}$, где $A_{ij}$~--- алгебраическое дополнение $a_{ij}$.

Если $|A| \neq 0$, то
\begin{equation*}
A^{-1} = \frac1{|A|} \cdot \hat A^T
\end{equation*}
\end{theorem}
\begin{proof}
\begin{equation*}
A \cdot \left( \frac1{|A|} \cdot \hat A^T \right) =
\frac1{|A|} \cdot A \cdot \hat A^T =
\frac1{|A|} \cdot
\begin{Vmatrix}
a_{11} & a_{12} & \cdots & a_{1n} \\
a_{21} & a_{22} & \cdots & a_{2n} \\
\vdots & \vdots & \ddots & \vdots \\
a_{n1} & a_{n2} & \cdots & a_{nn}
\end{Vmatrix} \cdot
\begin{Vmatrix}
A_{11} & A_{21} & \cdots & A_{n1} \\
A_{12} & A_{22} & \cdots & A_{n2} \\
\vdots & \vdots & \ddots & \vdots \\
A_{1n} & A_{2n} & \cdots & A_{nn}
\end{Vmatrix} =
\end{equation*}
\begin{equation*}
= \frac1{|A|} \cdot
\begin{Vmatrix}
\sum\limits_{k=1}^n a_{1k} A_{1k} & \sum\limits_{k=1}^n a_{1k} A_{2k} & \cdots & \sum\limits_{k=1}^n a_{1k} A_{nk} \\
\sum\limits_{k=1}^n a_{2k} A_{1k} & \sum\limits_{k=1}^n a_{2k} A_{2k} & \cdots & \sum\limits_{k=1}^n a_{2k} A_{nk} \\
\vdots & \vdots & \ddots & \vdots \\
\sum\limits_{k=1}^n a_{nk} A_{1k} & \sum\limits_{k=1}^n a_{nk} A_{2k} & \cdots & \sum\limits_{k=1}^n a_{nk} A_{nk} \\
\end{Vmatrix} =
\frac1{|A|} \cdot
\begin{Vmatrix}
|A| & 0 & \cdots & 0 \\
0 & |A| & \cdots & 0 \\
\vdots & \vdots & \ddots & \vdots \\
0 & 0 & \cdots & |A|
\end{Vmatrix} = E \Rightarrow
\end{equation*}
\begin{equation*}
\Rightarrow \frac1{|A|} \cdot \hat A^T = A^{-1}
\end{equation*}
\end{proof}

\begin{theorem}[метод Гаусса"--~Жордана]
\index{Метод!Гаусса---Жордана}
Пусть дана невырожденная матрица $A = \|a_{ij}\|_{\begin{smallmatrix}
i = \overline{1, n} \\
j = \overline{1, n}
\end{smallmatrix}}$.

Присоединим к~ней единичную матрицу:
\begin{equation*}
B = \begin{Vmatrix}
a_{11} & a_{12} & \cdots & a_{1n} & 1 & 0 & \cdots & 0 \\
a_{21} & a_{22} & \cdots & a_{2n} & 0 & 1 & \cdots & 0 \\
\vdots & \vdots & \ddots & \vdots & \vdots & \vdots & \ddots & \vdots \\
a_{n1} & a_{n2} & \cdots & a_{nn} & 0 & 0 & \cdots & 1
\end{Vmatrix}
\end{equation*}
и с помощью элементарных преобразований только над строками полученной матрицы (или только над столбцами) приведём её левую часть к единичной матрице.
Тогда правая часть будет обратной к~$A$ матрицей.
\end{theorem}
\begin{proof}
Каждое элементарное преобразование квадратной матрицы~$A$ эквивалентно её умножению на некоторую матрицу~$T$ того~же порядка:
\begin{itemize}
	\item \begin{equation*}
	\begin{Vmatrix}
	1 & 0 & \ldots & 0 & \ldots & 0 & \ldots & 0 \\
	0 & 1 & \ldots & 0 & \ldots & 0 & \ldots & 0 \\
	\vdots & \vdots & \ddots & \vdots & \ddots & \vdots & \ddots & \ldots \\
	0 & 0 & \ldots & 0 & \ldots & 1 & \ldots & 0 \\
	\vdots & \vdots & \ddots & \vdots & \ddots & \vdots & \ddots & \ldots \\
	0 & 0 & \ldots & 1 & \ldots & 0 & \ldots & 0 \\
	\vdots & \vdots & \ddots & \vdots & \ddots & \vdots & \ddots & \ldots \\
	0 & 0 & \ldots & 0 & \ldots & 0 & \ldots & 1
	\end{Vmatrix} \cdot
	\begin{Vmatrix}
	a_{11} & a_{12} & \ldots & a_{1i} & \ldots & a_{1j} & \ldots & a_{1n} \\
	a_{21} & a_{22} & \ldots & a_{2i} & \ldots & a_{2j} & \ldots & a_{2n} \\
	\vdots & \vdots & \ddots & \vdots & \ddots & \vdots & \ddots & \ldots \\
	a_{i1} & a_{i2} & \ldots & a_{ii} & \ldots & a_{ij} & \ldots & a_{in} \\
	\vdots & \vdots & \ddots & \vdots & \ddots & \vdots & \ddots & \ldots \\
	a_{j1} & a_{j2} & \ldots & a_{ji} & \ldots & a_{jj} & \ldots & a_{jn} \\
	\vdots & \vdots & \ddots & \vdots & \ddots & \vdots & \ddots & \ldots \\
	a_{1n} & a_{2n} & \ldots & a_{in} & \ldots & a_{jn} & \ldots & a_{nn}
	\end{Vmatrix} =
	\end{equation*}
	\begin{equation*}
	= \begin{Vmatrix}
	a_{11} & a_{12} & \ldots & a_{1i} & \ldots & a_{1j} & \ldots & a_{1n} \\
	a_{21} & a_{22} & \ldots & a_{2i} & \ldots & a_{2j} & \ldots & a_{2n} \\
	\vdots & \vdots & \ddots & \vdots & \ddots & \vdots & \ddots & \ldots \\
	a_{j1} & a_{j2} & \ldots & a_{ji} & \ldots & a_{jj} & \ldots & a_{jn} \\
	\vdots & \vdots & \ddots & \vdots & \ddots & \vdots & \ddots & \ldots \\
	a_{i1} & a_{i2} & \ldots & a_{ii} & \ldots & a_{ij} & \ldots & a_{in} \\
	\vdots & \vdots & \ddots & \vdots & \ddots & \vdots & \ddots & \ldots \\
	a_{1n} & a_{2n} & \ldots & a_{in} & \ldots & a_{jn} & \ldots & a_{nn}
	\end{Vmatrix}
	\end{equation*}
	
	\item \begin{equation*}
	\begin{Vmatrix}
	1 & 0 & \ldots & 0 & \ldots & 0 \\
	0 & 1 & \ldots & 0 & \ldots & 0 \\
	\vdots & \vdots & \ddots & \vdots & \ddots & \ldots \\
	0 & 0 & \ldots & \lambda & \ldots & 0 \\
	\vdots & \vdots & \ddots & \vdots & \ddots & \ldots \\
	0 & 0 & \ldots & 0 & \ldots & 1
	\end{Vmatrix} \cdot
	\begin{Vmatrix}
	a_{11} & a_{12} & \ldots & a_{1i} & \ldots & a_{1n} \\
	a_{21} & a_{22} & \ldots & a_{2i} & \ldots & a_{2n} \\
	\vdots & \vdots & \ddots & \vdots & \ddots & \ldots \\
	a_{i1} & a_{i2} & \ldots & a_{ii} & \ldots & a_{in} \\
	\vdots & \vdots & \ddots & \vdots & \ddots & \ldots \\
	a_{1n} & a_{2n} & \ldots & a_{in} & \ldots & a_{nn}
	\end{Vmatrix} =
	\begin{Vmatrix}
	a_{11} & a_{12} & \ldots & a_{1i} & \ldots & a_{1n} \\
	a_{21} & a_{22} & \ldots & a_{2i} & \ldots & a_{2n} \\
	\vdots & \vdots & \ddots & \vdots & \ddots & \ldots \\
	\lambda a_{i1} & \lambda a_{i2} & \ldots & \lambda a_{ii} & \ldots & \lambda a_{in} \\
	\vdots & \vdots & \ddots & \vdots & \ddots & \ldots \\
	a_{1n} & a_{2n} & \ldots & a_{in} & \ldots & a_{nn}
	\end{Vmatrix}
	\end{equation*}
	
	\item \begin{equation*}
	\begin{Vmatrix}
	1 & 0 & \ldots & 0 & \ldots & 0 & \ldots & 0 \\
	0 & 1 & \ldots & 0 & \ldots & 0 & \ldots & 0 \\
	\vdots & \vdots & \ddots & \vdots & \ddots & \vdots & \ddots & \ldots \\
	0 & 0 & \ldots & 1 & \ldots & 0 & \ldots & 0 \\
	\vdots & \vdots & \ddots & \vdots & \ddots & \vdots & \ddots & \ldots \\
	0 & 0 & \ldots & \lambda & \ldots & 1 & \ldots & 0 \\
	\vdots & \vdots & \ddots & \vdots & \ddots & \vdots & \ddots & \ldots \\
	0 & 0 & \ldots & 0 & \ldots & 0 & \ldots & 1
	\end{Vmatrix} \cdot
	\begin{Vmatrix}
	a_{11} & a_{12} & \ldots & a_{1i} & \ldots & a_{1j} & \ldots & a_{1n} \\
	a_{21} & a_{22} & \ldots & a_{2i} & \ldots & a_{2j} & \ldots & a_{2n} \\
	\vdots & \vdots & \ddots & \vdots & \ddots & \vdots & \ddots & \ldots \\
	a_{i1} & a_{i2} & \ldots & a_{ii} & \ldots & a_{ij} & \ldots & a_{in} \\
	\vdots & \vdots & \ddots & \vdots & \ddots & \vdots & \ddots & \ldots \\
	a_{j1} & a_{j2} & \ldots & a_{ji} & \ldots & a_{jj} & \ldots & a_{jn} \\
	\vdots & \vdots & \ddots & \vdots & \ddots & \vdots & \ddots & \ldots \\
	a_{1n} & a_{2n} & \ldots & a_{in} & \ldots & a_{jn} & \ldots & a_{nn}
	\end{Vmatrix} =
	\end{equation*}
	\begin{equation*}
	= \begin{Vmatrix}
	a_{11} & a_{12} & \ldots & a_{1i} & \ldots & a_{1j} & \ldots & a_{1n} \\
	a_{21} & a_{22} & \ldots & a_{2i} & \ldots & a_{2j} & \ldots & a_{2n} \\
	\vdots & \vdots & \ddots & \vdots & \ddots & \vdots & \ddots & \ldots \\
	a_{i1} & a_{i2} & \ldots & a_{ii} & \ldots & a_{ij} & \ldots & a_{in} \\
	\vdots & \vdots & \ddots & \vdots & \ddots & \vdots & \ddots & \ldots \\
	\lambda a_{i1} + a_{j1} & \lambda a_{i2} + a_{j2} & \ldots & \lambda a_{ii} + a_{ji} & \ldots & \lambda a_{ij} + a_{jj} & \ldots & \lambda a_{in} + a_{jn} \\
	\vdots & \vdots & \ddots & \vdots & \ddots & \vdots & \ddots & \ldots \\
	a_{1n} & a_{2n} & \ldots & a_{in} & \ldots & a_{jn} & \ldots & a_{nn}
	\end{Vmatrix}
	\end{equation*}
\end{itemize}

Т.\,о., результат последовательных элементарных преобразований матрицы~$B$ можно представить в виде $T_k \opbr\cdot \ldots \opbr\cdot T_1 \opbr\cdot B$.
Рассматривая отдельно левую и правую части матрицы~$B$, получим:
\begin{equation*}
\begin{cases}
T_k \cdot \ldots \cdot T_1 \cdot A = E \\
T_k \cdot \ldots \cdot T_1 \cdot E = A_1
\end{cases}
\Rightarrow A_1 \cdot A = E \Rightarrow A_1 = A^{-1}
\end{equation*}
\end{proof}

\begin{theorem}
Если $A$~--- квадратная матрица, то $\exists A^{-1} \Leftrightarrow \det A \neq 0$.
\end{theorem}
\begin{proof}
\begin{enumerate}
	\item $\Rightarrow$. $A \cdot A^{-1} = E \Rightarrow
	\det A \cdot \det A^{-1} = 1 \Rightarrow
	\det A \neq 0$
	
	\item $\Leftarrow$. $\exists A^{-1}$ по теореме~\ref*{th:inverse_matrix_by_matrix_of_cofactors}.
\end{enumerate}
\end{proof}
\chapter{Теория многочленов}
\section{Многочлены от одной переменной}
\subsection{Деление многочленов}
\textbf{Общим делителем} многочленов $f(x)$ и $g(x)$ называется многочлен~$h(x)$, на который и~$f$, и~$g$ делятся без остатка: $f = ph$, $g = qh$.

\textbf{Наибольшим} называется общий делитель наибольшей степени и обозначается $\NOD$.

\begin{theorem}[алгоритм Евклида]
\index{Алгоритм!Евклида}
Любые два многочлена имеют единственный $\NOD$.
\end{theorem}
\begin{proof}
Будем делить многочлены следующим образом:
\begin{equation*}
f = q_1 g + r_1, \
g = q_2 r_1 + r_2, \
r_1 = q_3 r_2 + r_3, \ \ldots,
\end{equation*}
\begin{equation*}
r_{n-1} = q_{n+1} r_n + r_{n+1}, \
r_n = q_{n+2} r_{n+1} + r_{n+2} = q_{n+2} r_{n+1},
\end{equation*}
\begin{equation*}
\deg g > \deg r_1 > \deg r_2 > \ldots > \deg r_{n+1} > \deg r_{n+2} = -\infty
\end{equation*}

Докажем, что $r_{n+1}$~--- общий делитель $f$ и $g$.
\begin{equation*}
r_n \mult r_{n+1} \Rightarrow
r_{n-1} \mult r_{n+1} \Rightarrow
\ldots \Rightarrow
r_1 \mult r_{n+1} \Rightarrow
g \mult r_{n+1} \Rightarrow
f \mult r_{n+1}
\end{equation*}

Докажем, что $\forall h \ (f \mult h \lAnd g \mult h \Rightarrow r_{n+1} \mult h)$.
\begin{equation*}
f \mult h \lAnd g \mult h \Rightarrow r_1 \mult h, \
g \mult h \lAnd r_1 \mult h \Rightarrow r_2 \mult h, \
r_1 \mult h \lAnd r_2 \mult h \Rightarrow r_3 \mult h, \ \ldots, \
r_{n-1} \mult h \lAnd r_n \mult h \Rightarrow r_{n+1} \mult h
\end{equation*}

Значит, $r_{n+1} = \NOD(f, g)$.
\end{proof}

\subsection{Корень многочлена}
\begin{theorem}[основная теорема алгебры]
\label{th:fundamental_th_of_algebra}
\index{Теорема!основная т. алгебры}
Если $f(x)$~--- многочлен, отличный от константы, то он имеет хотя~бы один комплексный корень.
\end{theorem}%
Доказательство теоремы слишком сложно, поэтому здесь не приводится.

\begin{consequent}
\label{conseq:n_roots_of_polynomial}
Многочлен $n$\nobreakdash-й степени имеет ровно $n$ комплексных корней с учётом их кратности.
\end{consequent}
\begin{proof}
Пусть $f(x)$~--- многочлен $n$\nobreakdash-й степени.
По основной теореме алгебры он имеет корень~$a$, тогда по следствию~\ref*{conseq:division_by_binomial} $f(x) = g(x)(x - a)$, где $g(x)$~--- многочлен степени $n - 1$, который также имеет корень.
Будем повторять деление до~тех пор, пока не получим константу.
Т.\,о., получим $n$~корней.
\end{proof}

\begin{consequent}
Любой многочлен~$f(x)$ $n$\nobreakdash-й степени представим в~виде
\begin{equation*}
f(x) = a(x - x_0)(x - x_1) \cdot \ldots \cdot (x - x_{n-1})
\end{equation*}
где $a$~--- число, $x_0, \ldots, x_{n-1}$~--- корни $f(x)$.
\end{consequent}

\begin{lemma}
Если $f(x)$~--- многочлен с действительными коэффициентами, $z \in \mathbb C$, то $\overline{f(z)} = f(\overline z)$.
\end{lemma}
\begin{proof}
Пусть $z_1 = a_1 + b_1 i$, $z_2 = a_2 + b_2 i$, $a_1, b_1, a_2, b_2 \in \mathbb R$.
Многочлен строится при помощи операций сложения и умножения, поэтому достаточно доказать следующее:
\begin{enumerate}
	\item $\overline{z_1 + z_2} = \overline{z_1} + \overline{z_2}$
	\begin{equation*}
	\overline{z_1 + z_2} = \overline{(a_1 + a_2) + (b_1 + b_2)i} = (a_1 + a_2) - (b_1 + b_2)i =
	(a_1 - b_1 i) + (a_2 - b_2 i) = \overline{z_1} + \overline{z_2}
	\end{equation*}
	
	\item $\overline{z_1 z_2} = \overline{z_1} \cdot \overline{z_2}$
	\begin{equation*}
	\overline{z_1 z_2} = \overline{(a_1 a_2 - b_1 b_2) + (a_1 b_2 + a_2 b_1)i} =
	(a_1 a_2 - b_1 b_2) - (a_1 b_2 + a_2 b_1)i = (a_1 - b_1 i)(a_2 - b_2 i) =
	\overline{z_1} \cdot \overline{z_2}
	\end{equation*}
\end{enumerate}
Тогда $\overline{a_n z^n + \ldots + a_1 z + a_0} = a_n \overline z^n + \ldots + a_1 \overline z + a_0$ при $a_0, a_1, \ldots, a_n \in \mathbb R$.
\end{proof}

\begin{theorem}
\label{th:polynomial_factorization}
Любой многочлен с действительными коэффициентами можно разложить на линейные и квадратные множители с действительными коэффициентами.
\end{theorem}
\begin{proof}
Пусть $f(x)$~--- многочлен с действительными коэффициентами, тогда если $f(z) = 0$, то $f(\overline z) \opbr= \overline{f(z)} \opbr= \overline 0 \opbr= 0$.
Значит, если $a + bi$~--- корень~$f(x)$, то $a - bi$~--- тоже корень~$f(x)$.
Имеем:
\begin{equation*}
f(x) = a \prod_{j=1}^{m} (x - x_j) \cdot \prod_{j=1}^{n} (x - (a_j + b_j i))(x - (a_j - b_j i)) =
a \prod_{j=1}^{m} (x - x_j) \cdot \prod_{j=1}^{n} (x^2 - 2a_j x + a_j^2 + b_j^2)
\end{equation*}
где $a, x_1, \ldots, x_m, a_1, \ldots, a_n, b_1, \ldots, b_n \in \mathbb R$,
$x_1, \ldots, x_m, a_1 + b_1 i, \ldots, a_n + b_n i$~--- корни~$f(x)$.
\end{proof}

\begin{theorem}[формулы Виета]
\index{Формула!Виета}
Пусть 
\begin{equation}
\label{eq:formulas_in_Vieta_theorem}
f(x) = a_n x^n + a_{n-1} x^{n-1} + \ldots + a_1 x + a_0 = a_n(x - x_0)(x - x_1) \cdot \ldots \cdot (x - x_{n-1})
\end{equation}
тогда
\begin{equation*}
a_{n-1} = -a_n \sum_{i=0}^{n-1} x_i, \
a_{n-2} = a_n \sum_{i=0}^{n-1} \sum_{j=i+1}^{n-1} x_i x_j, \
a_{n-3} = -a_n \sum_{i=0}^{n-1} \sum_{j=i+1}^{n-1} \sum_{k=j+1}^{n-1} x_i x_j x_k, \ \ldots,
\end{equation*}
\begin{equation*}
a_2 = (-1)^{n-1} \cdot a_n \sum_{i=0}^{n-1} x_0 x_1 \cdot \ldots \cdot x_{i-1} x_{i+1} \cdot \ldots \cdot x_{n-1}, \
a_1 = (-1)^n \cdot a_n x_0 x_1 \cdot \ldots \cdot x_{n-1}
\end{equation*}
\end{theorem}%
Для доказательства достаточно раскрыть скобки в~правой части равенства (\ref*{eq:formulas_in_Vieta_theorem}).

\begin{theorem}
Пусть на плоскости даны $n + 1$~точек, никакие две из которых не лежат на прямой, паралелльной оси ординат, тогда через них проходит единственная кривая $n$-го порядка.
\end{theorem}
\begin{proof}
Пусть данные точки заданы координатами $(a_0, b_0), (a_1, b_1), \ldots, (a_n, b_n)$.
\begin{enumerate}
	\item Докажем существование.
	Рассмотрим многочлен~$f(x)$, называемый \textbf{интерполяционным многочленом Лагранжа}:
	\begin{equation*}
	f(x) = \sum_{i=0}^n b_i \frac
	{(x - a_0) \cdot \ldots \cdot (x - a_{i-1})(x - a_{i+1}) \cdot \ldots \cdot (x - a_n)}
	{(a_i - a_0) \cdot \ldots \cdot (a_i - a_{i-1})(a_i - a_{i+1}) \cdot \ldots \cdot (a_i - a_n)}
	\end{equation*}
	
	Докажем, что кривая, задаваемая функцией~$f$, проходит через все данные точки.
	Рассмотрим точку~$(a_k, b_k)$.
	Подставим $x = a_k$, тогда $k$-е (считая с нуля) слагаемое равно $b_k$, а остальные~--- 0.
	
	\item Докажем единственность.
	Предположим, что существуют многочлены $f(x)$ и $g(x)$ $n$-й степени такие, что $f(a_i) = g(a_i) = b_i$, где $i = 0, 1, \ldots, n$.
	Рассмотрим $h(x) = f(x) - g(x) \Rightarrow \deg h \leqslant n \Rightarrow$ $h(x)$ имеет не более $n$~корней.
	При этом $h(x) = 0$ в $n + 1$~точках $\Rightarrow$ $h(x)$ тождественно равен нулю $\Rightarrow$ $f(x) = g(x)$.
\end{enumerate}
\end{proof}
\section{Многочлены от нескольких переменных}
\index{Многочлен!от нескольких переменных}
\begin{enumerate}
	\item В~многочлене $a_n x^n + a_{n-1} x^{n-1} + \ldots + a_0$ подставим $a_i = P_i(y)$, $i = 0, 1, \ldots, n$~--- многочлен от~$y$.
	Получим многочлен от $x$ и $y$.
	
	\item Пусть имеем многочлен от $n$~переменных.
	Подставим вместо его коэффициентов многочлен от одной переменной, получим многочлен от $n + 1$~переменных.
\end{enumerate}

Одночлены многочлена будем записывать в~лексикографическом порядке степеней переменных (члены с б\'{о}ль\-ши\-ми степенями идут раньше).

\begin{statement}
Старший член произведения многочленов равен произведению старших членов множителей.
\end{statement}
\begin{proof}
Перемножая члены с наибольшими показателями старшей переменной, получим член с наибольшим показателем при этой переменной.
Проведя аналогичные рассуждения для остальных переменных, придём к выводу, что полученный член является старшим.
\end{proof}

Аналогично доказывается следующее утверждение.
\begin{statement}
Младший член произведения многочленов равен произведению младших членов множителей.
\end{statement}
\chapter{Теория множеств}
\index{Инъекция} Функция $f \colon A \to B$ называется \textbf{инъективной (инъекцией)}, если
$\forall x, y \in A	\ \allowbreak (x \neq y \Rightarrow f(x) \neq f(y))$.

\index{Сюръекция} Функция $f \colon A \to B$ называется \textbf{сюръективной (сюръекцией)}, если
$\forall b \in B \ \allowbreak \exists a \in A \colon \allowbreak f(a) = b$.

\index{Биекция} Функция $f \colon A \to B$ называется \textbf{биективной (биекцией)}, если она инъективная и сюръективная.
\section{Мощность множеств}
Множества $A$ и $B$ называются \textbf{равномощными (имеют одинаковую мощность)}, если существует биекция~$f \colon A \to B$, иначе~--- \textbf{неравномощными}.

Для конечных множеств это означает, что у них одинаковое количество элементов.

\index{Мощность} \textbf{Мощностью конечного множества~$A$} называется количество его элементов и обозначается $|A|$.

Множество всех подмножеств множества~$A$ обозначается $\mathcal P(A) = \{ x \mid x \subseteq A \}$.

Множество всех подмножеств множества~$A$ мощности~$k$ обозначается $\mathcal P_k(A) = \{ x \subseteq A \mid |x| = k \}$.

\begin{theorem}[Кантора]
\index{Теорема!Кантора}
Множества $A$ и $\mathcal P(A)$ не~равномощны.
\end{theorem}
\begin{proofcontra}
Пусть $f \colon A \to \mathcal P(A)$~--- биекция. Рассмотрим множество
\begin{equation*}
X = \{ a \in A \mid a \notin f(a) \} \Rightarrow X \subset A \Rightarrow X \in \mathcal P(A)
\end{equation*}

$f$~--- биекция, тогда $\exists b \in A \colon f(b) = X$.
Возможны два случая:
\begin{enumerate}
	\item Пусть $b \in X \Rightarrow b \in f(b) \Rightarrow b \notin X$.
	Противоречие.
	\item Пусть $b \notin X \Rightarrow b \in f(b) \Rightarrow b \in X$.
	Противоречие.
\end{enumerate}

В~обоих случаях получили противоречие.
\end{proofcontra}

\begin{theorem}
Пусть дано множество~$A \colon |A| = n$, тогда $|\mathcal P_k(A)| = C_n^k$.
\end{theorem}
\begin{proofmathind}
	\indbase $n = 0$:
	\begin{equation*}
	|A| = 0 \Rightarrow A = \varnothing \Rightarrow \mathcal P(A) = \{ \varnothing \} \Rightarrow
	|\mathcal P_0(A)| = 1 = C_0^0
	\end{equation*}
	\indstep Пусть теорема верна для $n$.
	Докажем её для $n + 1$.
	Пусть $X \subset A$, $|X| = k$, $a \in A$.
	Подсчитаем количество таких $X$.
	Возможны два случая:
	\begin{enumerate}
		\item Пусть $a \notin X \Rightarrow X \subset A \setminus \{ a \}$, тогда таких $X$ $C_n^k$.
		\item Пусть $a \in X$, тогда таких $X$ столько~же, сколько множеств
		$X \setminus \{ a \} \subset A \setminus \{ a \}$, т.\,е.~$C_n^{k-1}$.
	\end{enumerate}

	Тогда $|\mathcal P(A)| = C_n^{k-1} + C_n^k = C_{n+1}^k$. \indend
\end{proofmathind}

\subsection{Мощность числовых множеств}
Множество называется \textbf{счётным}, если оно равномощно множеству натуральных чисел.
Бесконечное множество, не являющееся счётным, называется \textbf{несчётным}.

\begin{statement}
$\mathbb Z$ счётно.
\end{statement}
\begin{proof}
Построим биекцию~$f \colon \mathbb Z \to \mathbb N$:
\begin{equation*}
f(n) =
\begin{cases}
-2n - 1, \ n < 0 \\
2n, \ n \geqslant 0
\end{cases}
\end{equation*}

Тогда $|\mathbb Z| = |\mathbb N|$.
\end{proof}

\begin{statement}
$\mathbb Q$ счётно.
\end{statement}
\begin{proof}
Составим таблицу, в~верхней строке которой стоят $p_i \in \mathbb Z$, где $i = 1, 2, \ldots$, в~левом столбце~--- $q_j \in \mathbb N$, где $j = 1, 2, \ldots$, а на пересечении столбца и строки~--- $\frac{p_i}{q_j}$.
Обходя таблицу в~указанном порядке, будем нумеровать очередной элемент, только если он не встречался ранее:
\begin{center}
\begin{tikzpicture}
\matrix (table) [matrix of math nodes,
	column sep = 3 mm,
	row sep = 3 mm]
{
  & 0 & -1 & 1 & -2 & \cdots \\
1 & |[draw,circle]| 0 & |[draw,circle]| -1 & |[draw,circle]| 1 & |[draw,circle]| -2 & \cdots \\
2 & 0 & |[draw,circle]| -\dfrac12 & |[draw,circle]| \dfrac12 & -1 & \cdots \\
3 & 0 & |[draw,circle]| -\dfrac13 & |[draw,circle]| \dfrac13 & \ddots \\
4 & 0 & |[draw,circle]| -\dfrac14 & \ddots \\
\vdots & \vdots & \vdots \\
};

\draw (table.west |- table-1-6.south east) -- (table-1-6.south east);
\draw (table.north -| table-6-1.south east) -- (table-6-1.south east);

{ [start chain, every on chain/.style = {join = by ->}]
\chainin (table-2-2); \chainin (table-2-3); \chainin (table-3-2);
\chainin (table-4-2); \chainin (table-3-3); \chainin (table-2-4);
\chainin (table-2-5); \chainin (table-3-4); \chainin (table-4-3); \chainin (table-5-2);
\chainin (table-6-2); \chainin (table-5-3); \chainin (table-4-4); \chainin (table-3-5); \chainin (table-2-6); }
\end{tikzpicture}
\end{center}

Ясно, что таким образом можно пронумеровать все элементы~$\mathbb Q$, причём ни один из них не будет пронумерован дважды, значит, $\mathbb Q$ счётно.
\end{proof}

\begin{statement}
$(0; 1)$ несчётно.
\end{statement}
\begin{proofcontra}
Пусть все числа из интервала~$(0; 1)$ можно пронумеровать.
Тогда представим каждое число в~виде десятичной дроби и расположим эти дроби в~соответствии с нумерацией:
\begin{enumerate}
	\item $0{,}a_{11}a_{12} \ldots$
	\item $0{,}a_{21}a_{22} \ldots$
	
	\ldots
\end{enumerate}
где $a_{11}, a_{12}, \ldots, a_{21}, a_{22}, \ldots$~--- цифры.
Рассмотрим дробь $0{,}b_1 b_2 \ldots$, где $b_1, b_2, \ldots$~--- цифры такие, что $b_1 \neq a_{11}$, \linebreak $b_2 \neq a_{22}$, \ldots.
Такая дробь отличается от каждой из пронумерованных хотя~бы в~одной позиции, значит, она не пронумерована.
Противоречие.
\end{proofcontra}

\begin{statement}
Отрезок~$[a; b]$ равномощен отрезку~$[c; d]$.
\end{statement}
\begin{proof}
Рассмотрим функцию
\begin{equation*}
f(x) = \frac{c - d}{a - b} (x - a) + c, \ D(f) = [a; b]
\end{equation*}

$E(f) = [c; d]$, $f$~--- биекция, значит, любые два отрезка равномощны друг другу.
\end{proof}

\begin{statement}
Множество~$\mathbb R$ равномощно интервалу~$(0; 1)$.
\end{statement}
\begin{proof}
Рассмотрим функцию
\begin{equation*}
f(x) =
\begin{cases}
\dfrac1x - 2, \ 0 < x \leqslant \dfrac12 \\
\dfrac1{x - 1} + 2, \ \dfrac12 < x < 1
\end{cases}
\end{equation*}

$D(f) = (0; 1)$, $E(f) = \mathbb R$, $f$~--- биекция, значит, интервал~$(0; 1)$ равномощен $\mathbb R$.
\end{proof}

$|\mathbb R|$ называется \textbf{континуумом}.

\printindex
\end{document}