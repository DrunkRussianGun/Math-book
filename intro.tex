Здесь содержатся знания maxim4133 о~математике. Принятые обозначения:
\begin{itemize}
	\item $\forall$~--- \textbf{квантор всеобщности}.
	Обозначение условия, которое верно для~всех указанных элементов.
	Читается как~<<для всех>>, <<для каждого>>, <<для любого>> или~<<все>>, <<каждый>>, <<любой>>.
	
	\item $\exists$~--- \textbf{квантор существования}.
	Обозначение условия, которое верно хотя~бы для~одного из~указанных элементов.
	Читается как~<<существует>>, <<найдётся>>.
	
	\item $\exists!$~--- \textbf{квантор существования и единственности}.
	Обозначение условия, которое верно ровно для~одного из~указанных элементов.
	Читается как~<<существует единственный>>.
	
	\item $\colon$~--- <<что>>, <<такой (такие)>>, <<что>>, <<так, что>>, <<обладающий свойством>>.
	
	\item $\lAnd$~--- знак конъюнкции. Читается как <<и>>.
	
	\item $\lOr$~--- знак дизъюнкции. Читается как <<или>>.
	
	\item $\Rightarrow$~--- знак импликации (следствия).
	Читается как~<<если\dots, то\dots>>, <<значит>>, <<тогда>>.
	
	\item $\Leftrightarrow$~--- знак эквивалентности (равносильности).
	Читается как <<тогда и~только тогда, когда>>, <<ровно/в~точности тогда, когда>>.
	
	\item $\scriptstyle \blacksquare$~--- Q.E.D. (лат. quod erat demonstrandum, рус. что и~требовалось доказать).
	Обозначение конца доказательства.
\end{itemize}

Приоритет связок в порядке от высшего к низшему: $\lAnd, \lOr, \Rightarrow, \Leftrightarrow$.