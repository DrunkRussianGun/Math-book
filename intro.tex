	Здесь содержатся знания maxim4133 о математике. Принятые обозначения:
\begin{itemize}
	\item $\forall$~--- \textbf{квантор всеобщности}. Обозначение условия, которое верно для~всех
	указанных элементов. Читается как~<<для всех>>, <<для каждого>>, <<для любого>> или~<<все>>,
	<<каждый>>, <<любой>>.
	\item $\exists$~--- \textbf{квантор существования}. Обозначение условия, которое верно хотя~бы
	для~одного из~указанных элементов. Читается как~<<существует>>, <<найдётся>>.
	\item $\exists!$~--- \textbf{квантор существования и единственности}. Обозначение условия,
	которое верно ровно для~одного из~указанных элементов. Читается как~<<существует единственный>>.
	\item $\colon$~--- <<что>>, <<такой (такие)>>, <<что>>, <<так, что>>, <<обладающий свойством>>.
	\item $\Rightarrow$~--- символ следствия. Читается как~<<если\dots, то\dots>>.
	\item $\Leftrightarrow$~--- символ эквивалентности (равносильности). Читается как
	~<<тогда и~только тогда, когда>>, <<ровно/в~точности тогда, когда>>.
	\item $\land$~--- знак конъюнкции. Высказывание, полученное при~связывании двух других
	высказываний конъюнкцией, истинно ровно тогда, когда оба связываемых высказываний истинны.
	\item $\lor$~--- знак дизъюнкции. Высказывание, полученное при~связывании двух других
	высказываний дизъюнкцией, истинно ровно тогда, когда истинно хотя~бы одно из~связываемых
	высказываний. Дизъюнкция имеет более низкий приоритет по~сравнению с~конъюнкцией.
\end{itemize}