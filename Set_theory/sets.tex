\section{Множества}
	Множество~--- основное понятие. Некоторые числовые множества:
\begin{itemize}
	\item $\mathbb N = \{ 1, 2, 3, \dots \}$~--- множество натуральных чисел.
	\item $\mathbb Z = \{ \dots, -2, -1, 0, 1, 2, \dots \}$~--- множество целых чисел.
	\item $\mathbb Q = \left\{ \dfrac{m}n \mid m \in \mathbb Z \land n \in \mathbb N \right\}$
	~--- множество рациональных чисел.
	\item $\mathbb I$~--- множество иррациональных чисел.
	\item $\mathbb R$~--- множество действительных (вещественных) чисел.
	\item $\mathbb C$~--- множество комплексных чисел.
\end{itemize}

\subsection{Отношения между множествами}
	Пусть $A, B$~--- множества. Между~ними определены следующие отношения:
\begin{itemize}
	\item $A$ включено в~$B$ (является \textbf{подмножеством}~$B$):
	\[ A \subseteq B \Leftrightarrow \forall a \in A \ a \in B \]
	Нередко вместо знака~$\subseteq$ пишется знак~$\subset$\,.
	\item $A$ равно $B$:
	\[ A = B \Leftrightarrow \forall a \ (a \in A \Leftrightarrow a \in B) \]
	\item $A$ строго включено в~$B$:
	\[ A \subset B \Leftrightarrow A \subseteq B \land A = B \]
\end{itemize}

\subsection{Операции над множествами}
	Пусть $A, B$~--- множества. Над~ними определены следующие операции:
\begin{itemize}
	\item \textbf{Объединение}:
	\[ A \cup B = \{ x \mid x \in A \lor x \in B \} \]
	\item \textbf{Пересечение}:
	\[ A \cap B = \{ x \mid x \in A \land x \in B \} \]
	\item \textbf{Разность}:
	\[ A \setminus B = \{ x \mid x \in A \land x \notin B \} \]
	\item \textbf{Симметрическая разность}:
	\[ A \triangle B = \{ x \mid x \in A \land x \notin B \lor x \notin A \land x \in B \} \]
	\item \textbf{Дополнение} до~$U$, где~$A \subseteq U$:
	\[ \overline A = \{ x \in U \mid x \notin A \} \]
	\item \textbf{Декартово произведение}:
	\[ A \times B = \{ (x, y) \mid x \in A \land y \in B \} \]
	\item \textbf{Декартова степень}:
	\[ A^n = \underbrace{A \times A \times \ldots \times A}_n \]
\end{itemize}

\subsection{Функции}
Пусть $A$ и $B$~--- множества.
\textbf{Функцией}~f называется правило, ставящее в~соответствие каждому элементу~$a \in A$ единственный элемент~$f(a) \in B$.
$A$ называется \textbf{областью определения} функции~$f$, $B$~--- \textbf{областью значений} функции~$f$.
$a$ называется \textbf{прообразом}~$f(a)$, $f(a)$~--- \textbf{образом}~$a$.

Функция $f \colon A \to B$ называется \textbf{инъективной (инъекцией)}, если
$\forall x, y \in A	\ \allowbreak (x \neq y \Rightarrow f(x) \neq f(y))$.

Функция $f \colon A \to B$ называется \textbf{сюръективной (сюръекцией)}, если
$\forall b \in B \ \allowbreak \exists a \in A \colon \allowbreak f(a) = b$.

Функция $f \colon A \to B$ называется \textbf{биективной (биекцией)}, если она инъективная
и~сюръективная.

\textbf{Последовательностью} называется функция, заданная на~множестве~$X \subseteq \mathbb N$, и~обозначается~$(x_n)$.
\section{Мощность множеств}
Множества $A$ и $B$ называются \textbf{равномощными}, если существует биекция~$f \colon A \to B$, иначе~--- \textbf{неравномощными}.

Для конечных множеств это означает, что у них одинаковое количество элементов.

\index{Мощность} \textbf{Мощностью конечного множества~$A$} называется количество его элементов и обозначается $|A|$.

Множество всех подмножеств множества~$A$ обозначается $2^A = \mathcal P(A) = \{ x \mid x \subseteq A \}$.

Множество всех подмножеств множества~$A$ мощности~$k$ обозначается $\mathcal P_k(A) = \{ x \subseteq A \mid |x| = k \}$.

\index{Теорема!Кантора}
\begin{theorem}[Кантора]
Множества $A$ и $\mathcal P(A)$ не~равномощны.
\end{theorem}
\begin{proofcontra}
Пусть $f \colon A \to \mathcal P(A)$~--- биекция. Рассмотрим множество
\begin{equation*}
X = \{ a \in A \mid a \notin f(a) \} \Rightarrow X \subset A \Rightarrow X \in \mathcal P(A)
\end{equation*}

$f$~--- биекция, тогда $\exists b \in A \colon f(b) = X$.
Возможны два случая:
\begin{enumerate}
	\item Пусть $b \in X \Rightarrow b \in f(b) \Rightarrow b \notin X$.
	Противоречие.
	\item Пусть $b \notin X \Rightarrow b \in f(b) \Rightarrow b \in X$.
	Противоречие.
\end{enumerate}

В~обоих случаях получили противоречие.
\end{proofcontra}

\begin{theorem}
Пусть дано множество~$A \colon |A| = n$, тогда $|\mathcal P_k(A)| = C_n^k$.
\end{theorem}
\begin{proofmathind}
	\indbase $n = 0$:
	\begin{equation*}
	|A| = 0 \Rightarrow A = \varnothing \Rightarrow \mathcal P(A) = \{ \varnothing \} \Rightarrow
	|\mathcal P_0(A)| = 1 = C_0^0
	\end{equation*}
	\indstep Пусть теорема верна для $n$.
	Докажем её для $n + 1$.
	Пусть $X \subset A$, $|X| = k$, $a \in A$.
	Подсчитаем количество таких $X$.
	Возможны два случая:
	\begin{enumerate}
		\item Пусть $a \notin X \Rightarrow X \subset A \setminus \{ a \}$, тогда таких $X$ $C_n^k$.
		\item Пусть $a \in X$, тогда таких $X$ столько~же, сколько множеств
		$X \setminus \{ a \} \subset A \setminus \{ a \}$, т.\,е.~$C_n^{k-1}$.
	\end{enumerate}

	Тогда $|\mathcal P(A)| = C_n^{k-1} + C_n^k = C_{n+1}^k$. \indend
\end{proofmathind}

\subsection{Мощность числовых множеств}
Множество называется \textbf{счётным}, если оно равномощно множеству натуральных чисел.
Бесконечное множество, не являющееся счётным, называется \textbf{несчётным}.

\begin{statement}
$\mathbb Z$ счётно.
\end{statement}
\begin{proof}
Построим биекцию~$f \colon \mathbb Z \to \mathbb N$:
\begin{equation*}
f(n) =
\begin{cases}
-2n - 1, \ n < 0 \\
2n, \ n \geqslant 0
\end{cases}
\end{equation*}

Тогда $|\mathbb Z| = |\mathbb N|$.
\end{proof}

\begin{statement}
$\mathbb Q$ счётно.
\end{statement}
\begin{proof}
Составим таблицу, в~верхней строке которой стоят $p_i \in \mathbb Z$, где $i = 1, 2, \ldots$, в~левом столбце~--- $q_j \in \mathbb N$, где $j = 1, 2, \ldots$, а на пересечении столбца и строки~--- $\frac{p_i}{q_j}$.
Обходя таблицу в~указанном порядке, будем нумеровать очередной элемент, только если он не встречался ранее:
\begin{center}\noindent
\begin{tikzpicture}
\matrix (table) [matrix of math nodes,
	column sep = 3 mm,
	row sep = 3 mm]
{
  & 0 & -1 & 1 & -2 & \cdots \\
1 & |[draw,circle]| 0 & |[draw,circle]| -1 & |[draw,circle]| 1 & |[draw,circle]| -2 & \cdots \\
2 & 0 & |[draw,circle]| -\dfrac12 & |[draw,circle]| \dfrac12 & -1 & \cdots \\
3 & 0 & |[draw,circle]| -\dfrac13 & |[draw,circle]| \dfrac13 & \ddots \\
4 & 0 & |[draw,circle]| -\dfrac14 & \ddots \\
\vdots & \vdots & \vdots \\
};

\draw (table.west |- table-1-6.south east) -- (table-1-6.south east);
\draw (table.north -| table-6-1.south east) -- (table-6-1.south east);

{ [start chain, every on chain/.style = {join = by ->}]
\chainin (table-2-2); \chainin (table-2-3); \chainin (table-3-2);
\chainin (table-4-2); \chainin (table-3-3); \chainin (table-2-4);
\chainin (table-2-5); \chainin (table-3-4); \chainin (table-4-3); \chainin (table-5-2);
\chainin (table-6-2); \chainin (table-5-3); \chainin (table-4-4); \chainin (table-3-5); \chainin (table-2-6); }
\end{tikzpicture}
\end{center}

Ясно, что таким образом можно пронумеровать все элементы~$\mathbb Q$, причём ни один из них не будет пронумерован дважды, значит, $\mathbb Q$ счётно.
\end{proof}

\begin{statement}
$(0; 1)$ несчётно.
\end{statement}
\begin{proofcontra}
Пусть все числа из интервала~$(0; 1)$ можно пронумеровать.
Тогда представим каждое число в~виде десятичной дроби и расположим эти дроби в~соответствии с нумерацией:
\begin{enumerate}
	\item $0{,}a_{11}a_{12} \ldots$
	\item $0{,}a_{21}a_{22} \ldots$
	
	\ldots
\end{enumerate}
где $a_{11}, a_{12}, \ldots, a_{21}, a_{22}, \ldots$~--- цифры.
Рассмотрим дробь $0{,}b_1 b_2 \ldots$, где $b_1, b_2, \ldots$~--- цифры такие, что $b_1 \neq a_{11}$, \linebreak $b_2 \neq a_{22}$, \ldots.
Такая дробь отличается от каждой из пронумерованных хотя~бы в~одной позиции, значит, она не пронумерована.
Противоречие.
\end{proofcontra}

\begin{statement}
Отрезок~$[a; b]$ равномощен отрезку~$[c; d]$.
\end{statement}
\begin{proof}
Рассмотрим функцию
\begin{equation*}
f(x) = \frac{c - d}{a - b} (x - a) + c, \ D(f) = [a; b]
\end{equation*}

$E(f) = [c; d]$, $f$~--- биекция, значит, любые два отрезка равномощны друг другу.
\end{proof}

\begin{statement}
Множество~$\mathbb R$ равномощно интервалу~$(0; 1)$.
\end{statement}
\begin{proof}
Рассмотрим функцию
\begin{equation*}
f(x) =
\begin{cases}
\dfrac1x - 2, \ 0 < x \leqslant \dfrac12 \\
\dfrac1{x - 1} + 2, \ \dfrac12 < x < 1
\end{cases}
\end{equation*}

$D(f) = (0; 1)$, $E(f) = \mathbb R$, $f$~--- биекция, значит, интервал~$(0; 1)$ равномощен $\mathbb R$.
\end{proof}

$|\mathbb R|$ называется \textbf{континуумом}.