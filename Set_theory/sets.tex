\section{Множества}
	Множество~--- основное понятие. Некоторые числовые множества:
\begin{itemize}
	\item $\mathbb N = \{ 1, 2, 3, \dots \}$~--- множество натуральных чисел.
	\item $\mathbb Z = \{ \dots, -2, -1, 0, 1, 2, \dots \}$~--- множество целых чисел.
	\item $\mathbb Q = \left\{ \dfrac{m}n \mid m \in \mathbb Z \land n \in \mathbb N \right\}$
	~--- множество рациональных чисел.
	\item $\mathbb I$~--- множество иррациональных чисел.
	\item $\mathbb R$~--- множество действительных (вещественных) чисел.
	\item $\mathbb C$~--- множество комплексных чисел.
\end{itemize}

\subsection{Отношения между множествами}
	Пусть $A, B$~--- множества. Между~ними определены следующие отношения:
\begin{itemize}
	\item $A$ включено в~$B$ (является \textbf{подмножеством}~$B$):
	\[ A \subseteq B \Leftrightarrow \forall a \in A \ a \in B \]
	Нередко вместо знака~$\subseteq$ пишется знак~$\subset$\,.
	\item $A$ равно $B$:
	\[ A = B \Leftrightarrow \forall a \ (a \in A \Leftrightarrow a \in B) \]
	\item $A$ строго включено в~$B$:
	\[ A \subset B \Leftrightarrow A \subseteq B \land A = B \]
\end{itemize}

\subsection{Операции над множествами}
	Пусть $A, B$~--- множества. Над~ними определены следующие операции:
\begin{itemize}
	\item \textbf{Объединение}:
	\[ A \cup B = \{ x \mid x \in A \lor x \in B \} \]
	\item \textbf{Пересечение}:
	\[ A \cap B = \{ x \mid x \in A \land x \in B \} \]
	\item \textbf{Разность}:
	\[ A \setminus B = \{ x \mid x \in A \land x \notin B \} \]
	\item \textbf{Симметрическая разность}:
	\[ A \triangle B = \{ x \mid x \in A \land x \notin B \lor x \notin A \land x \in B \} \]
	\item \textbf{Дополнение} до~$U$, где~$A \subseteq U$:
	\[ \overline A = \{ x \in U \mid x \notin A \} \]
	\item \textbf{Декартово произведение}:
	\[ A \times B = \{ (x, y) \mid x \in A \land y \in B \} \]
	\item \textbf{Декартова степень}:
	\[ A^n = \underbrace{A \times A \times \ldots \times A}_n \]
\end{itemize}

\subsection{Функции}
Пусть $A$ и $B$~--- множества.
\textbf{Функцией}~f называется правило, ставящее в~соответствие каждому элементу~$a \in A$ единственный элемент~$f(a) \in B$.

$A$ называется \textbf{областью определения} функции~$f$.

$B$ называется \textbf{областью значений} функции~$f$.

$a$ называется \textbf{прообразом}~$f(a)$.

$f(a)$ называется \textbf{образом}~$a$.

Функция $f \colon A \to B$ называется \textbf{инъективной (инъекцией)}, если
$\forall x, y \in A	\ \allowbreak (x \neq y \Rightarrow f(x) \neq f(y))$.

Функция $f \colon A \to B$ называется \textbf{сюръективной (сюръекцией)}, если
$\forall b \in B \ \allowbreak \exists a \in A \colon \allowbreak f(a) = b$.

Функция $f \colon A \to B$ называется \textbf{биективной (биекцией)}, если она инъективная
и~сюръективная.
\subsection{Мощность множеств}
Множества $A$ и $B$ называются \textbf{равномощными (имеют одинаковую мощность)}, если существует биекция~$f \colon A \to B$, иначе~--- \textbf{неравномощными}.

Для конечных множеств это означает, что у~них одинаковое количество элементов.

\textbf{Мощностью} конечного множества~$A$ называется количество~$|A|$ его элементов.

Множество всех подмножеств множества~$A$ обозначается
\[ \mathcal P(A) = \{ x \mid x \subseteq A \} \]

Множество всех подмножеств множества~$A$ мощности~$k$ обозначается
\[ \mathcal P_k(A) = \{ x \subseteq A \mid |x| = k \} \]

\begin{theorem}[Кантора]
Множества $A$ и $\mathcal P(A)$ не~равномощны.
\end{theorem}
\begin{proofcontra}
Пусть $f \colon A \to \mathcal P(A)$~--- биекция. Рассмотрим множество
\[ X = \{ a \in A \mid a \notin f(a) \} \Rightarrow X \subset A \Rightarrow X \in \mathcal P(A) \]

$f$~--- биекция, тогда $\exists b \in A \colon f(b) = X$.
Возможны два случая:
\begin{enumerate}
	\item Пусть $b \in X \Rightarrow b \in f(b) \Rightarrow b \notin X$.
	Противоречие.
	\item Пусть $b \notin X \Rightarrow b \in f(b) \Rightarrow b \in X$.
	Противоречие.
\end{enumerate}

В~обоих случаях получили противоречие.
\end{proofcontra}

\begin{theorem}
Пусть дано множество~$A \colon |A| = n$, тогда $|\mathcal P_k(A)| = C_n^k$.
\end{theorem}
\begin{proofmathind}
	\indbase $n = 0$:
	\[ |A| = 0 \Rightarrow A = \varnothing \Rightarrow \mathcal P(A) = \{ \varnothing \} \Rightarrow |\mathcal P_0(A)| = 1 = C_0^0 \]
	\indstep Пусть теорема верна для~$n$.
	Докажем её для~$n + 1$.
	Пусть $X \subset A$, $|X| = k$, $a \in A$.
	Подсчитаем количество таких $X$.
	Возможны два случая:
	\begin{enumerate}
		\item Пусть $a \notin X \Rightarrow X \subset A \setminus \{ a \}$, тогда таких~$X$~$C_n^k$.
		\item Пусть $a \in X$, тогда таких $X$ столько~же, сколько множеств
		$X \setminus \{ a \} \subset A \setminus \{ a \}$, т.\,е.~$C_n^{k-1}$.
	\end{enumerate}

	Тогда $|\mathcal P(A)| = C_n^{k-1} + C_n^k = C_{n+1}^k$. \indend

\end{proofmathind}