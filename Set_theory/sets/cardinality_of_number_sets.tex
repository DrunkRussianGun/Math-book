\subsection{Мощность числовых множеств}
Множество называется \textbf{счётным}, если оно равномощно множеству натуральных чисел.
Бесконечное множество, не являющееся счётным, называется \textbf{несчётным}.

\begin{statement}
$\mathbb Z$ счётно.
\end{statement}
\begin{proof}
Построим биекцию~$f \colon \mathbb Z \to \mathbb N$:
\begin{equation*}
f(n) =
\begin{cases}
-2n - 1, \ n < 0 \\
2n, \ n \geqslant 0
\end{cases}
\end{equation*}

Тогда $|\mathbb Z| = |\mathbb N|$.
\end{proof}

\begin{statement}
$\mathbb Q$ счётно.
\end{statement}
\begin{proof}
Составим таблицу, в~верхней строке которой стоят $p_i \in \mathbb Z$, в~левом столбце~--- $q_i \in \mathbb N$, а на пересечении столбца и строки~--- $\dfrac{p_i}{q_i}$.
Обходя таблицу в~указанном порядке, будем нумеровать очередной элемент, только если он не встречался ранее:

\xymatrix{
  & 0 & -1 & 1 & -2 & -2 & \cdots \\
1 & *+[o]+[F]{0} \ar[r] & *+[o]+[F]{-1} \ar[ld] & *+[o]+[F]{1} \ar[r] & *+[o]+[F]{-2} \ar[ld] & *+[o]+[F]{2} \ar[r] & \cdots \ar[ld] \\
2 & 0 \ar[d] & *+[o]+[F]{-\dfrac12} \ar[ur] & *+[o]+[F]{\dfrac12} \ar[ld] & -1 \ar[ur] & 1 \ar[ld] & \cdots \\
3 & 0 \ar[ur] & *+[o]+[F]{-\dfrac13} \ar[ld] & *+[o]+[F]{\dfrac13} \ar[ur] & *+[o]+[F]{-\dfrac23} \ar[ld] & *+[o]+[F]{\dfrac23} \ar[ur] & \cdots \ar[ld] \\
4 & 0 \ar[d] & *+[o]+[F]{-\dfrac14} \ar[ur] & *+[o]+[F]{\dfrac14} \ar[ld] & -\dfrac12 \ar[ur] & \dfrac12 \ar[ld] & \cdots \\
5 & 0 \ar[ur] & *+[o]+[F]{-\dfrac15} \ar[ld] & *+[o]+[F]{\dfrac15} \ar[ur] & *+[o]+[F]{-\dfrac25} \ar[ld] & *+[o]+[F]{\dfrac25} \ar[ur] & \cdots \\
\vdots & \vdots & \vdots \ar[ur] & \vdots & \vdots \ar[ur] & \vdots & \ddots
}

Ясно, что таким образом можно пронумеровать все элементы~$\mathbb Q$, причём ни один из них не будет пронумерован дважды, значит, $\mathbb Q$ счётно.
\end{proof}

\begin{statement}
$(0; 1)$ несчётно.
\end{statement}
\begin{proofcontra}
Пусть все числа из интервала~$(0; 1)$ можно пронумеровать.
Тогда представим каждое число в~виде десятичной дроби и расположим эти дроби в~соответствии с нумерацией:
\begin{enumerate}
	\item $0{,}a_{11}a_{12} \ldots$
	\item $0{,}a_{21}a_{22} \ldots$
	
	\ldots
\end{enumerate}
где $a_{11}, a_{12}, \ldots, a_{21}, a_{22}, \ldots$~--- цифры.
Рассмотрим дробь $0{,}b_1 b_2 \ldots$, где $b_1, b_2, \ldots$~--- цифры такие, что $b_1 \neq a_{11}, b_2 \neq a_{22}, \ldots$.
Такая дробь отличается от каждой из пронумерованных хотя~бы в~одной позиции, значит, она не пронумерована.
Противоречие.
\end{proofcontra}

\begin{statement}
$\left| [a; b] \right| = \left| [c; d] \right|$
\end{statement}
\begin{proof}
Рассмотрим функцию
\begin{equation*}
f(x) = \frac{c - d}{a - b} (x - a) + c, \ x \in [a; b]
\end{equation*}

$f$ переводит $[a; b] \to [c; d]$ и является биекцией, значит, любые два отрезка равномощны друг другу.
\end{proof}

\begin{statement}
$|\mathbb R| = |(0; 1)|$.
\end{statement}
\begin{proof}
Рассмотрим функцию
\begin{equation*}
f(x) =
\begin{cases}
\dfrac1x - 2, \ 0 < x \leqslant \dfrac12 \\
\dfrac1{x - 1} + 2, \ \dfrac12 < x < 1
\end{cases}
\end{equation*}

$f$ переводит $(0; 1) \to \mathbb R$ и является биекцией, значит, интервал~$(0; 1)$ равномощен $\mathbb R$.
\end{proof}

$|\mathbb R|$ называется \textbf{континуумом}.