\subsection{Мощность множеств}
Множества $A$ и $B$ называются \textbf{равномощными (имеют одинаковую мощность)}, если существует биекция~$f \colon A \to B$, иначе~--- \textbf{неравномощными}.

Для конечных множеств это означает, что у~них одинаковое количество элементов.

\textbf{Мощностью} конечного множества~$A$ называется количество~$|A|$ его элементов.

Множество всех подмножеств множества~$A$ обозначается
\[ \mathcal P(A) = \{ x \mid x \subseteq A \} \]

Множество всех подмножеств множества~$A$ мощности~$k$ обозначается
\[ \mathcal P_k(A) = \{ x \subseteq A \mid |x| = k \} \]

\begin{theorem}[Кантора]
Множества $A$ и $\mathcal P(A)$ не~равномощны.
\end{theorem}
\begin{proofcontra}
Пусть $f \colon A \to \mathcal P(A)$~--- биекция. Рассмотрим множество
\[ X = \{ a \in A \mid a \notin f(a) \} \Rightarrow X \subset A \Rightarrow X \in \mathcal P(A) \]

$f$~--- биекция, тогда $\exists b \in A \colon f(b) = X$.
Возможны два случая:
\begin{enumerate}
	\item Пусть $b \in X \Rightarrow b \in f(b) \Rightarrow b \notin X$.
	Противоречие.
	\item Пусть $b \notin X \Rightarrow b \in f(b) \Rightarrow b \in X$.
	Противоречие.
\end{enumerate}

В~обоих случаях получили противоречие.
\end{proofcontra}

\begin{theorem}
Пусть дано множество~$A \colon |A| = n$, тогда $|\mathcal P_k(A)| = C_n^k$.
\end{theorem}
\begin{proofmathind}
	\indbase $n = 0$:
	\[ |A| = 0 \Rightarrow A = \varnothing \Rightarrow \mathcal P(A) = \{ \varnothing \} \Rightarrow |\mathcal P_0(A)| = 1 = C_0^0 \]
	\indstep Пусть теорема верна для~$n$.
	Докажем её для~$n + 1$.
	Пусть $X \subset A$, $|X| = k$, $a \in A$.
	Подсчитаем количество таких $X$.
	Возможны два случая:
	\begin{enumerate}
		\item Пусть $a \notin X \Rightarrow X \subset A \setminus \{ a \}$, тогда таких~$X$~$C_n^k$.
		\item Пусть $a \in X$, тогда таких $X$ столько~же, сколько множеств
		$X \setminus \{ a \} \subset A \setminus \{ a \}$, т.\,е.~$C_n^{k-1}$.
	\end{enumerate}

	Тогда $|\mathcal P(A)| = C_n^{k-1} + C_n^k = C_{n+1}^k$. \indend

\end{proofmathind}