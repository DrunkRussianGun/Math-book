% beta
\section{Деревья}
Граф без~циклов называется \textbf{лесом}.

Связный лес называется \textbf{деревом}.

Ребро называется \textbf{мостом}, если при~его удалении увеличивается число компонент связности.

\begin{statement}
Ребро~--- мост ровно тогда, когда оно не~содержится в~цикле.
\end{statement}
\begin{proof}
\begin{enumerate}
	\item Докажем методом от противного, что если ребро содержится в цикле, то оно не является мостом. Пусть ребро $e$ содержится в цикле $W = v_0 e_0 \ldots u e v \ldots v_k$, $u'$ и $v'$~--- смежные вершины.
	\begin{enumerate}
		\item Если в этом маршруте нет ребра $e$, то при его удалении из графа $u'$ и $v'$ останутся смежными.
		\item Если $u' = v_0' e_0' \ldots u e v \ldots e_m v_m' = v'$~--- маршрут, соединяющий $u'$ и $v'$, тогда при удалении $e$ из графа $u'$ и $v'$ соединяет маршрут $u' = v_0' e_0' \ldots u \ldots e_0 v_0 = v_k e_{k-1} \ldots v \ldots e_m v_m' = v'$.
	\end{enumerate}
	\item Пусть $e = (u, v)$ не является мостом, тогда $u$, $v$ лежат в одной компоненте связности. Удалим $e$ из графа, тогда число компонент связности не изменилось, значит, $u$ и $v$ также лежат в одной компоненте связности, т./,е. существует цепь, соединяющая $u$ и $v$: $u = v_0 e_0 \ldots e_{k-1} v_k = v$. Тогда в исходном графе существует цикл $u = v_0 e_0 \ldots e_{k-1} v_k = v e u$.
\end{enumerate}
\end{proof}

\begin{theorem}
Следующие утверждения о графе $G$ с $n$ вершинами эквивалентны:
\begin{enumerate}
	\item $G$~--- дерево.
	\item $G$ связный и имеет $n - 1$ ребро.
	\item $G$ связный и каждое его ребро~--- мост.
	\item $G$ не содержит циклов и имеет $n - 1$ ребро.
	\item Любые две вершины графа $G$ соединены ровно одной простой цепью.
	\item $G$ не содержит циклов и добавление ребра приводит к появлению цикла.
\end{enumerate}
\end{theorem}
\begin{proof}
\begin{itemize}
	\item Докажем 1) $\Rightarrow$ 3). Связность следует из определения дерева. В силу пред. утв. каждое ребро~--- мост.
	
	\item Докажем 3) $\Rightarrow$ 2). Связность по предположению. Докажем методом математической индукции, что в графе $n - 1$ ребро.
	\indbase Для $n = 1, 2$ очевидно.
	\indstep Пусть для графов с числом вершин, меньшим $n$,  Возьмём мост $e$ и удалим его. Получим две компоненты связности $G_1 = (V_1, E_1)$, $G_2 = (V_2, E_2)$. По предположению индукции $|E_1| = |V_1| - 1$, $|E_2| = |V_2| - 1$. В исходном графе рёбер $|E_1| + |E_2| + 1 = |V_1| + |V_2| - 1 = n - 1$.
	\indend
	
	\item Докажем 2) $\Rightarrow$ 4). В $G$ $n - 1$ ребро по предположению. Докажем методом математической индукции, что $G$ не содержит циклов.
	\indbase Для $n = 1, 2$ очевидно.
	\indstep Докажем, что в графе есть вершина степени 1. $\forall u \ deg u \geqslant 1$. $\forall u \ deg u \geqslant 2 \Rightarrow 2|E| = \sum_{u \in V} deg u \geqslant 2n \Rightarrow n - 1 = |E| \geqslant n$. Значит, в графе найдётся вершина степени 1. Удалим её и инцидентное ей ребро. Полученный граф содержит $n - 1$ вершину и удовлетворяет утверждению 2). По предположению индукции он не содержит циклов, тогда и исходный граф не содержит циклов.
	\indend
	
	\item Докажем 4) $\Rightarrow$ 5). Докажем связность методом математической индукции.
	\indbase Для $n = 1, 2$ очевидно.
	\indstep Пусть в графе $k$ компонент связности: $G_1 = (V_1, E_1)$, $G_2 = (V_2, E_2)$, \ldots, $G_k = (V_k, E_k)$. Они являются деревьями.
	\indend
	
	$|E_1| = |V_1| - 1$, $|E_2| = |V_2| - 1$, \ldots, $|E_k| = |V_k| - 1$. $n - 1 = |E_1| + \ldots + |E_k| = n - k \Rightarrow k = 1$, значит, граф связный.
	
	Пусть существуют вершины $u$, $v$ такие, что их соединяют две простые цепи, тогда в графе есть цикл, что противоречит предположению. Тогда эти вершины соединены ровно одной простой цепью.
	
	\item Докажем 5) $\Rightarrow$ 6). Предположим, что в графе есть цикл $v_0 e_0 v_1 e_1 \ldots v_k = v_0$, тогда есть две простые цепи $v_0 e_0 \ldots v_{k-1}$ и $v_{k-1} e_k v_k = v_0$, соединяющие $v_0$ и $v_{k-1}$, что противоречит предположению.
	
	Докажем, что добавление ребра приводит к появлению ровно одного цикла. Рассмотрим несоседних вершины $u$ и $v$. По предположению есть цепь $u = v_0 e_0 \ldots v_k = v$, соединяющая их. Тогда $u = v_0 e_0 \ldots v_k = v e u$~--- цикл, где $e$~--- $(u, v)$-маршрут. Пусть есть 2 цикла, соединяющих $u$ и $v$. Удалим $e$, цикл останется. Получили исходный граф, в котором нет циклов. Противоречие.
	
	\item 6) $\Rightarrow$ 1). Докажем связность. Рассмотрим вершины $u$ и $v$. Если они не соединены ребром, то соединим и по предположению получим цикл $v_0 e_0 \ldots u e v \ldots e_{k-1} v_k = v_0$. Тогда $u \ldots e_0 v_0 = v_k e_{k-1} \ldots v$~--- $(u, v)$-маршрут. Противоречие.
\end{itemize}
\end{proof}