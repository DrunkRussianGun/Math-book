\section{Связность графов}
\begin{definition}
	Вершины $u$ и~$v$ называются \textbf{связанными}, если существует $(u, v)$-маршрут, иначе~---
	\textbf{несвязанными}.
\end{definition}

\begin{definition}
	Граф называется \textbf{связным}, если в~нём любые две вершины связаны, иначе~--- \textbf{несвязным}.
\end{definition}

\begin{definition}
	Граф~$G' = (V', E')$ называется \textbf{подграфом} графа~$G = (V, E)$, если $V' \subseteq V$ и
	~$E' \subseteq E$.
\end{definition}

\begin{definition}
	\textbf{Компонентой связности} графа называется его максимальный (относительно включения)
	связный подграф.
\end{definition}

\subsection{Эйлеровы графы}
\begin{definition}
	Цикл, содержащий все рёбра графа, называется \textbf{эйлеровым}.
\end{definition}

\begin{definition}
	Граф, содержащий эйлеров цикл, называется \textbf{эйлеровым}.
\end{definition}

\begin{theorem}
	Связный граф эйлеров ровно тогда, когда степени всех вершин чётны.
\end{theorem}
\begin{proof}
\begin{enumerate}
	\item Пусть в~графе есть эйлеров цикл. Выберем вершину~$v_0$ в~этом цикле и~начнём обходить
	его. При~каждом посещении вершины~$v \neq v_0$ степень вершины увеличивается на~2. Т.\,о., если
	посетить её $k$~раз, то $deg v = 2k$.
	
	Для~$v_0$ степень увеличивается на~1 в~начале обхода, на~1 в~конце обхода и~на~2 при
	~промежуточных посещениях. Т.\,о., её степень чётна.
	
	\item Пусть степени всех вершин чётны. Выберём цепь~$C = v_0, e_0, v_1, e_1, \ldots, e_{k-1},
	v_k$ наибольшей длины.
	
	$v_0 = v_k$
	
	Все рёбра, инцидентные~$v_k$, присутствуют в~этой цепи. Иначе, если $e = (v_k, v)$ не~
	присутствует, то цепь~$С' = v_0, e_0, \ldots, v_k, e, v$ длиннее, что противоречит выбору $C$.
	
	$v_0 \neq v_k$. При прохождении вершины~$v_i = v_k$, $k > i > 0$, степень $v_k$ увеличивается на~2. Также проходим по ребру $e_{k-1}$, тогда степень~$v_k$ нечётна. Противоречие.
	
	Пусть найдётся ребро $e = (u, v)$, не входящее в цикл и не входящее в цепь~$C$. Существует $(v_0, u)$-маршрут. Возьмём первое ребро $e' = (v_i, v')$, не входящее в $C$. Тогда цепь
	~$C' = v', e', v_i, e_i, \ldots, e_{k-1}, v_k = v_0, e_0, v_1, e_1, \ldots, v_{i-1}$ длиннее,
	чем $C$. Противоречие.
\end{enumerate}
\end{proof}

\subsection{Алгоритм нахождения эйлерова цикла}
\begin{enumerate}
	\item \textbf{Алгоритм Флери (очень медленный)}.
	\begin{enumerate}
		\item Выберем произвольную вершину.
		\item Пусть находимся в вершине $v$. Выберем ребро, инцидентное ей, которое должно быть мостом, только если не осталось других рёбер.
		\item Проходим по выбранному ребру и вычёркиваем его.
		\item Повторяем, пока есть рёбра.
	\end{enumerate}
	\item \textbf{Алгоритм объединения циклов}.
	\begin{enumerate}
		\item Выберем произвольную вершину.
		\item Выбираем любое непосещённое ребро и идём по нему.
		\item Повторяем, пока не вернёмся в начальную вершину.
		\item Получим цикл~$C$. Если он не эйлеров, то $\exists u \in C, \ \exists e = (u, u') \colon u' \notin C$. Выбираем $u$ и повторяем шаг 2. Получим цикл $C'$, рёбра которого не совпадают с рёбрами $C$, объединяем их. Повторяем шаг 4.
	\end{enumerate}
\end{enumerate}

\begin{definition}
	Граф называется \textbf{полуэйлеровым}, если в нём есть цепь, не являющаяся циклом и содержащая все рёбра.
\end{definition}

\begin{theorem}[критерий полуэйлерова графа]
content...
\end{theorem}

\subsection{Гамильтоновы графы}
\begin{definition}
	Простой цикл, содержащий все вершины графа, называется \textbf{гамильтоновым}.
\end{definition}

\begin{definition}
	Граф называется \textbf{гамильтоновым}, если в нём есть гамильтонов цикл.
\end{definition}

\begin{theorem}[Дирака]
	Если в графе с $n$ вершинами, $n \geqslant 3$, $\forall u \ deg u \geqslant \frac{n}2$, то граф гамильтонов.
\end{theorem}
\begin{proof}
\begin{enumerate}
	\item Докажем методом от противного, что граф связный. Пусть он несвязный. Выберем компоненту
	связности $G' = (V', E')$ с наименьшим числом вершин, тогда $|V'| \leqslant \frac{n}2$.
	Возьмём $v \in V'$, тогда $deg v \leqslant |V'| - 1 < \frac{n}2$. Противоречие с условием.
	\item Выберем цепь~$C = v_0, e_0, v_1, e_1, \ldots, e_{k-1}, v_k$ максимальной длины. Тогда все вершины, соседние с $v_0$, лежат в этой цепи, иначе можно увеличить длину цепи. Аналогично для $v_0$. Среди $v_1, v_2, \ldots, v_k$ вершин, соседних с $v_0$, не менее $\frac{n}2$.
	
	Пусть $v_i$ соседняя с $v_0$. Рассмотрим $v_{i-1}$, их не менее $\frac{n}2$, расположенных среди $v_0, \ldots, v_{k-1}$. Среди них не менее $\frac{n}2$, соседних с $v_k$. Найдётся $v_{i-1}$ такая, что $v_{i-1}$ соседняя с $v_k$. $v_i$ соседняя с $v_0$.
	
	Докажем, что $v_i, e_{i+1}, \ldots, v_k, e_k, v_{i-1}, e_{i-1}, \ldots, v_0, e_k', v_i$~---
	гамильтонов цикл, методом от противного. Предположим обратное, тогда есть вершина $u$, не входящая в цикл, и существует $(v_0, u)$-маршрут, значит, существует ребро, инцидентное одной из вершин цикла, но не входящее в него. Можно получить более длинную цепь.
\end{enumerate}
\end{proof}