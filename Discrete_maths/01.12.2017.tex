



\subsection{Гамильтоновы графы}
\begin{definition}
	Простой цикл, содержащий все вершины графа, называется \textbf{гамильтоновым}.
\end{definition}

\begin{definition}
	Граф называется \textbf{гамильтоновым}, если в нём есть гамильтонов цикл.
\end{definition}

\begin{theorem}[Дирака]
	Если в графе с $n$ вершинами, $n \geqslant 3$, $\forall u \ deg u \geqslant \frac{n}2$, то граф гамильтонов.
\end{theorem}
\begin{proof}
\begin{enumerate}
	\item Докажем методом от противного, что граф связный. Пусть он несвязный. Выберем компоненту
	связности $G' = (V', E')$ с наименьшим числом вершин, тогда $|V'| \leqslant \frac{n}2$.
	Возьмём $v \in V'$, тогда $deg v \leqslant |V'| - 1 < \frac{n}2$. Противоречие с условием.
	\item Выберем цепь~$C = v_0, e_0, v_1, e_1, \ldots, e_{k-1}, v_k$ максимальной длины. Тогда все вершины, соседние с $v_0$, лежат в этой цепи, иначе можно увеличить длину цепи. Аналогично для $v_0$. Среди $v_1, v_2, \ldots, v_k$ вершин, соседних с $v_0$, не менее $\frac{n}2$.
	
	Пусть $v_i$ соседняя с $v_0$. Рассмотрим $v_{i-1}$, их не менее $\frac{n}2$, расположенных среди $v_0, \ldots, v_{k-1}$. Среди них не менее $\frac{n}2$, соседних с $v_k$. Найдётся $v_{i-1}$ такая, что $v_{i-1}$ соседняя с $v_k$. $v_i$ соседняя с $v_0$.
	
	Докажем, что $v_i, e_{i+1}, \ldots, v_k, e_k, v_{i-1}, e_{i-1}, \ldots, v_0, e_k', v_i$~---
	гамильтонов цикл, методом от противного. Предположим обратное, тогда есть вершина $u$, не входящая в цикл, и существует $(v_0, u)$-маршрут, значит, существует ребро, инцидентное одной из вершин цикла, но не входящее в него. Можно получить более длинную цепь.
\end{enumerate}
\end{proof}