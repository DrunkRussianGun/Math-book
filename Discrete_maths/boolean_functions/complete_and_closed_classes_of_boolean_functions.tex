\subsection{Полнота и замкнутость классов булевых функций}
Множество~$F$ булевых функций (также называемое \textbf{классом}) называется \textbf{полным}, если любая булева функция реализуется формулой над~$F$.

Множество, состоящее из всех булевых функций, обозначается $P_2$.

\begin{theorem}
Если $F = \{ f_1, f_2, \ldots \}$~--- полный набор булевых функций, $G = \{ g_1, g_2, \ldots \}$~--- набор булевых функций, причём каждая функция из~$F$ реализуется формулой над~$G$, то $G$~--- полный набор.
\end{theorem}
\begin{proof}
\begin{equation*}
f_1 = \Phi_1(g_1, g_2, \ldots), f_2 = \Phi_2(g_1, g_2, \ldots), \ldots \Rightarrow
\end{equation*}
\begin{equation*}
\Rightarrow \forall f = \Phi(f_1, f_2, \ldots) =
\Phi(\Phi_1(g_1, g_2, \ldots), \Phi_2(g_1, g_2, \ldots), \ldots) =
\Phi(g_1, g_2, \ldots)
\end{equation*}

Значит, $G$~--- полный набор.
\end{proof}

\index{Замыкание} \textbf{Замыканием} множества~$F$ булевых функций называется множество всех булевых функций, реализуемых формулами над~$F$, и обозначается $[F]$.
Свойства замыкания:
\begin{enumerate}
	\item $F \subseteq [F]$;
	\item $[[F]] = [F]$, т.\,к. $[[F]]$~--- множество функций, реализуемых формулами над~$[F]$, которые реализуются формулами над~$F$;
	\item $F \subseteq K \Rightarrow [F] \subseteq [K]$, т.\,к. формула над~$F$ является формулой над~$K$.
\end{enumerate}

\begin{statement}
Множество~$F$ булевых функций полно $\Leftrightarrow [F] = P_2$.
\end{statement}

\subsubsection{Класс \texorpdfstring{$T_0$}{} функций, сохраняющих константу 0}
\begin{equation*}
T_0 = \{ f(x_1, \ldots, x_n) \mid f(0, \ldots, 0) = 0 \}
\end{equation*}
\begin{proof}[замкнутости]
Пусть $f, f_1, \ldots, f_n \in T_0$, тогда $\Phi =
f(f_1(0, \ldots, 0), \ldots, f_n(0, \ldots, 0)) \opbr=
f(0, \ldots, 0) \opbr= 0$, значит, $\Phi$ реализует функцию из~$T_0$.
\end{proof}

$T_0$ содержит $\dfrac{2^{2^n}}2 = 2^{2^n - 1}$~функций от $n$~переменных.

\subsubsection{Класс \texorpdfstring{$T_1$}{} функций, сохраняющих константу 1}
\begin{equation*}
T_1 = \{ f(x_1, \ldots, x_n) \mid f(1, \ldots, 1) = 1 \}
\end{equation*}
\begin{proof}[замкнутости]
Пусть $f, f_1, \ldots, f_n \in T_1$, тогда $\Phi =
f(f_1(1, \ldots, 1), \ldots, f_n(1, \ldots, 1)) \opbr=
f(1, \ldots, 1) \opbr= 1$, значит, $\Phi$ реализует функцию из~$T_1$.
\end{proof}

$T_1$ содержит $\dfrac{2^{2^n}}2 = 2^{2^n - 1}$~функций от $n$~переменных.

\subsubsection{Класс \texorpdfstring{$S$}{} самодвойственных функций}
\index{Функция!булева!самодвойственная} Булева функция~$f$ называется \textbf{самодвойственной}, если $f^* = f$.
\begin{equation*}
S = \{ f(x_1, \ldots, x_n) \mid
f(x_1, \ldots, x_n) = f^*(x_1, \ldots, x_n) \} \Leftrightarrow
\end{equation*}
\begin{equation*}
\Leftrightarrow S = \{ f(x_1, \ldots, x_n) \mid
f(x_1, \ldots, x_n) = \overline f(\overline{x_1}, \ldots, \overline{x_n}) \}
\end{equation*}
\begin{proof}[замкнутости]
Пусть $f, f_1, \ldots, f_n \in S$, тогда $\Phi = f(f_1, \ldots, f_n) \opbr= f^*(f_1^*, \ldots, f_n^*) \opbr= \Phi^*$, значит, $\Phi$ реализует функцию из~$S$.
\end{proof}

$S$ содержит $2^{2^{n-1}} = \sqrt{2^{2^n}}$~функций от $n$~переменных.

\begin{lemma}[о несамодвойственной функции]
Если булева функция $f \notin S$, то подстановкой $x, \overline x$ вместо переменных можно получить константу.
\end{lemma}
\begin{proof}
\begin{equation*}
f(x_1, \ldots, x_n) \notin S \Leftrightarrow
\exists \alpha_1, \ldots, \alpha_n \colon
f(\alpha_1, \ldots, \alpha_n) \neq \overline f(\overline{\alpha_1}, \ldots, \overline{\alpha_n})
\end{equation*}

Подставим в~$f$ $x^{\alpha_i}$ вместо $x_i$.
\begin{enumerate}
	\item При $x = 0$ получим $f(0^{\alpha_1}, \ldots, 0^{\alpha_n}) =
	f(\overline{\alpha_1}, \ldots, \overline{\alpha_n}) =
	f(\alpha_1, \ldots, \alpha_n)$.
	
	\item При $x = 1$ получим $f(1^{\alpha_1}, \ldots, 1^{\alpha_n}) =
	f(\alpha_1, \ldots, \alpha_n)$.
\end{enumerate}

$f(0^{\alpha_1}, \ldots, 0^{\alpha_n}) = f(1^{\alpha_1}, \ldots, 1^{\alpha_n})$, значит, $f(x^{\alpha_1}, \ldots, x^{\alpha_n})$~--- константа.
\end{proof}

\subsubsection{Класс \texorpdfstring{$M$}{} монотонных функций}
Введём отношение~$\preccurlyeq$: $\overline\alpha = (\alpha_1, \ldots, \alpha_n) \preccurlyeq \overline\beta = (\beta_1, \ldots, \beta_n) \Leftrightarrow
\alpha_1 \leqslant \beta_1, \ldots, \alpha_n \leqslant \beta_n$.

\index{Функция!булева!монотонная} Булева функция~$f$ называется \textbf{монотонной}, если
$\forall \overline\alpha, \overline\beta \
\overline\alpha \preccurlyeq \overline\beta \Rightarrow f(\overline\alpha) \leqslant f(\overline\beta)$.
\begin{equation*}
M = \{ f(x_1, \ldots, x_n) \mid
\forall \overline\alpha, \overline\beta \
\overline\alpha \preccurlyeq \overline\beta \Rightarrow f(\overline\alpha) \leqslant f(\overline\beta) \} \Leftrightarrow
\end{equation*}
\begin{equation*}
\Leftrightarrow M = \{ f(x_1, \ldots, x_n) \mid
(\forall i \ \alpha_i \leqslant \beta_i) \Rightarrow f(\alpha_1, \ldots, \alpha_n) \leqslant f(\beta_1, \ldots, \beta_n) \}
\end{equation*}
\begin{proof}[замкнутости]
Пусть $f, f_1, \ldots, f_n \in S$, $\Phi = f(f_1, \ldots, f_n)$.
\begin{equation*}
\overline\alpha \preccurlyeq \overline\beta \Rightarrow
f(\overline\alpha) \leqslant f(\overline\beta), \
f_1(\overline\alpha) \leqslant f_1(\overline\beta), \ \ldots, \
f_n(\overline\alpha) \leqslant f_n(\overline\beta) \Rightarrow
\end{equation*}
\begin{equation*}
\Rightarrow (f_1(\overline\alpha), \ldots, f_n(\overline\alpha)) \preccurlyeq (f_1(\overline\beta), \ldots, f_n(\overline\beta)) \Rightarrow
f(f_1(\overline\alpha), \ldots, f_n(\overline\alpha)) \leqslant f(f_1(\overline\beta), \ldots, f_n(\overline\beta))
\end{equation*}

Значит, $\Phi$ реализует функцию из~$M$.
\end{proof}

\begin{consequent}
Если некоторая ДНФ для булевой функции~$f$ не содержит отрицаний, то $f$ монотонна.
\end{consequent}
\begin{proof}
ДНФ без отрицаний — это формула над~$\{ \And, \lor \}$.
$\And, \lor$~--- монотонные функции, значит, ДНФ без отрицаний тоже монотонна, тогда и $f$ монотонна.
\end{proof}

\begin{lemma}[о немонотонной функции]
Если булева функция~$f \notin M$, то подстановкой $0, 1, x$ вместо переменных можно получить $\overline x$.
\end{lemma}
\begin{proof}
\begin{equation*}
f(x_1, \ldots, x_n) \notin M \Leftrightarrow
\exists \overline\alpha, \overline\beta \colon
\overline\alpha \preccurlyeq \overline\beta, \ f(\overline\alpha) \nleqslant f(\overline\beta) \Leftrightarrow
f(\overline\alpha) = 1, \ f(\overline\beta) = 0
\end{equation*}
\begin{enumerate}
	\item Пусть $\overline\alpha$ и $\overline\beta$ отличаются в нескольких позициях:
	$\overline\alpha = (\ldots, \gamma_1, \ldots, \gamma_2, \ldots, \gamma_k, \ldots)$, где
	$\gamma_1, \ldots, \gamma_k$~--- позиции, в которых $\overline\alpha$ и $\overline\beta$ отличаются, тогда $\gamma_1 = \gamma_2 = \ldots = \gamma_k = 0$.
	
	Рассматривая эти позиции, введём наборы значений
	\begin{equation*}
	\overline\alpha_0 = \overline\alpha = (\ldots, 0, \ldots, 0, \ldots, 0, \ldots), \
	\overline\alpha_1 = (\ldots, 1, \ldots, 0, \ldots, 0, \ldots),
	\end{equation*}
	\begin{equation*}
	\overline\alpha_2 = (\ldots, 1, \ldots, 1, \ldots, 0, \ldots), \ \ldots, \
	\overline\alpha_k = (\ldots, 1, \ldots, 1, \ldots, 1, \ldots)
	\end{equation*}
	
	Легко показать методом от противного, что $f(\overline\alpha) > f(\overline\beta) \Rightarrow
	\exists 0 \leqslant i < k \colon f(\overline\alpha_i) > f(\overline\alpha_{i+1})$.
	Т.\,о., этот случай сведён к следующему случаю.
	
	\item Пусть $\overline\alpha$ и $\overline\beta$ отличаются только в одной позиции.
	\begin{equation*}
	f(\alpha_1, \ldots, \alpha_{i-1}, 0, \alpha_{i+1}, \ldots, \alpha_n) = 1, \
	f(\alpha_1, \ldots, \alpha_{i-1}, 1, \alpha_{i+1}, \ldots, \alpha_n) = 0 \Rightarrow
	\end{equation*}
	\begin{equation*}
	\Rightarrow f(\alpha_1, \ldots, \alpha_{i-1}, x, \alpha_{i+1}, \ldots, \alpha_n) = \overline x
	\end{equation*}
\end{enumerate}
\end{proof}

\begin{theorem}
Если булева функция~$f \in M$ и $K$~--- простой импликант $f$, то $K$ не содержит отрицаний.
\end{theorem}
\begin{proofcontra}
Пусть дана $f(x_1, \ldots, x_n, y)$, $K = x_1^{\sigma_1} \cdot \ldots \cdot x_k^{\sigma_k} \overline y$.
\begin{equation*}
K(\sigma_1, \ldots, \sigma_k, 0) = 1 \Rightarrow
f(\sigma_1, \ldots, \sigma_k, x_{k+1}, \ldots, x_n, 0) = 1 \Rightarrow
f(\sigma_1, \ldots, \sigma_k, x_{k+1}, \ldots, x_n, 1) = 1
\end{equation*}

Пусть $K' = x_1^{\sigma_1} \cdot \ldots \cdot x_k^{\sigma_k} \Rightarrow
K(\sigma_1, \ldots, \sigma_k) = 1$, $f(\sigma_1, \ldots, \sigma_k, x_{k+1}, \ldots, x_n, y) = 1 \Rightarrow$ $K'$~--- импликант $f \Rightarrow K$ не является простым импликантом $f$.
Противоречие.
\end{proofcontra}

\begin{theorem}
Булева функция $f \in M \Leftrightarrow$ сокращённая ДНФ для неё не содержит отрицаний.
\end{theorem}
\begin{proof}
\begin{enumerate}
	\item $\Rightarrow$. Простые импликанты $f$ не содержат отрицаний.
	Сокращённая ДНФ~--- дизъюнкция простых импликантов, значит, она не содержит отрицаний.
	\item $\Leftarrow$. $f \in M$ как функция, реализуемая формулой над монотонными функциями.
\end{enumerate}
\end{proof}

\begin{theorem}
Сокращённая ДНФ для монотонной булевой функции является тупиковой.
\end{theorem}
\begin{proof}
Пусть $K = x_1 \And \ldots \And x_k$~--- простой импликант монотонной булевой функции~$f(x_1, \ldots, x_n)$, $f = K \lor f'$.
Каждый импликант $f'$ должен содержать один из литералов $x_{k+1}, \ldots, x_n$, т.\,к. $K$~--- простой импликант.
Тогда при $x_1 = \ldots = x_k = 1$, $x_{k+1} = \ldots = x_n = 0$ имеем $f(x_1, \ldots, x_n) = 1$, $f'(x_1, \ldots, x_n) = 0 \Rightarrow f \neq f' \Rightarrow$ из сокращённой ДНФ нельзя вычеркнуть ни один из конъюнктов, поэтому она тупиковая.
\end{proof}

\subsubsection{Класс \texorpdfstring{$L$}{} линейных функций}
\index{Функция!булева!линейная} Булева функция~$f$ называется \textbf{линейной}, если она реализуется формулой над~$\{ +, 1 \}$.
\begin{equation*}
L = \{ f(x_1, \ldots, x_n) \mid
f(x_1, \ldots, x_n) = a_0 + a_1 x_1 + \ldots + a_n x_n, \ a_i \in B \}
\end{equation*}
\begin{proof}[замкнутости]
Пусть
\begin{equation*}
f(x_1, \ldots, x_n) = c_0 + c_1 x_1 + \ldots + c_n x_n,
\end{equation*}
\begin{equation*}
f_1(x_1, \ldots, x_n) = c_{10} + c_{11} x_1 + \ldots + c_{1n} x_n, \ \ldots, \
f_n(x_1, \ldots, x_n) = c_{n0} + c_{n1} x_1 + \ldots + c_{nn} x_n
\end{equation*}
\begin{equation*}
\Phi
= f(f_1, \ldots, f_n) =
c_0 + c_1(c_{10} + c_{11} x_1 + \ldots + c_{1n} x_n) + \ldots + c_n(c_{n0} + c_{n1} x_1 + \ldots + c_{nn} x_n) =
\end{equation*}
\begin{equation*}
= c_0 + c_1 c_{10} + \ldots + c_n c_{n0} + (c_1 c_{11} + \ldots + c_n c_{n1}) x_1 + \ldots + (c_1 c_{1n} + \ldots + c_n c_{nn}) x_n
\end{equation*}

Значит, $\Phi$ реализует функцию из~$L$.
\end{proof}

$L$ содержит $2^{n+1}$~функций от $n$~переменных.

\begin{lemma}[о нелинейной функции]
Если булева функция~$f \notin L$, то подстановкой $0, 1, x, y, \overline x, \overline y$ вместо переменных и, возможно, инверсией значения $f$ можно получить $x \And y$.
\end{lemma}
\begin{proof}
$f \notin L$, тогда без ограничения общности можно считать, что $f$ содержит $x_1 x_2$.
Пусть
\begin{equation*}
f(x_1, \ldots, x_n) = x_1 x_2 f_1(x_3, \ldots, x_n) + x_1 f_2(x_2, x_3, \ldots, x_n) + x_2 f_3(x_1, x_3, \ldots, x_n) + f_4(x_3, \ldots, x_n) \Rightarrow
\end{equation*}
\begin{equation*}
\Rightarrow \exists \alpha_3, \ldots, \alpha_n \colon f_1(\alpha_3, \ldots, \alpha_n) = 1
\end{equation*}

Рассмотрим
\begin{equation*}
\varphi(x_1, x_2) = f(x_1, x_2, \alpha_3, \ldots, \alpha_n) = x_1 x_2 + \alpha x_1 + \beta x_2 + \gamma,
\end{equation*}
\begin{equation*}
\psi(x_1, x_2) = \varphi(x_1 + \beta, x_2 + \alpha) + \alpha\beta + \gamma =
\end{equation*}
\begin{equation*}
= x_1 x_2 + \alpha x_1 + \beta x_2 + \alpha\beta + \alpha x_1 + \alpha\beta + \beta x_2 + \alpha\beta + \gamma + \alpha\beta + \gamma =
x_1 x_2
\end{equation*}

$\psi$ получена из~$\varphi$ подстановкой $x, y, \overline x, \overline y$ и, возможно, инверсией значения $\varphi$, которая, в свою очередь, получена из~$f$ подстановкой $0, 1$.
\end{proof}