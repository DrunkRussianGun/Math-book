\subsection{Логические операции}
Пусть $x, y \in B$.
\index{Операции!логические} Основные логические операции (в порядке убывания приоритета выполнения):
\begin{enumerate}
	\item \index{Инверсия} \index{Отрицание} \textbf{Отрицание}, или \textbf{инверсия}: $\lnot x$, $\overline x$.
	\item \index{Конъюнкция} \textbf{Конъюнкция}, или \textbf{логическое И}: $x \land y$, $x \And y$, $x \cdot y$, $xy$.
	\item \index{Дизъюнкция} \textbf{Дизъюнкция}, или \textbf{логическое ИЛИ}: $x \lor y$
\end{enumerate}

Следующие операции не имеют общепринятого приоритета выполнения, но обычно выполняются после вышеуказанных.
\begin{enumerate}
	\item \textbf{Импликация}: $x \rightarrow y$.
	\item \textbf{Эквиваленция}: $x \leftrightarrow y$, $x \sim y$, $x \equiv y$.
	\item \textbf{Сложение по модулю~2}, или \textbf{исключающее ИЛИ}: $x \oplus y$, $x + y$.
	\item \textbf{Штрих Шеффера}: $x \mid y$.
	\item \textbf{Стрелка Пирса}: $x \downarrow y$.
\end{enumerate}

Приведём таблицу истинности рассмотренных логических операций:
\begin{equation*}
\begin{array}{|c|c|c|c|c|c|c|c|c|c|}
\hline
x & \overline x & y & x \And y & x \lor y & x \rightarrow y & x \sim y & x + y & x \mid y & x \downarrow y \\
\hline
0 & 1 & 0 & 0 & 0 & 1 & 1 & 0 & 1 & 1 \\
\hline
0 & 1 & 1 & 0 & 1 & 1 & 0 & 1 & 1 & 0 \\
\hline
1 & 0 & 0 & 0 & 1 & 0 & 0 & 1 & 1 & 0 \\
\hline
1 & 0 & 1 & 1 & 1 & 1 & 1 & 0 & 0 & 0 \\
\hline
\end{array}
\end{equation*}

\index{Степень!булева} Также определяется \textbf{булева степень}:
\begin{equation*}
x^\sigma =
\begin{cases}
\overline x, \ \sigma = 0 \\
x, \ \sigma = 1
\end{cases}
\end{equation*}
где $\sigma \in B$~--- параметр.