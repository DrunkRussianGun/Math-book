\subsection{Конъюнктивная нормальная форма}
\index{Дизъюнкт} \textbf{Элементарным дизъюнктом} называется дизъюнкция литералов, в которую каждая переменная входит не более одного раза.
Элементарный дизъюнкт называется \textbf{полным}, если он содержит все рассматриваемые переменные.

\textbf{Конъюнктивной нормальной формой}, или \textbf{КНФ}, называется конъюнкция элементарных дизъюнктов.

\textbf{Совершенной конъюнктивной нормальной формой}, или \textbf{СКНФ}, называется конъюнкция полных элементарных дизъюнктов.

\begin{statement}
Булева функция~$f(x_1, \ldots, x_n)$, не равная тождественно~$1$, представима в~виде СКНФ.
\end{statement}
\begin{proof}
\begin{equation*}
f^*(x_1, \ldots, x_n) = \bigvee_{\begin{smallmatrix}
(\sigma_1, \ldots, \sigma_n) \\
f^*(\sigma_1, \ldots, \sigma_n) = 1
\end{smallmatrix}} x_1^{\sigma_1} \And \ldots \And x_n^{\sigma_n} \Leftrightarrow
\end{equation*}
\begin{equation*}
\Leftrightarrow f^{**}(x_1, \ldots, x_n) =
\bigwedge_{\begin{smallmatrix}
(\sigma_1, \ldots, \sigma_n) \\
f^*(\sigma_1, \ldots, \sigma_n) = 1
\end{smallmatrix}} x_1^{\sigma_1} \lor \ldots \lor x_n^{\sigma_n} =
\end{equation*}
\begin{equation*}
= \bigwedge_{\begin{smallmatrix}
(\sigma_1, \ldots, \sigma_n) \\
f(\overline{\sigma_1}, \ldots, \overline{\sigma_n}) = 0
\end{smallmatrix}} x_1^{\sigma_1} \lor \ldots \lor x_n^{\sigma_n} =
\bigwedge_{\begin{smallmatrix}
(\sigma_1, \ldots, \sigma_n) \\
f(\sigma_1, \ldots, \sigma_n) = 0
\end{smallmatrix}} x_1^{\overline{\sigma_1}} \lor \ldots \lor x_n^{\overline{\sigma_n}} \Leftrightarrow
\end{equation*}
\begin{equation*}
\Leftrightarrow f(x_1, \ldots, x_n)
= \bigwedge_{\begin{smallmatrix}
(\sigma_1, \ldots, \sigma_n) \\
f(\sigma_1, \ldots, \sigma_n) = 0
\end{smallmatrix}} x_1^{\overline{\sigma_1}} \lor \ldots \lor x_n^{\overline{\sigma_n}}
\end{equation*}
\end{proof}