\subsection{Остовы}
\textbf{Остовом} графа $G = (V, E)$ называется его подграф~$G' = (V', E')$ такой, что $V = V'$ и~$G'$~--- дерево.

\begin{statement}
Любой связный граф содержит остов.
\end{statement}

\begin{statement}
Если граф не~является деревом, то в~нём несколько остовов.
\end{statement}

Пусть $G = (V, E)$~--- граф.
\textbf{Весом} называется функция~$\alpha \colon E \to \mathbb R^+$.
\textbf{Весом ребра}~$e \in E$ называется $\alpha(e)$.
\textbf{Весом графа} называется $\displaystyle \sum_{e \in E} \alpha(e)$.

\subsubsection{Алгоритмы нахождения остова минимального веса}
Пусть дан граф~$G = (V, E)$, $n = |V|$ и~весовая функция $\alpha \colon E \to R^+$.
Строим остов наименьшего веса $T = (V, P)$.
\begin{enumerate}
	\item\textbf{Алгоритм Краскала}
	\begin{enumerate}
		\item Выбираем ребро $e \in E$ с~наименьшим весом: $P_1 = \{ e \}$, $T_1 = (V, P_1)$.
		\item\label{list:Kruskal's_alg_step2} Выбираем ребро~$e \in E$ с~наименьшим весом такое, что $e \notin P_i$ и~добавление этого ребра не~приводит к~образованию цикла в~$T$: $T_{i+1} = (V, P_i \cup \{ e \})$.
		\item Повторяем шаг \ref{list:Kruskal's_alg_step2} $n - 2$~раз.
		$T_n$~--- искомый остов.
	\end{enumerate}
	\begin{proof}
		Пусть $T = (V, P)$~--- построенный остов, где $P = \{ e_1, e_2, \ldots, e_{n-1} \}$, $e_1, e_2, \ldots, \allowbreak e_{n-1}$~--- рёбра в~порядке их добавления в~остов, а также $D = (V, M)$~--- другой остов, где $M = \{ e_1', e_2', \ldots, \allowbreak e_{n-1}' \}$, $e_1', e_2', \ldots, \allowbreak e_{n-1}'$~--- рёбра в~порядке неубывания их весов.
		
		Если $T \neq D$, то пусть $i$~--- наименьшее число такое, что $e_i \neq e_i'$.
		%Рассмотрим $D' = (V, M \cup \{ e_i \})$.
		%В этом графе ровно один цикл, причём $e_i$ входит в цикл.
		%
		%Данный цикл содержит ребро $e' \notin \{ e_1, \ldots, e_i \} \colon \alpha(e_i) \leqslant \alpha(e')$, т.\,к. $e_1, \ldots, e_i$ не образуют цикл.
		Если $\alpha(e_i') < \alpha(e_i)$, то на~$i$\nobreakdash-м шаге алгоритм выбрал~бы $e_i'$ вместо $e_i$, т.\,к. $e_1, \ldots, e_{i-1}, e_i'$ не~образуют цикл, иначе $D$ содержал~бы цикл, значит, $\alpha(e_i) \leqslant \alpha(e_i')$.
		
		Пусть $D_1 = (V, M \cup \{ e_i \} \setminus \{ e_i' \})$.
		Этот граф~--- остов, причём $\alpha(D_1) \leqslant \alpha(D)$ и~у~$T$ и~$D_1$ на~$1$ общее ребро больше, чем у~$T$ и~$D$.
		Повторяя, получим $D_k = T$.
		Значит, вес построенного остова не~превосходит веса любого другого остова.
	\end{proof}

	\item\textbf{Алгоритм Прима}
	
	Строится последовательность деревьев $S_1 \subset S_2 \subset \ldots \subset S_n = T$.
	\begin{enumerate}
		\item Выбираем произвольную вершину $v$. $S_1 = (\{ v \}, \varnothing)$.
		\item\label{list:Prim's_alg_step2} Пусть построено $S_i = (V_i, E_i)$.
		Находим ребро $e = \{ u, v_i \} \in E$, где $u \in V_i$, $v_i \notin V_i$, наименьшего веса, добавление которого не~приводит к~образованию цикла: $S_{i+1} = (V_i \cup \{ v_i \}, E_i \cup \{ e \})$.
		\item Повторяем шаг \ref{list:Prim's_alg_step2} $n - 1$~раз.
		$S_n$~--- искомый остов.
	\end{enumerate}
\end{enumerate}