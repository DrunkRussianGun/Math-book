\subsection{Эйлеровы графы}
Цикл, содержащий все рёбра графа, называется \textbf{эйлеровым}.

Граф, содержащий эйлеров цикл, называется \textbf{эйлеровым}.

\begin{theorem}
Связный граф эйлеров $\Leftrightarrow$ степени всех вершин чётны.
\end{theorem}
\begin{proof}
\begin{enumerate}
	\item $\Rightarrow$. Пусть в~графе есть эйлеров цикл.
	Выберем вершину~$v_0$ в~этом цикле и~начнём обходить его.
	При~каждом посещении вершины~$v \neq v_0$ её степень увеличивается на~$2$.
	Т.\,о., если посетить её $k$~раз, то $\deg v = 2k \mult 2$.
	
	Для~$v_0$ степень увеличивается на~$1$ в~начале обхода, на~$1$ в~конце обхода и~на~$2$ при~промежуточных посещениях.
	Т.\,о., её степень чётна.
	
	\item $\Leftarrow$. Пусть степени всех вершин чётны.
	Выберём цепь~$C = (v_0; e_0; v_1; e_1; \ldots; e_{k-1}; v_k)$ наибольшей длины.
	Все рёбра, инцидентные~$v_k$, присутствуют в~этой цепи, иначе её можно было~бы удлинить.
	
	Докажем методом от~противного, что $v_0 = v_k$.
	Пусть $v_0 \neq v_k$.
	При прохождении вершины~$v_i = v_k$, где $0 < i < k$, степень~$v_k$ увеличивается на~$2$.
	Также проходим по~ребру~$e_{k-1}$, тогда степень~$v_k$ нечётна.
	Противоречие.
	
	Докажем методом от~противного, что $C$ содержит все рёбра.	
	Пусть найдётся ребро~$e = \{ u, v \}$, не~входящее в~$C$.
	Возьмём первое ребро~$e' = \{ v_i, v' \}$ из~$(v_0, u)$\nobreakdash-\hspace{0pt}маршрута, не~входящее в~$C$.
	Тогда цепь~$(v'; e'; v_i; e_i; \ldots; e_{k-1}; v_k = v_0; e_0; v_1; e_1; \ldots; v_{i-1})$ длиннее, чем~$C$.
	Противоречие.
\end{enumerate}
\end{proof}

\subsubsection{Алгоритмы нахождения эйлерова цикла}
\begin{enumerate}
	\item\textbf{Алгоритм Флёри (очень медленный)}.
	\begin{enumerate}
		\item Выберем произвольную вершину.
		\item Пусть находимся в~вершине~$v$.
		Выберем ребро, инцидентное ей, которое должно быть мостом, только если не~осталось других рёбер.
		\item Проходим по~выбранному ребру и~вычёркиваем его.
		\item Повторяем, пока есть рёбра.
	\end{enumerate}
	\item\textbf{Алгоритм объединения циклов}.
	\begin{enumerate}
		\item\label{list:cycle_alg_step1} Выберем произвольную вершину.
		\item Выбираем любое непосещённое ребро и~идём по~нему.
		\item\label{list:cycle_alg_step3} Повторяем, пока не~вернёмся в~начальную вершину.
		\item\label{list:cycle_alg_step4} Получим цикл~$C$.
		Если он не~эйлеров, то $\exists u \in C, \ e = \{ u, u' \} \colon u' \notin C$.
		Повторяем шаги \ref{list:cycle_alg_step1}--\ref{list:cycle_alg_step3} для~начальной вершины~$u$.
		Получим цикл~$C'$, рёбра которого не~совпадают с~рёбрами $C$.
		Объединим эти циклы и~получим новый.
		Повторяем шаг \ref{list:cycle_alg_step4}.
	\end{enumerate}
\end{enumerate}

Цепь называется \textbf{эйлеровым путём}, если она не~является циклом и~содержит все рёбра.

Граф называется \textbf{полуэйлеровым}, если в~нём есть эйлеров путь.

\begin{theorem}
Связный граф полуэйлеров $\Leftrightarrow$ степени двух вершин нечётны, а остальных~--- чётны.
\end{theorem}
\begin{proof}
\begin{enumerate}
	\item $\Rightarrow$. Пусть в~графе есть эйлеров путь.
	Соединив его концы ребром, получим эйлеров цикл.
	Степени соединённых вершин увеличились каждая на~$1$, значит, они были нечётными, а степени остальных вершин~--- чётными.
	\item $\Leftarrow$. Пусть степени двух вершин нечётны, а остальных~--- чётны.
	Соединим нечётные вершины ребром, тогда можно получить эйлеров цикл.
	Убрав из~него добавленное ребро, получим эйлеров путь.
\end{enumerate}
\end{proof}