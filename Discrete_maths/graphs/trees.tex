\subsection{Деревья}
Граф без~циклов называется \textbf{лесом}.

Связный лес называется \textbf{деревом}.

Ребро называется \textbf{мостом}, если при~его удалении увеличивается число компонент связности.

\begin{statement}
\label{st:criterion_of_bridge_in_graph}
Ребро~--- мост $\Leftrightarrow$ оно не~содержится в~цикле.
\end{statement}
\begin{proof}
\begin{enumerate}
	\item $\Rightarrow$. Докажем методом от~противного, что если ребро содержится в~цикле, то оно не~является мостом.
	Пусть ребро $e$ содержится в~цикле $W = (v_0; e_0; \ldots; u; e; v; \ldots; v_k)$, $u'$ и~$v'$~--- связные вершины.
	\begin{enumerate}
		\item Если в~$(u', v')$\nobreakdash-\hspace{0pt}маршруте нет ребра $e$, то при~его удалении из~графа $u'$ и~$v'$ останутся связными.
		\item Пусть $(u' = v_0'; e_0'; \ldots; u; e; v; \ldots; e_m'; v_m' = v')$~--- маршрут, соединяющий $u'$ и~$v'$, тогда при~удалении $e$ из~графа $u'$ и~$v'$ соединяет маршрут\newline
		$(u' = v_0'; e_0'; \ldots; u; \ldots; e_0; v_0 = v_k; e_{k-1}; \ldots; v; \ldots; e_m'; v_m' = v')$.
	\end{enumerate}
	
	\item $\Leftarrow$. Пусть $e = \{ u, v \}$ не~является мостом, тогда $u$, $v$ лежат в~одной компоненте связности.
	Удалим $e$ из~графа.
	Число компонент связности не~изменится, значит, $u$ и~$v$ также лежат в~одной компоненте связности, т.\,е. существует цепь, соединяющая $u$ и~$v$: $(u = v_0; e_0; \ldots; e_{k-1}; v_k = v)$.
	Тогда в~исходном графе существует цикл $(u = v_0; e_0; \ldots; e_{k-1}; v_k = v; e; u)$.
\end{enumerate}
\end{proof}

\begin{theorem}
Следующие утверждения о~графе~$G = (V, E)$ с~$n$ вершинами эквивалентны:
\begin{enumerate}
	\item\label{st:about_tree_1} $G$~--- дерево.
	\item\label{st:about_tree_2} $G$ связный и~имеет $n - 1$~ребро.
	\item\label{st:about_tree_3} $G$ связный и~каждое его ребро~--- мост.
	\item\label{st:about_tree_4} $G$ не~содержит циклов и~имеет $n - 1$~ребро.
	\item\label{st:about_tree_5} Любые две вершины графа~$G$ соединены ровно одной простой цепью.
	\item\label{st:about_tree_6} $G$ не~содержит циклов и~добавление ребра приводит к~появлению ровно одного цикла.
\end{enumerate}
\end{theorem}
\begin{proof}
\begin{itemize}
	\item Докажем (\ref{st:about_tree_1}) $\Rightarrow$ (\ref{st:about_tree_3}).
	Связность следует из~определения дерева.
	
	В~силу утверждения (\ref{st:criterion_of_bridge_in_graph}) каждое ребро~--- мост.
	
	\item Докажем (\ref{st:about_tree_3}) $\Rightarrow$ (\ref{st:about_tree_2}).
	Связность следует из~предположения.
	
	Докажем методом математической индукции, что в~графе $n - 1$~ребро.
	\indbase Для~$n = 1, 2$ очевидно.
	\indstep Пусть утверждение верно для~чисел, меньших $n$.
	Возьмём мост~$e$ и~удалим его.
	Получим две компоненты связности $G_1 = (V_1, E_1)$, $G_2 = (V_2, E_2)$.
	По~предположению индукции $|E_1| = |V_1| - 1$, $|E_2| = |V_2| - 1$.
	Тогда в~исходном графе рёбер $|E_1| + |E_2| + 1 = |V_1| + |V_2| - 1 = n - 1$. \indend
	
	\item Докажем (\ref{st:about_tree_2}) $\Rightarrow$ (\ref{st:about_tree_4}).
	$G$ имеет $n - 1$~ребро по~предположению.
	
	Докажем методом математической индукции, что $G$ не~содержит циклов.
	\indbase Для~$n = 1, 2$ очевидно.
	\indstep Пусть утверждение верно для~чисел, меньших $n$.
	Докажем методом от~противного, что в~графе есть вершина степени~$1$.
	Пусть
	\begin{equation*}
	\forall u \in V \ \deg u \geqslant 2 \Rightarrow 2|E| = \sum_{u \in V} \deg u \geqslant 2n \Rightarrow n - 1 = |E| \geqslant n \Rightarrow -1 \geqslant 0
	\end{equation*}
	Противоречие, значит, в~графе найдётся вершина степени~$1$.
	
	Удалим её и~инцидентное ей ребро.
	Полученный граф содержит $n - 1$~вершину и~удовлетворяет утверждению~(\ref{st:about_tree_2}).
	По~предположению индукции он не~содержит циклов, тогда и~исходный граф не~содержит циклов. \indend
	
	\item Докажем (\ref{st:about_tree_4}) $\Rightarrow$ (\ref{st:about_tree_5}).
	
	Пусть в~графе $k$~компонент связности: $G_1 = (V_1, E_1)$, $G_2 = (V_2, E_2)$, \ldots, $G_k = (V_k, E_k)$.
	Они не~содержат циклов по~предположению, тогда они являются деревьями.
	\begin{equation*}
	|E_1| = |V_1| - 1, |E_2| = |V_2| - 1, \ldots, |E_k| = |V_k| - 1,
	n - 1 = |E_1| + \ldots + |E_k| = n - k \Rightarrow k = 1
	\end{equation*}
	Значит, граф связный.
	
	Пусть существуют вершины $u$ и~$v$ такие, что их соединяют две простые цепи, тогда по~лемме~(\ref{lemma:existence_of_simple_cycle}) в~графе есть цикл, что противоречит предположению.
	Значит, эти вершины соединены ровно одной простой цепью.
	
	\item Докажем (\ref{st:about_tree_5}) $\Rightarrow$ (\ref{st:about_tree_6}).
	
	Докажем методом от~противного, что в~графе нет циклов.
	Предположим, что есть цикл $(v_0; e_0; v_1; \allowbreak \ldots; \allowbreak v_k = v_0)$, тогда есть две простые цепи $(v_0; e_0; \ldots; v_{k-1})$ и~$(v_{k-1}; e_k; v_k = v_0)$, соединяющие $v_0$ и~$v_{k-1}$, что противоречит предположению.
	
	Докажем, что добавление ребра приводит к~появлению ровно одного цикла.
	Рассмотрим несоседние вершины $u$ и~$v$.
	По~предположению есть цепь $(u = v_0; e_0; \ldots; v_k = v)$, соединяющая их.
	Тогда, добавив $e = \{ u, v \}$, получим цикл $(u = v_0; e_0; \ldots; v_k = v; e; u)$.
	
	Пусть есть $2$~цикла, соединяющих $u$ и~$v$.
	Удалим $e$, тогда один цикл останется.
	Получим исходный граф, в~котором не~должно быть циклов.
	Противоречие.
	
	\item (\ref{st:about_tree_6}) $\Rightarrow$ (\ref{st:about_tree_1}).
	
	Докажем связность методом от~противного.
	Рассмотрим несвязные вершины $u$ и~$v$.
	Соединим их и~по~предположению получим цикл $(v_0; e_0; \ldots; u; e; v; \ldots; e_{k-1}; v_k = v_0)$.
	Тогда в~исходном графе $(u; \ldots; e_0; v_0 = v_k; e_{k-1}; \ldots; v)$~--- $(u, v)$-маршрут.
	Противоречие.
\end{itemize}
\end{proof}