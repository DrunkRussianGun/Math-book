\subsection{Геометрическое представление вещественных чисел}
Самой распространённой интерпретацией множества~$\mathbb R$ является бесконечная прямая.
\begin{figure}[h] \centering
\noindent
$\begin{xy} /r5mm/:
(0, 0) *{\bullet} *++!U{0};
(5, 0) **@{-};
(-5, 0) **@{-};
(1, 0) *{\bullet} *++!U{1};
(2, 0) *{\bullet} *++!U{2};
(3, 0) *{\bullet} *++!U{3};
(-1, 0) *{\bullet} *++!U{-1};
(-2, 0) *{\bullet} *++!U{-2};
(-3, 0) *{\bullet} *++!U{-3};
\end{xy}$
\caption{Множество~$\mathbb R$ в~виде прямой}
\end{figure}

Множество~$\mathbb R$ также можно представить в~виде окружности, одна точка которой соответствует нулю, а другая~--- бесконечности.
\begin{figure}[h] \centering
\noindent
$\begin{xy} /r7mm/:
(0, 0) *\cir<7mm>{};
(-0.8660, -0.5) *{\bullet} *++!R{-2};
(-0.5, -0.8660) *{\bullet} *+!UR{-1};
(0, -1) *{\bullet} *++!U{0};
(0.5, -0.8660) *{\bullet} *+!UL{1};
(0.8660, -0.5) *{\bullet} *++!L{2};
(0, 1) *{\bullet} *++!D{\infty};
\end{xy}$
\caption{Множество~$\mathbb R$ в~виде окружности}
\end{figure}

\begin{floatingfigure}[r]{54mm}
\noindent
$\begin{xy} /r5mm/:
(0, 0) *{\bullet} *++!U{0};
(5, 0) **@{-} *++!D{a};
(-5, 0) **@{-};
(0, 1) *\cir<5mm>{};
(0, 2) = "inf" *{\bullet} *++!D{\infty};
(3.75, -0.5) **@{-};
"inf"; (-2.5, -0.5) **@{-};
\end{xy}$
\end{floatingfigure}
Покажем, что эти интерпретации взаимозаменяемы.
Изобразим их так, чтобы точка, соответствующая нулю на прямой~$a$, совпадала с точкой, соответствующей нулю на окружности.
Теперь из точки, соответствующей бесконечности на окружности, проведём все возможные прямые.
Каждая из них пересекает одну точку на прямой~$a$ и одну точку на окружности и таким образом устанавливает взаимно однозначное соответствие, при этом $-\infty$ и $+\infty$ означают движение к одной и той~же точке на окружности, соответствующей бесконечности.