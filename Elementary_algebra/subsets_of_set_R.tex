\section{Подмножества множества \texorpdfstring{$\mathbb R$}{}}
\hypertarget{def:interval}{\textbf{Промежутком}} называется множество вещественных чисел, которое вместе с~любыми двумя числами содержит любое число между~ними.
Типы промежутков:
\begin{itemize}
	\item \hypertarget{def:closed_interval}{} $[a; b] = \{ x \in \mathbb R \mid a \leqslant x \leqslant b \}$~--- \textbf{отрезок}
	\item \hypertarget{def:open_interval}{} $(a; b) = \{ x \in \mathbb R \mid a < x < b \}$~--- \textbf{интервал}
	\item $[a; b) = \{ x \in \mathbb R \mid a \leqslant x < b \}$~--- \textbf{полуинтервал}
	\item $(a; b] = \{ x \in \mathbb R \mid a < x \leqslant b \}$~--- \textbf{полуинтервал}
\end{itemize}

Полагая $a = \pm\infty$ или~$b = \pm\infty$, можно определить бесконечные промежутки. Например:
\begin{itemize}
	\item $(-\infty; +\infty) = \mathbb R$
	\item $(0; +\infty) = \mathbb R^+$
\end{itemize}

\hypertarget{def:neighbourhood}{\textbf{Окрестностью}} точки~$x \in \mathbb R$ называется интервал~$(a; b) \colon x \in (a; b)$.

\textbf{$\varepsilon$-окрестностью} $U_\varepsilon(x)$ точки~$x \in \mathbb R$ называется интервал~$(x - \varepsilon; x + \varepsilon)$.

\textbf{Проколотой $\varepsilon$-окрестностью} $\breve U_\varepsilon(x)$ точки~$x \in \mathbb R$ называется $U_\varepsilon(x) \setminus \{ x \}$.