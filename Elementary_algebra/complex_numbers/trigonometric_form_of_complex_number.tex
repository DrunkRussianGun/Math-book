\subsection{Тригонометрическая форма комплексного числа}
\textbf{Тригонометрической формой} комплексного числа~$z$ называется его представление в~виде
\begin{equation*}
z = |z|(\cos \varphi + i\sin \varphi), \varphi = \Arg z
\end{equation*}

При~использовании тригонометрических форм операции умножения и~деления комплексных чисел
$z_1 = |z_1|(\cos \alpha + i\sin \alpha), z_2 = |z_2|(\cos \beta + i\sin \beta)$ упрощаются:
\begin{equation*}
z_1 z_2 = |z_1| |z_2|(\cos \alpha \cos \beta - \sin \alpha \sin \beta + i(\sin \alpha \cos \beta + \cos \alpha \sin \beta)) =
|z_1| |z_2|(\cos (\alpha + \beta) + i\sin (\alpha + \beta))
\end{equation*}
\begin{equation*}
\frac{z_1}{z_2} = \frac{|z_1|}{|z_2|} \cdot
\frac{(\cos \alpha \cos \beta + \sin \alpha \sin \beta + i(\sin \alpha \cos \beta - \cos \alpha \sin \beta))}
{\cos^2 \beta + \sin^2 \beta} =
\frac{|z_1|}{|z_2|} (\cos (\alpha - \beta) + i\sin (\alpha - \beta))
\end{equation*}

\begin{theorem}[формула Эйлера]
\label{eq:Euler's_formula}
\begin{equation*}
\cos x + i\sin x = e^{ix}
\end{equation*}
\end{theorem}
\begin{proof}
Воспользуемся разложением $\cos x$, $\sin x$ и $e^{ix}$ в~ряд Маклорена:
\begin{equation*}
\cos x + i\sin x = 1 + \frac{ix}{1!} - \frac{x^2}{2!} - \frac{ix^3}{3!} + \frac{x^4}{4!} + \frac{ix^5}{5!} + \ldots =
1 + \frac{ix}{1!} + \frac{i^2 x^2}{2!} + \frac{i^3 x^3}{3!} + \frac{i^4 x^4}{4!} + \frac{i^5 x^5}{5!} + \ldots = e^{ix}
\end{equation*}
\end{proof}

При~подстановке $x = \pi$ в~формулу Эйлера (\ref{eq:Euler's_formula}) получим замечательное \textbf{тождество Эйлера}, связывающее пять фундаментальных математических констант:
\begin{equation*}
e^{i\pi} + 1 = 0
\end{equation*}

\begin{theorem}[формула Муавра]
\label{eq:de_Moivre's_formula}
Если $z = |z|(\cos \varphi + i\sin \varphi)$, $n \in \mathbb R$, то
\begin{equation*}
z^n = |z|^n(\cos n\varphi + i\sin n\varphi)
\end{equation*}
\end{theorem}
\begin{proof}
Для~$n \in \mathbb N$ формулу можно доказать методом математической индукции, тогда показать истинность формулы для~$n \in \mathbb Z$ несложно.
Мы~же докажем формулу сразу для~$n \in \mathbb R$, пользуясь формулой Эйлера (\ref{eq:Euler's_formula}):
\begin{equation*}
z^n = |z|^n(\cos \varphi + i\sin \varphi)^n =
|z|^n e^{i\varphi n} = |z|^n (\cos n\varphi + i\sin n\varphi)
\end{equation*}
\end{proof}

Пользуясь формулой Муавра (\ref{eq:de_Moivre's_formula}), можно извлекать корни из~комплексного числа~$z = |z|(\cos \varphi \opbr+ i\sin \varphi)$:
\begin{equation*}
\sqrt[n]{z} = \sqrt[n]{|z|}(\cos \frac{\varphi}n + i\sin \frac{\varphi}n)
\end{equation*}
Следует не~забывать, что $\varphi$ определено с~точностью до~$2\pi k, k \in \mathbb Z$, поэтому комплексный корень имеет не~одно, а $n$~значений (что можно показать, пользуясь следствием (\ref{conseq:n_roots_of_polynomial})).