\subsection{Геометрическое представление комплексного числа}
Комплексное число~$a + bi$ принято изображать на~координатной плоскости точкой~$(a; b)$, а также радиус-вектором, соединяющим начало координат с~этой точкой.
Такая плоскость называется \textbf{комплексной}.

\textbf{Модулем} комплексного числа~$z = a + bi$, или его \textbf{абсолютной величиной}, называется длина соответствующего радиус-вектора комплексной плоскости, равная
\begin{equation*}
|z| = \sqrt{a^2 + b^2}
\end{equation*}

\textbf{Аргументом} комплексного числа~$z = a + bi$ называется угол соответствующего радиус-вектора на~комплексной плоскости:
\begin{equation*}
a = |z| \cos \Arg z, \ b = |z| \sin \Arg z
\end{equation*}
\textbf{Главным} аргументом называется значение~$\Arg z \cap (-\pi; \pi]$ и~обозначается $\arg z$.