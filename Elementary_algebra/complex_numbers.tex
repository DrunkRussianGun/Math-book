\section{Комплексные числа}
\index{i@$i$} \index{Мнимая единица} \textbf{Мнимой единицей} называется число, квадрат которого равен $-1$, и обозначается $i$.

\index{Число!комплексное} \textbf{Комплексным} называется число вида $a + bi$, $a, b \in \mathbb R$.
\index{Число!мнимое} Если $a = 0$, то такое число называется \textbf{мнимым}, или \textbf{чисто мнимым}.
\index{C@$\mathbb C$} Множество комплексных чисел обозначается $\mathbb C$.

Если $z = a + bi$, то $\overline z = a - bi$ называется \textbf{сопряжённым к~$z$}.

Следующие операции над комплексными числами $z_1 = a_1 + b_1 i, z_2 = a_2 + b_2 i, \ a_1, b_1, a_2, b_2 \in \mathbb R$ осуществляются так же, как над вещественными, и обладают теми~же свойствами:
\begin{itemize}
	\item\textbf{Сложение}
	\begin{equation*}
	z_1 + z_2 = (a_1 + b_1 i) + (a_2 + b_2 i) = (a_1 + a_2) + (b_1 + b_2)i
	\end{equation*}
	
	\item\textbf{Умножение}
	\begin{equation*}
	z_1 \cdot z_2 = (a_1 + b_1 i)(a_2 + b_2 i) = (a_1 a_2 - b_1 b_2) + (a_1 b_2 + a_2 b_1)i
	\end{equation*}
	
	\item\textbf{Деление}
	\begin{equation*}
	\frac{z_1}{z_2} = \frac{a_1 + b_1 i}{a_2 + b_2 i} =
	\frac{(a_1 + b_1 i)(a_2 - b_2 i)}{a_2^2 + b_2^2} =
	\frac{a_1 a_2 + b_1 b_2}{a_2^2 + b_2^2} + \frac{a_2 b_1 - a_1 b_2}{a_2^2 + b_2^2} i
	\end{equation*}
\end{itemize}