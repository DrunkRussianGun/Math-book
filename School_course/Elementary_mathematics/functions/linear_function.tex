\subsection{Линейная функция}
\index{Функция!линейная} \textbf{Линейной} называется функция вида $y = kx + b$.
$k$ называется \textbf{угловым коэффициентом}.

Очевидно, что $D(y) = E(y) = \mathbb R$.

Докажем монотонность линейной функции при $k \neq 0$.
\begin{enumerate}
	\item Пусть $k < 0, x_1 < x_2$, тогда
	\begin{equation*}
	f(x_1) - f(x_2) =
	k x_1 + b - k x_2 - b =
	k(x_1 - x_2) > 0 \Rightarrow
	f(x_1) > f(x_2)
	\end{equation*}
	
	Значит, функция убывает.
	
	\item Аналогичным образом легко доказать, что при~$k > 0$ функция возрастает.
\end{enumerate}

График линейной функции~--- прямая.
\begin{center}
$\begin{xy} /r8mm/:
(-3, 0); (3, 0) **@{-} *@{>} *++!U{x};
(0, -2); (0, 3) **@{-} *@{>} *++!R{y};
(-3, 2); (3, -1) **@{-} *+++!R{\scriptstyle y = kx + b, \ k < 0};
{(1, 0) \ellipse<2mm>:a(90),_,:a(-63.435){}} *++!LD{\alpha};
(-1, 1) = "M"; (-1, 0) **@{--} *++!U{x_0};
"M"; (0, 1) **@{--} *++!L{y_0};
\end{xy}$
$\begin{xy} /r8mm/:
(-3, 0); (3, 0) **@{-} *@{>} *++!U{x};
(0, -2); (0, 3) **@{-} *@{>} *++!R{y};
(-2, -2); (3, 2.5) **@{-} *+!DR{\scriptstyle y = kx + b, \ k > 0};
{(0.222, 0) \ellipse<3mm>:a(48.013),^,:a(90){}};
(0.9, 0.25) *{\alpha};
(1.5, 1.15) = "M"; (1.5, 0) **@{--} *++!U{x_0};
"M"; (0, 1.15) **@{--} *++!R{y_0};
\end{xy}$
\end{center}

Докажем, что $k = \tg \alpha$.
Пусть $y_0 = k x_0 + b$.
Заметим, что при $k \neq 0$ график линейной функции пересекает ось абсцисс в точке~$(-\frac{b}k, 0)$.
\begin{enumerate}
	\item Если $k < 0$, то для определённости предположим, что $x_0 < -\frac{b}k$.
	\begin{equation*}
	\tg \alpha =
	-\tg (\pi - \alpha) =
	-\frac{y_0}{-\frac{b}k - x_0} =
	\frac{k y_0}{k x_0 + b} =
	k
	\end{equation*}
	
	\item При $k = 0$ $\tg \alpha = 0$.
	
	\item Если $k > 0$, то для определённости предположим, что $x_0 > -\frac{b}k$.
	\begin{equation*}
	\tg \alpha =
	\frac{y_0}{x_0 - (-\frac{b}k)} =
	\frac{k y_0}{k x_0 + b} =
	k
	\end{equation*}
\end{enumerate}