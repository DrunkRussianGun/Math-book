\subsection{Квадратичная функция}
\index{Функция!квадратичная} \textbf{Квадратичной} называется функция вида $y = ax^2 + bx + c$, где $a \neq 0$.

Очевидно, что $D(y) = \mathbb R$.

Для исследования функции выделим в правой части уравнения полный квадрат:
\begin{equation*}
y = a \left( x^2 + \frac{b}a x \right) + c \Leftrightarrow
y = a \left( x + \frac{b}{2a} \right)^2 + c - \frac{b^2}{4a} \Leftrightarrow
y = a \left( x + \frac{b}{2a} \right)^2 - \frac{b^2 - 4ac}{4a}
\end{equation*}
\begin{itemize}
	\item Пусть $a < 0$.
	
	$\displaystyle E(y) = \left( -\infty; -\frac{b^2 - 4ac}{4a} \right]$
	
	Функция возрастает на~$\bigl( -\infty; -\frac{b}{2a} \bigr]$ и убывает на~$\bigl[ -\frac{b}{2a}; +\infty \bigr)$.
	
	\item Пусть $a > 0$.
	
	$\displaystyle E(y) = \left[ -\frac{b^2 - 4ac}{4a}; +\infty \right)$
	
	Функция убывает на~$\bigl( -\infty; -\frac{b}{2a} \bigr]$ и возрастает на~$\bigl[ -\frac{b}{2a}; +\infty \bigr)$.
\end{itemize}

Найдём нули функции:
\begin{equation*}
a \left( x + \frac{b}{2a} \right)^2 - \frac{b^2 - 4ac}{4a} = 0 \Leftrightarrow
\left( x + \frac{b}{2a} \right)^2 = \frac{b^2 - 4ac}{4a^2} \Leftrightarrow
x + \frac{b}{2a} = \pm\frac{\sqrt{b^2 - 4ac}}{2a} \Leftrightarrow
\end{equation*}
\begin{equation*}
\Leftrightarrow x = \frac{-b \pm \sqrt{b^2 - 4ac}}{2a}, \ b^2 - 4ac \geqslant 0
\end{equation*}

Если $b^2 - 4ac < 0$, то график квадратичной функции не пересекает ось абсцисс.
\index{Дискриминант} Выражение $D = b^2 - 4ac$ называется \textbf{дискриминантом квадратного многочлена}.

\index{Парабола} График квадратичной функции называется \textbf{параболой}.
При $a < 0$ её \textbf{ветви} направлены вниз, а при $a > 0$~--- вверх.
Точка наименьшего или наибольшего значения функции называется \textbf{вершиной параболы} (на рисунках обозначена через $M$), которая имеет координаты~$\bigl( -\frac{b}{2a}, -\frac{b^2 - 4ac}{4a} \bigr)$.
\begin{center}
$\begin{xy} /r8mm/:
(-2, 0); (2, 0) **@{-} *@{>} *++!U{x};
(0, -3); (0, 3) **@{-} *@{>} *++!R{y};
(-2, -3); (1.5, -3) **\crv{(-0.25, 5)} *++!L{\scriptstyle y = ax^2 + bx + c, \ a < 0};
(-0.25, 1) *{\bullet} *+!DR{M};
\end{xy}$
$\begin{xy} /r8mm/:
(-2, 0); (2, 0) **@{-} *@{>} *++!U{x};
(0, -3); (0, 3) **@{-} *@{>} *++!R{y};
(-1, 3); (2, 3) **\crv{(0.5, -2)} *++!L{\scriptstyle y = ax^2 + bx + c, \ a > 0};
(0.5, 0.5) *{\bullet} *+!UL{M};
\end{xy}$
\end{center}