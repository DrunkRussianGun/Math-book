\subsection{Ограниченные функции}
\index{Функция!ограниченная} Функция~$f(x)$ называется \textbf{ограниченной сверху на множестве~$X$}, если $\exists M \colon \forall x \in X \ f(x) \leqslant M$.
Если $X \opbr= D(f)$, то $f(x)$ называется \textbf{ограниченной сверху}.

Функция~$f(x)$ называется \textbf{ограниченной снизу на множестве~$X$}, если $\exists m \colon \forall x \in X \ f(x) \geqslant m$.
Если $X = D(f)$, то $f(x)$ называется \textbf{ограниченной снизу}.

Функция называется \textbf{ограниченной на множестве}, если она ограничена и сверху, и снизу на этом множестве.
Если данное множество совпадает с областью определения этой функции, то она называется \textbf{ограниченной}.