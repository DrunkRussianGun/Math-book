\subsection{Обратные функции}
\index{Функция!обратимая} Функция называется \textbf{обратимой}, если $\forall x_1, x_2 \in D(f) \ (x_1 \neq x_2 \Rightarrow f(x_1) \neq f(x_2))$.

\index{Функция!обратная} Функция~$g \colon Y \to X$ называется \textbf{обратной к} обратимой \textbf{функции~$f \colon X \to Y$}, если $\forall x \in X \ g(f(x)) = x \opbr\lAnd \forall y \in Y \ f(g(y)) = y$, и обозначается $f^{-1}$.

\begin{statement}
$(f^{-1})^{-1} = f$.
\end{statement}
\begin{proof}
Пусть даны функции $f \colon X \to Y$ и $g \colon Y \to X$, причём $g = f^{-1}$, тогда
$\forall x \in Y \ f(g(x)) = x \opbr\lAnd \forall y \in X \ g(f(y)) = y \Rightarrow f = g^{-1} = (f^{-1})^{-1}$.
\end{proof}

Функции $f$ и $g = f^{-1}$ называются \textbf{взаимно обратными}.

\begin{theorem}
Графики взаимно обратных функций симметричны относительно прямой~$y = x$.
\end{theorem}
\begin{proof}
Пусть даны функции $f \colon X \to Y$ и $g \colon Y \to X$, причём $g = f^{-1}$, тогда $f(a) = b$, $g(b) \opbr= g(f(a)) \opbr= a$, где $a \in X$, $b \in Y$.
Т.\,о., точка~$(a, b)$ принадлежит графику функции~$f$ $\Leftrightarrow$ точка~$(b, a)$ принадлежит графику функции~$g$.

Найдём расстояния от точек~$(a, b)$ и $(b, a)$ до произвольной точки~$(c, c)$ прямой~$y = x$:
\begin{equation*}
d_1 = \sqrt{(a - c)^2 + (b - c)^2}, \
d_2 = \sqrt{(b - c)^2 + (a - c)^2}
\end{equation*}

$d_1 = d_2$, значит, прямая~$y = x$~--- серединный перпендикуляр к отрезку, концами которого являются точки $(a, b)$ и $(b, a)$, поэтому они симметричны относительно $y = x$.
В силу того, что $a$ может принимать любое значение из множества~$X$, графики функций $f$ и $g$ симметричны относительно прямой~$y = x$.
\end{proof}