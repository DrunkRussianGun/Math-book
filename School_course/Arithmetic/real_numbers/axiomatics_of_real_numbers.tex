\subsection{Аксиоматика вещественных чисел}
Аксиомы сложения:
\begin{enumerate}
	\item Коммутативность сложения:
	\begin{equation*}
	\forall a, b \in \mathbb R \
	a + b = b + a
	\end{equation*}
	
	\item Ассоциативность сложения:
	\begin{equation*}
	\forall a, b, c \in \mathbb R \
	a + (b + c) = (a + b) + c
	\end{equation*}
	
	\item Существование \textbf{нуля}:
	\begin{equation*}
	\exists 0 \in \mathbb R \colon
	\forall a \in \mathbb R \
	a + 0 = a
	\end{equation*}
	
	\item Существование \textbf{противоположного} числа:
	\begin{equation*}
	\forall a \in \mathbb R \
	\exists (-a) \in \mathbb R \colon
	a + (-a) = 0
	\end{equation*}
\end{enumerate}

Аксиомы умножения:
\begin{enumerate}
	\item Коммутативность умножения:
	\begin{equation*}
	\forall a, b \in \mathbb R \
	a \cdot b = b \cdot a
	\end{equation*}
	
	\item Ассоциативность умножения:
	\begin{equation*}
	\forall a, b, c \in \mathbb R \
	a \cdot (b \cdot c) = (a \cdot b) \cdot c
	\end{equation*}
	
	\item Дистрибутивность умножения относительно сложения:
	\begin{equation*}
	\forall a, b, c \in \mathbb R \
	a \cdot (b + c) = a \cdot b + a \cdot c
	\end{equation*}
	
	\item Существование \textbf{единицы}:
	\begin{equation*}
	\exists 1 \in \mathbb R \colon
	\forall a \in \mathbb R \
	a \cdot 1 = a
	\end{equation*}
	
	\item Существование \textbf{обратного} числа:
	\begin{equation*}
	\forall a \in \mathbb R \setminus \{ 0 \} \
	\exists \frac1a = a^{-1} \in \mathbb R \colon
	a \cdot a^{-1} = 1
	\end{equation*}
\end{enumerate}

Аксиомы порядка:
\begin{enumerate}
	\item Рефлексивность:
	\begin{equation*}
	\forall a \in \mathbb R \
	a \leqslant a
	\end{equation*}
	
	\item Антисимметричность:
	\begin{equation*}
	\forall a, b \in \mathbb R \
	a \leqslant b, \ b \leqslant a \Rightarrow a = b
	\end{equation*}
	
	\item Транзитивность:
	\begin{equation*}
	\forall a, b, c \in \mathbb R \
	a \leqslant b, \ b \leqslant c \Rightarrow a \leqslant c
	\end{equation*}
	
	\item Линейная упорядоченность:
	\begin{equation*}
	\forall a, b \in \mathbb R \
	a \leqslant b \text{ или } b \leqslant a
	\end{equation*}
	
	\item Связь сложения и порядка:
	\begin{equation*}
	\forall a, b, c \in \mathbb R \
	a \leqslant b \Rightarrow a + c \leqslant b + c
	\end{equation*}
	
	\item Связь умножения и порядка:
	\begin{equation*}
	\forall a, b \in \mathbb R \
	0 \leqslant a, \ 0 \leqslant b \Rightarrow 0 \leqslant a \cdot b
	\end{equation*}
\end{enumerate}

Аксиома нетривиальности:
\begin{equation*}
0 \neq 1
\end{equation*}

\hypertarget{eq:continuity_axiom}{Аксиома непрерывности}:
\begin{equation*}
\forall A, B \subset \mathbb R \colon
(\forall a \in A, \ b \in B \ a \leqslant b) \
\exists x \in \mathbb R \colon
a \leqslant x \leqslant b
\end{equation*}