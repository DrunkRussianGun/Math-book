\section{Прогрессии}
\subsection{Арифметическая прогрессия}
\index{Прогрессия!арифметическая} \textbf{Арифметической прогрессией} называется последовательность~$(a_n)$, если $\exists d \colon \forall n \in \mathbb N \ a_{n+1} = a_n + d$.
$d$ называется \textbf{разностью арифметической прогрессии}.
Если $d < 0$, то прогрессия называется \textbf{убывающей}, а если $d > 0$, то \textbf{возрастающей}.

\begin{statement}
$\forall n \in \mathbb N \ a_n = a_1 + d(n - 1)$.
\end{statement}
\begin{proofmathind}
	\indbase $a_1 = a_1 + d \cdot (1 - 1)$
	\indstep Пусть $a_n = a_1 + d(n - 1)$, тогда $a_{n+1} = a_n + d = a_1 + dn$. \indend
\end{proofmathind}

\begin{theorem}[характеристическое свойство арифметической прогрессии]
$(a_n)$~--- арифметическая прогрессия $\Leftrightarrow$ $\forall n \in \mathbb N \setminus \{ 1 \} \ a_n = \dfrac{a_{n-1} + a_{n+1}}2$.
\end{theorem}
\begin{proof}
\begin{enumerate}
	\item $\Rightarrow$.
	\begin{equation*}
	a_n = \frac{a_1 + d(n - 1)}2 + \frac{a_1 + d(n - 1)}2 =
	\frac{a_1 + d(n - 2) + a_1 + dn}2 =
	\frac{a_{n-1} + a_{n+1}}2
	\end{equation*}
	
	\item $\Leftarrow$. Пусть $d = a_2 - a_1$.
	Докажем методом математической индукции, что $\forall n \in \mathbb N \ a_{n+1} - a_n = d$.
		\indbase $a_2 - a_1 = d$ по определению.
		\indstep Пусть $a_{n+1} - a_n = d$.
		\begin{equation*}
		a_{n+1} = \frac{a_n + a_{n+2}}2 \Leftrightarrow
		2 a_{n+1} = a_n + a_{n+2} \Leftrightarrow
		a_{n+2} - a_{n+1} = a_{n+1} - a_n = d
		\end{equation*}
		\indend
		
	Тогда $\forall n \in \mathbb N \ a_{n+1} = a_n + d \Rightarrow (a_n)$~--- арифметическая прогрессия.
\end{enumerate}
\end{proof}

\begin{lemma}
\begin{equation*}
\forall i, j \in \mathbb N \ i + j = n + 1 \Rightarrow a_i + a_j = a_1 + a_n
\end{equation*}
\end{lemma}
\begin{proof}
\begin{equation*}
a_i + a_j =
a_1 + d(i - 1) + a_1 + d(j - 1) =
2 a_1 + d(i + j - 2) =
2 a_1 + d(n + 1 - 2) =
a_1 + a_1 + d(n - 1) =
a_1 + a_n
\end{equation*}
\end{proof}

\begin{theorem}
\begin{equation*}
\sum_{k=1}^n a_k = \frac{a_1 + a_n}2 \cdot n
\end{equation*}
\end{theorem}
\begin{proof}
\begin{equation*}
2 \sum_{k=1}^n a_k =
\sum_{k=1}^n (a_k + a_k) =
\sum_{k=1}^n (a_k + a_{n-k+1}) \;
\left| k + (n - k + 1) = n + 1 \right| =
\sum_{k=1}^n (a_1 + a_n) =
n(a_1 + a_n) \Leftrightarrow
\end{equation*}
\begin{equation*}
\Leftrightarrow \sum_{k=1}^n a_k = \frac{a_1 + a_n}2 \cdot n
\end{equation*}
\end{proof}

\subsection{Геометрическая прогрессия}
\index{Прогрессия!геометрическая} \textbf{Геометрической прогрессией} называется последовательность~$(b_n)$, если $b_1 \neq 0 \lAnd \exists q \neq 0 \colon \forall n \in \mathbb N \ b_{n+1} = b_n q$.
$q$~называется \textbf{знаменателем геометрической прогрессии}.

\begin{statement}
$\forall n \in \mathbb N \ b_n = b_1 q^{n-1}$
\end{statement}
\begin{proofmathind}
	\indbase $b_1 = b_1 q^{1-1}$
	\indstep Пусть $b_n = b_1 q^{n-1}$, тогда $b_{n+1} = b_n q = b_1 q^n$. \indend
\end{proofmathind}

\begin{theorem}[характеристическое свойство геометрической прогрессии]
$(b_n)$~--- геометрическая прогрессия $\Leftrightarrow \forall n \in \mathbb N \setminus \{ 1 \} \ |b_n| = \sqrt{b_{n-1} b_{n+1}} \neq 0$.
\end{theorem}
\begin{proof}
\begin{enumerate}
	\item $\Rightarrow$.
	\begin{equation*}
	|b_n| = \sqrt{b_n^2} =
	\sqrt{b_1 q^{n-2} b_1 q^n} =
	\sqrt{b_{n-1} b_{n+1}}
	\end{equation*}
	
	\item $\Leftarrow$. Пусть $q = \dfrac{b_2}{b_1}$.
	Докажем методом математической индукции, что $\forall n \in \mathbb N \ \dfrac{b_{n+1}}{b_n} = q$.
		\indbase $\dfrac{b_2}{b_1} = q$ по определению.
		\indstep Пусть $\dfrac{b_{n+1}}{b_n} = q$.
		\begin{equation*}
		|b_{n+1}| = \sqrt{b_n b_{n+2}} \Leftrightarrow
		b_{n+1}^2 = b_n b_{n+2} \Leftrightarrow
		\frac{b_{n+2}}{b_{n+1}} = \frac{b_{n+1}}{b_n} = q
		\end{equation*}
		\indend
		
	Тогда $\forall n \in \mathbb N \ b_{n+1} = b_n q$ $\Rightarrow$ $(b_n)$~--- геометрическая прогрессия.
\end{enumerate}
\end{proof}

\begin{theorem}
\begin{equation*}
\sum_{k=1}^n b_k = \frac{b_1(q^n - 1)}{q - 1} = \frac{b_n q - b_1}{q - 1}, \ q \neq 1
\end{equation*}
\end{theorem}
\begin{proof}
\begin{equation*}
(q - 1) \sum_{k=1}^n b_k =
\sum_{k=1}^n (b_{k+1} - b_k) =
b_{n+1} - b_1 \Leftrightarrow
\sum_{k=1}^n b_k = \frac{b_n q - b_1}{q - 1} = \frac{b_1(q^n - 1)}{q - 1}
\end{equation*}
\end{proof}

Если $q = 1$, то очевидно, что $\sum\limits_{k=1}^n b_k = b_1 n$.

Геометрическая прогрессия называется \textbf{бесконечно убывающей}, если модуль её знаменателя меньше~$1$.
В этом случае $\sum\limits_{k=1}^\infty b_k = \frac{b_1}{1 - q}$.