\subsection{Корень многочлена}
\index{Корень} \textbf{Корнем многочлена~$f(x)$} называется такое число~$a$, что $f(a) = 0$.

\index{Теорема!Безу}
\begin{theorem}[Безу]
Остаток от деления многочлена~$f(x)$ на двучлен~$x - a$ равен $f(a)$.
\end{theorem}
\begin{proof}
\begin{equation*}
f(x) = g(x)(x - a) + r \Rightarrow f(a) = g(a)(a - a) + r \Leftrightarrow r = f(a)
\end{equation*}
\end{proof}

\begin{consequent}
\label{conseq:division_by_binomial}
Если $a$~--- корень~$f(x)$, то $f(x)$ делится на~$x - a$ без остатка.
\end{consequent}

\textbf{Кратностью корня~$a$} многочлена~$f(x)$ называется число~
$m \colon f(x) \mult (x - a)^m, \ f(x) \notmult (x - a)^{m+1}$.

\begin{theorem}
Если многочлен~$P(x) = a_n x^n + a_{n-1} x^{n-1} + \ldots + a_1 x + a_0$, где $a_0, \ldots, a_n \in \mathbb Z$, имеет рациональный корень, то этот корень равен частному делителя числа~$a_0$ и делителя числа~$a_n$.
\end{theorem}
\begin{proof}
Пусть $\frac{p}q$~--- несократимая дробь, являющаяся корнем $P(x)$.
Тогда
\begin{equation*}
a_n \left( \frac{p}q \right)^n + a_{n-1} \left( \frac{p}q \right)^{n-1} + \ldots + a_1 \frac{p}q + a_0 = 0 \Leftrightarrow
a_n p^n + a_{n-1} p^{n-1} q + \ldots + a_1 pq^{n-1} a_0 q^n = 0
\end{equation*}

Учитывая, что $\NOD(p, q) = 1$, получим
\begin{equation*}
a_n p^n = -q (a_{n-1} p^{n-1} + \ldots + a_1 pq^{n-2} + a_0 q^{n-1}) \Rightarrow
a_n \mult q
\end{equation*}
\begin{equation*}
a_0 q^n = -p (a_n p^{n-1} + a_{n-1} p^{n-2} q + \ldots + a_1 q^{n-1}) \Rightarrow
a_0 \mult p
\end{equation*}
\end{proof}