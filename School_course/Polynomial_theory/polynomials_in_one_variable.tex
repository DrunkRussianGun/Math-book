\section{Многочлены от одной переменной}
\index{Одночлен} \index{Моном} \textbf{Одночленом}, или \textbf{мономом}, называется произведение числового множителя и нуля и более переменных, взятых каждая в~неотрицательной степени.

\textbf{Степенью одночлена} называется сумма степеней входящих в~него переменных.
Степень тождественного нуля равна~$-\infty$.

\index{Многочлен} \index{Полином} \textbf{Многочленом}, или \textbf{полиномом}, \textbf{от одной переменной} называется сумма вида
\begin{equation*}
a_n x^n + a_{n-1} x^{n-1} + \ldots + a_1 x + a_0, \ a_n \neq 0
\end{equation*}
где $x_1, \ldots, x_n$~--- переменные.

\index{deg} \textbf{Степенью многочлена~$f$} называется максимальная из степеней его одночленов и обозначается $\deg f$.
Многочлен $1$-й степени называется \textbf{линейным}, $2$-й степени~--- \textbf{квадратным}, $3$-й степени~--- \textbf{кубическим}.

\begin{statement}
Пусть $f$ и $g$~--- многочлены, тогда $\deg (f + g) \leqslant \max \{ \deg f, \deg g \}$.
\end{statement}
\begin{statement}
Пусть $f$ и $g$~--- многочлены, тогда $\deg fg = \deg f + \deg g$.
\end{statement}

\subsection{Деление многочленов}
\begin{theorem}
Пусть $f(x)$ и $g(x) \neq 0$~--- многочлены, тогда существуют единственные многочлены $q(x)$ и $r(x)$ такие, что $f = qg + r$, причём $\deg r < \deg g$.
\end{theorem}
\begin{proof}
Пусть $\deg f = n$, $\deg g = m$.
\begin{enumerate}
	\item Докажем существование.
	Если $m = 0$, то $q = \frac{f}g$, $r = 0$.
	Пусть $m > 0$.
	Если $n < m$, то $q = 0$, $r = f$.
	Пусть $n \geqslant m$.
	В~таком случае докажем существование $q$ и $r$ методом математической индукции.
		\indbase $n = m$.
		Пусть $f = a x^n + f_1$, $g = b x^n + g_1$, тогда $\deg f_1, \deg g_1 < n$.
		$\deg (f_1 - \frac{a}b g_1) \opbr\leqslant \max \{ \deg f_1, \deg g_1 \} \opbr< \deg g$, тогда $q = \frac{a}b$, $r = f_1 - \frac{a}b g_1$.
		\indstep Пусть $n > m$, теорема верна для $k < n$ и $f = a x^n + f_1$, $g = b x^m + g_1$.
		Рассмотрим
		\begin{equation*}
		h_1(x) = \frac{a}b x^{n-m} \Rightarrow
		h_1 g = a x^n + h_1 g_1 \Rightarrow
		f - h_1 g = a x^n + f_1 - a x^n - h_1 g_1 = f_1 - h_1 g_1
		\end{equation*}
		
		Тогда
		\begin{equation*}
		\deg h_1 g_1 = \deg h_1 + \deg g_1 < (n - m) + m = n \Rightarrow
		\end{equation*}
		\begin{equation*}
		\Rightarrow \deg (f - h_1 g) = \deg (f_1 - h_1 g_1) \leqslant \max \{ \deg f_1, \deg h_1 g_1 \} < n
		\end{equation*}
		
		По предположению индукции
		\begin{equation*}
		f - h_1 g = q_1 g + r \Rightarrow
		f = (h_1 + q_1)g + r, \ \deg r < \deg g
		\end{equation*}
		\indend
		
	\item Докажем единственность.
	Пусть
	\begin{equation*}
	f = q_1 g + r_1 = q_2 g + r_2 \Rightarrow
	(q_1 - q_2)g = r_2 - r_1, \ \deg r_1, \deg r_2 < \deg g
	\end{equation*}
	
	Возможны два случая:
	\begin{enumerate}
		\item $q_1 \neq q_2 \Rightarrow
		\deg (r_1 - r_2) \leqslant \max \{ \deg r_1, \deg r_2 \} < \deg g \leqslant \deg (q_1 - q_2)g$
		
		Противоречие.
		
		\item $q_1 = q_2 \Rightarrow r_1 = r_2$
	\end{enumerate}
\end{enumerate}
\end{proof}

Многочлен~$q$ называется \textbf{частным}, а $r$~--- \textbf{остатком от деления $\frac{f}g$}.
Если $r = 0$, то говорят, что \textbf{$f$ делится на $g$ без остатка}, и пишут $f \mult g$.

\subsection{Корень многочлена}
\index{Корень} \textbf{Корнем многочлена~$f(x)$} называется такое число~$a$, что $f(a) = 0$.

\index{Теорема!Безу}
\begin{theorem}[Безу]
Остаток от деления многочлена~$f(x)$ на двучлен~$x - a$ равен $f(a)$.
\end{theorem}
\begin{proof}
\begin{equation*}
f(x) = g(x)(x - a) + r \Rightarrow f(a) = g(a)(a - a) + r \Leftrightarrow r = f(a)
\end{equation*}
\end{proof}

\begin{consequent}
\label{conseq:division_by_binomial}
Если $a$~--- корень~$f(x)$, то $f(x)$ делится на~$x - a$ без остатка.
\end{consequent}

\textbf{Кратностью корня~$a$} многочлена~$f(x)$ называется число~
$m \colon f(x) \mult (x - a)^m, \ f(x) \notmult (x - a)^{m+1}$.

\begin{theorem}
Если многочлен~$P(x) = a_n x^n + a_{n-1} x^{n-1} + \ldots + a_1 x + a_0$, где $a_0, \ldots, a_n \in \mathbb Z$, имеет рациональный корень, то этот корень равен частному делителя числа~$a_0$ и делителя числа~$a_n$.
\end{theorem}
\begin{proof}
Пусть $\frac{p}q$~--- несократимая дробь, являющаяся корнем $P(x)$.
Тогда
\begin{equation*}
a_n \left( \frac{p}q \right)^n + a_{n-1} \left( \frac{p}q \right)^{n-1} + \ldots + a_1 \frac{p}q + a_0 = 0 \Leftrightarrow
a_n p^n + a_{n-1} p^{n-1} q + \ldots + a_1 pq^{n-1} a_0 q^n = 0
\end{equation*}

Учитывая, что $\NOD(p, q) = 1$, получим
\begin{equation*}
a_n p^n = -q (a_{n-1} p^{n-1} + \ldots + a_1 pq^{n-2} + a_0 q^{n-1}) \Rightarrow
a_n \mult q
\end{equation*}
\begin{equation*}
a_0 q^n = -p (a_n p^{n-1} + a_{n-1} p^{n-2} q + \ldots + a_1 q^{n-1}) \Rightarrow
a_0 \mult p
\end{equation*}
\end{proof}