\subsection{Возрастающие и убывающие функции}
\index{Функция!возрастающая} Функция~$f$ называется \textbf{возрастающей}, или \textbf{монотонной}, \textbf{на множестве~$X$}, если $\forall x_1, x_2 \in X \ x_1 \opbr< x_2 \opbr\Rightarrow f(x_1) \opbr< f(x_2)$.

\index{Функция!убывающая} Функция~$f$ называется \textbf{убывающей}, или \textbf{монотонной}, \textbf{на множестве~$X$}, если $\forall x_1, x_2 \in X \ x_1 \opbr< x_2 \opbr\Rightarrow f(x_1) \opbr> f(x_2)$.

Если функция возрастает (убывает) на всей области определения, то её называют \textbf{возрастающей (убывающей)}, или \textbf{монотонной}.

Свойства монотонных функций:
\begin{enumerate}
	\item Если $f(x)$ и $g(x)$~--- возрастающие (убывающие) на множестве~$X$ функции, то $h(x) = f(x) + g(x)$~--- возрастающая (убывающая) на~$X$ функция.
	\begin{proof}
	Пусть $f(x)$ и $g(x)$ возрастают на~$X$, $x_1, x_2 \in X \colon x_1 < x_2$.
	\begin{equation*}
	x_1 < x_2 \Rightarrow
	f(x_1) < f(x_2) \lAnd g(x_1) < g(x_2) \Rightarrow
	f(x_1) + g(x_1) < f(x_2) + g(x_2) \Rightarrow
	h(x_1) < h(x_2)
	\end{equation*}
	
	Значит, $h(x)$ возрастает на~$X$.
	
	Доказательство для случая убывания аналогично.
	\end{proof}
	
	\item Если $f(x)$~--- возрастающая (убывающая) на множестве~$X$ функция, то:
	\begin{itemize}
		\item при $C < 0$ $h(x) = Cf(x)$~--- убывающая (возрастающая) на~$X$ функция;
		\item при $C > 0$ $h(x) = Cf(x)$~--- возрастающая (убывающая) на~$X$ функция.
	\end{itemize}
	\begin{proof}
	Пусть $f(x)$ возрастает на~$X$, $C < 0$, $x_1, x_2 \in X \colon x_1 < x_2$.
	\begin{equation*}
	x_1 < x_2 \Rightarrow
	f(x_1) < f(x_2) \Rightarrow
	C f(x_1) > C f(x_2) \Rightarrow
	h(x_1) > h(x_2)
	\end{equation*}
	
	Значит, $h(x)$ убывает на~$X$.
	
	Доказательства для остальных трёх случаев аналогичны.
	\end{proof}
	
	\item Если $f(x)$ и $g(x)$~--- возрастающие (убывающие) на множестве~$X$ функции и
	\begin{itemize}
		\item $f(x), g(x) < 0$ на~$X$, то $h(x) = f(x)g(x)$~--- убывающая (возрастающая) на~$X$ функция;
		\item $f(x), g(x) > 0$ на~$X$, то $h(x) = f(x)g(x)$~--- возрастающая (убывающая) на~$X$ функция.
	\end{itemize}
	\begin{proof}
	Пусть $f(x)$ и $g(x)$ возрастают на~$X$, $f(x), g(x) < 0$ на~$X$, $x_1, x_2 \in X \colon x_1 < x_2$.
	\begin{equation*}
	x_1 < x_2 \Rightarrow
	f(x_1) < f(x_2) \lAnd g(x_1) < g(x_2) \Rightarrow
	\end{equation*}
	\begin{equation*}
	\Rightarrow -f(x_1) > -f(x_2) \And -g(x_1) > -g(x_2) \Rightarrow
	f(x_1)g(x_1) > f(x_2)g(x_2) \Rightarrow
	h(x_1) > h(x_2)
	\end{equation*}
	
	Значит, $h(x)$ убывает на~$X$.
	
	Доказательства для остальных трёх случаев аналогичны.
	\end{proof}
	
	\item Если $f(x) \neq 0$~--- функция, возрастающая (убывающая) на множестве~$X$ и сохраняющая на нём знак, то $h(x) = \frac1{f(x)}$~--- убывающая (возрастающая) на~$X$ функция.
	\begin{proof}
	Пусть $f(x)$ возрастает на~$X$, $x_1, x_2 \in X \colon x_1 < x_2$.
	\begin{equation*}
	x_1 < x_2 \Rightarrow
	f(x_1) < f(x_2) \Rightarrow
	\frac{f(x_1)}{f(x_1)f(x_2)} < \frac{f(x_2)}{f(x_1)f(x_2)} \Rightarrow
	\frac1{f(x_1)} > \frac1{f(x_2)} \Rightarrow
	h(x_1) > h(x_2)
	\end{equation*}
	
	Значит, $h(x)$ убывает на~$X$.
		
	Доказательство для случая убывания аналогично.
	\end{proof}
	
	\item Если $f(x)$ и $g(x)$~--- обе возрастающие или обе убывающие на множестве~$X$ функции, то $h(x) \opbr= g(f(x))$~--- возрастающая на~$X$ функция.
	\begin{proof}
	Пусть $f(x)$ и $g(x)$ возрастают на~$X$, $x_1, x_2 \in X \colon x_1 < x_2$.
	\begin{equation*}
	x_1 < x_2 \Rightarrow
	f(x_1) < f(x_2) \Rightarrow
	g(f(x_1)) < g(f(x_2)) \Rightarrow
	h(x_1) < h(x_2)
	\end{equation*}
	
	Значит, $h(x)$ возрастает на~$X$.
	
	Доказательство для случая убывания аналогично.
	\end{proof}
\end{enumerate}