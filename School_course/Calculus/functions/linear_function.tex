\subsection{Линейная функция}
\index{Функция!линейная} \textbf{Линейной} называется функция вида $y = kx + b$, где $k \neq 0$.

Очевидно, что $D(y) = E(y) = \mathbb R$.

Докажем монотонность линейной функции.
\begin{enumerate}
	\item Пусть $k < 0, x_1 < x_2$, тогда
	\begin{equation*}
	f(x_1) - f(x_2) =
	k x_1 + b - k x_2 - b =
	k(x_1 - x_2) > 0 \Rightarrow
	f(x_1) > f(x_2)
	\end{equation*}
	
	Значит, функция убывает.
	
	\item Аналогичным образом легко доказать, что при~$k > 0$ функция возрастает.
\end{enumerate}

\begin{center}
$\begin{xy} /r8mm/:
(-3, 0); (3, 0) **@{-} *@{>} *++!U{x};
(0, -3); (0, 3) **@{-} *@{>} *++!R{y};
(-3, 2); (3, -1) **@{-} *+++!R{\scriptstyle y = kx + b, \ k < 0};
\end{xy}$
$\begin{xy} /r8mm/:
(-3, 0); (3, 0) **@{-} *@{>} *++!U{x};
(0, -3); (0, 3) **@{-} *@{>} *++!R{y};
(-3, -1); (3, 2) **@{-} *+!DR{\scriptstyle y = kx + b, \ k > 0};
\end{xy}$
\end{center}