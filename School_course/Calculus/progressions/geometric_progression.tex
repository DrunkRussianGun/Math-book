\subsection{Геометрическая прогрессия}
\index{Прогрессия!геометрическая} \textbf{Геометрической прогрессией} называется последовательность~$(b_n)$, если $b_1 \neq 0 \lAnd \exists q \neq 0 \colon \allowbreak \forall n \in \mathbb N \allowbreak b_{n+1} \opbr= b_n q$.
$q$ называется \textbf{знаменателем геометрической прогрессии}.

\begin{statement}
$\forall n \in \mathbb N \ b_n = b_1 q^{n-1}$
\end{statement}
\begin{proofmathind}
	\indbase $b_1 = b_1 q^{1-1}$
	\indstep Пусть $b_n = b_1 q^{n-1}$, тогда $b_{n+1} = b_n q = b_1 q^n$. \indend
\end{proofmathind}

\begin{theorem}[характеристическое свойство геометрической прогрессии]
$(b_n)$~--- геометрическая прогрессия $\Leftrightarrow \lAnd \forall n \in \mathbb N \setminus \{ 1 \} \ |b_n| = \sqrt{b_{n-1} b_{n+1}} \neq 0$.
\end{theorem}
\begin{proof}
\begin{enumerate}
	\item $\Rightarrow$.
	\begin{equation*}
	|b_n| = \sqrt{b_n^2} =
	\sqrt{b_1 q^{n-2} b_1 q^n} =
	\sqrt{b_{n-1} b_{n+1}}
	\end{equation*}
	
	\item $\Leftarrow$. Пусть $q = \frac{b_2}{b_1}$.
	Докажем методом математической индукции, что $\forall n \in \mathbb N \ \frac{b_{n+1}}{b_n} = q$.
		\indbase $\frac{b_2}{b_1} = q$ по определению.
		\indstep Пусть $\frac{b_{n+1}}{b_n} = q$.
		\begin{equation*}
		|b_{n+1}| = \sqrt{b_n b_{n+2}} \Leftrightarrow
		b_{n+1}^2 = b_n b_{n+2} \Leftrightarrow
		\frac{b_{n+2}}{b_{n+1}} = \frac{b_{n+1}}{b_n} = q
		\end{equation*}
		\indend
		
	Тогда $\forall n \in \mathbb N \ b_{n+1} = b_n q \Rightarrow (b_n)$~--- геометрическая прогрессия.
\end{enumerate}
\end{proof}

\begin{theorem}
\begin{equation*}
\sum_{i=1}^n b_i = \frac{b_1(q^n - 1)}{q - 1} = \frac{b_n q - b_1}{q - 1}, \ q \neq 1
\end{equation*}
\end{theorem}
\begin{proof}
\begin{equation*}
(q - 1) \sum_{i=1}^n b_i =
\sum_{i=1}^n (b_{i+1} - b_i) =
b_{n+1} - b_1 \Leftrightarrow
\sum_{i=1}^n b_i = \frac{b_n q - b_1}{q - 1} = \frac{b_1(q^n - 1)}{q - 1}
\end{equation*}
\end{proof}

Если $q = 1$, то очевидно, что $\displaystyle \sum_{i=1}^n b_i = b_1 n$.

Геометрическая прогрессия называется \textbf{бесконечно убывающей}, если модуль её знаменателя меньше~$1$.
В этом случае $\displaystyle \sum_{i=1}^\infty b_i = \frac{b_1}{1 - q}$.