\section{Функции}
\index{Функция} Пусть $A$ и $B$~--- множества.
\textbf{Функцией~$f$} называется правило, ставящее в~соответствие каждому элементу~$a \in A$ единственный элемент~$f(a) \in B$.
$A$ называется \textbf{областью определения функции~$f$} и обозначается $D(f)$, $B$~--- \textbf{областью значений функции~$f$} и обозначается $E(f)$.
$a$ называется \textbf{прообразом~$f(a)$}, $f(a)$~--- \textbf{образом~$a$}.

\index{Последовательность} \textbf{Последовательностью} называется функция, заданная на множестве~$X \subseteq \mathbb N$, и обозначается~$(x_n)$.

\textbf{Подпоследовательностью последовательности~$(x_n)$} называется последовательность~$(x_{n_k})$, если $\forall k \in \mathbb N \ n_k < n_{k+1}$.