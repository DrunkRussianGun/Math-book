\section{Замкнутые классы булевых функций}
\begin{theorem}[Поста о функциональной полноте]
\index{Теорема!Поста}
Класс~$F$ булевых функций полон $\Leftrightarrow$ он не лежит целиком ни в одном из классов $T_0, T_1, S, M, L$.
\end{theorem}
\begin{proof}
\begin{enumerate}
	\item $\Rightarrow$. Докажем методом от противного.
	Пусть среди классов $T_0, T_1, S, M, L$ найдётся класс~$K \colon F \subseteq K$, тогда $[F] \subseteq [K] = K \neq P_2$, значит, $F$ не является полным.
	Противоречие.
	
	\item $\Leftarrow$.
	\begin{equation*}
	\exists f_0, f_1, f_S, f_M, f_L \in F \colon
	f_0 \notin T_0 \lAnd f_1 \notin T_1 \lAnd f_S \notin S \lAnd f_M \notin M \lAnd f_L \notin L
	\end{equation*}
	\begin{enumerate}
		\item Покажем, что $0$ и $1$ реализуются формулой над~$\{ f_0, f_1, f_S \}$.
		Пусть $\varphi(x) = f_0(x, \ldots, x) \opbr\Rightarrow \varphi(0) = 1$.
		\begin{enumerate}
			\item Пусть $\varphi(1) = 0 \Rightarrow \varphi(x) = \overline x$.
			Подставляя $x, \overline x$ в~$f_S$, получим одну из констант.
			Другую константу можно выразить через полученную константу и~$\overline x$.
			\item Пусть $\varphi(1) = 1 \Rightarrow \varphi(x) = 1 \Rightarrow
			f_1(\varphi(x), \ldots, \varphi(x)) = f(1, \ldots, 1) = 0$.
		\end{enumerate}
		
		\item Подставляя $0, 1, x$ в~$f_M$, получим $\overline x$.
		
		\item Подставляя $0, 1, x, \overline x, y, \overline y$ в~$f_L$ и, возможно, изменяя её значение на противоположное, получим~$x \land y$.
	\end{enumerate}
	
	Т.\,о., функции из полного набора~$\{ \neg, \land \}$ реализуются формулами над~$\{ f_0, f_1, f_S, f_M, f_L \} \subseteq F$, значит, $F$~--- полный набор.
\end{enumerate}
\end{proof}

\begin{statement}
$T_0, T_1, S, M, L$ попарно различны.
\end{statement}
\begin{proof}
Составим таблицу, в которой $+$ означает принадлежность функции классу, а $-$ означает её отсутствие в классе.
\begin{equation*}
\begin{array}{|c|c|c|c|c|c|}
\hline
  & T_0 & T_1 & S & M & L \\
  \hline
0 & + & - & - & + & + \\
\hline
1 & - & + & - & + & + \\
\hline
\overline x & - & - & + & - & + \\
\hline
\end{array}
\end{equation*}
\end{proof}

Класс~$F$ булевых функций называется \textbf{предполным}, если $[F] \neq P_2$ и $\forall f \notin F \colon [F \cup \{ f \}] = P_2$.

\begin{statement}
Существует ровно 5 предполных классов булевых функций: $T_0, T_1, S, M, L$.
\end{statement}
\begin{proof}
\begin{enumerate}
	\item $\Rightarrow$. Пусть $K \in \{ T_0, T_1, S, M, L \}$, $f \notin K$.
	$K \cup \{ f \}$ не лежит целиком ни в одном из классов $T_0, T_1, S, M, L$, значит, $[K \cup \{ f \}] = P_2$, т.\,е. $K$~--- предполный класс.
	\item $\Leftarrow$. Пусть $K$~--- предполный класс $\Rightarrow$ $K$ не является полным $\Rightarrow$ $K$ лежит в одном из классов $T_0, T_1, S, M, L$.
	
	Докажем методом от противного, что $K$ равен одному из них.
	Пусть $K_1 \in \{ T_0, T_1, S, M, L \}$, $K \subset K_1$.
	$f \notin K \lAnd f \in K_1 \Rightarrow K \cup \{ f \} \subseteq K_1 \neq P_2$, значит, $K$ не является предполным классом.
	Противоречие.
\end{enumerate}
\end{proof}

\begin{consequent}
Любой замкнутый класс булевых функций $F \neq P_2$ целиком лежит в одном из классов $T_0, T_1, S, M, L$.
\end{consequent}
\begin{proofcontra}
Пусть $F$ не лежит ни в одном из классов $T_0, T_1, S, M, L$, тогда $F = [F] = P_2$.
Противоречие.
\end{proofcontra}

Пусть $F$~--- замкнутый набор булевых функций. Набор~$M \subseteq F$ называется \textbf{полным в~$F$}, если $[M] = F$.

\index{Базис!класса булевых функций} Набор~$K$ булевых функций называется \textbf{базисом замкнутого класса~$F$ булевых функций}, если $K$ полон в~$F$ и при удалении из него любой булевой функции он перестаёт быть полным.