\section{Разложение булевых функций по переменным}
\begin{theorem}
Булева функция~$f(x_1, \ldots, x_n)$ может быть записана в~виде
\begin{equation*}
f(x_1, \ldots, x_n) = \bigvee_{(\sigma_1, \ldots, \sigma_m)}
x_1^{\sigma_1} \And \ldots \And x_m^{\sigma_m} \And f(\sigma_1, \ldots, \sigma_m, x_{m+1}, \ldots, x_n)
\end{equation*}
\end{theorem}
\begin{proof}
Рассмотрим произвольный набор $(\alpha_1, \ldots, \alpha_n)$ и покажем, что левая и правая части данного соотношения принимают на нём одно и то~же значение:
\begin{enumerate}
	\item Для левой части получим $f(\alpha_1, \ldots, \alpha_n)$.
	\item Для правой части получим
	\begin{equation*}
	\bigvee_{(\sigma_1, \ldots, \sigma_m)}
	\alpha_1^{\sigma_1} \And \ldots \And \alpha_m^{\sigma_m} \And f(\sigma_1, \ldots, \sigma_m, \alpha_{m+1}, \ldots, \alpha_n) =
	\end{equation*}
	\begin{equation*}
	\left| \alpha_1^{\sigma_1} \And \ldots \And \alpha_m^{\sigma_m} = 1 \Leftrightarrow
	\sigma_1 = \alpha_1, \ \ldots, \ \sigma_m = \alpha_m \right|
	\end{equation*}
	\begin{equation*}
	= \alpha_1^{\alpha_1} \And \ldots \And \alpha_m^{\alpha_m} \And f(\alpha_1, \ldots, \alpha_m, \alpha_{m+1}, \ldots, \alpha_n)
	= f(\alpha_1, \ldots, \alpha_n)
	\end{equation*}
\end{enumerate}
\end{proof}

\begin{consequent}
Любая булева функция может быть реализована формулой над~$\{ \neg, \And, \lor \}$.
\end{consequent}