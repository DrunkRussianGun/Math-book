\section{Дизъюнктивная нормальная форма}
\index{Литерал} \textbf{Литералом} называется переменная или её отрицание.

\index{Конъюнкт} \textbf{Элементарным конъюнктом} называется конъюнкция литералов, в которую каждая переменная входит не более одного раза.
Элементарный конъюнкт называется \textbf{полным}, если он содержит все рассматриваемые переменные.

\index{Импликант} Элементарный конъюнкт~$K$ называется \textbf{импликантом} булевой функции~$f$, если $K \lor f = f$.
Импликант называется \textbf{простым}, если вычёркиванием литералов из него нельзя получить импликант $f$.

\begin{statement}
Элементарный конъюнкт~$K$~--- импликант булевой функции~$f$ $\Leftrightarrow K \rightarrow f = 1$.
\end{statement}
\begin{proof}
$K \lor f = f \Leftrightarrow
\overline K \lor K \lor f = f \lor \overline K \Leftrightarrow
1 = K \rightarrow f$.
\end{proof}

\index{Сокращения!ДНФ} \textbf{Дизъюнктивной нормальной формой} (\textbf{ДНФ}) называется дизъюнкция элементарных конъюнктов.

\index{Сокращения!СДНФ} \textbf{Совершенной дизъюнктивной нормальной формой} (\textbf{СДНФ}) называется дизъюнкция полных элементарных конъюнктов.

\begin{statement}
Булева функция~$f$ от $n$~переменных, не равная тождественно~$0$, представима в~виде СДНФ.
\end{statement}%
Для доказательства достаточно разложить её по всем переменным.

ДНФ булевой функции~$f$ называется \textbf{сокращённой}, если все её конъюнкты~--- простые импликанты $f$.

\begin{statement}
Булева функция представима в~виде сокращённой ДНФ, причём единственным образом.
\end{statement}
\begin{proof}
Пусть $f(x_1, \ldots, x_n)$~--- булева функция, $D = K_1 \lor K_2 \lor \ldots \lor K_m$~--- дизъюнкция всех простых импликантов $f$.
Возможны два случая:
\begin{enumerate}
	\item Пусть $f = 0 \Leftrightarrow
	K_i = 0 \Leftrightarrow
	D = 0 \Rightarrow
	D = f$, где $i = 1, \ldots, m$.
	
	\item Пусть $f = 1$.
	Запишем $f$ в~виде СДНФ:
	\begin{equation*}
	1 = f = \bigvee_{\begin{smallmatrix}
	(\sigma_1, \ldots, \sigma_n) \\
	f(\sigma_1, \ldots, \sigma_n) = 1
	\end{smallmatrix}} x_1^{\sigma_1} \land x_2^{\sigma_2} \land \ldots \land x_n^{\sigma_n} \Rightarrow
	\exists \alpha_1, \ldots, \alpha_n \colon x_1^{\alpha_1} \land x_2^{\alpha_2} \land \ldots \land x_n^{\alpha_n} = 1
	\end{equation*}
	
	Получили импликант.
	Из него можно получить простой импликант~$K$ вычёркиванием литералов, причём $K = 1$.
	$K$ входит в~$D$, тогда $D = 1 = f$.
\end{enumerate}
\end{proof}