\section{Построение тупиковых ДНФ}
Пусть дана булева функция~$f$.
\begin{enumerate}
	\item Находим сокращённую ДНФ $D = K_1 \lor \ldots \lor K_m$ для~$f$.
	
	\item Пусть $N_f = \{ P_1, \ldots, P_r \}$.
	\index{Таблица!Квайна} Составляем следующую таблицу, называемую \textbf{таблицей Квайна}:
	\begin{equation*}
	\begin{array}{|c|c|c|c|c|}
	\hline
	    & P_1 & P_2 & \cdots & P_r \\
    \hline
	K_1 & \sigma_{11} & \sigma_{12} & \cdots & \sigma_{1r} \\
	\hline
	K_2 & \sigma_{21} & \sigma_{22} & \cdots & \sigma_{2r} \\
	\hline
	\vdots & \vdots & \vdots & \ddots & \vdots \\
	\hline
	K_m & \sigma_{m1} & \sigma_{m2} & \cdots & \sigma_{mr} \\
	\hline
	\end{array}\,, \
	\sigma_{ij} =
	\begin{cases}
	0, \ P_j \notin N_{K_i} \\
	1, \ P_j \in N_{K_i}
	\end{cases}
	\end{equation*}
	
	\item Составляем выражение $\bigand\limits_{j=1}^r (\sigma_{1j} K_1 \lor \sigma_{2j} K_2 \lor \ldots \lor \sigma_{mj} K_m)$ и раскрываем в нём скобки по формулам $(A \lor B)C \opbr= AC \lor BC$, $A \lor BA = A$.
	В полученной ДНФ относительно переменных $K_1, \ldots, K_m$ каждому конъюнкту однозначно соответствует тупиковая ДНФ для~$f$: $K_{i_1} \land \ldots \land K_{i_p} \to K_{i_1} \lor \ldots \lor K_{i_p}$.
\end{enumerate}
\begin{proof}
$\sigma_{1j} K_1 \lor \ldots \lor \sigma_{mj} K_m = 1 \Rightarrow
P_j \in N_{\sigma_{1j} K_1} \cup \ldots \cup N_{\sigma_{mj} K_m}$, тогда если
$\bigand\limits_{j=1}^r (\sigma_{1j} K_1 \lor \ldots \lor \sigma_{mj} K_m) = 1$, то набор полученных граней покрывает все вершины из~$N_f$.
\end{proof}