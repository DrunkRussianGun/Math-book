\section{Операции типа I и II}
Пусть даны ДНФ~$D$ и эквивалентная ей $D'$.
Говорят, что:
\begin{itemize}
	\item ДНФ~$D'$ получается из~$D$ \textbf{операцией типа I}, если $D'$ получена из~$D$ вычёркиванием элементарного конъюнкта;
	\item ДНФ~$D'$ получается из~$D$ \textbf{операцией типа II}, если $D'$ получена из~$D$ вычёркиванием одного или нескольких литералов в каком-либо элементарном конъюнкте.
\end{itemize}

ДНФ~$D$ называется \textbf{тупиковой относительно операций типа I и II}, если они к ней неприменимы.

\begin{statement}
ДНФ~$D$ тупиковая относительно операций типа I и II $\Leftrightarrow$ $D$ тупиковая в геометрическом смысле.
\end{statement}
\begin{proof}
\begin{enumerate}
	\item $\Rightarrow$. Если к~$D$ неприменимы операции типа I и II, то ей соответствует неприводимое покрытие, поэтому $D$ тупиковая в геометрическом смысле.
	\item $\Leftarrow$.
	\begin{enumerate}
		\item Операции типа I соответствует удаление грани из покрытия, соответствующего $D$.
		Удалить грань нельзя, значит, операция типа I неприменима к~$D$.
		\item Пусть $D = K_1 \lor \ldots \lor K_m$, $K_i = x_j^{\sigma_j} K_i'$, тогда $N_{K_i'} \subset N_{K_i}$.
		Если $N_f = N_{K_1} \cup \ldots \cup N_{K_i'} \cup \ldots \cup N_{K_m}$, то $N_{K_i}$ не является максимальной гранью, что неверно, значит, операция типа II неприменима к~$D$.
	\end{enumerate}
	
	Тогда $D$ тупиковая относительно операций типа I и II.
\end{enumerate}
\end{proof}