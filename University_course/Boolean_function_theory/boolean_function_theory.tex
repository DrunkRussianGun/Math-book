\index{Множество!булево} \textbf{Булевым} называется множество~$B = \{ 0, 1 \}$.

\index{Функция!булева} \textbf{Булевой} называется функция~$f \colon B^n \to B$.
Булеву функцию можно задать таблицей, называемой \textbf{таблицей истинности}.

\begin{statement}
Количество булевых функций от $n$~переменных равно $2^{2^n}$.
\end{statement}
\begin{proof}
Число всех возможных наборов аргументов булевой функции от $n$~переменных равно $2^n$, тогда число всех возможных таких функций равно $2^{2^n}$.
\end{proof}

\index{Переменная!существенная} \index{Переменная!фиктивная} Пусть $f(x_1, \ldots, x_n)$~--- булева функция.
Переменная $x_i$ называется \textbf{существенной}, если
\begin{equation*}
\exists a_1, \ldots, a_{i-1}, a_{i+1}, \ldots, a_n \colon
f(a_1, \ldots, a_{i-1}, 0, a_{i+1}, \ldots, a_n) \neq f(a_1, \ldots, a_{i-1}, 1, a_{i+1}, \ldots, a_n)
\end{equation*}
иначе~--- \textbf{несущественной} (\textbf{фиктивной}).

Две булевы функции называются \textbf{равными}, если одну из них можно получить из другой последовательным удалением или введением несущественных переменных.