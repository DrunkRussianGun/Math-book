\section{Уравнения второго порядка}
Уравнение вида $a_{11} x^2 + a_{22} y^2 + a_{33} z^2 + 2 a_{12} xy + 2 a_{13} xz + 2 a_{23} yz + 2 a_{14} x + 2 a_{24} y + 2 a_{34} z + a_{44} = 0$, где $x, y, z$~--- переменные, называется \textbf{уравнением второй степени}.

\textbf{Инвариантами} называются следующие определители:
\begin{enumerate}
	\item $I_1 = a_{11} + a_{22} + a_{33}$
	
	\item $I_2 =
	\begin{vmatrix}
	a_{11} & a_{12} \\
	a_{12} & a_{22}
	\end{vmatrix} +
	\begin{vmatrix}
	a_{11} & a_{13} \\
	a_{13} & a_{33}
	\end{vmatrix} +
	\begin{vmatrix}
	a_{22} & a_{23} \\
	a_{23} & a_{33}
	\end{vmatrix}$
	
	\item $I_3 =
	\begin{vmatrix}
	a_{11} & a_{12} & a_{13} \\
	a_{12} & a_{22} & a_{23} \\
	a_{13} & a_{23} & a_{33}
	\end{vmatrix}$
	
	\item $I_4 =
	\begin{vmatrix}
	a_{11} & a_{12} & a_{13} & a_{14} \\
	a_{12} & a_{22} & a_{23} & a_{24} \\
	a_{13} & a_{23} & a_{33} & a_{34} \\
	a_{14} & a_{24} & a_{34} & a_{44}
	\end{vmatrix}$
\end{enumerate}

Применив к поверхности, задаваемой данным уравнением, параллельный перенос на вектор~$(a, b, c)$, получим
\begin{equation*}
a_{11} (x + a)^2 + a_{22} (y + b)^2 + a_{33} (z + c)^2 + 2 a_{12} (x + a)(y + b) + 2 a_{13} (x + a)(z + c) + 2 a_{23} (y + b)(z + c) + {}
\end{equation*}
\begin{equation*}
{} + 2 a_{14} (x + a) + 2 a_{24} (y + b) + 2 a_{34} (z + c) + a_{44} = 0 \Leftrightarrow
\end{equation*}
\begin{equation*}
\Leftrightarrow a_{11} x^2 + a_{22} y^2 + a_{33} z^2 + 2 a_{12} xy + 2 a_{13} xz + 2 a_{23} yz + {}
\end{equation*}
\begin{equation*}
{} + 2 (a_{11} a + a_{12} b + a_{13} c + a_{14}) x + 2 (a_{12} a + a_{22} b + a_{23} c + a_{24}) y + 2 (a_{13} a + a_{23} b + a_{33} c + a_{34}) z + {}
\end{equation*}
\begin{equation*}
{} + a_{11} a^2 + a_{22} b^2 + a_{33} c^2 + 2 a_{12} ab + 2 a_{13} ac + 2 a_{23} bc + 2 a_{14} a + 2 a_{24} b + 2 a_{34} c + a_{44} = 0
\end{equation*}

В дальнейшем при преобразованиях новые коэффициенты в уравнении могут обозначаться $a_{11}', a_{12}', \ldots$ или же для удобства записи $a_{11}, a_{12}, \ldots$

\begin{theorem}
$I_1$, $I_2$, $I_3$, $I_4$ не меняются при параллельном переносе и поворотах.
\end{theorem}

Если $I_3 \neq 0$, то $\exists! (x_0, y_0, z_0) \colon
\begin{cases}
a_{11} x_0 + a_{12} y_0 + a_{13} z_0 + a_{14} = 0 \\
a_{12} x_0 + a_{22} y_0 + a_{23} z_0 + a_{24} = 0 \\
a_{13} x_0 + a_{23} y_0 + a_{33} z_0 + a_{34} = 0 
\end{cases}$ и систему координат можно перенести на вектор~$(x_0, y_0, z_0)$, получив уравнение
\begin{equation*}
a_{11} x^2 + a_{22} y^2 + a_{33} z^2 + 2 a_{12} xy + 2 a_{13} xz + 2 a_{23} yz + a_{44} = 0
\end{equation*}

В таком случае поверхность называется \textbf{центральной}, а точка~$(x_0, y_0, z_0)$~--- её \textbf{центром}.

Найдём такой поворот, при котором $a_{13}' = a_{23}' = 0$.
Для этого введём функцию
\begin{equation*}
F(x, y, z) = a_{11} x^2 + a_{22} y^2 + a_{33} z^2 + 2 a_{12} xy + 2 a_{13} xz + 2 a_{23} yz, \
D(F) = \{ (x, y, z) \mid x^2 + y^2 + z^2 = 1 \}
\end{equation*}
и найдём точку $P \colon F(P) = \max F(x, y, z)$.

Теперь осуществим такой поворот, чтобы $Oz' = OP$, и рассмотрим плоскость $Ox'z'$.
В этой плоскости
\begin{equation*}
F(x, z) =
F(\sin \Theta, \cos \Theta) =
F(\Theta) =
a_{11} \sin^2 \Theta + a_{33} \cos^2 \Theta + 2 a_{13} \sin \Theta \cos \Theta =
\end{equation*}
\begin{equation*}
= \frac{a_{11}}2 (1 - \cos 2\Theta) + \frac{a_{33}}2 (1 + \cos 2\Theta) + a_{13} \sin 2\Theta =
a_{13} \sin 2\Theta - \frac{a_{11} - a_{33}}2\,\cos 2\Theta + \frac{a_{11} + a_{33}}2
\end{equation*}
\begin{equation*}
F'(\Theta) = 2 a_{13} \cos 2\Theta + (a_{11} - a_{33}) \sin 2\Theta
\end{equation*}

Тогда
\begin{equation*}
P = \max F \Rightarrow
F'(0) = 0 \Rightarrow
a_{13} = 0
\end{equation*}

Аналогично рассматривая плоскость, получим $a_{23} = 0$.
Затем применим поворот, описанный в разделе~\ref{sect:second_degree_equations_on_plane}, и получим $a_{12} = 0$.

Т.\,о., далее можно ограничиться рассмотрением уравнений вида $a_{11} x^2 + a_{22} y^2 + a_{33} z^2 + 2 a_{14} x + 2 a_{24} y + 2 a_{34} z \opbr+ a_{44} = 0$.
\begin{enumerate}
	\item Пусть $I_3 \neq 0$, тогда, найдя корни $\lambda_1, \lambda_2, \lambda_3$ характеристического многочлена, получим уравнение
	\begin{equation*}
	\lambda_1 x^2 + \lambda_2 y^2 + \lambda_3 z^2 + \frac{I_4}{I_3} = 0
	\end{equation*}
	\begin{enumerate}
		\item Пусть $I_4 \neq 0$, тогда получим
		\begin{equation*}
		\frac{x^2}{-\frac{I_4}{\lambda_1 I_3}} + \frac{y^2}{-\frac{I_4}{\lambda_2 I_3}} + \frac{z^2}{-\frac{I_4}{\lambda_3 I_3}} = 1
		\end{equation*}
		
		В зависимости от знаков коэффициентов получим
		\begin{itemize}
			\item Эллипсоид:
			$\displaystyle \frac{x^2}{a^2} + \frac{y^2}{b^2} + \frac{z^2}{c^2} = 1$
			\item Пустое множество:
			$\displaystyle \frac{x^2}{a^2} + \frac{y^2}{b^2} + \frac{z^2}{c^2} = -1$
			\item Однополостный гиперболоид:
			$\displaystyle \frac{x^2}{a^2} + \frac{y^2}{b^2} - \frac{z^2}{c^2} = 1 \Leftrightarrow
			\frac{x^2}{a^2} + \frac{y^2}{b^2} = \frac{z^2}{c^2} + 1$
			\item Двуполостный гиперболоид:
			$\displaystyle \frac{x^2}{a^2} + \frac{y^2}{b^2} - \frac{z^2}{c^2} = -1 \Leftrightarrow
			\frac{x^2}{a^2} + \frac{y^2}{b^2} = \frac{z^2}{c^2} - 1$
		\end{itemize}
		
		\item Пусть $I_4 = 0$, тогда получим
		\begin{equation*}
		\frac{x^2}{\frac1{\lambda_1}} + \frac{y^2}{\frac1{\lambda_2}} + \frac{z^2}{\frac1{\lambda_3}} = 0
		\end{equation*}
		
		В зависимости от знаков коэффициентов получим
		\begin{itemize}
			\item Вырожденный конус (точка~$(0, 0, 0)$):
			$\displaystyle \frac{x^2}{a^2} + \frac{y^2}{b^2} + \frac{z^2}{c^2} = 0$
			\item Конус:
			$\displaystyle \frac{x^2}{a^2} + \frac{y^2}{b^2} - \frac{z^2}{c^2} = 0$
		\end{itemize}
	\end{enumerate}
	
	\item Пусть $I_3 = 0$, тогда
	\begin{equation*}
	\begin{vmatrix}
	a_{11} & 0 & 0 \\
	0 & a_{22} & 0 \\
	0 & 0 & a_{33}
	\end{vmatrix} = 0 \Leftrightarrow
	a_{11} a_{22} a_{33} = 0
	\end{equation*}
	\begin{enumerate}
		\item Пусть $a_{11}, a_{22} \neq 0$, $a_{33} = 0$, тогда
		\begin{equation*}
		a_{11} x^2 + a_{22} y^2 + 2 a_{14} x + 2 a_{24} y + 2 a_{34} z + a_{44} = 0 \Leftrightarrow
		\end{equation*}
		\begin{equation*}
		\Leftrightarrow a_{11} \left(x + \frac{a_{14}}{a_{11}}\right)^2 + a_{22} \left(y + \frac{a_{24}}{a_{22}}\right)^2 + 2 a_{34} z + a_{44} - \frac{a_{14}^2}{a_{11}} - \frac{a_{24}^2}{a_{22}} = 0 \Rightarrow
		\end{equation*}
		\begin{equation*}
		\Rightarrow a_{11} x^2 + a_{22} y^2 + 2 p z + q = 0
		\end{equation*}
		\begin{itemize}
			\item Пусть $p = q = 0$, тогда $a_{11} x^2 + a_{22} y^2 = 0$.
			\begin{itemize}
				\item Если $a_{11} a_{22} < 0$, то получим две пересекающиеся плоскости:
				$\displaystyle \frac{x^2}{a^2} - \frac{y^2}{b^2} = 0 \Leftrightarrow
				y = \pm\frac{b}a\,x$
				\item Если $a_{11} a_{22} > 0$, то получим прямую:
				$\displaystyle \frac{x^2}{a^2} + \frac{y^2}{b^2} = 0 \Leftrightarrow
				x = y = 0$
			\end{itemize}
			
			\item Пусть $p = 0$, $q \neq 0$, тогда
			\begin{equation*}
			a_{11} x^2 + a_{22} y^2 + q = 0 \Leftrightarrow
			\frac{x^2}{-\frac{q}{a_{11}}} + \frac{y^2}{-\frac{q}{a_{22}}} = 1
			\end{equation*}
			
			В зависимости от знаков коэффициентов получим
			\begin{itemize}
				\item Пустое множество:
				$\displaystyle -\frac{x^2}{a^2} - \frac{y^2}{b^2} = 1$
				\item Гиперболический цилиндр:
				$\displaystyle \frac{x^2}{a^2} - \frac{y^2}{b^2} = 1$
				\item Эллиптический цилиндр:
				$\displaystyle \frac{x^2}{a^2} + \frac{y^2}{b^2} = 1$
			\end{itemize}
			
			\item Пусть $p \neq 0$, тогда
			\begin{equation*}
			a_{11} x^2 + a_{22} y^2 + 2 pz + q = 0 \Leftrightarrow
			a_{11} x^2 + a_{22} y^2 + 2 p \left(z + \frac{q}{2p}\right) = 0 \Rightarrow
			a_{11} x^2 + a_{22} y^2 + 2 pz \Leftrightarrow
			\end{equation*}
			\begin{equation*}
			\Leftrightarrow z = \frac{x^2}{-\frac{2p}{a_{11}}} + \frac{y^2}{-\frac{2p}{a_{22}}}
			\end{equation*}
			
			В зависимости от знаков коэффициентов получим
			\begin{itemize}
				\item Эллиптический параболоид:
				$\displaystyle z = \frac{x^2}{a^2} + \frac{y^2}{b^2}$
				\item Гиперболический параболоид:
				$\displaystyle z = \frac{x^2}{a^2} - \frac{y^2}{b^2}$
			\end{itemize}
		\end{itemize}
		
		\item Пусть $a_{11} = a_{22} = 0$, $a_{33} \neq 0$, тогда
		\begin{equation*}
		a_{33} z^2 + 2 a_{14} x + 2 a_{24} y + 2 a_{34} z + a_{44} = 0 \Leftrightarrow
		a_{33} \left(z + \frac{a_{34}}{a_{33}}\right)^2 + 2 a_{14} x + 2 a_{24} y + a_{44} - \frac{a_{34}^2}{a_{33}} = 0 \Rightarrow
		\end{equation*}
		\begin{equation*}
		\Rightarrow a_{33} z^2 + 2 p x + 2 q y + r = 0
		\end{equation*}
		\begin{itemize}
			\item Пусть $p = q = 0$, тогда $a_{33} z^2 + r = 0$.
			\begin{itemize}
				\item Если $a_{33} r \leqslant 0$, то получим две параллельные плоскости:
				$\displaystyle z = \pm\sqrt{-\frac{r}{a_{33}}}$
				\item Если $a_{33} r > 0$, то получим пустое множество:
				$\displaystyle \frac{z^2}{c^2} = -1$
			\end{itemize}
			
			\item Пусть $p \neq 0 \lOr q \neq 0$.
			Подставив~$z = 0$, получим прямую, которую с помощью поворота можно преобразовать в прямую, параллельную оси~$Ox$.
			Применив такой поворот, получим параболический цилиндр:
			\begin{equation*}
			a_{33} z^2 + 2 q y + r = 0 \Leftrightarrow
			z^2 = -\frac{2q}{a_{33}} \left(y + \frac{r}{2q} \right) \Rightarrow
			z^2 = 2py
			\end{equation*}
		\end{itemize}
	\end{enumerate}
\end{enumerate}

\subsection{Приведение уравнений к каноническому виду через инварианты}
Пусть дано уравнение $a_{11} x^2 + a_{22} y^2 + a_{33} z^2 + 2 a_{12} xy + 2 a_{13} xz + 2 a_{23} yz + 2 a_{14} x + 2 a_{24} y + 2 a_{34} z + a_{44} = 0$.

\textbf{Полуинвариантами} называются следующие определители:
\begin{enumerate}
	\item $K_2 =
	\begin{vmatrix}
	a_{11} & a_{14} \\
	a_{14} & a_{44}
	\end{vmatrix} +
	\begin{vmatrix}
	a_{22} & a_{24} \\
	a_{24} & a_{44}
	\end{vmatrix} +
	\begin{vmatrix}
	a_{33} & a_{34} \\
	a_{34} & a_{44}
	\end{vmatrix}$
	
	\item $K_3 =
	\begin{vmatrix}
	a_{11} & a_{12} & a_{14} \\
	a_{12} & a_{22} & a_{24} \\
	a_{14} & a_{24} & a_{44}
	\end{vmatrix} +
	\begin{vmatrix}
	a_{11} & a_{13} & a_{14} \\
	a_{13} & a_{33} & a_{34} \\
	a_{14} & a_{34} & a_{44}
	\end{vmatrix} +
	\begin{vmatrix}
	a_{22} & a_{23} & a_{24} \\
	a_{23} & a_{33} & a_{34} \\
	a_{24} & a_{34} & a_{44}
	\end{vmatrix}$
\end{enumerate}

\begin{theorem}
Для канонической формы поверхности второго порядка $a_{11}, a_{22}, a_{33}$~--- корни характеристического многочлена $I_3$.
\end{theorem}

Пусть $\lambda_1, \lambda_2, \lambda_3$~--- корни характеристического многочлена~$I_3$.
Заметим, что $I_3 = \lambda_1 \lambda_2 \lambda_3$.
Приведём уравнение к каноническому виду.
\begin{enumerate}
	\item Пусть $I_3 \neq 0$, тогда, используя параллельный перенос, получим
	\begin{equation*}
	a_{11} x^2 + a_{22} y^2 + a_{33} z^2 + a_{44} = 0
	\end{equation*}
	
	Найдём $a_{44}$:
	\begin{equation*}
	I_4 =
	\begin{vmatrix}
	a_{11} & 0 & 0 & 0 \\
	0 & a_{22} & 0 & 0 \\
	0 & 0 & a_{33} & 0 \\
	0 & 0 & 0 & a_{44}
	\end{vmatrix} = a_{11} a_{22} a_{33} a_{44} \Rightarrow
	a_{44} = \frac{I_4}{I_3}
	\end{equation*}
	
	Получили
	\begin{equation*}
	\lambda_1 x^2 + \lambda_2 y^2 + \lambda_3 z^2 + \frac{I_4}{I_3} = 0
	\end{equation*}
	
	\item Пусть $I_3 = 0$, $I_4 \neq 0$, тогда $(a_{11} \neq 0 \lOr a_{22} \neq 0) \lAnd a_{33} = 0$.
	Считая, что $a_{11} \neq 0$, преобразуем уравнение так, что $a_{14} = 0$, и получим
	\begin{equation*}
	a_{11} x^2 + a_{22} y^2 + 2 a_{24} y + 2 a_{34} z + a_{44} = 0
	\end{equation*}
	
	Найдём неизвестные коэффициенты.
	\begin{equation*}
	I_4 =
	\begin{vmatrix}
	a_{11} & 0 & 0 & 0 \\
	0 & a_{22} & 0 & a_{24} \\
	0 & 0 & 0 & a_{34} \\
	0 & a_{24} & a_{34} & a_{44}
	\end{vmatrix} = -a_{11} a_{22} a_{34}^2
	\end{equation*}
	
	$I_4 \neq 0 \Rightarrow a_{22}, a_{34} \neq 0$, тогда уравнение можно преобразовать так, что $a_{24} = a_{44} = 0$.
	Найдём $a_{34}$.
	\begin{equation*}
	I_2 =
	\begin{vmatrix}
	a_{11} & 0 \\
	0 & a_{22}
	\end{vmatrix} +
	\begin{vmatrix}
	a_{11} & 0 \\
	0 & 0
	\end{vmatrix} +
	\begin{vmatrix}
	a_{22} & 0 \\
	0 & 0
	\end{vmatrix} =
	a_{11} a_{22}
	\end{equation*}
	\begin{equation*}
	a_{34} = \pm\sqrt{-\frac{-a_{11} a_{22} a_{34}^2}{a_{11} a_{22}}} =
	\pm\sqrt{-\frac{I_4}{I_2}}
	\end{equation*}
	
	Получили
	\begin{equation*}
	\lambda_1 x^2 + \lambda_2 y^2 \pm 2\sqrt{-\frac{I_4}{I_2}}z = 0
	\end{equation*}
	
	\item Пусть $I_3 = I_4 = 0$, $I_2 \neq 0$, тогда $a_{11} a_{22} \neq 0 \lAnd -a_{11} a_{22} a_{34}^2 = 0 \Rightarrow a_{34} = 0$.
	Кроме того, уравнение можно преобразовать так, что $a_{14} = a_{24} = 0$.
	Получим
	\begin{equation*}
	a_{11} x^2 + a_{22} y^2 + a_{44} = 0
	\end{equation*}
	
	Найдём $a_{44}$:
	\begin{equation*}
	K_3 =
	\begin{vmatrix}
	a_{11} & 0 & 0 \\
	0 & a_{22} & 0 \\
	0 & 0 & a_{44}
	\end{vmatrix} +
	\begin{vmatrix}
	a_{11} & 0 & 0 \\
	0 & 0 & 0 \\
	0 & 0 & a_{44}
	\end{vmatrix} +
	\begin{vmatrix}
	a_{22} & 0 & 0 \\
	0 & 0 & 0 \\
	0 & 0 & a_{44}
	\end{vmatrix} = a_{11} a_{22} a_{44}
	\end{equation*}
	\begin{equation*}
	a_{44} = \frac{a_{11} a_{22} a_{44}}{a_{11} a_{22}} =
	\frac{K_3}{I_2}
	\end{equation*}
	
	Получили
	\begin{equation*}
	\lambda_1 x^2 + \lambda_2 y^2 + \frac{K_3}{I_2} = 0
	\end{equation*}
	
	\item Пусть $I_2 = I_3 = I_4 = 0$, $K_3 \neq 0$, тогда $a_{11} \neq 0$, $a_{22} = a_{33} = 0$.
	Преобразуем уравнение так, что $a_{14} = a_{44} = 0$, и получим
	\begin{equation*}
	a_{11} x^2 + 2 a_{24} y + 2 a_{34} z = 0
	\end{equation*}
	
	Осуществим такой поворот, что $a_{34} = 0$, и найдём $a_{24}$:
	\begin{equation*}
	K_3 =
	\begin{vmatrix}
	a_{11} & 0 & 0 \\
	0 & 0 & a_{24} \\
	0 & a_{24} & 0
	\end{vmatrix} +
	\begin{vmatrix}
	a_{11} & 0 & 0 \\
	0 & 0 & 0 \\
	0 & 0 & 0
	\end{vmatrix} +
	\begin{vmatrix}
	0 & 0 & a_{24} \\
	0 & 0 & 0 \\
	a_{24} & 0 & 0
	\end{vmatrix} = -a_{11} a_{24}^2
	\end{equation*}
	\begin{equation*}
	a_{24} = \pm\sqrt{-\frac{-a_{11} a_{24}^2}{a_{11}}} = \pm\sqrt{-\frac{K_3}{I_1}}
	\end{equation*}
	
	Получили
	\begin{equation*}
	I_1 x^2 \pm 2\sqrt{-\frac{K_3}{I_1}} y = 0
	\end{equation*}
	
	\item Пусть $I_2 = I_3 = I_4 = K_3 = 0$, тогда $a_{24} = 0$.
	Получим
	\begin{equation*}
	a_{11} x^2 + a_{44} = 0
	\end{equation*}
	
	Найдём $a_{44}$:
	\begin{equation*}
	K_2 =
	\begin{vmatrix}
	a_{11} & 0 \\
	0 & a_{44}
	\end{vmatrix} +
	\begin{vmatrix}
	0 & 0 \\
	0 & a_{44}
	\end{vmatrix} +
	\begin{vmatrix}
	0 & 0 \\
	0 & a_{44}
	\end{vmatrix} = a_{11} a_{44}
	\end{equation*}
	\begin{equation*}
	a_{44} = \frac{a_{11} a_{44}}{a_{44}} = \frac{K_2}{I_1}
	\end{equation*}
	
	Получили
	\begin{equation*}
	I_1 x^2 + \frac{K_2}{I_1} = 0
	\end{equation*}
\end{enumerate}