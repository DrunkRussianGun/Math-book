\subsection{Физический смысл производной}
Пусть зависимость пути, пройденного некоторой точкой, от времени выражается функцией~$S(t)$.
Чтобы найти среднюю скорость движения в~промежутке времени $[t_0; t_0 + \Delta t]$, достаточно вычислить $\dfrac{S(t_0 + \Delta t) - S(t_0)}{\Delta t}$.
Перейдём к пределу при~$\Delta t \to 0$, тогда $[t_0; t_0 + \Delta t]$ выродится в~точку, а средняя скорость движения превратится в~мгновенную скорость в точке~$t_0$.
Т.\,о., производная функции~$S(t)$ представляет зависимость мгновенной скорости от времени.