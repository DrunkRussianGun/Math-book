\subsection{Теоремы о дифференцируемых функциях}
\index{Теорема!Ролля}
\begin{theorem}[Ролля]
Если функция~$f(x)$ непрерывна на~$[a; b]$, дифференцируема на~$(a; b)$, причём $f(a) = f(b)$, то $\exists c \in (a; b) \colon f'(c) = 0$.
\end{theorem}
%\begin{floatingfigure}[r]{58mm}
\begin{center}
\noindent
$\begin{xy} /r8mm/:
(-0.5, 0); (4, 0) **@{-} *@{>} *++!U{x};
(0, -0.5); (0, 2) **@{-} *@{>} *++!R{y};
(0.5, 0.5) = "a"; (3, 0.5) = "b" **\crv{(1, 2)};
"a"; (0.5, 0) **@{--} *++!U{a};
"b"; (3, 0) **@{--} *++!U{b};
"b"; (0, 0.5) **@{--} *++!R{f(a) = f(b)};
(0.5, 1.2515); (3, 1.2515) **@{-};
(1.377, 1.2515); (1.377, 0) **@{--} *++!U{c};
\end{xy}$
\end{center}
\begin{proof}
Если $f(x) = f(a)$, то в~качестве точки~$c$ можно взять любую точку из~$(a; b)$.

Пусть $f(x)$ не является константой на~$[a; b]$, тогда по свойству~\ref{st:continuous_function_takes_inf_and_sup} непрерывной функции $\displaystyle \exists c \in (a; b) \colon \allowbreak f(c) = \inf_{x \in [a; b]} f(x)$ или $\displaystyle f(c) = \sup_{x \in [a; b]} f(x)$.
Для определённости предположим, что
\begin{equation*}
f(c) = \inf_{x \in [a; b]} f(x) \Leftrightarrow
\forall x \in [a; b] \ f(x) - f(c) > 0 \Rightarrow
\begin{cases}
\dfrac{f(x) - f(c)}{x - c} < 0, \ x < c \\
\dfrac{f(x) - f(c)}{x - c} > 0, \ x > c
\end{cases}
\end{equation*}

$f(x)$ дифференцируема в точке~$c$, тогда $\displaystyle \exists f'(c) = \lim_{x \to c} \frac{f(x) - f(c)}{x - c}$.
\begin{equation*}
\lim_{x \to c-0} \frac{f(x) - f(c)}{x - c} < 0, \
\lim_{x \to c+0} \frac{f(x) - f(c)}{x - c} > 0 \Rightarrow
f'(c) = 0
\end{equation*}

Доказательство в~случае $\displaystyle f(c) = \sup_{x \in [a; b]} f(x)$ аналогично.
\end{proof}

\index{Теорема!Коши о среднем значении}
\begin{theorem}[Коши о среднем значении]
\label{th:Cauchy's_mean_value}
Если функции $f(x)$ и $g(x)$ непрерывны на~$[a; b]$, дифференцируемы на~$(a; b)$, $g(a) \neq g(b)$, то
\begin{equation*}
\exists c \in (a; b) \colon \frac{f(b) - f(a)}{g(b) - g(a)} = \frac{f'(c)}{g'(c)}
\end{equation*}
\end{theorem}
\begin{proof}
Пусть
\begin{equation*}
F(x) = f(x) - \frac{f(b) - f(a)}{g(b) - g(a)}(g(x) - g(a))
\end{equation*}

$F(x)$ дифференцируема на~$(a; b)$, $F(a) = F(b) = f(a)$, тогда по теореме Ролля
\begin{equation*}
\exists c \in (a; b) \colon 0 = F'(c) = f'(c) - \frac{f(b) - f(a)}{g(b) - g(a)} g'(c) \Rightarrow
\frac{f(b) - f(a)}{g(b) - g(a)} = \frac{f'(c)}{g'(c)}
\end{equation*}
\end{proof}

Полагая $g(x) = x$, получим \textbf{формулу конечных приращений}:
\index{Теорема!Лагранжа о среднем значении}\index{Формула!конечных приращений}
\begin{theorem}[Лагранжа о среднем значении]
\label{th:mean_value}
Если функция~$f(x)$ непрерывна на~$[a; b]$, дифференцируема на~$(a; b)$, то
\begin{equation*}
\exists c \in (a; b) \colon f(b) - f(a) = f'(c)(b - a)
\end{equation*}
\end{theorem}

\begin{center}
\noindent
$\begin{xy} /r8mm/:
(-0.5, 0); (5.5, 0) **@{-} *@{>} *++!U{x};
(0, -0.5); (0, 3) **@{-} *@{>} *++!R{y};
(0.5, -0.5); (5, 2.75) **\crv{(1, 6) & (3.5, -3)} *+!L{y = f(x)};
(0.6125, 0.5) = "a"; (4.93, 2.5) = "b" **@{-};
"a"; (0.6125, 0) **@{--} *++!U{a};
"b"; (4.93, 0) **@{--} *++!U{b};
(-0.5, 1.1522); (3, 2.7735) **@{-}; % касательная
(1.21, 1.9443); (1.21, 0) **@{--} *++!U{c};
\end{xy}$
\end{center}