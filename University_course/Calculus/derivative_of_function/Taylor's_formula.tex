\subsection{Формула Тейлора}
\index{Формула!Тейлора}
\begin{theorem}[формула Тейлора]
\label{eq:Taylor_series}
Если функция~$f(x)$ в~некоторой окрестности~$U(a)$ имеет все производные порядка $n + 1$ и ниже, то
\begin{equation*}
\forall x \in U(a) \
f(x) = \sum_{k=0}^n \frac{f^{(k)} (x - a)^k}{k!} + R(x), \
R(x) = \frac{f^{(n + 1)}(a + \Theta(x - a))}{(n + 1)!}(x - a)^{n + 1}, \
\Theta \in (0; 1)
\end{equation*}
\end{theorem}

$R(x)$ называется \textbf{остаточным членом} в~форме Лагранжа и используется для оценки ошибки. Также его можно представить в~форме Пеано~--- $R(x) = o((x - a)^n)$~--- которая используется при вычислении пределов.

\index{Формула!Маклорена}
Подставив $a = 0$ в~формулу Тейлора, получим \textbf{формулу Маклорена}:
\begin{equation}
\label{eq:Maclaurin_series}
f(x) = \sum_{k=0}^n \frac{f^{(k)} x^k}{k!} + R(x), \
R(x) = \frac{f^{(n + 1)}(\Theta x)}{(n + 1)!} x^{n + 1}, \
\Theta \in (0; 1)
\end{equation}

\subsubsection{Разложения некоторых функций в ряд Маклорена}
\begin{itemize}
	\item $f(x) = e^x, \
	f^{(n)}(x) = e^x, \
	f^{(n)}(0) = 1$
	\begin{equation*}
	\forall n \in \mathbb N \ f(x) = 1 + x + \frac{x^2}{2!} + \ldots + \frac{x^n}{n!} + R(x), \
	R(x) = \frac{e^{\Theta x}}{(n + 1)!} x^{n + 1}
	\end{equation*}
	\begin{equation*}
	|R(x)| \leqslant e^{\max \{ 0, x \}} \cdot \frac{|x|^{n + 1}}{(n + 1)!} \Rightarrow
	\begin{cases}
	\displaystyle |R(x)| \leqslant \frac{|x|^{n + 1}}{(n + 1)!}, x < 0 \\
	\displaystyle |R(x)| \leqslant 3^x \cdot \frac{|x|^{n + 1}}{(n + 1)!}, x > 0
	\end{cases}
	\end{equation*}
	
	\item $f(x) = \sin x, \
	f^{(n)}(x) = \sin \left( x + \dfrac\pi{2} n \right), \
	f^{(n)}(0) = \sin \dfrac\pi{2} n =
	\begin{cases}
	0, \ n \mult 2 \\
	1, \ \exists k \in \mathbb Z \colon n = 4k + 1 \\
	-1, \ \exists k \in \mathbb Z \colon n = 4k + 3
	\end{cases}$
	\begin{equation*}
	\forall n \in \mathbb N \ f(x) = x - \frac{x^3}{3!} + \frac{x^5}{5!} - \ldots + \frac{(-1)^{n-1} \cdot x^{2n-1}}{(2n - 1)!} + R(x), \
	R(x) = \frac{\sin \left( \Theta x + \frac\pi{2} (2n + 1) \right)}{(2n + 1)!} x^{2n+1}
	\end{equation*}
	\begin{equation*}
	|R(x)| \leqslant \frac{|x|^{2n+1}}{(2n + 1)!}
	\end{equation*}
	
	\item $f(x) = \cos x,
	f^{(n)}(x) = \cos \left( x + \dfrac\pi{2} n \right), \
	f^{(n)}(0) = \cos \dfrac\pi{2} n =
	\begin{cases}
	0, \ n \notmult 2 \\
	1, \ \exists k \in \mathbb Z \colon n = 4k \\
	-1, \ \exists k \in \mathbb Z \colon n = 4k + 2
	\end{cases}$
	\begin{equation*}
	\forall n \in \mathbb N \ f(x) = 1 - \frac{x^2}{2!} + \frac{x^4}{4!} - \ldots + \frac{(-1)^{n-1} x^{2n-2}}{(2n - 2)!} + R(x), \
	R(x) = \frac{\cos (\Theta x + \pi n)}{(2n)!} x^{2n}
	\end{equation*}
	\begin{equation*}
	|R(x)| \leqslant \frac{x^{2n}}{(2n)!}
	\end{equation*}
	
	\item $f(x) = \ln (1 + x), \
	f^{(n)}(x) = \dfrac{(-1)^{n-1} (n - 1)!}{(1 + x)^n}, \
	f^{(n)}(0) = (-1)^{n-1} (n - 1)!$
	\begin{equation*}
	\forall n \in \mathbb N \ f(x) = x - \frac{x^2}2 + \frac{x^3}3 - \ldots + \frac{(-1)^{n-1} x^n}n + R(x), x \in (-1; 1], \
	R(x) = \frac{(-1)^n x^{n+1}}{(n + 1)(1 + \Theta x)^{n+1}}
	\end{equation*}
	
	Для вычисления $\ln a$, $a \neq -1$, можно воспользоваться формулой
	\begin{equation*}
	\forall n \in \mathbb N \ \ln \frac{1 + x_0}{1 - x_0} = \ln (1 + x_0) - \ln (1 - x_0) =
	2 \left( x_0 + \frac{x_0^3}3 + \frac{x_0^5}5 + \ldots \right)
	\end{equation*}
	\begin{equation*}
	a = \frac{1 + x_0}{1 - x_0} \Leftrightarrow
	a - a x_0 = 1 + x_0 \Leftrightarrow
	x_0 = \frac{a - 1}{a + 1}
	\end{equation*}
	
	\item $f(x) = (1 + x)^\alpha, \
	f^{(n)}(x) = \alpha (\alpha - 1) (\alpha - 2) \cdot \ldots \cdot (\alpha - n + 1) (1 + x)^{\alpha - n}, \
	f^{(n)}(0) = \alpha (\alpha - 1) (\alpha - 2) \cdot \ldots \cdot (\alpha - n + 1)$
	\begin{equation*}
	\forall n \in \mathbb N \ f(x) = 1 + \alpha x + \frac{\alpha (\alpha - 1) x^2}{2!} + \ldots + \frac{\alpha (\alpha - 1) \cdot \ldots \cdot (\alpha - n + 1) x^n}{n!} + R(x), |x| < 1,
	\end{equation*}
	\begin{equation*}
	R(x) = \frac{\alpha (\alpha - 1) \cdot \ldots \cdot (\alpha - n) (1 + \Theta x)^{\alpha-n-1}}{(n + 1)!} x^{n+1}
	\end{equation*}
\end{itemize}