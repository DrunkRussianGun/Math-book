\section{Бесконечно малые и бесконечно большие функции}
\index{Функция!бесконечно малая} Функция~$\alpha(x)$ называется \textbf{бесконечно малой} при~$x \to x_0$, если $\displaystyle \lim_{x \to x_0} \alpha(x) = 0$.

\index{Функция!бесконечно большая} Функция~$A(x)$ называется \textbf{бесконечно большой} при~$x \to x_0$, если $\displaystyle \lim_{x \to x_0} A(x) = \infty$.

Очевидны следующие утверждения.
\begin{statement}
Если $\alpha(x)$~--- бесконечно малая функция, то $\dfrac1{\alpha(x)}$~--- бесконечно большая функция.
\end{statement}

\begin{statement}
Если $A(x)$~--- бесконечно большая функция, то $\dfrac1{A(x)}$~--- бесконечно малая функция.
\end{statement}

Функции $\alpha(x)$ и $\beta(x)$ называются \textbf{бесконечно малыми одного порядка малости} при~$x \to x_0$, если
$\displaystyle 0 < \left| \lim_{x \to x_0} \frac{\alpha(x)}{\beta(x)} \right| < \infty$.

Функции $\alpha(x)$ и $\beta(x)$ называются \textbf{эквивалетнтными бесконечно малыми} при~$x \to x_0$, если
$\displaystyle \lim_{x \to x_0} \frac{\alpha(x)}{\beta(x)} = 1$.
При этом пишут $f(x) \sim g(x)$.

Функция $\alpha(x)$ называется \textbf{бесконечно малой более высокого порядка малости}, чем $\beta(x)$, при~$x \to x_0$, если
$\displaystyle \lim_{x \to x_0} \frac{\alpha(x)}{\beta(x)} = 0$, и обозначается $\alpha(x) = o(\beta(x))$.
Следует помнить, что это не равенство в~обычном смысле, т.\,е. запись $o(\beta(x)) = \alpha(x)$ бессмысленна.