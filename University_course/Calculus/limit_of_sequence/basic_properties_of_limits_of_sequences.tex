\subsection{Основные свойства пределов последовательностей}
\begin{enumerate}
	\item Из ограниченной последовательности можно выбрать сходящуюся подпоследовательность.
	\begin{proof}
	Пусть $A$~--- множество значений, принимаемых членами ограниченной последовательности~$(x_n)$.
	\begin{enumerate}
		\item Пусть $A$ конечно.
		Тогда бесконечное множество членов последовательности~$(x_n)$ принимает хотя~бы одно значение из~$A$, значит, подпоследовательность, состоящая из~них, сходится к этому значению.
		
		\item Пусть $A$ бесконечно, тогда оно ограничено, значит, по \hyperref[th:Weierstrass]{теореме Вейерштрасса} оно имеет предельную точку~$a$.
		В окрестности~$\breve U_1(a)$ содержится хотя~бы одна точка из множества~$A$, а соответствующее значение принимает член~$x_{n_1}$.
		
		Рассмотрим множество~$A_1$, полученное из~$A$ удалением значений, принимаемых членами $x_1, x_2, \ldots, x_{n_1}$.
		$A_1$ бесконечно и имеет предельную точку~$a$, поэтому в окрестности~$\breve U_\frac12(a)$ найдётся значение, принимаемое членом~$x_{n_2}$, причём $n_1 < n_2$.
		
		Рассмотрим множество~$A_2$, полученное из~$A_1$ удалением значений, принимаемых членами $x_{n_1 + 1}, x_{n_1 + 2}, \allowbreak \ldots, x_{n_2}$.
		$A_2$ бесконечно и имеет предельную точку~$a$, поэтому в окрестности~$\breve U_\frac13(a)$ найдётся значение, принимаемое членом~$x_{n_3}$, причём $n_2 < n_3$.
		
		Продолжая, получим последовательность~$(x_{n_k}) \colon |x_{n_k} - a| < \dfrac1k$. По следствию~\ref{conseq:small_rational_exists} $\displaystyle \lim_{k \to \infty} x_{n_k} \opbr= a$.
	\end{enumerate}
	\end{proof}
	
	\item Монотонная ограниченная последовательность~$(x_n)$ сходится.
	\begin{proof}
	Для опредёленности предположим, что $\forall n \in \mathbb N \ x_n \leqslant x_{n+1}$.
	Последовательность ограничена, поэтому множество~$A$ её значений имеет супремум~$a = \sup A$.
	По утверждению~\ref{st:inequality_of_supremum}
	\begin{equation*}
	\forall \varepsilon > 0 \ \exists k \in \mathbb N \colon a - \varepsilon < x_k \leqslant a \Rightarrow
	\forall n > k \ a - \varepsilon < x_k \leqslant x_n \leqslant a \Rightarrow
	|x_n - a| < \varepsilon \Rightarrow \lim_{n \to \infty} x_n = a
	\end{equation*}
	\end{proof}
	
	\item\index{Лемма!о~вложенных отрезках} \begin{lemma}[о~вложенных отрезках]
	\label{lemma:about_nested_intervals}
	Пусть $(a_n), (b_n)$~--- последовательности концов последовательно вложенных друг в~друга отрезков (т.\,е. $[a_n; b_n] \subset [a_{n-1}; b_{n-1}]$), причём $\displaystyle \lim_{n \to \infty} (b_n - a_n) = 0$.
	Тогда $\displaystyle \bigcap_{k=1}^\infty [a_k; b_k] = \{ a \}$.
	\end{lemma}
	\begin{proof}
	Очевидно, что $(a_n)$ монотонна и ограничена сверху, $(b_n)$ монотонна и ограничена снизу, тогда $\displaystyle \lim_{n \to \infty} a_n = a$, $\displaystyle \lim_{n \to \infty} b_n = b$.
	Имеем:
	\begin{equation*}
	b = \lim_{n \to \infty} b_n = \lim_{n \to \infty} (b_n - a_n + a_n) = \lim_{n \to \infty} (b_n - a_n) + \lim_{n \to \infty} a_n = a
	\end{equation*}
	
	Отрезки последовательно вложены друг в~друга, поэтому $\displaystyle \bigcap_{k=1}^n [a_k; b_k] = [a_n; b_n]$.
	\begin{equation*}
	\bigcap_{k=1}^\infty [a_k; b_k] = \lim_{n \to \infty} \bigcap_{k=1}^n [a_k; b_k] = \lim_{n \to \infty} [a_n; b_n] = \{ a \}
	\end{equation*}
	\end{proof}
\end{enumerate}