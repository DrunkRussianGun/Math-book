\section{Функции нескольких переменных}
\index{Функция!нескольких переменных} \textbf{Функцией от $n$~переменных} называется функция~$f \colon D \to E$, где $D \subseteq \mathbb R^n$, $E \subseteq \mathbb R$, и обозначается $f(\overline x)$, или $f(x_1, x_2, \ldots, x_n)$.

\index{r@$\rho$} \index{Расстояние между точками} \textbf{Расстоянием между точками} $\overline a = (a_1, \ldots, a_n)$ и $\overline b = (b_1, \ldots, b_n)$ называется величина
\begin{equation*}
\rho(\overline a, \overline b) = \sqrt{\sum_{i=0}^n (a_i - b_i)^2}
\end{equation*}

\subsection{Предел функции нескольких переменных}
Число~$a$ называется \textbf{пределом функции~$f(\overline x)$ в точке~$\overline x_0$}, если
\begin{equation*}
\forall \varepsilon > 0 \ \exists \delta > 0 \colon \forall \overline x \ (\rho(\overline x, \overline x_0) < \delta \Rightarrow |f(\overline x) - a| < \varepsilon)
\end{equation*}

Для вычисления предела функции двух переменных удобно перейти в~полярные координаты, трёх переменных~--- в~сферические.

\subsection{Непрерывность функции нескольких переменных}
\index{Функция!нескольких переменных!непрерывная} Функция~$f(\overline x)$ называется \textbf{непрерывной в точке~$\overline x_0$}, если
$\lim\limits_{\rho(\overline x, \overline x_0) \to 0} f(\overline x) = f(\overline x_0)$.

Функция~$f(\overline x)$ называется \textbf{непрерывной на множестве}, если она непрерывна в~каждой точке этого множества.

\subsection{Дифференцируемость функции нескольких переменных}
\index{Функция!нескольких переменных!дифференцируемая} Функция~$f(x_1, \ldots, x_n)$ называется \textbf{дифференцируемой в точке~$\overline x_0 = (x_{10}, \ldots, x_{n0})$}, если
\begin{equation*}
\exists (a_1, \ldots, a_n) \colon \forall \Delta \overline x = (\Delta x_1, \ldots, \Delta x_n) \
f(\overline x_0 + \Delta \overline x) - f(\overline x_0) =
a_1 \Delta x_1 + \ldots + a_n \Delta x_n + o(\rho(\overline x_0 + \Delta \overline x, \overline x_0))
\end{equation*}

\index{Матрица!производная} Матрица-строка~$\begin{Vmatrix}
a_1 & a_2 & \ldots & a_n
\end{Vmatrix}$ называется \textbf{производной матрицей}.

Пусть функция~$f(x_1, \ldots, x_n)$ дифференцируема в точке~$\overline x_0 = (x_{10}, \ldots, x_{n0})$, $\Delta \overline x = (0, \ldots, 0, \Delta x_k, 0, \ldots, 0)$, тогда
\begin{equation*}
\rho(\overline x_0 + \Delta \overline x, \overline x_0) = |\Delta x_k|, \
f(\overline x_0 + \Delta \overline x) - f(\overline x_0) =
a_k \Delta x_k + o(|\Delta x_k|)
\end{equation*}

Это означает, что функция~$g(x) = f(x_{10}, \ldots, x_{k-1\, 0}, x_k, x_{k+1\, 0}, \ldots, x_{n0})$ дифференцируема в точке~$x_{k0}$ и $a_k \opbr= g'(x_{k0})$.
\index{Производная!частная} $g(x)$ называется \textbf{частной производной функции~$f(\overline x)$} и обозначается $\dfrac{\partial f}{\partial x_k}$, или $f_{x_k}'(\overline x)$.
Т.\,о., производная матрица функции~$f(\overline x)$ имеет вид $(f_{x_1}', f_{x_2}', \ldots, f_{x_n}')$.

Следует обратить внимание, что обозначение~$\dfrac{\partial f}{\partial x_k}$ следует понимать как цельный символ, а не как отношение некоторых величин.
Например, $\dfrac{\partial f}{\partial x} \cdot \dfrac{\partial x}{\partial t} \neq \dfrac{\partial f}{\partial t}$.

Существование частных производных функции в~некоторой точке не является достаточным условием дифференцируемости этой функции в~данной точке.
Например, функция~$f(x, y) =
\begin{cases}
0, \ xy = 0 \\
1, \ xy \neq 0
\end{cases}$
имеет частные производные в точке~$(0, 0)$: $f_x'(0, 0) = f_y'(0, 0) = 0$,~--- однако не является дифференцируемой в~этой точке, т.\,к., очевидно, терпит в~ней разрыв.

Частная производная~$f_1(x_1, \ldots, x_n) = f_{x_k}'$ функции~$f(x_1, \ldots, x_n)$ также является функцией.
Частная производная~$f_{1\, x_l}'$ называется \textbf{частной производной функции~$f(\overline x)$ второго порядка} и обозначается $f_{x_k x_l}'' = \dfrac{\partial^2 f}{\partial x_k \partial x_l}$.
При этом, если $k \neq l$, то такая производная называется \textbf{смешанной}.
Частные производные большего порядка определяются по индукции.

\index{Теорема!Шварца}
\begin{theorem}[Шварца]
Пусть дана функция~$f(x_1, \ldots, x_n)$.
Если $f_{x_i x_j}''$ и~$f_{x_j x_i}''$ непрерывны, то $f_{x_i x_j}'' = f_{x_j x_i}''$.
\end{theorem}

\textbf{Дифференциалом функции~$f(\overline x)$ в точке~$\overline x_0$} называется величина $\sum\limits_{k=1}^n f_{x_k}'(\overline x_0) dx_k$, являющаяся линейной частью приращения функции и обозначаемая $df(\overline x_0)$.
По аналогии с частными производными произвольного порядка вводятся дифференциалы произвольного порядка.

\hyperref[eq:Taylor_series]{Формулу Тейлора} можно обобщить на случай нескольких переменных:
\begin{equation}
\label{eq:Taylor_series_for_several_variables}
f(\overline x) - f(\overline x_0) =
df(\overline x_0) + \frac{d^2 f(\overline x_0)}{2!} + \frac{d^3 f(\overline x_0)}{3!} + \ldots + \frac{d^n f(\overline x_0)}{n!} + o(\rho(\overline x, \overline x_0))
\end{equation}

\index{Производная!по направлению} Пусть функция~$f(x_1, \ldots, x_n)$ дифференцируема в точке~$\overline x_0$, $\overline e = (e_1, \ldots, e_n)$~--- единичный вектор.
\textbf{Производной функции~$f(\overline x)$ по направлению~$\overline e$} называется величина
\begin{equation*}
\frac{\partial f}{\partial \overline e} = \sum_{i=1}^n \frac{\partial f}{\partial x_i} \cdot e_i
\end{equation*}

\index{grad} \index{Градиент} \textbf{Градиентом функции $f(x_1, \ldots, x_n)$}, дифференцируемой в точке~$\overline x_0$, называется вектор $\grad f = (f_{x_1}'(\overline x_0), \ldots, \allowbreak f_{x_n}'(\overline x_0))$.

\subsection{Экстремумы функции нескольких переменных}
\index{Минимум!функции} Точка~$\overline x_0 = (x_{10}, \ldots, x_{n0})$ называется \textbf{точкой локального минимума функции~$f(\overline x) = f(x_1, \ldots, x_n)$}, если существует проколотая окрестность
$\breve U(\overline x_0) \colon \forall \overline x \in \breve U(\overline x_0) \
f(\overline x) > f(\overline x_0)$.
	
\index{Максимум!функции} Точка~$\overline x_0 = (x_{10}, \ldots, x_{n0})$ называется \textbf{точкой локального максимума функции~$f(\overline x) = f(x_1, \ldots, x_n)$}, если существует проколотая окрестность
$\breve U(\overline x_0) \colon \forall \overline x \in \breve U(\overline x_0) \
f(\overline x) < f(\overline x_0)$.

\index{Экстремум} Точки локального минимума и максимума называются \textbf{точками локального экстремума}.
	
\begin{theorem}
В~точке локального экстремума частные производные функции равны нулю или не существуют.
\end{theorem}
\begin{proof}
Пусть $\overline x_0 = (x_{10}, \ldots, x_{n0})$~--- точка локального экстремума функции~$f(\overline x)$, дифференцируемой в точке~$\overline x_0$.
Рассмотрим $g(x) = f(x_{10}, \ldots, x_{k-1\,0}, x, x_{k+1\,0}, \ldots, x_{n0})$.
$\overline x_0$~--- точка экстремума~$f(\overline x)$, тогда $x_{k0}$~--- точка экстремума~$g(x)$, значит, $g'(x_{k0}) = 0$ или не существует.
Тогда $f_{x_k}'(\overline x_0) = 0$ или не существует.
\end{proof}

\begin{theorem}
Пусть дана функция~$f(x, y)$. Если
\begin{itemize}
	\item $\displaystyle f_x'(x_0, y_0) = f_y'(x_0, y_0) = 0$
	\item $\displaystyle (f_{xy}''(x_0, y_0))^2 - f_{xx}''(x_0, y_0) f_{yy}''(x_0, y_0) < 0$
\end{itemize}
то $(x_0, y_0)$~--- точка локального экстремума $f(x, y)$.
	
\begin{enumerate}
	\item $(x_0, y_0)$~--- точка локального минимума,
	если $f_{xx}''(x_0, y_0) \opbr> 0 \lOr f_{yy}''(x_0, y_0) \opbr> 0$.
	\item $(x_0, y_0)$~--- точка локального максимума,
	если $f_{xx}''(x_0, y_0) \opbr< 0 \lOr f_{yy}''(x_0, y_0) \opbr< 0$.
\end{enumerate}
\end{theorem}
\begin{proof}
По \hyperref[eq:Taylor_series_for_several_variables]{формуле Тейлора}
\begin{equation*}
f(x, y) - f(x_0, y_0) = f(x_0, y_0) + df(x_0, y_0) + \frac12 d^2f(x_0, y_0) + o(\rho^2((x, y), (x_0, y_0))) - f(x_0, y_0) =
\end{equation*}
\begin{equation*}
= \frac12 d^2f(x_0, y_0) + o(\rho^2((x, y), (x_0, y_0)))
\end{equation*}
значит, $f(x, y) - f(x_0, y_0)$ сохраняет знак, если $d^2f(x_0, y_0)$ сохраняет знак.

\begin{equation*}
d^2f(x_0, y_0) = f_{xx}''(x_0, y_0)\,dx^2 + 2f_{xy}''(x_0, y_0)\,dx\,dy + f_{yy}''(x_0, y_0)\,dy^2 =
\end{equation*}
\begin{equation*}
= \left(f_{xx}''(x_0, y_0) + 2f_{xy}''(x_0, y_0) \frac{dy}{dx} + f_{yy}''(x_0, y_0) \left(\frac{dy}{dx}\right)^2\right) dx^2
\end{equation*}

Т.\,о., $(x_0, y_0)$~--- точка локального экстремума $f(x, y)$,
если $d^2f(x_0, y_0)$ сохраняет знак, т.\,е. при
\begin{equation*}
(f_{xy}''(x_0, y_0))^2 - f_{xx}''(x_0, y_0) f_{yy}''(x_0, y_0) < 0
\end{equation*}

\begin{equation*}
(f_{xy}''(x_0, y_0))^2 - f_{xx}''(x_0, y_0) f_{yy}''(x_0, y_0) < 0 \Leftrightarrow
\end{equation*}
\begin{equation*}
\Leftrightarrow f_{xx}''(x_0, y_0) f_{yy}''(x_0, y_0) > (f_{xy}''(x_0, y_0))^2 \Rightarrow f_{xx}''(x_0, y_0) f_{yy}''(x_0, y_0) > 0
\end{equation*}
значит, $f_{xx}''(x_0, y_0)$ и~$f_{yy}''(x_0, y_0)$ одного знака.

\begin{enumerate}
	\item Если $f_{xx}''(x_0, y_0) > 0 \lOr f_{yy}''(x_0, y_0) > 0$, то $d^2 f(x_0, y_0) \opbr> 0$, тогда $(x_0, y_0)$~--- точка локального минимума.
	\item Если $f_{xx}''(x_0, y_0) < 0 \lOr f_{yy}''(x_0, y_0) < 0$, то $d^2 f(x_0, y_0) \opbr< 0$, тогда $(x_0, y_0)$~--- точка локального максимума.
\end{enumerate}
\end{proof}

\begin{theorem}
\label{th:sufficient_condition_of_local_extremum}
Пусть дана функция~$f(\overline x) = f(x_1, \ldots, x_n)$ и $f_{x_1}'(\overline x_0) = \ldots = f_{x_n}'(\overline x_0) = 0$ в некоторой точке~$\overline x_0 \opbr= (x_{10}, \ldots, x_{n0})$.
\begin{itemize}
	\item $\overline x_0$~--- точка локального минимума, если $\sum\limits_{
	\begin{smallmatrix}
	i = \overline{1, n} \\
	j = \overline{1, n}
	\end{smallmatrix}
	} f_{x_i x_j}''(\overline x_0)\,dx_i\,dx_j > 0$.
	\item $\overline x_0$~--- точка локального максимума, если $\sum\limits_{
	\begin{smallmatrix}
	i = \overline{1, n} \\
	j = \overline{1, n}
	\end{smallmatrix}
	} f_{x_i x_j}''(\overline x_0)\,dx_i\,dx_j < 0$.
\end{itemize}
\end{theorem}
\begin{proof}
По \hyperref[eq:Taylor_series_for_several_variables]{формуле Тейлора}
\begin{equation*}
f(\overline x) - f(\overline x_0) =
f(\overline x_0) + df(\overline x_0) + \frac{d^2 f(\overline x_0)}{2!} + o(\rho^2(\overline x, \overline x_0)) - f(\overline x_0) =
\frac{d^2 f(\overline x_0)}{2!} + o(\rho^2(\overline x, \overline x_0))
\end{equation*}
значит, $f(\overline x) - f(\overline x_0)$ сохраняет знак, если $d^2 f(\overline x_0)$ сохраняет знак.
	
\begin{equation*}
d^2 f(\overline x_0) = \sum_{
\begin{smallmatrix}
i = \overline{1, n} \\
j = \overline{1, n}
\end{smallmatrix}
} f_{x_i x_j}''(\overline x_0)\,dx_i\,dx_j
\end{equation*}
	
\begin{enumerate}
	\item Если $\displaystyle \sum_{
	\begin{smallmatrix}
	i = \overline{1, n} \\
	j = \overline{1, n}
	\end{smallmatrix}
	} f_{x_i x_j}''(\overline x_0)\,dx_i\,dx_j > 0$, то	$d^2 f(\overline x_0) > 0$, тогда $\overline x_0$~--- точка локального минимума.
	\item Если $\displaystyle \sum_{
	\begin{smallmatrix}
	i = \overline{1, n} \\
	j = \overline{1, n}
	\end{smallmatrix}
	} f_{x_i x_j}''(\overline x_0)\,dx_i\,dx_j < 0$, то	$d^2 f(\overline x_0) < 0$, тогда $\overline x_0$~--- точка локального максимума.
\end{enumerate}
\end{proof}
	
При практическом применении теоремы~\ref*{th:sufficient_condition_of_local_extremum} полезен критерий Сильвестра.
	
\subsubsection{Метод наименьших квадратов}
\index{Метод!наименьших квадратов}
Пусть даны точки $x_1, \ldots, x_n$ и требуется найти аппроксимирующую прямую для значений некоторой функции~$f(x)$ в~этих точках.
Уравнение прямой~--- $y = Ax + B$.
Найдём точку, в~которой сумма
\begin{equation*}
S(A, B) = \sum_{i=1}^n (A x_i + B - f(x_i))^2
\end{equation*}
принимает наименьшее значение.
	
\begin{equation*}
S_A' = \sum 2 x_i (A x_i + B - f(x_i))
\end{equation*}
\begin{equation*}
S_B' = \sum 2 (A x_i + B - f(x_i))
\end{equation*}
\begin{equation*}
\begin{cases}
\displaystyle S_A' = 0 \\
\displaystyle S_B' = 0
\end{cases}
\Leftrightarrow
\begin{cases}
\displaystyle A \sum x_i^2 + B \sum x_i = \sum x_i f(x_i) \\
\displaystyle A \sum x_i + Bn = \sum f(x_i)
\end{cases}
\Leftrightarrow
\end{equation*}
\begin{equation*}
\Leftrightarrow
\begin{cases}
\displaystyle \left(n \sum x_i^2 - \left(\sum x_i\right)^2\right)A = n \sum x_i f(x_i) - \sum x_i \sum f(x_i) \\
\displaystyle Bn = \sum f(x_i) - A \sum x_i
\end{cases}
\Leftrightarrow
\end{equation*}
\begin{equation*}
\Leftrightarrow
\begin{cases}
\displaystyle A = \frac{\displaystyle n \sum x_i f(x_i) - \sum x_i \sum f(x_i)}{\displaystyle n \sum x_i^2 - \left(\sum x_i\right)^2} \\
\displaystyle B = \frac{\displaystyle \sum x_i^2 \sum f(x_i) - \sum x_i \sum x_i f(x_i)}{\displaystyle n \sum x_i^2 - \left(\sum x_i\right)^2}
\end{cases}
\end{equation*}
	
Найденные значения~$A$ и $B$~--- искомые коэффициенты в~уравнении аппроксимирующей прямой.
Для оценки точности аппроксимации можно найти коэффициент корреляции по формуле
\begin{equation*}
r = \sqrt{\frac
{\sum (f(x_i) - \tilde y)^2 - \sum (f(x_i) - \tilde{y_i})^2}
{\sum (f(x_i) - \tilde y)^2}} =
\sqrt{1 - \frac{\sum (f(x_i) - \tilde{y_i})^2}{\sum (f(x_i) - \tilde y)^2}}
\end{equation*}
где $\tilde y = \frac1n \sum f(x_i)$, $\tilde{y_i} = A x_i + B$, а значение коэффициента~$r$ тем ближе к единице, чем точнее аппроксимация.
	
\subsubsection{Метод множителей Лагранжа}
Пусть дана функция $f(x_1, \ldots, x_n)$, переменные которой удовлетворяют условиям
\begin{equation*}
\begin{cases}
g_1(x_1, \ldots, x_n) = 0 \\
\ldots \\
g_m(x_1, \ldots, x_n) = 0
\end{cases}
\end{equation*}

Для нахождения её экстремумов (называемых \textbf{условными}) введём функцию Лагранжа
\begin{equation*}
L(x_1, \ldots, x_n, \lambda_1, \ldots, \lambda_m) =
f(x_1, \ldots, x_n) + \lambda_1 g_1(x_1, \ldots, x_n) + \ldots + \lambda_m g_m(x_1, \ldots, x_n)
\end{equation*}
и исследуем её.
Её экстремумы являются условными экстремумами функции~$f$.