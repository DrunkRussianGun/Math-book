\subsection{Интегрирование дробно-рациональных выражений}
Пусть $P_n(x)$ и $Q_m(x)$~--- многочлены $n$-й и $m$-й степеней соответственно, $n < m$.
По теореме~\ref{th:polynomial_factorization} $Q_m(x)$ можно разложить на множители
\begin{equation*}
Q_m(x) = \prod_{i=1}^{p} (x - a_i)^{\alpha_i} \cdot \prod_{i=1}^{q} ((x - b_i)^2 + c_i^2)^{\beta_i}
\end{equation*}

Тогда дробь~$\dfrac{P_n(x)}{Q_m(x)}$ может быть представлена в~виде суммы простейших дробей
\begin{equation*}
\frac{P_n(x)}{Q_m(x)} = \sum_{i=1}^p \sum_{j=1}^{\alpha_i} \frac{A_{ij}}{(x - a_i)^j} + \sum_{i=1}^q \sum_{j=1}^{\beta_q} \frac{B_{ij}x + C_{ij}}{((x - b_i)^2 + c_i^2)^j}
\end{equation*}

Т.\,о., интегрирование дробно-рациональных выражений сводится к интегрированию простейших дробей и, в~случае $n \geqslant m$, многочленов от переменной~$x$.