\subsection{Предел функции на бесконечности}
Пусть $f(x)$~--- функция.
Следующие определения эквивалентны:
\begin{enumerate}
	\item \textbf{Определение по Гейне}
	
	Число~$a$ называется \textbf{пределом функции~$f(x)$ на бесконечности}, если $\displaystyle \lim_{n \to \infty} f(x_n) = a$ для любой последовательности~$\displaystyle (x_n): \lim_{n \to \infty} x_n = \infty$.
	
	\item \textbf{Определение по Коши}
	
	Число~$a$ называется \textbf{пределом функции~$f(x)$ на бесконечности}, если
	\begin{equation*}
	\forall \varepsilon > 0 \ \exists M > 0 \colon \forall x \ |x| > M \Rightarrow |f(x) - b| < \varepsilon
	\end{equation*}
\end{enumerate}

Предел функции~$f(x)$ на бесконечности обозначается $\displaystyle \lim_{x \to \infty} f(x)$.
\begin{proof}[эквивалентности]
\begin{enumerate}
	\item (2) $\Rightarrow$ (1).
	Пусть $\displaystyle \lim_{n \to \infty} x_n = \infty$, тогда
	\begin{equation*}
	\forall M > 0 \ \exists n_0 \in \mathbb N \colon \forall n > n_0 \ |x_n| > M \Rightarrow
	\end{equation*}
	\begin{equation*}
	\Rightarrow \forall \varepsilon > 0 \ |f(x_n) - a| < \varepsilon \Rightarrow
	\lim_{n \to \infty} f(x_n) = a
	\end{equation*}
	
	\item (1) $\Rightarrow$ (2).
	Докажем методом от противного, что условия определения~(2) выполняются.
	Пусть
	\begin{equation*}
	\exists \varepsilon_0 > 0 \colon \forall M > 0 \ \exists x_0 \colon |x_0| > M, \ |f(x_0) - a| \geqslant \varepsilon_0
	\end{equation*}
	
	Тогда
	\begin{equation*}
	\forall n \in \mathbb N \ \exists x_n \colon |x_n| > n, \ |f(x_n) - a| \geqslant \varepsilon_0
	\end{equation*}
	
	Получили последовательность $\displaystyle (x_n) \colon \lim_{n \to \infty} x_n = \infty, \ \lim_{n \to \infty} f(x_n) \neq a$.
	Противоречие.
\end{enumerate}
\end{proof}

Аналогично доказывается эквивалентность следующих определений:
\begin{enumerate}
	\item \textbf{Определение по Гейне}
	
	Число~$a$ называется \textbf{пределом функции~$f(x)$ на плюс (минус) бесконечности}, если \allowbreak
	$\displaystyle \lim_{n \to \infty} f(x_n) \opbr= a$ для любой последовательности~$\displaystyle (x_n): \lim_{n \to \infty} x_n = +\infty \ (\lim_{n \to \infty} x_n = -\infty)$.
	
	\item \textbf{Определение по Коши}
	
	Число~$a$ называется \textbf{пределом функции~$f(x)$ на плюс (минус) бесконечности}, если
	\begin{equation*}
	\forall \varepsilon > 0 \ \exists M > 0 \colon \forall x \ x > M \ (x < -M) \Rightarrow |f(x) - b| < \varepsilon
	\end{equation*}
\end{enumerate}

Предел функции на бесконечности обладает теми же свойствами, что и предел функции в~точке.