\subsection{Предел функции в точке}
Пусть $a$~--- предельная точка области определения функции~$f(x)$.
Следующие определения эквивалентны:
\begin{enumerate}
	\item \textbf{Определение по Гейне}
	
	Число~$b$ называется \textbf{пределом функции~$f(x)$ в точке~$a$}, если $\displaystyle \lim_{n \to \infty} f(x_n) = b$ для любой последовательности~$\displaystyle (x_n): \lim_{n \to \infty} x_n = a$.
	
	\item \textbf{Определение по Коши}
	
	Число~$b$ называется \textbf{пределом функции~$f(x)$ в точке~$a$}, если
	\begin{equation*}
	\forall \varepsilon > 0 \ \exists \delta > 0 \colon \forall x \ |x - a| < \delta \Rightarrow |f(x) - b| < \varepsilon
	\end{equation*}
\end{enumerate}

Предел функции~$f(x)$ в точке~$a$ обозначается $\displaystyle \lim_{x \to a} f(x)$.
\begin{proof}[эквивалентности]
\begin{enumerate}
	\item (2) $\Rightarrow$ (1).
	Пусть $\displaystyle \lim_{n \to \infty} x_n = a$, тогда
	\begin{equation*}
	\forall \delta > 0 \ \exists n_0 \in \mathbb N \colon \forall n > n_0 \ |x_n - a| < \delta \Rightarrow
	\forall \varepsilon > 0 \ |f(x_n) - b| < \varepsilon \Rightarrow
	\lim_{n \to \infty} f(x_n) = b
	\end{equation*}
	
	\item (1) $\Rightarrow$ (2).
	Докажем методом от противного, что условия определения~(2) выполняются.
	Пусть
	\begin{equation*}
	\exists \varepsilon_0 > 0 \colon \forall \delta > 0 \ \exists x_0 \colon |x_0 - a| < \delta, \ |f(x_0) - b| \geqslant \varepsilon_0
	\end{equation*}
	
	Тогда
	\begin{equation*}
	\forall n \in \mathbb N \ \exists x_n \colon |x_n - a| < \frac1n, \ |f(x_n) - b| \geqslant \varepsilon_0
	\end{equation*}
	
	Получили последовательность $\displaystyle (x_n) \colon \lim_{n \to \infty} x_n = a, \ \lim_{n \to \infty} f(x_n) \neq b$.
	Противоречие.
\end{enumerate}
\end{proof}

Также можно определить односторонние пределы.

Число~$b$ называется \textbf{левым пределом}, или \textbf{пределом слева}, \textbf{функции~$f(x)$ в точке~$a$}, если
\begin{equation*}
\forall \varepsilon > 0 \ \exists \delta > 0 \colon \forall x \ 0 < a - x < \delta \Rightarrow |f(x) - b| < \varepsilon
\end{equation*}
и обозначается $\displaystyle \lim_{x \to a-0} f(x)$.

Число~$b$ называется \textbf{правым пределом}, или \textbf{пределом справа}, \textbf{функции~$f(x)$ в точке~$a$}, если
\begin{equation*}
\forall \varepsilon > 0 \ \exists \delta > 0 \colon \forall x \ 0 < x - a < \delta \Rightarrow |f(x) - b| < \varepsilon
\end{equation*}
и обозначается $\displaystyle \lim_{x \to a+0} f(x)$.

Т.\,о., $\displaystyle \lim_{x \to a} f(x) = b \Leftrightarrow \lim_{x \to a-0} f(x) = \lim_{x \to a+0} f(x) = b$.

С помощью определения по Гейне и свойств предела последовательности доказываются свойства предела функции в~точке.

Элементарные свойства:
\begin{enumerate}
	\item Функция может иметь не более одного предела в одной точке.
	\item \begin{theorem}[о двух милиционерах]
	\index{Теорема!о двух милиционерах}
	\label{th:about_two_policemen}
	Если в окрестности точки~$a$ $f(x) \leqslant g(x) \leqslant h(x)$, $\displaystyle \lim_{x \to a} f(x) = \lim_{x \to a} h(x) = b$, то $\displaystyle \lim_{x \to a} g(x) = b$.
	\end{theorem}
	\item Если в окрестности точки~$a$ $f(x) \geqslant 0$, $\displaystyle \lim_{x \to a} f(x) = b$, то $b \geqslant 0$.
	\item $\displaystyle \lim_{x \to a} f(x) = b \Rightarrow \lim_{x \to a} |f(x)| = |b|$.
\end{enumerate}

Арифметические свойства.
Пусть $\displaystyle \lim_{x \to x_0} f(x) = a$, $\displaystyle \lim_{x \to x_0} g(x) = b$.
\begin{enumerate}
	\item $\displaystyle \lim_{x \to x_0} (f(x) + g(x)) = a + b$
	\item $\displaystyle \lim_{x \to x_0} f(x)g(x) = ab$
	\item Если $a \neq 0$, то $\displaystyle \lim_{x \to x_0} \frac1{f(x)} = \frac1a$
	\item Если $b \neq 0$, то $\displaystyle \lim_{x \to x_0} \frac{f(x)}{g(x)} = \frac{a}b$
\end{enumerate}