\subsection{Замечательные пределы}
Замечательными пределами называют два тождества, часто используемых при нахождении других пределов.

\subsubsection{Первый замечательный предел}
\begin{statement}
\begin{equation*}
\lim_{x \to 0} \frac{\sin x}x = 1
\end{equation*}
\end{statement}
\begin{minipage}[c]{122.5mm}\parindent=15pt
\begin{proof}
Пусть $x > 0$.
Рассмотрим сектор~$AOB$ единичного круга ($OA = OB = 1$) с углом~$x$ и касательную~$BC$ к нему.
\begin{equation*}
S_{AOB} < S_\text{сект} < S_{BOC} \Leftrightarrow
\sin x < x < \tg x \Leftrightarrow
1 < \frac{x}{\sin x} < \frac1{\cos x}
\end{equation*}

Применяя \hyperref[th:about_two_policemen]{теорему о двух милиционерах}, получим:
\begin{equation*}
\lim_{x \to 0} 1 = \lim_{x \to 0} \frac1{\cos x} = 1 \Rightarrow
\lim_{x \to 0} \frac{\sin x}x = \lim_{x \to 0} \frac{x}{\sin x} = 1
\end{equation*}

Для $x < 0$ $\displaystyle \lim_{x \to 0} \frac{\sin x}x = \lim_{x \to 0} \frac{\sin (-x)}{-x} = 1$.
\end{proof}
\end{minipage}
\begin{minipage}[c]{57.5mm}
$\begin{xy} /r3cm/:
(0, 0) = "O" *{\bullet} *++!R{O};
(1.4142, 0) = "C" **@{-} *{\bullet} *++!L{C};
(0.5, 0.9142) **@{-}; % касательная
(0.7071, 0.7071) = "B" *{\bullet} *++!DL{B};
(1, 0) = "A" **@{-} *{\bullet} *++!UR{A};
"B";  "O" **@{-};
{"O" \ellipse<3cm>:a(210),^,=:a(75){}};
{"O" \ellipse<5mm>:a(90),_,:a(45){}};
(0.23, 0.075) *{x};
\end{xy}$
\end{minipage}

Следствия:
\begin{itemize}
	\item \begin{equation*}
	\lim_{x \to 0} \frac{\sin ax}x = a\lim_{x \to 0} \frac{\sin ax}{ax} = a, \ a \neq 0
	\end{equation*}

	\item \begin{equation*}
	\lim_{x \to 0} \frac{\tg ax}x = a\lim_{x \to 0} \frac{\sin ax}{ax \cos ax} = a, \ a \neq 0
	\end{equation*}
	
	\item \begin{equation*}
	\lim_{x \to 0} \frac{1 - \cos ax}{x^2} =
	\lim_{x \to 0} \frac{2\sin^2 \dfrac{ax}2}{x^2} =
	\frac{a^2}2 \lim_{x \to 0} \frac{\sin^2 \dfrac{ax}2}{\left( \dfrac{ax}2 \right)^2} =
	\frac{a^2}2, \ a \neq 0
	\end{equation*}
	
	\item \begin{equation*}
	\lim_{x \to 0} \frac{\arcsin ax}x \;
	\left| \text{Пусть } ax = \sin y \right| =
	a\lim_{y \to 0} \frac{y}{\sin y} = a, \ a \neq 0
	\end{equation*}
	
	\item \begin{equation*}
	\lim_{x \to 0} \frac{\arctg ax}x \;
	\left| \text{Пусть } x = \tg y \right| =
	a\lim_{y \to 0} \frac{y}{\tg y} = a, \ a \neq 0
	\end{equation*}
\end{itemize}

\subsubsection{Второй замечательный предел}
\begin{statement}
\begin{equation*}
\lim_{x \to \infty} \left( 1 + \frac1x \right)^x = e
\end{equation*}
\end{statement}
\begin{proof}
\begin{enumerate}
	\item Пусть $x > 0$.
	По определению числа Эйлера
	$\displaystyle \lim_{x \to +\infty} \left( 1 + \frac1{[x]} \right)^{[x]} =
	\lim_{x \to +\infty} \left( 1 + \frac1{[x] + 1} \right)^{[x] + 1} =
	e$.
	\begin{equation*}
	\left( 1 + \frac1{[x] + 1} \right)^{[x] + 1} \left( 1 + \frac1{[x] + 1} \right)^{-1} =
	\left( 1 + \frac1{[x] + 1} \right)^{[x]} <
	\end{equation*}
	\begin{equation*}
	< \left( 1 + \frac1x \right)^x <
	\left( 1 + \frac1{[x]} \right)^{[x] + 1} =
	\left( 1 + \frac1{[x]} \right)^{[x]} \left( 1 + \frac1{[x]} \right)
	\end{equation*}
	
	Применяя теорему о двух милиционерах, получим:
	\begin{equation*}
	\lim_{x \to +\infty} \left( 1 + \frac1{[x] + 1} \right)^{[x]} =
	\lim_{x \to +\infty} \left( 1 + \frac1{[x] + 1} \right)^{[x] + 1} \left( 1 + \frac1{[x] + 1} \right)^{-1} = e,
	\end{equation*}
	\begin{equation*}
	\lim_{x \to +\infty} \left( 1 + \frac1{[x]} \right)^{[x] + 1} =
	\lim_{x \to +\infty} \left( 1 + \frac1{[x]} \right)^{[x]} \left( 1 + \frac1{[x]} \right) = e \Rightarrow
	\end{equation*}
	\begin{equation*}
	\Rightarrow \lim_{x \to +\infty} \left( 1 + \frac1x \right)^x = e
	\end{equation*}
	
	\item Пусть $x < 0$, $y = -x$, тогда
	\begin{equation*}
	\lim_{x \to -\infty} \left( 1 + \frac1x \right)^x =
	\lim_{y \to +\infty} \left( 1 - \frac1y \right)^{-y} =
	\lim_{y \to +\infty} \left( \frac{y}{y - 1} \right)^y =
	\lim_{y \to +\infty} \left( 1 + \frac1{y - 1} \right)^y =
	\end{equation*}
	\begin{equation*}
	= \lim_{y \to +\infty} \left( 1 + \frac1{y - 1} \right)^{y-1} \left( 1 + \frac1{y - 1} \right) = e
	\end{equation*}
\end{enumerate}
\end{proof}

Следствия:
\begin{itemize}
	\item \begin{equation*}
	\lim_{x \to 0} (1 + ax)^{\tfrac1x} \;
	\left| \text{Пусть } y = \frac1{ax} \right| =
	\lim_{y \to \infty} \left( \left( 1 + \frac1y \right)^y \right)^a =
	e^a, \ a \neq 0
	\end{equation*}
	
	\item \begin{equation*}
	\lim_{x \to \infty} \left( 1 + \frac{a}x \right)^x =
	\lim_{x \to \infty} \left( \left( 1 + \frac{a}x \right)^{\tfrac{x}a} \right)^a =
	e^a, \ a \neq 0
	\end{equation*}
	
	\item \begin{equation*}
	\lim_{x \to 0} \frac{\ln (1 + ax)}{bx} =
	\lim_{x \to 0} \ln (1 + ax)^{\tfrac1{bx}} =
	\ln \lim_{x \to 0} \left( (1 + ax)^{\tfrac1{ax}} \right)^{\tfrac{a}b} =
	\ln e^{\tfrac{a}b} =
	\frac{a}b, \ a, b \neq 0
	\end{equation*}
	
	\item \begin{equation*}
	\lim_{x \to 0} \frac{c^{ax} - 1}{bx} =
	\end{equation*}
	\begin{equation*}
	\left| \text{Пусть } c^{ax} - 1 = y \Leftrightarrow ax\ln c = \ln (y + 1) \Leftrightarrow x = \frac{\ln (y + 1)}{a\ln c} \right|
	\end{equation*}
	\begin{equation*}
	= \frac{a\ln c}b \lim_{y \to 0} \frac{y}{\ln (y + 1)} =
	\frac{a}b \ln c, \ a, b \neq 0, \ c > 0
	\end{equation*}
	
	\item \begin{equation*}
	\lim_{x \to 0} \frac{(1 + ax)^n - 1}{bx} \;
	\left| \text{Пусть } 1 + ax = e^y \right| =
	\frac{a}b \lim_{y \to 0} \frac{e^{ny} - 1}{e^y - 1} =
	\frac{an}b \lim_{y \to 0} \frac{e^{ny} - 1}{ny} \cdot \frac{y}{e^y - 1} =
	\frac{an}b, \ a, b, n \neq 0
	\end{equation*}
\end{itemize}