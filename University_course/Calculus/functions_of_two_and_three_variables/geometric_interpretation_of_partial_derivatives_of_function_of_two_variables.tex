\subsection{Геометрическая интерпретация частных производных функции двух переменных}
Пусть функция~$f(x, y)$ имеет частные производные в точке~$(x_0, y_0)$.
Пересечением плоскости~$x = x_0$ с поверхностью~$z = f(x, y)$ является кривая~$z = f(x_0, y)$.
Т.\,о., значение~$f_y'(x_0, y_0)$ равно тангенсу угла между касательной к кривой~$z = f(x_0, y)$ в точке~$(x_0, y_0)$ и положительным направлением оси~$Oy$, а направляющий вектор этой касательной имеет координаты~$(0, 1, f_y'(x_0, y_0))$.

Аналогичный геометрический смысл имеет частная производная~$f_x'$.