\subsection{Дифференцируемость вектор-функции}
\index{Вектор-функция!дифференцируемая} Вектор-функция~$f(x_1, \ldots, x_n) = (f_1(\overline x), f_2(\overline x), \ldots, f_m(\overline x))$ называется \textbf{дифференцируемой в точке~$\overline x_0 = (x_{10}, \ldots, x_{n0})$}, если
\begin{equation*}
\exists A =
\begin{Vmatrix}
a_{11} & a_{12} & \cdots & a_{1n} \\
a_{21} & a_{22} & \cdots & a_{2n} \\
\vdots & \vdots & \ddots & \vdots \\
a_{m1} & a_{m2} & \cdots & a_{mn}
\end{Vmatrix} \colon
\forall \Delta \overline x = (\Delta x_1, \ldots, \Delta x_n) \
\begin{Vmatrix}
f_1(\overline x_0 + \Delta \overline x) - f_1(\overline x_0) \\
f_2(\overline x_0 + \Delta \overline x) - f_2(\overline x_0) \\
\vdots \\
f_m(\overline x_0 + \Delta \overline x) - f_m(\overline x_0)
\end{Vmatrix} =
A \cdot
\begin{Vmatrix}
\Delta x_1 \\
\Delta x_2 \\
\vdots \\
\Delta x_n
\end{Vmatrix} +
\begin{Vmatrix}
\alpha_1 \\
\alpha_2 \\
\vdots \\
\alpha_m
\end{Vmatrix},
\end{equation*}
\begin{equation*}
\lim_{\rho(\overline x_0 + \Delta \overline x, \overline x_0) \to 0}
\frac{\sqrt{\alpha_1^2 + \alpha_2^2 + \ldots + \alpha_m^2}}
{\rho(\overline x_0 + \Delta \overline x, \overline x_0)} = 0
\end{equation*}

\index{Матрица!производная} \index{Матрица!Якоби} Матрица~$A$ называется \textbf{производной матрицей}, или \textbf{матрицей Як\'{о}би}, и состоит из значений всех частных производных всех координатных функций в~данной точке:
\begin{equation*}
A =
\begin{Vmatrix}
f_{1\, x_1}'(\overline x_0) & f_{1\, x_2}'(\overline x_0) & \cdots & f_{1\, x_n}'(\overline x_0) \\
f_{2\, x_1}'(\overline x_0) & f_{2\, x_2}'(\overline x_0) & \cdots & f_{2\, x_n}'(\overline x_0) \\
\vdots & \vdots & \ddots & \vdots \\
f_{m\, x_1}'(\overline x_0) & f_{m\, x_2}'(\overline x_0) & \cdots & f_{m\, x_n}'(\overline x_0) \\
\end{Vmatrix}
\end{equation*}

\index{Якобиан} Если $f(x_1, \ldots, x_n)$~--- вектор-функция размерности~$n$, дифференцируемая в точке~$\overline x_0$, то \textbf{якобианом} называется определитель её производной матрицы и обозначается $\left. \dfrac{\partial f}{\partial \overline x} \right|_{\overline x_0}$.