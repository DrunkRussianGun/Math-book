\section{Функции двух и трёх переменных}
\subsection{Геометрическая интерпретация частных производных функции двух переменных}
Пусть функция~$f(x, y)$ имеет частные производные в точке~$(x_0, y_0)$.
Пересечением плоскости~$x = x_0$ с поверхностью~$z = f(x, y)$ является кривая~$z = f(x_0, y)$.
Т.\,о., значение~$f_y'(x_0, y_0)$ равно тангенсу угла между касательной к кривой~$z = f(x_0, y)$ в точке~$(x_0, y_0)$ и положительным направлением оси~$Oy$, а направляющий вектор этой касательной имеет координаты~$(0, 1, f_y'(x_0, y_0))$.

Аналогичный геометрический смысл имеет частная производная~$f_x'$.

\subsection{Уравнение касательной плоскости к поверхности}
\subsubsection{Поверхность, заданная явно}
Пусть поверхность задана уравнением~$z = f(x, y)$.
Проведём через точку~$(x_0, y_0, f(x_0, y_0))$ такую плоскость, что векторы $(0, 1, f_y'(x_0, y_0))$ и $(1, 0, f_x'(x_0, y_0))$ лежат в~ней.
Эта плоскость называется \textbf{касательной}.
Найдём вектор~$(A, B, C)$, перпендикулярный этим векторам, а значит, и проведённой плоскости:
\begin{equation*}
\begin{cases}
B + C f_y'(x_0, y_0) = 0 \\
A + C f_x'(x_0, y_0) = 0
\end{cases} \Leftrightarrow
\begin{cases}
A = -C f_x'(x_0, y_0) \\
B = -C f_y'(x_0, y_0)
\end{cases}
\end{equation*}

Вектор~$(f_x'(x_0, y_0), f_y'(x_0, y_0), -1)$ перпендикулярен проведённой плоскости, тогда её уравнение
\begin{equation*}
f_x'(x_0, y_0)(x - x_0) + f_y'(x_0, y_0)(y - y_0) - (z - f(x_0, y_0)) = 0 \Leftrightarrow
\end{equation*}
\begin{equation}
\label{eq:tangent_plane}
\Leftrightarrow z = f(x_0, y_0) + f_x'(x_0, y_0)(x - x_0) + f_y'(x_0, y_0)(y - y_0)
\end{equation}

\subsubsection{Поверхность, заданная параметрически}
Пусть поверхность задана функцией~$f(u, v) = (x(u, v), y(u, v), z(u, v))$, а в точке~$(x_0, y_0, z_0)$ к ней проведена касательная плоскость, причём $\dfrac{\partial(x, y)}{\partial(u, v)} \neq 0$.
Тогда $\exists u_0, v_0 \colon x(u_0, v_0) = x_0 \lAnd y(u_0, v_0) = y_0$.

Имеем явное и параметрическое задание одной и той~же поверхности: $z(u, v) = z(x(u, v), y(u, v))$.
Рассматривая производные матрицы этих функций, получим:
\begin{equation*}
\begin{Vmatrix}
z_u' & z_v'
\end{Vmatrix} =
\begin{Vmatrix}
z_x' & z_y'
\end{Vmatrix} \cdot
\begin{Vmatrix}
x_u' & x_v' \\
y_u' & y_v'
\end{Vmatrix} \Rightarrow
\begin{cases}
z_u' = z_x' x_u' + z_y' y_u' \\
z_v' = z_x' x_v' + z_y' y_v'
\end{cases}
\end{equation*}

Решая систему относительно $z_x'$ и $z_y'$, получим
\begin{equation*}
z_x'(x_0, y_0) = \frac
{\left. \frac{\partial(z, y)}{\partial(u, v)} \right|_{(u_0, v_0)}}
{\left. \frac{\partial(x, y)}{\partial(u, v)} \right|_{(u_0, v_0)}} =
-\frac
{\left. \frac{\partial(y, z)}{\partial(u, v)} \right|_{(u_0, v_0)}}
{\left. \frac{\partial(x, y)}{\partial(u, v)} \right|_{(u_0, v_0)}}, \
z_y'(x_0, y_0) = \frac
{\left. \frac{\partial(x, z)}{\partial(u, v)} \right|_{(u_0, v_0)}}
{\left. \frac{\partial(x, y)}{\partial(u, v)} \right|_{(u_0, v_0)}} =
-\frac
{\left. \frac{\partial(z, x)}{\partial(u, v)} \right|_{(u_0, v_0)}}
{\left. \frac{\partial(x, y)}{\partial(u, v)} \right|_{(u_0, v_0)}}
\end{equation*}

Подставим полученные значения в уравнение~(\ref*{eq:tangent_plane}) и получим уравнение касательной плоскости:
\begin{equation*}
z = z_0 - \frac
{\left. \frac{\partial(y, z)}{\partial(u, v)} \right|_{(u_0, v_0)}}
{\left. \frac{\partial(x, y)}{\partial(u, v)} \right|_{(u_0, v_0)}} (x - x_0) -
\frac
{\left. \frac{\partial(z, x)}{\partial(u, v)} \right|_{(u_0, v_0)}}
{\left. \frac{\partial(x, y)}{\partial(u, v)} \right|_{(u_0, v_0)}} (y - y_0)  \Leftrightarrow
\end{equation*}
\begin{equation*}
\Leftrightarrow \left. \frac{\partial(y, z)}{\partial(u, v)} \right|_{(u_0, v_0)} (x - x_0) +
\left. \frac{\partial(z, x)}{\partial(u, v)} \right|_{(u_0, v_0)} (y - y_0) +
\left. \frac{\partial(x, y)}{\partial(u, v)} \right|_{(u_0, v_0)} (z - z_0)
\end{equation*}