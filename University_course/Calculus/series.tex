\section{Числовые ряды}
Пусть дана бесконечная последовательность~$(a_n)$, $S_n = \sum\limits_{k=1}^n a_k$.

\index{Ряд!числовой} \textbf{Числовым рядом} называется совокупность последовательностей $(a_n)$ и $(S_n)$ и обозначается $\series a_k$.
$S_n$ называется \textbf{частичной суммой числового ряда}.

Если существует конечный предел~$\lim\limits_{n \to \infty} S_n = S$, то $S$ называется \textbf{суммой числового ряда}, а ряд называется \textbf{сходящимся}, иначе~--- \textbf{расходящимся}.

Свойства рядов:
\begin{enumerate}
	\item Ряд~$\series a_k$ сходится $\Rightarrow$ $\lim\limits_{n \to \infty} a_n = 0$.
	\begin{proof}
	\begin{equation*}
	a_n = S_n - S_{n-1}
	\left| \text{По \hyperref[th:Cauchy_criterion]{критерию Коши}} \right| \Rightarrow
	\lim_{n \to \infty} a_n = \lim_{n \to \infty} (S_n - S_{n-1}) = 0
	\end{equation*}
	\end{proof}
	
	\item Если ряды $\series a_k = S$ и $\series b_k = \sigma$, то $\series (\alpha a_k + \beta b_k) = \alpha S + \beta \sigma$.
	\begin{proof}
	\begin{equation*}
	\lim_{n \to \infty} \sum_{k=1}^n (\alpha a_k + \beta b_k) =
	\lim_{n \to \infty} (\alpha \sum_{k=1}^n a_k + \beta \sum_{k=1}^n b_k) =
	\alpha S + \beta \sigma
	\end{equation*}
	\end{proof}
	
	\item $\forall N \in \mathbb N$ $\series a_{k+N}$ сходится $\Leftrightarrow$ $\series a_k$ сходится.
	\begin{proof}
	$\sigma_n = S_{n+N} - S_N$, тогда
	\begin{equation*}
	\exists \lim_{n \to \infty} \sigma_n = \lim_{n \to \infty} S_n - S_N \Leftrightarrow
	\exists \lim_{n \to \infty} S_n
	\end{equation*}
	\end{proof}
\end{enumerate}

\subsection{Знакоположительные ряды}
\index{Ряд!знакоположительный} Числовой ряд называется \textbf{знакоположительным}, если все его члены положительны.

\begin{lemma}
\label{lemma:convergence_of_sign-positive_series}
Знакоположительный ряд сходится $\Leftrightarrow$ последовательность его частных сумм ограничена.
\end{lemma}
\begin{proof}
\begin{enumerate}
	\item $\Rightarrow$. Пусть $\exists \lim\limits_{n \to \infty} S_n = S$, тогда $\forall n \in N \ S_n \leqslant S$.
	\item $\Leftarrow$. Пусть $\exists T > 0 \colon \forall n \in \mathbb N \ S_n \leqslant T$, тогда по свойству~\ref{st:monotonic_bounded_sequence} предела последовательности ряд сходится.
\end{enumerate}
\end{proof}

\begin{theorem}[сравнения]
\label{th:direct_comparison_test1}
Пусть даны знакоположительные ряды $\series a_k$ и $\series b_k$, $\forall n \in \mathbb N \ a_n \leqslant b_n$.
\begin{itemize}
	\item Если $\series b_k$ сходится, то $\series a_k$ тоже сходится.
	\item Если $\series a_k$ расходится, то $\series b_k$ тоже расходится.
\end{itemize}
\end{theorem}
\begin{proof}
Пусть $(\sigma_n)$ и $(S_n)$~--- частичные суммы рядов $\series a_k$ и $\series b_k$ соответственно.
\begin{itemize}
	\item Пусть $\series b_k$ сходится.
	\begin{equation*}
	\exists T > 0 \colon \forall n \in N \ S_n \leqslant T \Rightarrow
	\sigma_n \leqslant S_n < T
	\end{equation*}
	
	По лемме~\ref*{lemma:convergence_of_sign-positive_series} $\series a_k$ сходится.
	
	\item Если $\series a_k$ расходится, то $\series b_k$ тоже расходится, что легко доказывается методом от противного.
\end{itemize}
\end{proof}

\begin{theorem}[сравнения]
Пусть даны знакоположительные ряды $\series a_k$ и $\series b_k$, $\lim\limits_{n \to \infty} \frac{a_n}{b_n} = c$.
Если $0 < c < \infty$, то $\series a_k$ сходится $\Leftrightarrow$ $\series b_k$ сходится.
\end{theorem}
\begin{proof}
\begin{equation*}
\forall \varepsilon > 0 \exists N \in \mathbb N \colon \forall k > N \ \left| \frac{a_k}{b_k} - L \right| < \varepsilon
\end{equation*}

Пусть $\varepsilon = \frac{c}2$, тогда
\begin{equation*}
\exists N_0 \in \mathbb N \colon \forall k > N_0 \ c - \frac{c}2 < \frac{a_k}{b_k} < \frac{c}2 + c \Rightarrow
a_k < \frac32 b_k c \lAnd b_k < \frac{2 a_k}c
\end{equation*}

По теореме~\ref*{th:direct_comparison_test1} получим:
\begin{itemize}
	\item Если $\series a_k$ сходится, то $\series \frac{2 a_k}c$ сходится $\Rightarrow$ $\series b_k$ сходится.
	\item Если $\series b_k$ сходится, то $\series \frac32 b_k c$ сходится $\Rightarrow$ $\series a_k$ сходится.
\end{itemize}
\end{proof}

\begin{theorem}[д'Аламбера]
Пусть $\series a_k$~--- знакоположительный ряд, $\lim\limits_{n \to \infty} \frac{a_{n+1}}{a_n} = p$.
Если $p < 1$, то $\series a_k$ сходится, а если $p > 1$, то расходится.
\end{theorem}
\begin{proof}
\begin{enumerate}
	\item Пусть $p < 1$.
	Выберём $\varepsilon > 0 \colon p + \varepsilon < 1$, тогда
	\begin{equation*}
	\exists N \in \mathbb N \colon \forall n > N \ \left| \frac{a_{n+1}}{a_n} - p \right| < \varepsilon \Rightarrow
	\frac{a_{n+1}}{a_n} < p + \varepsilon \Rightarrow
	\end{equation*}
	\begin{equation*}
	\Rightarrow \frac{a_{N+2}}{a_{N+1}} < p + \varepsilon \lAnd \frac{a_{N+3}}{a_{N+2}} < p + \varepsilon \Rightarrow
	a_{N+3} < (p + \varepsilon) a_{N+2} < (p + \varepsilon)^2 a_{N+1} \Rightarrow
	a_{N+k} < (p + \varepsilon)^{k-1} a_{N+1}
	\end{equation*}
	
	$\series (p + \varepsilon)^{k-1} a_{N+1}$ сходится $\Rightarrow$ $\series a_{N+k}$ сходится $\Rightarrow$ $\series a_k$ сходится.
	
	\item Пусть $p > 1$.
	Выберём $\varepsilon > 0 \colon p - \varepsilon > 1$, тогда
	\begin{equation*}
	\exists N \in \mathbb N \colon \forall n > N \ \left| \frac{a_{n+1}}{a_n} - p \right| < \varepsilon \Rightarrow
	\frac{a_{n+1}}{a_n} > p - \varepsilon \Rightarrow
	\end{equation*}
	\begin{equation*}
	\Rightarrow \frac{a_{N+2}}{a_{N+1}} > p - \varepsilon \lAnd \frac{a_{N+3}}{a_{N+2}} > p - \varepsilon \Rightarrow
	a_{N+3} > (p - \varepsilon) a_{N+2} > (p - \varepsilon)^2 a_{N+1} \Rightarrow
	a_{N+k} > (p - \varepsilon)^{k-1} a_{N+1}
	\end{equation*}
	
	$\series (p - \varepsilon)^{k-1} a_{N+1}$ расходится $\Rightarrow$ $\series a_{N+k}$ расходится $\Rightarrow$ $\series a_k$ расходится.
\end{enumerate}
\end{proof}

\begin{theorem}[радикальный признак Коши]
Пусть $\series a_k$~--- знакоположительный ряд и $\exists \lim\limits_{n \to \infty} \sqrt[n]{a_n} = p$.
Если $p < 1$, то $\series a_k$ сходится, а если $p > 1$, то расходится.
\end{theorem}
\begin{proof}
\begin{enumerate}
	\item Пусть $p < 1$.
	Выберём $\varepsilon > 0 \colon p + \varepsilon < 1$, тогда
	\begin{equation*}
	\exists N \in \mathbb N \colon \forall n > N \ \left| \sqrt[n]{a_n} - p \right| < \varepsilon \Rightarrow
	\sqrt[n]{a_n} < p + \varepsilon \Rightarrow
	a_n < (p + \varepsilon)^n \Rightarrow
	\frac{a_{n+1}}{a_n} < p + \varepsilon \Rightarrow
	\lim_{n \to \infty} \frac{a_{n+1}}{a_n} < 1
	\end{equation*}
	
	Тогда по признаку д'Аламбера $\series a_k$ сходится.
	
	\item Пусть $p > 1$.
	Выберём $\varepsilon > 0 \colon p - \varepsilon > 1$, тогда
	\begin{equation*}
	\exists N \in \mathbb N \colon \forall n > N \ \left| \sqrt[n]{a_n} - p \right| < \varepsilon \Rightarrow
	\sqrt[n]{a_n} > p - \varepsilon \Rightarrow
	a_n > (p - \varepsilon)^n \Rightarrow
	\frac{a_{n+1}}{a_n} > p - \varepsilon \Rightarrow
	\lim_{n \to \infty} \frac{a_{n+1}}{a_n} > 1
	\end{equation*}
	
	Тогда по признаку д'Аламбера $\series a_k$ расходится.
\end{enumerate}
\end{proof}