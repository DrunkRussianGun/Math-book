\subsection{Предел функции нескольких переменных}
Число~$a$ называется \textbf{пределом функции~$f(\overline x)$ в точке~$\overline x_0$}, если
\begin{equation*}
\forall \varepsilon > 0 \ \exists \delta > 0 \colon \forall \overline x \ (\rho(\overline x, \overline x_0) < \delta \Rightarrow |f(\overline x) - a| < \varepsilon)
\end{equation*}

Для вычисления предела функции двух переменных удобно перейти в~полярные координаты, трёх переменных~--- в~сферические.