\subsection{Выпуклость функции}
Кривая называется \textbf{выпуклой}, или \textbf{выпуклой вверх}, \textbf{в точке}, если в некоторой окрестности данной точки касательная к кривой в этой точке находится выше этой кривой.

Кривая называется \textbf{вогнутой}, или \textbf{выпуклой вниз}, \textbf{в точке}, если в некоторой окрестности данной точки касательная к кривой в этой точке находится ниже этой кривой.

\begin{theorem}
Пусть дана функция~$f(x)$.
Если $f''(x_0) < 0$, то кривая, задаваемая уравнением~$y \opbr= f(x)$, выпукла в точке~$x_0$.
Если же $f''(x_0) > 0$, то эта кривая вогнута в точке~$x_0$.
\end{theorem}
\begin{proof}
Касательная к кривой в точке~$(x_0, f(x_0))$ задаётся уравнением~$y = g(x)$, где
\begin{equation*}
g(x) = f(x_0) + f'(x_0)(x - x_0)
\end{equation*}

По \hyperref[eq:Taylor_series]{формуле Тейлора}
\begin{equation*}
f(x) = f(x_0) + f'(x_0)(x - x_0) + \frac12 f''(x_0)(x - x_0)^2 + o((x - x_0)^2)
\end{equation*}

Тогда
\begin{equation*}
f(x) - g(x) = \frac12 f''(x_0)(x - x_0)^2 + o((x - x_0)^2)
\end{equation*}

Т.\,о., знак разности~$f(x) - g(x)$ совпадает со знаком~$f''(x_0)$.
\begin{itemize}
	\item При $f''(x_0) < 0$ получим $f(x) < g(x)$ в некоторой окрестности точки~$x_0$, значит, кривая выпукла.
	\item При $f''(x_0) > 0$ получим $f(x) > g(x)$ в некоторой окрестности точки~$x_0$, значит, кривая вогнута.
\end{itemize}
\end{proof}

Пусть $f(x)$~--- функция.
\index{Точка!перегиба} Если $f_-''(x_0) \cdot f_+''(x_0) \leqslant 0$, то точка~$x_0$ называется \textbf{точкой перегиба}.