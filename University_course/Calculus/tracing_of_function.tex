\section{Исследование функции}
\subsection{Локальный экстремум функции}
\begin{theorem}
\label{th:criterion_of_monotonic_function}
Если функция~$f(x)$ дифференцируема на~$(a; b)$, то она не убывает (не возрастает) на~$(a; b) \opbr\Leftrightarrow \forall x \in (a; b) \ f'(x) \geqslant 0 \ (f'(x) \leqslant 0)$.
\end{theorem}
\begin{proof}
\begin{enumerate}
	\item $\Rightarrow$. Пусть $f(x)$ не убывает на~$(a; b)$, $x_1, x_2 \in (a; b)$.
	\begin{equation*}
	\frac{f(x_2) - f(x_1)}{x_2 - x_1} \geqslant 0 \Rightarrow
	\lim_{x_2 \to x_1} \frac{f(x_2) - f(x_1)}{x_2 - x_1} \geqslant 0 \Leftrightarrow
	f'(x_1) \geqslant 0
	\end{equation*}
	
	Доказательство в~случае невозрастания $f(x)$ аналогично.
	
	\item $\Leftarrow$. Пусть $\forall x \in (a; b) \ f'(x) \geqslant 0$, $a < x_1 < x_2 < b$.
	По \hyperref[th:mean_value]{теореме Лагранжа}
	\begin{equation*}
	\exists x_3 \in (x_1; x_2) \colon f(x_2) - f(x_1) = f'(x_3)(x_2 - x_1)
	\end{equation*}
	
	$f'(x_3)(x_2 - x_1) \geqslant 0 \Leftrightarrow
	f(x_2) - f(x_1) \geqslant 0 \Leftrightarrow
	f(x)$ не убывает на~$(a; b)$.
	
	Доказательство в~случае $f'(x) \leqslant 0$ аналогично.
\end{enumerate}
\end{proof}

\index{Максимум!функции} Точка~$x_0$ называется \textbf{точкой локального минимума функции}~$f(x)$, если существует проколотая окрестность~$\breve U(x_0) \colon \forall x \in \breve U(x_0) \ f(x) > f(x_0)$.

\index{Минимум!функции} Точка~$x_0$ называется \textbf{точкой локального максимума функции}~$f(x)$, если существует проколотая окрестность~$\breve U(x_0) \colon \forall x \in \breve U(x_0) \ f(x) < f(x_0)$.

\index{Экстремум} Точки локального минимума и максимума называются \textbf{точками локального экстремума}.

\begin{theorem}
Если $x_0$~--- точка локального экстремума функции~$f(x)$, то $\nexists f'(x_0) \lOr f'(x_0) = 0$.
\end{theorem}
\begin{proof}
Пусть $x_0$~--- точка локального минимума, $\exists f'(x_0)$, тогда
\begin{equation*}
\begin{cases}
\displaystyle \frac{f(x) - f(x_0)}{x - x_0} < 0, \ x < x_0 \\
\displaystyle \frac{f(x) - f(x_0)}{x - x_0} > 0, \ x > x_0
\end{cases} \Rightarrow
\begin{cases}
\displaystyle \lim_{x \to x_0-0} \frac{f(x) - f(x_0)}{x - x_0} \leqslant 0 \\
\displaystyle \lim_{x \to x_0+0} \frac{f(x) - f(x_0)}{x - x_0} \geqslant 0
\end{cases} \Rightarrow
\lim_{x \to x_0} \frac{f(x) - f(x_0)}{x - x_0} = 0 \Leftrightarrow
f'(x_0) = 0
\end{equation*}

Доказательство для локального максимума аналогично.
\end{proof}

\index{Точка!критическая} Точка, в~которой производная функции не существует или равна нулю, называется \textbf{критической}.

Существуют следующие признаки локального экстремума:
\begin{enumerate}
	\item Если $f_-'(x_0) \leqslant 0 \ (\geqslant 0) \lAnd f_+'(x_0) \geqslant 0 \ (\leqslant 0)$, то $x_0$~--- точка локального минимума (максимума).
	\begin{proof}
	Пусть
	\begin{equation*}
	f_-'(x_0) \leqslant 0 \lAnd f_+'(x_0) \geqslant 0 \Leftrightarrow
	\lim_{x \to x_0-0} \frac{f(x) - f(x_0)}{x - x_0} \leqslant 0 \lAnd
	\lim_{x \to x_0+0} \frac{f(x) - f(x_0)}{x - x_0} \geqslant 0
	\end{equation*}
	
	Значит, в некоторой окрестности точки~$x_0$ $f(x) \geqslant f(x_0)$, тогда $x_0$~--- точка локального минимума.
	
	Аналогичное доказательство для максимума.
	\end{proof}
	
	\item Если $f'(x_0) = f''(x_0) = \ldots = f^{(2n-1)}(x_0) = 0$, то
	\begin{itemize}
		\item $x_0$~--- точка локального максимума при~$f^{(2n)}(x_0) < 0$;
		\item $x_0$~--- точка локального минимума при~$f^{(2n)}(x_0) > 0$;
		\item $x_0$ не является точкой локального экстремума при~$f^{(2n)}(x_0) = 0$, $f^{(2n+1)}(x_0) \neq 0$.
	\end{itemize}
	\begin{proof}
	\begin{itemize}
		\item Пусть $f^{(2n)}(x_0) < 0$.
		По \hyperref[eq:Taylor_series]{формуле Тейлора}
		\begin{equation*}
		f(x) = f(x_0) + f'(x_0)(x - x_0) + \frac{f''(x_0)}{2!} (x - x_0)^2 + \ldots + \frac{f^{(2n)}(x_0)}{(2n)!} (x - x_0)^{2n} + o((x - x_0)^{2n}) \Leftrightarrow
		\end{equation*}
		\begin{equation*}
		\Leftrightarrow f(x) - f(x_0) = \frac{f^{(2n)}(x_0)}{(2n)!} (x - x_0)^{2n} + o((x - x_0)^{2n}) < 0
		\end{equation*}
		
		Тогда $x_0$~--- точка локального максимума.
		
		\item Случай при~$f^{(2n)}(x_0) > 0$ доказывается аналогично.
		
		\item Пусть $f^{(2n)}(x_0) = 0$, $f^{(2n+1)}(x_0) \neq 0$.
		По \hyperref[eq:Taylor_series]{формуле Тейлора}
		\begin{equation*}
		f(x) = f(x_0) + f'(x_0)(x - x_0) + \frac{f''(x_0)}{2!} (x - x_0)^2 + \ldots +
		\end{equation*}
		\begin{equation*}
		\vphantom1 + \frac{f^{(2n)}(x_0)}{(2n)!} (x - x_0)^{2n} + \frac{f^{(2n+1)}(x_0)}{(2n+1)!} (x - x_0)^{2n+1} + o((x - x_0)^{2n+1}) \Leftrightarrow
		\end{equation*}
		\begin{equation*}
		\Leftrightarrow f(x) - f(x_0) = \frac{f^{(2n+1)}(x_0)}{(2n+1)!} (x - x_0)^{2n+1} + o((x - x_0)^{2n+1})
		\end{equation*}
		
		Знак $f(x) - f(x_0)$ зависит от знака $x - x_0$, поэтому в точке~$x_0$ не может быть локального экстремума.
	\end{itemize}
	\end{proof}
\end{enumerate}

\subsection{Наименьшее и наибольшее значения функции}
Минимальное и максимальное значения функции на некотором отрезке не всегда находятся в~точках экстремума.
Для того, чтобы найти эти значения, необходимо вычислить значения функции в~критических и граничных точках и выбрать среди них наименьшее и наибольшее.

\subsection{Выпуклость функции}
Кривая называется \textbf{выпуклой}, или \textbf{выпуклой вверх}, \textbf{в точке}, если в некоторой окрестности данной точки касательная к кривой в этой точке находится выше этой кривой.

Кривая называется \textbf{вогнутой}, или \textbf{выпуклой вниз}, \textbf{в точке}, если в некоторой окрестности данной точки касательная к кривой в этой точке находится ниже этой кривой.

\begin{theorem}
Пусть дана функция~$f(x)$.
Если $f''(x_0) < 0$, то кривая, задаваемая уравнением~$y \opbr= f(x)$, выпукла в точке~$x_0$.
Если же $f''(x_0) > 0$, то эта кривая вогнута в точке~$x_0$.
\end{theorem}
\begin{proof}
Касательная к кривой в точке~$(x_0, f(x_0))$ задаётся уравнением~$y = g(x)$, где
\begin{equation*}
g(x) = f(x_0) + f'(x_0)(x - x_0)
\end{equation*}

По \hyperref[eq:Taylor_series]{формуле Тейлора}
\begin{equation*}
f(x) = f(x_0) + f'(x_0)(x - x_0) + \frac12 f''(x_0)(x - x_0)^2 + o((x - x_0)^2)
\end{equation*}

Тогда
\begin{equation*}
f(x) - g(x) = \frac12 f''(x_0)(x - x_0)^2 + o((x - x_0)^2)
\end{equation*}

Т.\,о., знак разности~$f(x) - g(x)$ совпадает со знаком~$f''(x_0)$.
\begin{itemize}
	\item При $f''(x_0) < 0$ получим $f(x) < g(x)$ в некоторой окрестности точки~$x_0$, значит, кривая выпукла.
	\item При $f''(x_0) > 0$ получим $f(x) > g(x)$ в некоторой окрестности точки~$x_0$, значит, кривая вогнута.
\end{itemize}
\end{proof}

Пусть $f(x)$~--- функция.
\index{Точка!перегиба} Если $f_-''(x_0) \cdot f_+''(x_0) \leqslant 0$, то точка~$x_0$ называется \textbf{точкой перегиба}.

\subsection{Асимптоты}
\index{Асимптота} Прямая называется \textbf{асимптотой кривой}, если расстояние от переменной точки кривой до данной прямой при удалении этой точки в бесконечность стремится к нулю.
Если указанное расстояние стремится к нулю при~$x \to \infty$, то такая асимптота называется \textbf{наклонной}, а если при~$y \to \infty$, то \textbf{вертикальной}.
Если наклонная асимптота задаётся уравнением $y = b$, то она называется \textbf{горизонтальной}.

\begin{theorem}
Кривая, задаваемая уравнением~$y = f(x)$, имеет наклонную асимптоту, задаваемую уравнением~$y = kx + b$, если $\lim\limits_{x \to \infty} \frac{f(x)}x = k$ и $\lim\limits_{x \to \infty} (f(x) - kx) = b$.
\end{theorem}
\begin{proof}
Из определения наклонной асимптоты $f(x) - (kx + b) = \alpha(x)$, где $\alpha(x)$~--- бесконечно малая при~$x \to \infty$.
Тогда
\begin{equation*}
\lim_{x \to \infty} \frac{f(x)}x =
\lim_{x \to \infty} \left( k + \frac{b}x + \frac{\alpha(x)}x \right) =
k, \
\lim_{x \to \infty} (f(x) - kx) =
\lim_{x \to \infty} (b + \alpha(x)) =
b
\end{equation*}
\end{proof}