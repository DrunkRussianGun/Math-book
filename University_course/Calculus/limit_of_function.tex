\section{Предел функции}
\subsection{Предел функции в точке}
\index{lim} \index{Предел!функции}

Пусть $a$~--- предельная точка области определения функции~$f(x)$.
Следующие определения эквивалентны:
\begin{enumerate}
	\item \textbf{Определение по Гейне}
	
	Число~$b$ называется \textbf{пределом функции~$f(x)$ в точке~$a$}, если $\displaystyle \lim_{n \to \infty} f(x_n) = b$ для любой последовательности~$\displaystyle (x_n) \colon \lim_{n \to \infty} x_n = a$.
	
	\item \textbf{Определение по Коши}
	
	Число~$b$ называется \textbf{пределом функции~$f(x)$ в точке~$a$}, если
	\begin{equation*}
	\forall \varepsilon > 0 \ \exists \delta > 0 \colon \forall x \ (|x - a| < \delta \Rightarrow |f(x) - b| < \varepsilon)
	\end{equation*}
\end{enumerate}

Предел функции~$f(x)$ в точке~$a$ обозначается $\displaystyle \lim_{x \to a} f(x)$.
\begin{proof}[эквивалентности]
\begin{enumerate}
	\item (2) $\Rightarrow$ (1).
	Пусть $\displaystyle \lim_{n \to \infty} x_n = a$, тогда
	\begin{equation*}
	\forall \delta > 0 \ \exists n_0 \in \mathbb N \colon \forall n > n_0 \ |x_n - a| < \delta \Rightarrow
	\forall \varepsilon > 0 \ |f(x_n) - b| < \varepsilon \Rightarrow
	\lim_{n \to \infty} f(x_n) = b
	\end{equation*}
	
	\item (1) $\Rightarrow$ (2).
	Докажем методом от противного, что условия определения~(2) выполняются.
	Пусть
	\begin{equation*}
	\exists \varepsilon_0 > 0 \colon \forall \delta > 0 \ \exists x_0 \colon |x_0 - a| < \delta \lAnd |f(x_0) - b| \geqslant \varepsilon_0
	\end{equation*}
	
	Тогда
	\begin{equation*}
	\forall n \in \mathbb N \ \exists x_n \colon |x_n - a| < \frac1n \lAnd |f(x_n) - b| \geqslant \varepsilon_0
	\end{equation*}
	
	Получили последовательность $\displaystyle (x_n) \colon \lim_{n \to \infty} x_n = a, \ \lim_{n \to \infty} f(x_n) \neq b$.
	Противоречие.
\end{enumerate}
\end{proof}

Также можно определить односторонние пределы.

Число~$b$ называется \textbf{левым пределом}, или \textbf{пределом слева}, \textbf{функции~$f(x)$ в точке~$a$}, если
\begin{equation*}
\forall \varepsilon > 0 \ \exists \delta > 0 \colon \forall x \ (0 < a - x < \delta \Rightarrow |f(x) - b| < \varepsilon)
\end{equation*}
и обозначается $\displaystyle \lim_{x \to a-0} f(x)$.

Число~$b$ называется \textbf{правым пределом}, или \textbf{пределом справа}, \textbf{функции~$f(x)$ в точке~$a$}, если
\begin{equation*}
\forall \varepsilon > 0 \ \exists \delta > 0 \colon \forall x \ (0 < x - a < \delta \Rightarrow |f(x) - b| < \varepsilon)
\end{equation*}
и обозначается $\displaystyle \lim_{x \to a+0} f(x)$.

Т.\,о., $\displaystyle \lim_{x \to a} f(x) = b \Leftrightarrow \lim_{x \to a-0} f(x) = \lim_{x \to a+0} f(x) = b$.

С помощью определения по Гейне и свойств предела последовательности доказываются свойства предела функции в~точке.

Элементарные свойства:
\begin{enumerate}
	\item Функция может иметь не более одного предела в одной точке.
	\item \begin{theorem}[о двух милиционерах]
	\index{Теорема!о двух милиционерах}
	\label{th:about_two_policemen}
	Если в окрестности точки~$a$ $f(x) \leqslant g(x) \leqslant h(x)$, $\displaystyle \lim_{x \to a} f(x) \opbr= \lim_{x \to a} h(x) \opbr= b$, то $\displaystyle \lim_{x \to a} g(x) = b$.
	\end{theorem}
	\item Если в окрестности точки~$a$ $f(x) \geqslant 0$, $\displaystyle \lim_{x \to a} f(x) = b$, то $b \geqslant 0$.
	\item $\displaystyle \lim_{x \to a} f(x) = b \Rightarrow \lim_{x \to a} |f(x)| = |b|$.
\end{enumerate}

Арифметические свойства.
Пусть $\displaystyle \lim_{x \to x_0} f(x) = a$, $\displaystyle \lim_{x \to x_0} g(x) = b$.
\begin{enumerate}
	\item $\displaystyle \lim_{x \to x_0} (f(x) + g(x)) = a + b$
	\item $\displaystyle \lim_{x \to x_0} f(x)g(x) = ab$
	\item Если $a \neq 0$, то $\displaystyle \lim_{x \to x_0} \frac1{f(x)} = \frac1a$
	\item Если $b \neq 0$, то $\displaystyle \lim_{x \to x_0} \frac{f(x)}{g(x)} = \frac{a}b$
\end{enumerate}

\subsection{Предел функции на бесконечности}
Пусть $f(x)$~--- функция.
Следующие определения эквивалентны:
\begin{enumerate}
	\item \textbf{Определение по Гейне}
	
	Число~$a$ называется \textbf{пределом функции~$f(x)$ на бесконечности}, если $\displaystyle \lim_{n \to \infty} f(x_n) = a$ для любой последовательности~$\displaystyle (x_n) \colon \lim_{n \to \infty} x_n = \infty$.
	
	\item \textbf{Определение по Коши}
	
	Число~$a$ называется \textbf{пределом функции~$f(x)$ на бесконечности}, если
	\begin{equation*}
	\forall \varepsilon > 0 \ \exists M > 0 \colon \forall x \ (|x| > M \Rightarrow |f(x) - b| < \varepsilon)
	\end{equation*}
\end{enumerate}

Предел функции~$f(x)$ на бесконечности обозначается $\displaystyle \lim_{x \to \infty} f(x)$.
\begin{proof}[эквивалентности]
\begin{enumerate}
	\item (2) $\Rightarrow$ (1).
	Пусть $\displaystyle \lim_{n \to \infty} x_n = \infty$, тогда
	\begin{equation*}
	\forall M > 0 \ \exists n_0 \in \mathbb N \colon \forall n > n_0 \ |x_n| > M \Rightarrow
	\end{equation*}
	\begin{equation*}
	\Rightarrow \forall \varepsilon > 0 \ |f(x_n) - a| < \varepsilon \Rightarrow
	\lim_{n \to \infty} f(x_n) = a
	\end{equation*}
	
	\item (1) $\Rightarrow$ (2).
	Докажем методом от противного, что условия определения~(2) выполняются.
	Пусть
	\begin{equation*}
	\exists \varepsilon_0 > 0 \colon \forall M > 0 \ \exists x_0 \colon |x_0| > M \lAnd |f(x_0) - a| \geqslant \varepsilon_0
	\end{equation*}
	
	Тогда
	\begin{equation*}
	\forall n \in \mathbb N \ \exists x_n \colon |x_n| > n \lAnd |f(x_n) - a| \geqslant \varepsilon_0
	\end{equation*}
	
	Получили последовательность $\displaystyle (x_n) \colon \lim_{n \to \infty} x_n = \infty, \ \lim_{n \to \infty} f(x_n) \neq a$.
	Противоречие.
\end{enumerate}
\end{proof}

Аналогично доказывается эквивалентность следующих определений:
\begin{enumerate}
	\item \textbf{Определение по Гейне}
	
	Число~$a$ называется \textbf{пределом функции~$f(x)$ на плюс (минус) бесконечности}, если $\displaystyle \lim_{n \to \infty} f(x_n) = a$ для любой последовательности~$\displaystyle (x_n) \colon \lim_{n \to \infty} x_n = +\infty \ (\lim_{n \to \infty} x_n = -\infty)$.
	
	\item \textbf{Определение по Коши}
	
	Число~$a$ называется \textbf{пределом функции~$f(x)$ на плюс (минус) бесконечности}, если
	\begin{equation*}
	\forall \varepsilon > 0 \ \exists M > 0 \colon \forall x \ (x > M \Rightarrow |f(x) - b| < \varepsilon)
	\end{equation*}
	\begin{equation*}
	(\forall \varepsilon > 0 \ \exists M > 0 \colon \forall x \ (x < -M \Rightarrow |f(x) - b| < \varepsilon))
	\end{equation*}
\end{enumerate}

Предел функции на бесконечности обладает теми же свойствами, что и предел функции в~точке.

\subsection{Замечательные пределы}
Замечательными пределами называют два тождества, часто используемых при нахождении других пределов.

\subsubsection{Первый замечательный предел}
\begin{statement}
\begin{equation*}
\lim_{x \to 0} \frac{\sin x}x = 1
\end{equation*}
\end{statement}
\begin{wrapfigure}{r}{0pt}
\noindent
\shorthandoff{"}
\begin{tikzpicture}[scale=3]
\clip (-0.15, -0.2) rectangle (1.6, 1.1);
\draw circle (1);
\draw (0:1) coordinate["$A$" {below right}] (A) node {$\bullet$}
	-- (0:0) coordinate["$O$" {left}] (O) node {$\bullet$}
	-- (45:1) coordinate["$B$" {above right}] (B) node {$\bullet$}
	-- cycle;
\draw pic[draw, "$x$", angle eccentricity=1.5, angle radius=3mm] {angle = A--O--B};

\draw[name path=tangent] (B) +(135:1) -- +(-45:2);
\path[name path=OA] (O) -- (0:2);
\draw[name intersections={of=OA and tangent, by=C}]
	(C) node {$\bullet$} node[above right] {$C$} -- (A);
\end{tikzpicture}
\shorthandon{"}
\end{wrapfigure}
\begin{proof}
Пусть $x > 0$.
Рассмотрим сектор~$AOB$ единичного круга ($OA = OB = 1$) с углом~$x$ и касательную~$BC$ к нему.

\begin{equation*}
S_{AOB} < S_\text{сект} < S_{BOC} \Leftrightarrow
\sin x < x < \tg x \Leftrightarrow
1 < \frac{x}{\sin x} < \frac1{\cos x}
\end{equation*}

Применяя \hyperref[th:about_two_policemen]{теорему о двух милиционерах}, получим:
\begin{equation*}
\lim_{x \to 0} 1 = \lim_{x \to 0} \frac1{\cos x} = 1 \Rightarrow
\lim_{x \to 0} \frac{\sin x}x = \lim_{x \to 0} \frac{x}{\sin x} = 1
\end{equation*}

Для $x < 0$ $\displaystyle \lim_{x \to 0} \frac{\sin x}x = \lim_{x \to 0} \frac{\sin (-x)}{-x} = 1$.
\end{proof}

Следствия:
\begin{itemize}
	\item $\displaystyle \lim_{x \to 0} \frac{\sin ax}x = a$, $a \neq 0$
	\begin{proof}
	\begin{equation*}
	\lim_{x \to 0} \frac{\sin ax}x = a\lim_{x \to 0} \frac{\sin ax}{ax} = a
	\end{equation*}
	\end{proof}

	\item $\displaystyle \lim_{x \to 0} \frac{\tg ax}x = a$, $a \neq 0$
	\begin{proof}
	\begin{equation*}
	\lim_{x \to 0} \frac{\tg ax}x = a\lim_{x \to 0} \frac{\sin ax}{ax \cos ax} = a
	\end{equation*}
	\end{proof}
	
	\item $\displaystyle \lim_{x \to 0} \frac{1 - \cos ax}{x^2} = \frac{a^2}2$, $a \neq 0$
	\begin{proof}
	\begin{equation*}
	\lim_{x \to 0} \frac{1 - \cos ax}{x^2} =
	\lim_{x \to 0} \frac{2\sin^2 \dfrac{ax}2}{x^2} =
	\frac{a^2}2 \lim_{x \to 0} \frac{\sin^2 \dfrac{ax}2}{\left( \dfrac{ax}2 \right)^2} =
	\frac{a^2}2
	\end{equation*}
	\end{proof}
	
	\item $\displaystyle \lim_{x \to 0} \frac{\arcsin ax}x = a$, $a \neq 0$
	\begin{proof}
	\begin{equation*}
	\lim_{x \to 0} \frac{\arcsin ax}x \;
	\left| \text{Пусть } ax = \sin y \right| =
	a\lim_{y \to 0} \frac{y}{\sin y} = a
	\end{equation*}
	\end{proof}
	
	\item $\displaystyle \lim_{x \to 0} \frac{\arctg ax}x = a$, $a \neq 0$
	\begin{proof}
	\begin{equation*}
	\lim_{x \to 0} \frac{\arctg ax}x \;
	\left| \text{Пусть } x = \tg y \right| =
	a\lim_{y \to 0} \frac{y}{\tg y} = a
	\end{equation*}
	\end{proof}
\end{itemize}

\subsubsection{Второй замечательный предел}
\begin{statement}
\begin{equation*}
\lim_{x \to \infty} \left( 1 + \frac1x \right)^x = e
\end{equation*}
\end{statement}
\begin{proof}
\begin{enumerate}
	\item Пусть $x > 0$.
	По определению числа Эйлера
	$\displaystyle \lim_{x \to +\infty} \left( 1 + \frac1{[x]} \right)^{[x]} =
	\lim_{x \to +\infty} \left( 1 + \frac1{[x] + 1} \right)^{[x] + 1} =
	e$.
	\begin{equation*}
	\left( 1 + \frac1{[x] + 1} \right)^{[x] + 1} \left( 1 + \frac1{[x] + 1} \right)^{-1} =
	\left( 1 + \frac1{[x] + 1} \right)^{[x]} <
	\end{equation*}
	\begin{equation*}
	< \left( 1 + \frac1x \right)^x <
	\left( 1 + \frac1{[x]} \right)^{[x] + 1} =
	\left( 1 + \frac1{[x]} \right)^{[x]} \left( 1 + \frac1{[x]} \right)
	\end{equation*}
	
	Применяя теорему о двух милиционерах, получим:
	\begin{equation*}
	\lim_{x \to +\infty} \left( 1 + \frac1{[x] + 1} \right)^{[x]} =
	\lim_{x \to +\infty} \left( 1 + \frac1{[x] + 1} \right)^{[x] + 1} \left( 1 + \frac1{[x] + 1} \right)^{-1} = e,
	\end{equation*}
	\begin{equation*}
	\lim_{x \to +\infty} \left( 1 + \frac1{[x]} \right)^{[x] + 1} =
	\lim_{x \to +\infty} \left( 1 + \frac1{[x]} \right)^{[x]} \left( 1 + \frac1{[x]} \right) = e \Rightarrow
	\end{equation*}
	\begin{equation*}
	\Rightarrow \lim_{x \to +\infty} \left( 1 + \frac1x \right)^x = e
	\end{equation*}
	
	\item Пусть $x < 0$, $y = -x$, тогда
	\begin{equation*}
	\lim_{x \to -\infty} \left( 1 + \frac1x \right)^x =
	\lim_{y \to +\infty} \left( 1 - \frac1y \right)^{-y} =
	\lim_{y \to +\infty} \left( \frac{y}{y - 1} \right)^y =
	\lim_{y \to +\infty} \left( 1 + \frac1{y - 1} \right)^y =
	\end{equation*}
	\begin{equation*}
	= \lim_{y \to +\infty} \left( 1 + \frac1{y - 1} \right)^{y-1} \left( 1 + \frac1{y - 1} \right) = e
	\end{equation*}
\end{enumerate}
\end{proof}

Следствия:
\begin{itemize}
	\item $\displaystyle \lim_{x \to 0} (1 + ax)^{\tfrac1x} = e^a$, $a \neq 0$
	\begin{proof}
	\begin{equation*}
	\lim_{x \to 0} (1 + ax)^{\tfrac1x} \;
	\left| \text{Пусть } y = \frac1{ax} \right| =
	\lim_{y \to \infty} \left( \left( 1 + \frac1y \right)^y \right)^a =
	e^a
	\end{equation*}
	\end{proof}
	
	\item $\displaystyle \lim_{x \to \infty} \left( 1 + \frac{a}x \right)^x = e^a$, $a \neq 0$
	\begin{proof}
	\begin{equation*}
	\lim_{x \to \infty} \left( 1 + \frac{a}x \right)^x =
	\lim_{x \to \infty} \left( \left( 1 + \frac{a}x \right)^{\tfrac{x}a} \right)^a =
	e^a
	\end{equation*}
	\end{proof}
	
	\item $\displaystyle \lim_{x \to 0} \frac{\ln (1 + ax)}{bx} = \frac{a}b$, $a, b \neq 0$
	\begin{proof}
	\begin{equation*}
	\lim_{x \to 0} \frac{\ln (1 + ax)}{bx} =
	\lim_{x \to 0} \ln (1 + ax)^{\tfrac1{bx}} =
	\ln \lim_{x \to 0} \left( (1 + ax)^{\tfrac1{ax}} \right)^{\tfrac{a}b} =
	\ln e^{\tfrac{a}b} =
	\frac{a}b
	\end{equation*}
	\end{proof}
	
	\item $\displaystyle \lim_{x \to 0} \frac{c^{ax} - 1}{bx} = \frac{a}b \ln c$, $a, b \neq 0$, $c > 0$
	\begin{proof}
	\begin{equation*}
	\lim_{x \to 0} \frac{c^{ax} - 1}{bx} =
	\end{equation*}
	\begin{equation*}
	\left| \text{Пусть } c^{ax} - 1 = y \Leftrightarrow ax\ln c = \ln (y + 1) \Leftrightarrow x = \frac{\ln (y + 1)}{a\ln c} \right|
	\end{equation*}
	\begin{equation*}
	= \frac{a\ln c}b \lim_{y \to 0} \frac{y}{\ln (y + 1)} =
	\frac{a}b \ln c
	\end{equation*}
	\end{proof}
	
	\item $\displaystyle \lim_{x \to 0} \frac{(1 + ax)^n - 1}{bx} = \frac{an}b$, $a, b, n \neq 0$
	\begin{proof}
	\begin{equation*}
	\lim_{x \to 0} \frac{(1 + ax)^n - 1}{bx} \;
	\left| \text{Пусть } 1 + ax = e^y \right| =
	\frac{a}b \lim_{y \to 0} \frac{e^{ny} - 1}{e^y - 1} =
	\frac{an}b \lim_{y \to 0} \frac{e^{ny} - 1}{ny} \cdot \frac{y}{e^y - 1} =
	\frac{an}b
	\end{equation*}
	\end{proof}
\end{itemize}