\section{Неопределённый интеграл}
\index{Первообразная} Первообразной функции~$f(x)$ называется функция~$F(x) \colon F'(x) = f(x)$.

\begin{theorem}
Если $F'(x) = G'(x) = f(x)$, то $F(x) - G(x) = C$, где $C$~--- некоторая константа.
\end{theorem}
\begin{proof}
Пусть $H(x) = F(x) - G(x)$, тогда по \hyperref[th:mean_value]{теореме Лагранжа}
\begin{equation*}
H(b) - H(a) = H'(c)(b - a) = 0 \Rightarrow
H(b) - H(a) = (F'(c) - G'(c))(b - a) = 0 \Rightarrow
H(x) = C
\end{equation*}
\end{proof}

\index{Интеграл!неопределённый} Множество всех первообразных функции~$f(x)$ называется \textbf{неопределённым интегралом} и обозначается $\int f(x)\,dx = F(x) + C$, где $F(x)$~--- первообразная~$f(x)$, $C$~--- произвольная константа.
$f(x)$ называется \textbf{подынтегральной функцией}, а $f(x)\,dx$~--- \textbf{подынтегральным выражением}.

\subsection{Свойства неопределённого интеграла}
Пусть $C$~--- произвольная константа.
\begin{enumerate}
	\item Пусть $F(x)$~--- первообразная функции~$f(x)$, тогда $d \left( \int f(x)\,dx \right) = f(x)\,dx$.
	
	\item $\int dF(x) = F(x) + C$.
	
	\item $\int (f(x) + g(x))\,dx = \int f(x)\,dx + \int g(x)\,dx$.
	\begin{proof}
	Пусть $F'(x) = f(x)$, $G'(x) = g(x)$, тогда $(F(x) + G(x))' = f(x) + g(x)$.
	Получим:
	\begin{equation*}
	\int f(x)\,dx + \int g(x)\,dx = F(x) + C_1 + G(x) + C_2 = (F(x) + G(x)) + C = \int (f(x) + g(x))\,dx
	\end{equation*}
	\end{proof}
	
	\item $\int a f(x)\,dx = a \int f(x)\,dx$.
	\begin{proof}
	Пусть $F'(x) = f(x)$, тогда $(a F(x))' = a f(x)$.
	Получим:
	\begin{equation*}
	a \int f(x)\,dx = a(F(x) + C_1) = a F(x) + C = \int a f(x)\,dx
	\end{equation*}
	\end{proof}
	
	\item Если $\int f(x)\,dx = F(x) + C$, $u(x)$~--- дифференцируемая функция, то $\int f(u)\,du = F(u) + C$.
\end{enumerate}

Пусть $u(x)$ и $v(x)$~--- дифференцируемые функции.
\index{Метод!интегрирования по частям} Существует \textbf{метод интегрирования по частям}, использующий следующее свойство:
$\int uv'\,dx = uv - \int u'v\,dx$.
\begin{proof}
\begin{equation*}
d(uv) = du\,v + u\,dv \Rightarrow
\int d(uv) = \int du\,v + u\,dv \Rightarrow
uv + C = \int du\,v + \int u\,dv \Rightarrow
\int uv'\,dx = u\,v - \int u'v\,dx
\end{equation*}
\end{proof}

\subsection{Таблица первообразных}
\begin{itemize}
	\item $\displaystyle \int x^n\,dx = \frac{x^{n+1}}{n + 1} + C$, $n \neq -1$
	
	\item $\displaystyle \int \frac{dx}{x + a} = \ln |x + a| + C$
	\begin{proof}
	\begin{equation*}
	(\ln |x + a| + C)' = (\ln \sqrt{(x + a)^2})' = \frac1{|x + a|} \cdot \frac1{2\sqrt{(x + a)^2}} \cdot 2(x + a) = \frac1{x + a}
	\end{equation*}
	\end{proof}
	
	\item $\displaystyle \int \frac{dx}{x^2 + a^2} = \frac1a \arctg \frac{x}a + C$, $a > 0$
	\begin{proof}
	\begin{equation*}
	\int \frac{dx}{x^2 + a^2} =
	\frac1{a^2} \int \frac{dx}{1 + \left( \frac{x}a \right)^2} =
	\frac1a \int \frac{d \left( \frac{x}a \right)}{1 + \left( \frac{x}a \right)^2} =
	\frac1a \arctg \frac{x}a + C
	\end{equation*}
	\end{proof}
	
	\item Т.\,н. <<\textbf{высокий логарифм}>>: $\displaystyle \int \frac{dx}{x^2 - a^2} = \frac1{2a} \ln \left| \frac{x - a}{x + a} \right| + C$, $a > 0$
	\begin{proof}
	\begin{equation*}
	\int \frac{dx}{x^2 - a^2} =
	\int \frac{dx}{2a(x - a)} - \int \frac{dx}{2a(x + a)} =
	\frac1{2a} ((\ln |x - a| + C_1) - (\ln |x + a| + C_2)) =
	\frac1{2a} \ln \left| \frac{x - a}{x + a} \right| + C
	\end{equation*}
	\end{proof}
	
	\item $\displaystyle \int \frac{x}{x^2 + a}\,dx = \frac12 \ln |x^2 + a| + C$, $a \neq 0$
	\begin{proof}
	\begin{equation*}
	\int \frac{x}{x^2 + a}\,dx =
	\frac12 \int \frac{d(x^2 + a)}{x^2 + a} =
	\frac12 \ln |x^2 + a| + C
	\end{equation*}
	\end{proof}
	
	\item Т.\,н. <<\textbf{длинный логарифм}>>: $\displaystyle \int \frac{dx}{\sqrt{x^2 + a}} = \ln \left| x + \sqrt{x^2 + a} \right| + C$, $a \neq 0$
	\begin{proof}
	Пусть $k = \sqrt{|a|}$.
	\begin{enumerate}
		\item Пусть $a < 0$, $x = \dfrac{k}{\sin t}$.
		\begin{equation*}
		\int \frac{dx}{\sqrt{x^2 - k^2}} =
		-\int \frac{\cos t}{\sin^2 t \cdot \sqrt{\frac1{\sin^2 t} - 1}}\,dt =
		-\int \frac{dt}{\sin t} =
		\int \frac{d(\cos t)}{1 - \cos^2 t} =
		-\frac12 \ln \left| \frac{\cos t - 1}{\cos t + 1} \right| + C_1 =
		\end{equation*}
		\begin{equation*}
		= -\frac12 \ln \left| \frac{\cos^2 t - 1}{\cos^2 t + 1 + 2\cos t} \right| + C_1 =
		\left| x = \frac{k}{\sin t} \Rightarrow
		\sqrt{1 - \cos^2 t} = \frac{k}x \Rightarrow
		\cos^2 t = 1 - \frac{k^2}{x^2} \right|
		\end{equation*}
		\begin{equation*}
		= -\frac12 \ln \left| \frac{-\frac{k^2}{x^2}}{-\frac{k^2}{x^2} + 2 + 2\sqrt{1 - \frac{k^2}{x^2}}} \right| + C_1 =
		\frac12 \ln \left| 1 - \frac{2x^2}{k^2} - \frac{2x}{k^2}\sqrt{x^2 - k^2} \right| + C_1 =
		\end{equation*}
		\begin{equation*}
		= \frac12 \ln \frac1{k^2} \left| 2x^2 + 2x\sqrt{x^2 + a} + a \right| + C_1 =
		\frac12 \ln \left| x^2 + 2x\sqrt{x^2 + a} + (x^2 + a) \right| + C =
		\end{equation*}
		\begin{equation*}
		= \ln \left| x + \sqrt{x^2 + a} \right| + C
		\end{equation*}
		
		\item Пусть $a > 0$, $x = k\tg t$.
		\begin{equation*}
		\int \frac{dx}{\sqrt{x^2 + k^2}} =
		\int \frac{dt}{\cos^2 t \cdot \sqrt{\tg^2 t + 1}} =
		\int \frac{dt}{\cos t} =
		\int \frac{d(\sin t)}{1 - \sin^2 t} =
		-\frac12 \ln \left| \frac{\sin t - 1}{\sin t + 1} \right| + C_1 =
		\end{equation*}
		\begin{equation*}
		= -\frac12 \ln \left| \frac{\sin^2 t - 1}{\sin^2 t + 1 + 2\sin t} \right| + C_1 =
		\end{equation*}
		\begin{equation*}
		\left| x = k\tg t \Rightarrow
		\sqrt{\frac1{\cos^2 t} - 1} = \frac{x}k \Rightarrow
		\cos^2 t = \frac{k^2}{x^2 + k^2} \Leftrightarrow
		\sin^2 t = \frac{x^2}{x^2 + k^2} \right|
		\end{equation*}
		\begin{equation*}
		= -\frac12 \ln \left| \frac{-\frac{k^2}{x^2 + k^2}}{\frac{2x^2 + k^2}{x^2 + k^2} + 2\sqrt{\frac{x^2}{x^2 + k^2}}} \right| + C_1 =
		\frac12 \ln \frac1{k^2} \left| 2x^2 + k^2 + 2x\sqrt{x^2 + k^2} \right| + C_1 =
		\end{equation*}
		\begin{equation*}
		= \frac12 \ln \left| x^2 + 2x\sqrt{x^2 + a} + (x^2 + a) \right| + C =
		\ln \left| x + \sqrt{x^2 + a} \right| + C
		\end{equation*}
	\end{enumerate}
	\end{proof}
	
	\item $\displaystyle \int \frac{dx}{\sqrt{a^2 - x^2}} = \arcsin \frac{x}a + C$, $a > 0$
	\begin{proof}
	Пусть $x = a \sin t$, тогда
	\begin{equation*}
	\int \frac{dx}{\sqrt{a^2 - x^2}} =
	a \int \frac{\cos t}{a \sqrt{1 - \sin^2 t}}\,dt =
	\int dt =
	t + C =
	\arcsin \frac{x}a + C
	\end{equation*}
	\end{proof}
	
	\item $\displaystyle \int \frac{x}{\sqrt{a^2 \pm x^2}}\,dx = \pm \sqrt{a^2 \pm x^2} + C$, $a \neq 0$
	\begin{proof}
	\begin{equation*}
	\int \frac{x}{\sqrt{a^2 \pm x^2}}\,dx =
	\pm \frac12 \int (a^2 \pm x^2)^{-\tfrac12}\,d(a^2 \pm x^2) =
	\pm \sqrt{a^2 \pm x^2} + C
	\end{equation*}
	\end{proof}
	
	\item $\displaystyle \int a^x\,dx = \frac{a^x}{\ln a} + C$
	
	\item $\displaystyle \int \ln x\,dx = x \ln x - x + C$
	\begin{proof}
	\begin{equation*}
	\int \ln x\,dx =
	\int \ln x \cdot 1 \cdot dx =
	x \ln x - \int \frac{x}x\,dx =
	x \ln x - x + C
	\end{equation*}
	\end{proof}
	
	\item $\displaystyle \int \sin x\,dx = -\cos x + C$
	
	\item $\displaystyle \int \cos x\,dx = \sin x + C$
	
	\item $\displaystyle \int \frac{dx}{\cos^2 x} = \tg x + C$
	
	\item $\displaystyle \int \frac{dx}{\sin^2 x} = -\ctg x + C$
	
	\item $\displaystyle \int \tg x\,dx = -\ln |\cos x| + C$
	\begin{proof}
	\begin{equation*}
	\int \tg x\,dx =
	\int \frac{\sin x}{\cos x}\,dx =
	-\int \frac{d(\cos x)}{\cos x} =
	-\ln |\cos x| + C
	\end{equation*}
	\end{proof}
	
	\item $\displaystyle \int \ctg x\,dx = \ln |\sin x| + C$
	\begin{proof}
	\begin{equation*}
	\int \ctg x\,dx =
	\int \frac{\cos x}{\sin x}\,dx =
	\int \frac{d(\sin x)}{\sin x} =
	\ln |\sin x| + C
	\end{equation*}
	\end{proof}
	
	\item $\displaystyle \int \arcsin x\,dx = x \arcsin x + \sqrt{1 - x^2} + C$
	\begin{proof}
	\begin{equation*}
	\int \arcsin x =
	\int \arcsin x \cdot 1 \cdot dx =
	x \arcsin x - \int \frac{x}{\sqrt{1 - x^2}}\,dx =
	\end{equation*}
	\begin{equation*}
	= x \arcsin x + \frac12 \int (1 - x^2)^{-\tfrac12}\,d(1 - x^2) =
	x \arcsin x + \sqrt{1 - x^2} + C
	\end{equation*}
	\end{proof}
	
	\item $\displaystyle \int \arccos x\,dx = x \arccos x - \sqrt{1 - x^2} + C$
	\begin{proof}
	\begin{equation*}
	\int \arccos x =
	\int \arccos x \cdot 1 \cdot dx =
	x \arccos x + \int \frac{x}{\sqrt{1 - x^2}}\,dx =
	\end{equation*}
	\begin{equation*}
	= x \arccos x - \frac12 \int (1 - x^2)^{-\tfrac12}\,d(1 - x^2) =
	x \arccos x - \sqrt{1 - x^2} + C
	\end{equation*}
	\end{proof}
	
	\item $\displaystyle \int \arctg x\,dx = x \arctg x - \frac12 \ln |1 + x^2| + C$
	\begin{proof}
	\begin{equation*}
	\int \arctg x =
	\int \arctg x \cdot 1 \cdot dx =
	x \arctg x - \int \frac{x}{1 + x^2}\,dx =
	\end{equation*}
	\begin{equation*}
	= x \arctg x - \frac12 \int \frac{d(1 + x^2)}{1 + x^2} =
	x \arctg x - \frac12 \ln |1 + x^2| + C
	\end{equation*}
	\end{proof}
	
	\item $\displaystyle \int \arcctg x\,dx = x \arcctg x + \frac12 \ln |1 + x^2| + C$
	\begin{proof}
	\begin{equation*}
	\int \arcctg x =
	\int \arcctg x \cdot 1 \cdot dx =
	x \arcctg x + \int \frac{x}{1 + x^2}\,dx =
	\end{equation*}
	\begin{equation*}
	= x \arcctg x + \frac12 \int \frac{d(1 + x^2)}{1 + x^2} =
	x \arcctg x + \frac12 \ln |1 + x^2| + C
	\end{equation*}
	\end{proof}
\end{itemize}

\subsection{Интегрирование простейших дробей}
\begin{itemize}
	\item $\displaystyle \int \frac{dx}{(x - a)^n} =
	\begin{cases}
	\frac{(x - a)^{1-n}}{1 - n} + C, \ n \neq 1 \\
	\ln |x - a| + C, \ n = 1
	\end{cases}$
	\begin{proof}
	\begin{equation*}
	\int \frac{dx}{(x - a)^n} =
	\int (x - a)^{-n}\,d(x - a) =
	\frac{(x - a)^{1-n}}{1 - n} + C, \ n \neq 1
	\end{equation*}
	\end{proof}
	
	\item $\displaystyle \int \frac{x + a}{(x - b)^2 + c^2}\,dx =
	\frac12 \ln ((x - b)^2 + c^2) + \frac{a + b}c \arctg \frac{x - b}c + C$
	\begin{proof}
	\begin{equation*}
	\int \frac{x + a}{(x - b)^2 + c^2}\,dx \;
	\left| \text{Пусть } t = x - b \Rightarrow dt = dx \right| =
	\int \frac{t + b + a}{t^2 + c^2}\,dt =
	\end{equation*}
	\begin{equation*}
	= \int \frac{t}{t^2 + c^2}\,dt + (a + b) \int \frac{dt}{t^2 + c^2} =
	\frac12 \ln ((x - b)^2 + c^2) + \frac{a + b}c \arctg \frac{x - b}c + C
	\end{equation*}
	\end{proof}
	
	\item Интеграл~$\displaystyle \int \frac{x + a}{((x - b)^2 + c^2)^n}\,dx$ при~$n \neq 1$ нельзя взять непосредственно.
	
	Пусть $\displaystyle I_n = \int \frac{dx}{((x - b)^2 + c^2)^n}$, тогда
	\begin{equation*}
	\int \frac{x + a}{((x - b)^2 + c^2)^n}\,dx =
	(a + b)I_n - \frac1{2(n - 1)((x - b)^2 + c^2)^{n-1}}
	\end{equation*}
	\begin{equation*}
	I_n = \left( 1 + \frac1{2(n - 1)} \right) \frac{I_{n-1}}{c^2} + \frac{x - b}{2(n - 1)c^2 ((x - b)^2 + c^2)^{n-1}}
	\end{equation*}
	\begin{proof}
	\begin{equation}
	\label{eq:int_of_simple_frac_proof1}
	\begin{gathered}
	\int \frac{x + a}{((x - b)^2 + c^2)^n}\,dx \;
	\left| \text{Пусть } t = x - b \Rightarrow dt = dx \right| =
	\int \frac{t + b + a}{(t^2 + c^2)^n}\,dt = \\
	\int \frac{t}{(t^2 + c^2)^n}\,dt + (a + b) \int \frac{dt}{(t^2 + c^2)^n} =
	\frac12 \int (t^2 + c^2)^{-n}\,d(t^2 + c^2) + (a + b) \int \frac{dt}{(t^2 + c^2)^n} = \\
	\left| \text{Пусть } I_n = \int \frac{dt}{(t^2 + c^2)^n} \right| =
	\frac{(t^2 + c^2)^{1-n}}{2(1 - n)} + (a + b)I_n
	\end{gathered}
	\end{equation}
	
	Найдём $I_n$:
	\begin{equation*}
	I_n = \frac1{c^2} \int \frac{(t^2 + c^2) - t^2}{(t^2 + c^2)^n}\,dt =
	\frac{I_{n-1}}{c^2} - \frac1{c^2} \int \frac{t^2}{(t^2 + c^2)^n}\,dt
	\end{equation*}
	
	Найдём $\displaystyle \int \frac{t^2}{(t^2 + c^2)^n}\,dt$:
	\begin{equation*}
	\int \frac{t^2}{(t^2 + c^2)^n}\,dt =
	\left| \text{Пусть } u = t, v' = \frac{t}{(t^2 + c^2)^n} \right|
	\end{equation*}
	\begin{equation*}
	= \frac{t(t^2 + c^2)^{1-n}}{2(1 - n)} - \int \frac{(t^2 + c^2)^{1-n}}{2(1 - n)}\,dt =
	\frac{t}{2(1 - n)(t^2 + c^2)^{n-1}} - \frac{I_{n-1}}{2(1 - n)}
	\end{equation*}
	
	Тогда
	\begin{equation*}
	I_n = \frac{I_{n-1}}{c^2} + \frac1{2(n - 1)c^2} \left( \frac{t}{(t^2 + c^2)^{n-1}} - I_{n-1} \right) =
	\left( 1 + \frac1{2(n - 1)} \right) \frac{I_{n-1}}{c^2} + \frac{t}{2(n - 1)c^2 (t^2 + c^2)^{n-1}}
	\end{equation*}
	
	Получим рекуррентную формулу:
	\begin{equation*}
	I_n = \left( 1 + \frac1{2(n - 1)} \right) \frac{I_{n-1}}{c^2} + \frac{x - b}{2(n - 1)c^2 ((x - b)^2 + c^2)^{n-1}}
	\end{equation*}
	
	Используя (\ref*{eq:int_of_simple_frac_proof1}), получим конечную формулу:
	\begin{equation*}
	\int \frac{x + a}{((x - b)^2 + c^2)^n}\,dx =
	\frac{(t^2 + c^2)^{1-n}}{2(1 - n)} + (a + b)I_n =
	(a + b)I_n - \frac1{2(n - 1)((x - b)^2 + c^2)^{n-1}}
	\end{equation*}
	\end{proof}
\end{itemize}

\subsection{Интегрирование дробно-рациональных выражений}
Пусть $P_n(x)$ и $Q_m(x)$~--- многочлены $n$-й и $m$-й степеней соответственно, $n < m$.
По теореме~\ref{th:polynomial_factorization} $Q_m(x)$ можно разложить на множители
\begin{equation*}
Q_m(x) = \prod_{i=1}^{p} (x - a_i)^{\alpha_i} \cdot \prod_{i=1}^{q} ((x - b_i)^2 + c_i^2)^{\beta_i}
\end{equation*}

Тогда дробь~$\dfrac{P_n(x)}{Q_m(x)}$ может быть представлена в~виде суммы простейших дробей
\begin{equation*}
\frac{P_n(x)}{Q_m(x)} = \sum_{i=1}^p \sum_{j=1}^{\alpha_i} \frac{A_{ij}}{(x - a_i)^j} + \sum_{i=1}^q \sum_{j=1}^{\beta_q} \frac{B_{ij}x + C_{ij}}{((x - b_i)^2 + c_i^2)^j}
\end{equation*}

Т.\,о., интегрирование дробно-рациональных выражений сводится к интегрированию простейших дробей и, в~случае $n \geqslant m$, многочленов от переменной~$x$.

\subsection{Интегрирование тригонометрических выражений}
Пусть $R(x_1, \ldots, x_n) = \dfrac{P(x_1, \ldots, x_n)}{Q(x_1, \ldots, x_n)}$, где $P(\overline x)$ и $Q(\overline x)$~--- многочлены.
\begin{itemize}
	\item $\displaystyle \int R(\sin x) \cos^{2k+1} x\,dx = \int R(\sin x) (1 - \sin^2 x)^k\,d(\sin x)$, $k \in \mathbb Z$
	
	\item $\displaystyle \int R(\cos x) \sin^{2k+1} x\,dx = -\int R(\cos x) (1 - \cos^2 x)^k\,d(\cos x)$, $k \in \mathbb Z$
	
	\item $\displaystyle \int R(\sin^2 x, \cos^2 x)\,dx = \int \frac
	{R\left( \frac{\tg^2 x}{1 + \tg^2 x}, \frac1{1 + \tg^2 x} \right)}
	{1 + \tg^2 x}\,d(\tg x)$
	
	\item $\displaystyle \int R(\tg x)\,dx = \int \frac{R(\tg x)}{1 + \tg^2 x}\,d(\tg x)$
	
	\item $\displaystyle \int R(\sin x, \cos x)\,dx = \int \frac
	{2R\left( \frac{2\tg \frac{x}2}{1 + \tg^2 \frac{x}2}, \frac{1 - \tg^2 \frac{x}2}{1 + \tg^2 \frac{x}2} \right)}
	{1 + \tg^2 \frac{x}2}\,d(\tg \frac{x}2)$
\end{itemize}

Т.\,о., интегрирование тригонометрических выражений сводится к интегрированию рациональных дробей.

\subsection{Интегрирование квадратичных иррациональностей}
Пусть $R(x_1, \ldots, x_n) = \dfrac{P(x_1, \ldots, x_n)}{Q(x_1, \ldots, x_n)}$, где $P(\overline x)$ и $Q(\overline x)$~--- многочлены.
\begin{equation*}
\int R(x, \sqrt{ax^2 + bx + c})\,dx =
\int R\left( x, \sqrt{a \left( x + \frac{b}{2a} \right)^2 + c - \frac{b^2}{4a}} \right) dx =
\end{equation*}
\begin{equation*}
\left| \text{Пусть } y = x + \frac{b}{2a}, \ z = c - \frac{b^2}{4a} \right| =
\int R\left( y - \frac{b}{2a}, \sqrt{ay^2 + z} \right) dy
\end{equation*}

Пусть $\alpha = \sqrt{|a|}$, $\beta = \sqrt{|z|}$.
Возможны три случая:
\begin{itemize}
	\item Если $a > 0$, $z > 0$
	\begin{equation*}
	\int R\left( y - \frac{b}{2a}, \sqrt{\alpha^2 y^2 + \beta^2} \right) dy =
	\left| \text{Пусть } y = \frac\beta\alpha \tg t \right|
	\end{equation*}
	\begin{equation*}
	= \frac\beta\alpha \int \frac
	{R\left( \frac\beta\alpha \tg t - \frac{b}{2a}, \beta \sqrt{\tg^2 t + 1} \right)}
	{\cos^2 t}\,dt =
	\frac\beta\alpha \int \frac
	{R\left( \frac{\beta \sin t}{\alpha \cos t} - \frac{b}{2a}, \frac\beta{\cos t} \right)}
	{\cos^2 t}\,dt
	\end{equation*}
	
	\item Если $a > 0$, $z < 0$
	\begin{equation*}
	\int R\left( y - \frac{b}{2a}, \sqrt{\alpha^2 y^2 - \beta^2} \right)\,dy =
	\left| \text{Пусть } y = \frac\beta{\alpha \sin t} \right|
	\end{equation*}
	\begin{equation*}
	= -\frac\beta\alpha \int \frac
	{R\left( \frac\beta{\alpha \sin t} - \frac{b}{2a}, \beta \sqrt{\frac1{\sin^2 t} - 1} \right) \cos t}
	{\sin^2 t}\,dt =
	-\frac\beta\alpha \int \frac
	{R\left( \frac\beta{\alpha \sin t} - \frac{b}{2a}, \frac{\beta \cos t}{\sin t} \right) \cos t}
	{\sin^2 t}\,dt
	\end{equation*}
	
	\item Если $a < 0$, $z > 0$
	\begin{equation*}
	\int R\left( y - \frac{b}{2a}, \sqrt{-\alpha^2 y^2 + \beta^2} \right) dy =
	\left| \text{Пусть } y = \frac\beta\alpha \sin t \right|
	\end{equation*}
	\begin{equation*}
	= \frac\beta\alpha \int R\left( \frac\beta\alpha \sin t - \frac{b}{2a}, \beta \sqrt{1 - \sin^2 t} \right) \cos t\,dt =
	\frac\beta\alpha \int R\left( \frac\beta\alpha \sin t - \frac{b}{2a}, \beta \cos t \right) \cos t\,dt
	\end{equation*}
\end{itemize}

Т.\,о., интегрирование квадратичных иррациональностей сводится к интегрированию тригонометрических выражений.