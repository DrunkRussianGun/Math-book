\section{Обыкновенные дифференциальные уравнения}
\index{Уравнения!дифференциальные!обыкновенные} \textbf{Обыкновенным дифференциальным уравнением} называется уравнение вида
\begin{equation*}
F(x, y(x), y'(x), y''(x), \ldots, y^{(n)}(x)) = 0
\end{equation*}

Число~$n$ называется \textbf{порядком дифференциального уравнения}.

Функция, удовлетворяющая дифференциальному уравнению, называется его \textbf{частным решением}.
Множество всех частных решений называется \textbf{общим решением}.

Задача нахождения решения дифференциального уравнения, удовлетворяющего \textbf{начальным условиям} $y(x_0) \opbr= y_0$, $y'(x_0) \opbr= y_1$, \ldots, $y^{(n-1)}(x_0) \opbr= y_{n-1}$, называется \textbf{задачей Коши}.

\subsection{Уравнения с разделяющимися переменными}
\index{Уравнения!дифференциальные!с разделяющимися переменными} \textbf{Дифференциальным уравнением с разделяющимися переменными} называется уравнение вида
\begin{equation*}
y' \opbr= f(x) \cdot g(y)
\end{equation*}

Решим его:
\begin{equation*}
y' = f(x)g(y) \Leftrightarrow
\frac{dy}{dx} = f(x)g(y) \Leftrightarrow
\frac{dy}{g(y)} = f(x)\,dx \Rightarrow
\int \frac{dy}{g(y)} = \int f(x)\,dx
\end{equation*}

\subsection{Однородные уравнения первого порядка}
\index{Уравнения!дифференциальные!однородные} \textbf{Однородным дифференциальным уравнением первого порядка} называется уравнение вида
\begin{equation*}
y' = F \left( \frac{y}x \right)
\end{equation*}

Пусть $p(x) = \frac{y}x$, тогда $y = xp \Rightarrow y' = p + xp'$.
Получим уравнение с разделяющимися переменными:
\begin{equation*}
y' = F \left( \frac{y}x \right) \Leftrightarrow
x p' = F(p) - p \Leftrightarrow
p' = \frac{F(p) - p}x
\end{equation*}

Решив его, найдём $p$, а затем $y$.

\subsection{Логистические уравнения}
\index{Уравнения!логистические} \index{Уравнения!Ферхюльста} Уравнение, характеризующее рост численности некоторой популяции (например, рыб в водоёме) при отсутствии внешнего влияния, называется \textbf{логистическим}, или \textbf{уравнением Ферхюльста}, и имеет вид~$y' = k \bigl( 1 - \frac{y}r \bigr) y$, где $y(t)$~--- зависимость численности популяции от времени, $k$~--- скорость роста численности, $r$~--- максимально возможная численность.

Решая его, получим
\begin{equation*}
y' = k(1 - \frac{y}r)y \Leftrightarrow
\frac{r\,dy}{(r - y)y} = k\,dt \Leftrightarrow
\frac1{r - y}\,dy + \frac1y\,dy = k\,dt \Rightarrow
\ln \left| \frac{y}{r - y} \right| = kt + \ln C \Rightarrow
\left| \frac{r}{r - y} - 1 \right| = C e^{kt}
\end{equation*}

Проанализируем результат:
\begin{equation*}
\lim_{t \to +\infty} \left| \frac{r}{r - y} - 1 \right| = +\infty \Rightarrow
\lim_{t \to +\infty} \frac1{r - y} = \pm\infty \Rightarrow
\lim_{t \to +\infty} y = r
\end{equation*}

Учитывая, что $y' = k \bigl( 1 - \frac{y}r \bigr) y$, видим, что $y$ возрастает к~$r$ при $y(0) < r$ и убывает к~$r$ при $y(0) > r$.

\subsubsection{Логистические уравнения с квотой отлова}
Пусть численность популяции уменьшается под воздействием внешних факторов на фиксированную величину, называемую \textbf{квотой отлова} (например, определённое число рыб вылавливается из водоёма).
Тогда уравнение примет вид~$y' = k \bigl( 1 - \frac{y}r \bigr) y - b$, где $b$~--- квота отлова.

Преобразуя его, получим
\begin{equation*}
y' = k \left( 1 - \frac{y}r \right) y - b \Leftrightarrow
-ry' = ky^2 - kry + rb \Rightarrow
-\int \frac{r\,dy}{ky^2 - kry + rb} = t \Leftrightarrow
-\int \frac{r\,dy}{\left( y - \frac{r}2 \right)^2 + r\left( \frac{b}k - \frac{r}4 \right)} = kt
\end{equation*}

Рассмотрим следующие случаи:
\begin{enumerate}
	\item Пусть $\frac{b}k - \frac{r}4 < 0$, тогда
	$\left( y - \frac{r}2 \right)^2 + r\left( \frac{b}k - \frac{r}4 \right) = (y - y_1)(y - y_2)$, где
	$y_1 = \frac{r}2 - \sqrt{\frac{r(r - 4b)}{4k}}$,
	$y_2 = \frac{r}2 + \sqrt{\frac{r(r - 4b)}{4k}}$.
	
	Получим
	\begin{equation*}
	-\int \frac{r\,dy}{(y - y_1)(y - y_2)} = kt \Leftrightarrow
	\frac{r}{y_2 - y_1} \left( \int \frac{dy}{y - y_1} - \int \frac{dy}{y - y_2} \right) = kt \Rightarrow
	\frac{r}{y_2 - y_1} \ln \left| \frac{y - y_1}{y - y_2} \right| = kt + C
	\end{equation*}
	
	Проанализируем результат:
	\begin{equation*}
	\lim_{t \to +\infty} \ln \left| \frac{y - y_1}{y - y_2} \right| = +\infty \Rightarrow
	\lim_{t \to +\infty} \left| \frac{y - y_1}{y - y_2} \right| = +\infty \Rightarrow
	\lim_{t \to +\infty} y = y_2
	\end{equation*}
	
	Дифференцируя, получим
	\begin{equation*}
	-\frac{ry'}{(y - y_1)(y - y_2)} = k
	\end{equation*}
	
	Значит,
	\begin{itemize}
		\item $y$ бесконечно убывает при $y(0) < y_1$;
		\item $y$ возрастает к~$y_2$ при $y_1 < y(0) < y_2$;
		\item $y$ убывает к~$y_2$ при $y(0) > y_2$.
	\end{itemize}
	
	\item Пусть $\frac{b}k - \frac{r}4 = 0$, тогда
	\begin{equation*}
	-\int \frac{r\,dy}{\left( y - \frac{r}2 \right)^2} = kt \Rightarrow
	\frac{r}{y - \frac{r}2} = kt + C \Rightarrow
	y = \frac{r}{kt + C} + \frac{r}2 \Rightarrow
	\lim_{t \to +\infty} y = \frac{r}2 \lAnd y' < 0
	\end{equation*}
	
	Значит,
	\begin{itemize}
		\item $y$ бесконечно убывает при $y(0) < \frac{r}2$;
		\item $y$ убывает к~$\frac{r}2$ при $y(0) > \frac{r}2$.
	\end{itemize}
	
	\item Пусть $\frac{b}k - \frac{r}4 > 0$, тогда
	\begin{equation*}
	-\int \frac{r\,dy}{\left( y - \frac{r}2 \right)^2 + \left( \sqrt\frac{r(4b - rk)}{4k} \right)^2} = kt \Rightarrow
	-\sqrt\frac{rk}{4b - rk} \arctg (2y - r)\sqrt\frac{k}{r(4b - rk)} = kt + C \Rightarrow
	\end{equation*}
	\begin{equation*}
	\Rightarrow (2y - r)\sqrt\frac{k}{r(4b - rk)} = -\tg (kt + C) \sqrt{\frac{4b}{rk} - 1} \Leftrightarrow
	y = \frac{r}2 - \sqrt{\frac{4br}{k} - r^2}\tg (kt + C) \sqrt{\frac{4b}{rk} - 1}
	\end{equation*}
	
	Значит, $y$ бесконечно убывает вне зависимости от начальных условий.
\end{enumerate}

\subsection{Уравнения в полных дифференциалах}
\index{Уравнения!дифференциальные!в полных дифференциалах} \textbf{Уравнением в полных дифференциалах} называется уравнение вида
\begin{equation*}
y' = -\frac{M(x, y)}{N(x, y)}
\end{equation*}

Если $M_y' = N_x'$, то $M(x, y) = F_x'$, $N(x, y) = F_y'$, тогда
\begin{equation*}
y' = -\frac{M(x, y)}{N(x, y)} \Leftrightarrow
M(x, y)\,dx + N(x, y)\,dy = 0 \Leftrightarrow
dF(x, y) = 0 \Leftrightarrow
F(x, y) = C
\end{equation*}

Пусть $M_y' \neq N_x'$, но $\exists \mu(x, y) \colon (\mu \cdot M)_y' = (\mu \cdot N)_x'$, тогда
\begin{equation*}
M(x, y)\,dx + N(x, y)\,dy = 0 \Leftrightarrow
\mu(x, y) M(x, y)\,dx + \mu(x, y) N(x, y)\,dy = 0 \Leftrightarrow
dF(x, y) = 0 \Leftrightarrow
F(x, y) = C
\end{equation*}

\index{Множитель!интегрирующий} $\mu(x, y)$ называется \textbf{интегрирующим множителем}.

\subsection{Линейное уравнение первого порядка}
\index{Уравнения!дифференциальные!линейные} \textbf{Линейным дифференциальным уравнением первого порядка} называется уравнение вида
\begin{equation*}
y' = a(x) y + b(x)
\end{equation*}

\index{Метод!вариации произвольной постоянной} Решим его \textbf{методом вариации произвольной постоянной}.
\begin{enumerate}
	\item Решим уравнение
	\begin{equation*}
	y_0' = a(x) y_0 \Leftrightarrow
	\frac{dy_0}{y_0} = a(x)\,dx \Rightarrow
	\ln y_0 = \varphi(x) + \ln C \Rightarrow
	y_0 = C e^{\varphi(x)}
	\end{equation*}
	
	\item Подставим $y = C(x) e^{\varphi(x)}$ в исходное уравнение:
	\begin{equation*}
	C'(x) e^{\varphi(x)} + C(x) e^{\varphi(x)} \varphi'(x) = a(x) C(x) e^{\varphi(x)} + b(x)
	\end{equation*}
	
	$y_0 = Ce^{\varphi(x)} \Rightarrow C e^{\varphi(x)} \varphi'(x) = a(x) C e^{\varphi(x)}$, тогда получим
	\begin{equation*}
	C'(x) e^{\varphi(x)} = b(x) \Leftrightarrow
	C'(x) = b(x) e^{-\varphi(x)} \Leftrightarrow
	C(x) = \int b(x) e^{-\varphi(x)}\,dx
	\end{equation*}
\end{enumerate}

Тогда $y = C(x) e^{\varphi(x)}$.

\subsection{Уравнение Бернулли}
\index{Уравнения!Бернулли} \textbf{Уравнением Бернулли} называется уравнение вида
\begin{equation*}
y' = a(x) y + b(x) y^n, \ n \neq 1
\end{equation*}

Пусть $\frac1{y^{n-1}} = z$, тогда
\begin{equation*}
y' = a(x) y + b(x) y^n \Leftrightarrow
\frac{y'}{y^n} = a(x) y^{1-n} + b(x) \Leftrightarrow
\frac{z'}{1 - n} = a(x) z + b(x)
\end{equation*}

Т.\,о., решение уравнения Бернулли сводится к решению линейного уравнения.
Решив его, найдём $z$, а затем $y$.