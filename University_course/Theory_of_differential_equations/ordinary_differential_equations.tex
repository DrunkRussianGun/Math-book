\section{Обыкновенные дифференциальные уравнения}
\index{Уравнения!дифференциальные!обыкновенные} \textbf{Обыкновенным дифференциальным уравнением} называется уравнение вида
\begin{equation*}
F(x, y(x), y'(x), y''(x), \ldots, y^{(n)}(x)) = 0
\end{equation*}

Число~$n$ называется \textbf{порядком дифференциального уравнения}.

Функция, удовлетворяющая дифференциальному уравнению, называется его \textbf{частным решением}.
Множество всех частных решений называется \textbf{общим решением}.

Задача нахождения решения дифференциального уравнения, удовлетворяющего \textbf{начальным условиям} $y(x_0) \opbr= y_0$, $y'(x_0) \opbr= y_1$, \ldots, $y^{(n-1)}(x_0) \opbr= y_{n-1}$, называется \textbf{задачей Коши}.

\subsection{Уравнения с разделяющимися переменными}
\index{Уравнения!дифференциальные!с разделяющимися переменными} \textbf{Дифференциальным уравнением с разделяющимися переменными} называется уравнение вида
\begin{equation*}
y' \opbr= f(x) \cdot g(y)
\end{equation*}

Решим его:
\begin{equation*}
y' = f(x)g(y) \Leftrightarrow
\frac{dy}{dx} = f(x)g(y) \Leftrightarrow
\frac{dy}{g(y)} = f(x)\,dx \Rightarrow
\int \frac{dy}{g(y)} = \int f(x)\,dx
\end{equation*}

\subsection{Однородное уравнение первого порядка}
\index{Уравнения!дифференциальные!однородные} \textbf{Однородным дифференциальным уравнением первого порядка} называется уравнение вида
\begin{equation*}
y' = F \left( \frac{y}x \right)
\end{equation*}

Пусть $p(x) = \frac{y}x$, тогда $y = xp \Rightarrow y' = p + xp'$.
Получим уравнение с разделяющимися переменными:
\begin{equation*}
y' = F \left( \frac{y}x \right) \Leftrightarrow
x p' = F(p) - p \Leftrightarrow
p' = \frac{F(p) - p}x
\end{equation*}

Решив его, найдём $p$, а затем $y$.