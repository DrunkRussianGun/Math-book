\section{Приближённое решение дифференциальных уравнений}
Рассмотрим методы приближённого решения уравнения~$y^{(n)} = F(x, y, y', \ldots, y^{(n-1)})$ с начальными условиями $y(x_0) = y_0$, $y'(x_0) = y_1$, \ldots, $y^{(n-1)}(x_0) = y_{n-1}$.

\subsection{Решение с помощью степенного ряда}
Найдём решение уравнения в окрестности точки~$x_0$:
\begin{equation*}
y(x) = y_0 + y_1(x - x_0) + \frac{y_2}{2!} (x - x_0)^2 + \ldots + \frac{y^{(n-1)}}{(n - 1)!} (x - x_0)^{n-1} + \frac{F(x_0, y_0, y_1, \ldots, y_{n-1})}{n!} (x - x_0)^n + c_{n+1} (x - x_0)^{n+1} + \ldots
\end{equation*}

Неизвестные коэффициенты можно определить подстановкой в исходное уравнение или его дифференцированием и подстановкой начальных условий.

\subsection{Метод Эйлера}
\index{Метод!Эйлера} Найдём решение уравнения первого порядка на отрезке~$[x_0; x_0 + b]$.
Разделим отрезок на $n$ равных частей длиной по~$\Delta$ каждая и обозначим $x_k = x_0 + \Delta k$.
Тогда, введя на каждом отрезке~$[x_k; x_{k+1}]$ функцию $y_k(x) = y_k + f(x_k, y_k) \* (x - x_k)$, где $y_k = y_{k-1} + f(x_{k-1}, y_{k-1}) \Delta$, получим приближённое решение.

Можно улучшить точность, если брать значения
\begin{equation*}
y_k = y_{k-1} + f \left(x_{k-1} + \frac\Delta2, y_{k-1} + f(x_{k-1}, y_{k-1}) \frac\Delta2\right) \Delta
\end{equation*}

\index{Метод!Рунге---Кутта} Метод Рунге"--~Кутта ещё больше увеличивает точность вычислений, приближая решение не прямыми, а параболами.

Отметим, что уравнения высших порядков можно свести к системе линейных уравнений.

\subsection{Графический метод}
\index{Метод!графический} Графический метод применим к уравнениям вида~$y' = f(x, y)$.
\index{Изоклина} Он заключается в нахождении \textbf{изоклин}~--- линий, в точках которых производная~$y'$ остаётся постоянной, и построении решения по ним.
На рисунке ниже синим цветом изображены изоклины уравнения $y' = y$, красным~--- одна из его интегральных кривых.

\begin{center}\noindent
\begin{tikzpicture}[scale=1.5]
\drawaxis{-0.5}{4}{-2}{2};

% рисуем траектории
\foreach \delta in {-2.25, -1.95, ..., 4}
	\draw[loosely dashed] (\delta, 0.1) ..
		controls (0.45 + \delta, 0.27) and (1.2 + \delta, 0.7) ..
		(1.75 + \delta, 2)
		(\delta, -0.1) ..
		controls (0.45 + \delta, -0.27) and (1.2 + \delta, -0.7) ..
		(1.75 + \delta, -2);
%	\draw[loosely dashed, domain=\delta:2 + \delta] plot function{0.4 * exp(x - \delta) - 0.3}
%		plot function{-0.4 * exp(x - \delta) + 0.3};

% рисуем изоклины
\foreach \y in {0.49, 0.99, ..., 2}
	\draw[blue] (\left_x, -\y) -- (\right_x, -\y)
		(\left_x, \y) -- (\right_x, \y);
		
% рисуем интегральную кривую
\draw[red] (-0.5, 0.05) ..
	controls (0.15, 0.27) and (0.8, 0.7) ..
	(1.4, 2);
\end{tikzpicture}
\end{center}