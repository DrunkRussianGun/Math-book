\section{Системы линейных дифференциальных уравнений}
\subsection{Однородные системы}
Однородная система линейных дифференциальных уравнений имеет вид
\begin{equation*}
\begin{cases}
y_1' = a_{11} y_1 + a_{12} y_2 + \ldots + a_{1n} y_n \\
y_2' = a_{21} y_1 + a_{22} y_2 + \ldots + a_{2n} y_n \\
\ldots \\
y_1' = a_{n1} y_1 + a_{n2} y_2 + \ldots + a_{nn} y_n \\
\end{cases}
\end{equation*}

Заметим, что можно ограничиться рассмотрением дифференциальных уравнений только первого порядка, т.\,к. уравнение $n$-го порядка можно свести к системе $n$ линейных уравнений.
Например, если в уравнении встречается производная~$y^{(n)}$, то можно ввести функцию~$z_{n-1} = y^{(n-1)}$ и заменить $y^{(n)} = z_{n-1}'$, затем заменить $y^{(n-1)} = z_{n-2}'$ и т.\,д.

Однородную систему можно заменить однородным уравнением высшего порядка.
Покажем это на примере системы из двух уравнений:
\begin{equation*}
\begin{cases}
y_1' = ay_1 + by_2 \\
y_2' = cy_1 + dy_2
\end{cases}
\Rightarrow
\begin{cases}
by_2' = bcy_1 + bdy_2 \\
by_2 = y_1' - ay_1 \\
by_2' = y_1'' - ay_1'
\end{cases}
\Rightarrow y_1'' - (a + d)y_1' + (ad - bc) y_1 = 0
\end{equation*}

Решив уравнение, найдём~$y_1$, а затем и~$y_2$.

Т.\,к. решение системы сводится к решению уравнения высшего порядка, можно представить её общее решение в виде $y_1(x) = M e^{kx}$, $y_2(x) = N e^{kx}$.
Тогда
\begin{equation}
\label{eq:homogeneous_system_after_substitution}
\begin{cases}
y_1' = ay_1 + by_2 \\
y_2' = cy_1 + dy_2
\end{cases}
\Leftrightarrow
\begin{cases}
M (a - k) e^{kx} + N b e^{kx} = 0 \\
M c e^{kx} + N (d - k) e^{kx} = 0
\end{cases}
\Leftrightarrow
\begin{cases}
(a - k) M + b N = 0 \\
c M + (d - k) N = 0
\end{cases}
\end{equation}

Чтобы получить нетривиальные решения, необходимо выполнение условия, называемого \textbf{характеристическим уравнением системы}:
\begin{equation*}
\begin{vmatrix}
a - k & b \\
c & d - k
\end{vmatrix} = 0
\end{equation*}

Решив его, найдём значения~$k$.
Подставляя их в систему~(\ref*{eq:homogeneous_system_after_substitution}), найдём соотношения между коэффициентами $M$ и $N$.

Решение в случае кратных и комплексных корней характеристического уравнения полностью аналогично решению однородного уравнения.

\subsection{Неоднородные системы}
Неднородная система линейных дифференциальных уравнений имеет вид
\begin{equation*}
\begin{cases}
y_1' = a_{11} y_1 + a_{12} y_2 + \ldots + a_{1n} y_n + f_1(x) \\
y_2' = a_{21} y_1 + a_{22} y_2 + \ldots + a_{2n} y_n + f_2(x) \\
\ldots \\
y_1' = a_{n1} y_1 + a_{n2} y_2 + \ldots + a_{nn} y_n + f_n(x) \\
\end{cases}
\end{equation*}

Как и в случае с неоднородным уравнением, общее решение неоднородной системы является суммой её частного решения и общего решения соответствующей однородной системы.

\subsubsection{Метод вариации произвольной постоянной}
\index{Метод!вариации произвольной постоянной} Рассмотрим способ решения методом вариации произвольной постоянной на примере неоднородной системы из двух уравнений:
\begin{equation*}
\begin{cases}
y_1' = ay_1 + by_2 + f_1(x) \\
y_2' = cy_1 + dy_2 + f_2(x)
\end{cases}
\end{equation*}

\begin{enumerate}
	\item Найдём решение $y_{10} = C_1 z_1 + C_2 z_2$, $y_{20} = M C_1 z_1 + N C_2 z_2$ соответствующей однородной системы, тогда решение исходной системы имеет вид $y_1 = C_1(x) z_1 + C_2(x) z_2$, $y_2 = M C_1(x) z_1 + N C_2(x) z_2$.
	
	Заметим, что верны равенства
	\begin{equation*}
	\begin{cases}
	C_1 z_1' + C_2 z_2' = a(C_1 z_1 + C_2 z_2) + b(M C_1 z_1 + N C_2 z_2) \\
	M C_1 z_1' + N C_2 z_2' = c(C_1 z_1 + C_2 z_2) + d(M C_1 z_1 + N C_2 z_2)
	\end{cases}
	\end{equation*}
	
	\item Подставив $y_1$ и $y_2$ в исходную систему и воспользовавшись полученными равенствами, получим
	\begin{equation*}
	\begin{cases}
	C_1'(x) z_1 + C_2'(x) z_2 = f_1(x) \\
	M C_1'(x) z_1 + N C_2'(x) z_2 = f_2(x) \\
	\end{cases}
	\end{equation*}
	
	Решая систему и интегрируя $C_1'(x)$ и $C_2'(x)$, получим $C_1(x)$ и $C_2(x)$.
\end{enumerate}