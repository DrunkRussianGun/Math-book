\section{Линейные дифференциальные уравнения с постоянными коэффициентами}
\index{Уравнения!дифференциальные!линейные} \textbf{Линейным дифференциальным уравнением $n$-го порядка с постоянными коэффициентами} называется уравнение вида
\begin{equation}
\label{eq:linear_differential_equation}
y^{(n)} + a_{n-1} y^{(n-1)} + \ldots + a_1 y' + a_0 y = f(x)
\end{equation}

Введём функции $y_0 = y$, $y_1 = y'$, \ldots, $y_{n-1} = y^{(n-1)}$, тогда
\begin{equation*}
y^{(n)} + a_{n-1} y^{(n-1)} + \ldots + a_1 y' + a_0 y = f(x) \Leftrightarrow
\begin{cases}
y_0' = y_1 \\
y_1' = y_2 \\
\ldots \\
y_{n-1}' = f(x) - a_0 y_0 - a_1 y_1 - \ldots - a_{n-1} y_{n-1}
\end{cases}
\end{equation*}

\index{Теорема!единственности}
\begin{theorem}[единственности решения дифференциального уравнения]
Если заданы начальные условия $y(x_0) = y_{00}$, $y'(x_0) = y_{10}$, \ldots, $y^{(n-1)}(x_0) = y_{n-1\,0}$, то уравнение~(\ref*{eq:linear_differential_equation}) имеет единственное решение в окрестности~$(x_0 - d; x_0 + d)$.
\end{theorem}
\begin{proof}
Введём вектор-функцию $Y(x) =
\begin{Vmatrix}
y_0 \\
y_1 \\
\ldots \\
y_{n-1}
\end{Vmatrix}$
и получим уравнение $Y' = A Y + F$, где
\begin{equation*}
A =
\begin{Vmatrix}
0 & 1 & 0 & \ldots & 0 \\
0 & 0 & 1 & \ldots & 0 \\
\vdots & \vdots & \vdots & \ddots & \vdots \\
-a_0 & -a_1 & -a_2 & \ldots & -a_{n-1}
\end{Vmatrix}, \
F =
\begin{Vmatrix}
0 \\
0 \\
\ldots \\
f(x)
\end{Vmatrix}
\end{equation*}

Аналогично теореме~\ref{th:differential_equation_has_single_solution} можно доказать, что единственное решение этого уравнения можно получить методом итераций:
\begin{equation*}
Y_k(x) = Y_0 + \int\limits_{x_0}^x (A Y_{k-1}(t) + F(t))\,dt
\end{equation*}
\end{proof}

\subsection{Однородное дифференциальное уравнение}
\index{Уравнения!дифференциальные!однородные} Если для уравнения~\ref{eq:linear_differential_equation} $f(x) = 0$, то оно называется \textbf{однородным} и принимает вид
\begin{equation}
\label{eq:homogeneous_differential_equation}
y^{(n)} + a_{n-1} y^{(n-1)} + \ldots + a_1 y' + a_0 y = 0
\end{equation}
иначе оно называется \textbf{неоднородным}.

\begin{statement}
Если $y_1$ и $y_2$~--- частные решения однородного дифференциального уравнения, то $\alpha y_1 \opbr+ \beta y_2$~--- также частное решение, где $\alpha, \beta \in \mathbb R$.
\end{statement}

Т.\,о., общее решение однородного уравнения имеет вид $y(x) = C_1 y_1 + \ldots + C_n y_n$.
Тогда начальные условия $y(x_0) = y_{00}$, $y'(x_0) = y_{10}$, \ldots, $y^{(n-1)}(x_0) = y_{n-1\,0}$ можно записать в виде системы уравнений относительно $C_1, \ldots, C_n$:
\begin{equation*}
\begin{cases}
C_1 y_1(x_0) + \ldots + C_n y_n(x_0) = y_{00} \\
C_1 y_1'(x_0) + \ldots + C_n y_n'(x_0) = y_{10} \\
\ldots \\
C_1 y_1^{(n-1)}(x_0) + \ldots + C_n y_n^{(n-1)}(x_0) = y_{n-1\,0}
\end{cases}
\end{equation*}

Чтобы получить решение задачи Коши, главный определитель этой системы должен быть отличен от нуля.

\index{Вронскиан} \index{Определитель!Вронского} \textbf{Определителем Вронского}, или \textbf{вронскианом}, называется определитель
\begin{equation*}
W(x) =
\begin{vmatrix}
y_1 & y_2 & \ldots & y_n \\
y_1' & y_2' & \ldots & y_n' \\
\vdots & \vdots & \ddots & \vdots \\
y_1^{(n-1)} & y_2^{(n-1)} & \ldots & y_n^{(n-1)}
\end{vmatrix}
\end{equation*}
где $y_1, \ldots, y_n$~--- частные решения уравнения~(\ref*{eq:homogeneous_differential_equation}).

Частные решения $y_1, y_2, \ldots, y_n$ уравнения~(\ref*{eq:homogeneous_differential_equation}) называются \textbf{линейно зависимыми}, если
\begin{equation*}
\exists \alpha_1, \ldots, \alpha_n \in \mathbb R \colon
\alpha_1 y_1 + \ldots + \alpha_n y_n = 0 \lAnd
\alpha_1^2 + \ldots + \alpha_n^2 \neq 0
\end{equation*}
иначе~--- \textbf{линейно независимыми}.

\begin{statement}
Если частные решения~$y_1, y_2, \ldots, y_n$ однородного дифференциального уравнения линейно независимы, то соответствующий определитель Вронского~$W \neq 0$.
\end{statement}
\begin{proofcontra}
Пусть $W(x_0) = 0$, тогда система
\begin{equation*}
\begin{cases}
C_1 y_1(x_0) + \ldots + C_n y_n(x_0) = 0 \\
C_1 y_1'(x_0) + \ldots + C_n y_n'(x_0) = 0 \\
\ldots \\
C_1 y_1^{(n-1)}(x_0) + \ldots + C_n y_n^{(n-1)}(x_0) = 0
\end{cases}
\end{equation*}
имеет нетривиальное решение данного уравнения, удовлетворяющее начальным условиям $y(x_0) = 0$, \ldots, $y^{(n-1)}(x_0) \opbr= 0$.
Но при этом существует и тривиальное решение, что противоречит теореме единственности.
\end{proofcontra}

\index{Уравнения!характеристические} Найдём частные решения уравнения~(\ref*{eq:homogeneous_differential_equation}).
Подставив $y(x) = e^{kx}$, получим \textbf{характеристическое уравнение}:
\begin{equation*}
(k^n + a_{n-1} k^{n-1} + \ldots + a_1 k + a_0) e^{kx} = 0 \Leftrightarrow
k^n + a_{n-1} k^{n-1} + \ldots + a_1 k + a_0 = 0
\end{equation*}

Его корням~$k_1, \ldots, k_n$ сопоставим линейно независимые частные решения~$y_1, y_2, \ldots, y_n$, тогда общим решением будет
\begin{equation*}
y(x) = C_1 y_1 + C_2 y_2 + \ldots + C_n y_n
\end{equation*}

\begin{enumerate}
	\item \textbf{Простой вещественный корень}
	
	Простому вещественному корню~$k_1$ соответствует решение~$y(x) = e^{k_1 x}$.
	При этом определитель Вронского имеет вид
	\begin{equation*}
	\begin{vmatrix}
	e^{k_1 x} & \ldots & e^{k_n x} \\
	k_1 e^{k_1 x} & \ldots & k_n e^{k_n x} \\
	\vdots & \ddots & \vdots \\
	k_1^{n-1} e^{k_1 x} & \ldots & k_n^{n-1} e^{k_n x} \\
	\end{vmatrix}
	= e^{(k_1 + \ldots + k_n) x} \cdot
	\begin{vmatrix}
	1 & \ldots & 1 \\
	k_1 & \ldots & k_n \\
	\vdots & \ddots & \vdots \\
	k_1^{n-1} & \ldots & k_n^{n-1}
	\end{vmatrix} \neq 0
	\end{equation*}
	
	\item \textbf{Кратный вещественный корень}
	
	Вещественному корню~$k_1$ кратности~$m$ соответствуют решения~$y(x) = x^j e^{k_1 x}$, где $j = 0, \ldots, m - 1$.
	Для доказательства прежде всего нужно заметить, что $k_1$ является корнем также и для любого из многочленов~$(k^n + a_{n-1} k^{n-1} + \ldots + a_1 k + a_0)^{(j)}$.
	
	Докажем, что решение $y(x) = x e^{k_1 x}$ удовлетворяет уравнению.
	\begin{equation*}
	(k_1^n x + n k_1^{n-1} + a_{n-1} (k_1^{n-1} x + (n - 1) k_1^{n-2}) + \ldots + a_1 (k_1 x + 1) + a_0 x) e^{k_1 x} = 0 \Leftrightarrow
	\end{equation*}
	\begin{equation*}
	\Leftrightarrow (k_1^n + a_{n-1} k_1^{n-1} + \ldots + a_1 k_1 + a_0) x + (n k_1^{n-1} + a_{n-1} (n - 1) k_1^{n-2} + \ldots + a_1) = 0 \Leftrightarrow
	0 = 0
	\end{equation*}
	
	То же аналогично доказывается для решений~$y(x) = x^j e^{k_1 x}$, где $j = 2, \ldots, m - 1$.
	
	\item \textbf{Простые сопряжённые комплексные корни}
	
	Комплексным корням $\alpha - i \beta$ и $\alpha + i \beta$ соответствуют решения~$y_1(x) = e^{(\alpha - i \beta) x}$ и $y_2(x) = e^{(\alpha + i \beta) x}$.
	Однако, чтобы избежать комплексных чисел в решении, используют решения $y_1(x) = e^{\alpha x} \cos \beta x$ и $y_2(x) = e^{\alpha x} \sin \beta x$.
	
	Используя \hyperref[eq:Euler's_formula]{формулу Эйлера}, покажем, что $C_1 e^{(\alpha - i \beta) x} + C_2 e^{(\alpha + i \beta) x} = D_1 e^{\alpha x} \cos \beta x + D_2 e^{\alpha x} \sin \beta x$.
	\begin{equation*}
	C_1 e^{(\alpha - i \beta) x} + C_2 e^{(\alpha + i \beta) x} =
	e^{\alpha x} (C_1 e^{i \beta x} + C_2 e^{-i \beta x}) =
	\end{equation*}
	\begin{equation*}
	= e^{\alpha x} ((C_1 + C_2) \cos \beta x + i(C_1 - C_2) \sin \beta x) =
	D_1 e^{\alpha x} \cos \beta x + D_2 e^{\alpha x} \sin \beta x
	\end{equation*}
	
	\item \textbf{Кратные комплексные корни}
	
	Комплексным корням $\alpha - i \beta$ и $\alpha + i \beta$ кратности~$m$ соответствуют решения $y_1(x) = x^j e^{\alpha x} \cos \beta x$ и $y_2(x) \opbr= x^j e^{\alpha x} \sin \beta x$, где $j = 0, \ldots, m - 1$.
\end{enumerate}

\subsection{Неоднородное дифференциальное уравнение}
\index{Уравнения!дифференциальные!неоднородные}
Рассмотрим уравнение
\begin{equation}
\label{eq:heterogeneous_differential_equation}
y^{(n)} + a_{n-1} y^{(n-1)} + \ldots + a_1 y' + a_0 y = f(x)
\end{equation}

Ему соответствует однородное уравнение $y^{(n)} + a_{n-1} y^{(n-1)} + \ldots + a_1 y' + a_0 y = 0$.

\begin{statement}
Общее решение уравнения~(\ref*{eq:heterogeneous_differential_equation}) равно сумме его частного решения и общего решения соответствующего однородного уравнения.
\end{statement}

\subsubsection{Метод вариации произвольной постоянной}
\index{Метод!вариации произвольной постоянной}
\begin{enumerate}
	\item Найдём решение $y_0(x) = C_1 y_1 + \ldots + C_n y_n$ соответствующего однородного уравнения, тогда решением исходного уравнения будет $y(x) = C_1(x) y_1 + \ldots + C_n(x) y_n$.
	
	\item Найдём $C_1(x), \ldots, C_n(x)$, решая систему
	\begin{equation*}
	\begin{cases}
	C_1'(x) y_1 + \ldots + C_n'(x) y_n = 0 \\
	C_1'(x) y_1' + \ldots + C_n'(x) y_n' = 0 \\
	\ldots \\
	C_1'(x) y_1^{(n-1)} + \ldots + C_n'(x) y_n^{(n-1)} = f(x) \\
	\end{cases}
	\end{equation*}
	и интегрируя полученные $C_1'(x), \ldots, C_n'(x)$.
\end{enumerate}
\begin{proof}[(для случая $n = 2$)]
Пусть дано уравнение $y'' + a_1 y' + a_0 = f(x)$ и $y_0(x) = C_1 y_1 + C_2 y_2$, тогда
\begin{equation*}
y(x) = C_1(x) y_1 + C_2(x) y_2 \Rightarrow
\end{equation*}
\begin{equation*}
\Rightarrow y'(x) = C_1'(x) y_1 + C_1(x) y_1' + C_2'(x) y_2 + C_2(x) y_2' \Rightarrow
\end{equation*}
\begin{equation*}
\Rightarrow y''(x) = C_1''(x) y_1 + 2 C_1'(x) y_1' + C_1(x) y_1'' +
C_2''(x) y_2 + 2 C_2'(x) y_2' + C_2(x) y_2''
\end{equation*}

Подставим в уравнение:
\begin{equation*}
C_1''(x) y_1 + 2 C_1'(x) y_1' + C_1(x) y_1'' +
C_2''(x) y_2 + 2 C_2'(x) y_2' + C_2(x) y_2'' + \vphantom{1}
\end{equation*}
\begin{equation*}
\vphantom{1} + a_1 (C_1'(x) y_1 + C_1(x) y_1' + C_2'(x) y_2 + C_2(x) y_2') +
a_0 (C_1(x) y_1 + C_2(x) y_2) = f(x) \Leftrightarrow
\end{equation*}
\begin{equation*}
\Leftrightarrow C_1(x) (y_1'' + a_1 y_1' + a_0 y_1) +
C_2(x) (y_2'' + a_1 y_2' + a_0 y_2) + \vphantom{1}
\end{equation*}
\begin{equation*}
\vphantom{1} + a_1 (C_1'(x) y_1 + C_2'(x) y_2) +
(C_1''(x) y_1 + 2 C_1'(x) y_1' + C_2''(x) y_2 + 2 C_2'(x) y_2') = f(x) \Leftrightarrow
\end{equation*}
\begin{equation*}
\left| \text{$y_1$ и $y_2$~--- частные решения однородного уравнения} \right|
\end{equation*}
\begin{equation*}
\Leftrightarrow a_1 (C_1'(x) y_1 + C_2'(x) y_2) +
(C_1''(x) y_1 + 2 C_1'(x) y_1' + C_2''(x) y_2 + 2 C_2'(x) y_2') = f(x)
\end{equation*}

Из системы
\begin{equation*}
\begin{cases}
C_1'(x) y_1 + C_2'(x) y_2 = 0 \\
C_1'(x) y_1' + C_2'(x) y_2' = f(x)
\end{cases}
\Rightarrow C_1''(x) y_1 + C_1'(x) y_1' + C_2''(x) y_2 + C_2'(x) y_2' = 0
\end{equation*}

Подставляя эти равенства в уравнение, получим $f(x) = f(x)$.
\end{proof}

\subsubsection{Метод неопределённых коэффициентов}
\index{Метод!неопределённых коэффициентов} Метод неопределённых коэффициентов применим в том случае, если
\begin{equation*}
f(x) = \sum_j e^{\alpha_j x} (M_j(x) \cos \beta_j x + N_j(x) \sin \beta_j x)
\end{equation*}
где $M_j(x), N_j(x)$~--- многочлены.
Тогда частное решение уравнения имеет вид
\begin{equation*}
\tilde y(x) = \sum_j x^{s_j} e^{\alpha_j x} (P_j(x) \cos \beta_j x + Q_j(x) \sin \beta_j x)
\end{equation*}
где $s_j$~--- кратность корня~$\alpha + i \beta$ характеристического уравнения, $P_j(x), Q_j(x)$~--- многочлены степени~$\max \{ \deg M_j(x), \allowbreak \deg N_j(x) \}$ с неопределёнными коэффициентами, которые находятся подстановкой~$\tilde y(x)$ в уравнение~(\ref*{eq:heterogeneous_differential_equation}).