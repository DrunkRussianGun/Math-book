\section{Предел последовательности}
\index{lim}\index{Предел!последовательности} \index{Сходимость} Число~$a$ называется \textbf{пределом последовательности~$(x_n)$}, если
\begin{equation*}
\forall \varepsilon > 0 \ \exists n_0 \in \mathbb N \colon \forall n > n_0 \ |x_n - a| < \varepsilon
\end{equation*}
и обозначается $\displaystyle \lim_{n \to \infty} x_n$.
Говорят, что последовательность~$(x_n)$ \textbf{сходится}, если $\displaystyle \exists\lim_{n \to \infty} x_n$, иначе говорят, что $(x_n)$ \textbf{расходится}.

Последовательность~$(x_n)$ называется \textbf{ограниченной}, или \textbf{ограниченной величиной}, если
$\exists a > 0 \colon \forall n \in \mathbb N \allowbreak |x_n| < a$.

\index{Бесконечно малая величина} \textbf{Бесконечно малой величиной} называется последовательность~$(x_n)$, если
\begin{equation*}
\forall \varepsilon > 0 \ \exists n_0 \in \mathbb N \colon \forall n > n_0 \ |x_n| < \varepsilon
\end{equation*}

Можно определить предел последовательности, используя понятие бесконечно малой величины.

Число~$a$ называется \textbf{пределом последовательности~$(x_n)$}, если $x_n = a + \alpha_n$, где $\alpha_n$~--- бесконечно малая величина.

Докажем эквивалентность этих определений.
\begin{proof}
\begin{enumerate}
	\item Пусть дана последовательность~$(x_n)$ такая, что
	\begin{equation*}
	\forall \varepsilon > 0 \ \exists n_0 \in \mathbb N \colon \forall n > n_0 \ |x_n - a| < \varepsilon
	\end{equation*}
	
	Докажем, что $x_n = a + \alpha_n$.
	В~самом деле, $\alpha_n = x_n - a$~--- бесконечно малая величина.
	Т.\,о., $x_n = a + \alpha_n$.
	
	\item Проведя те~же самые рассуждения в~обратную сторону, докажем обратное утверждение.
\end{enumerate}
\end{proof}

Также определяется бесконечный предел:
\begin{itemize}
	\item $\displaystyle \lim_{n \to \infty} x_n = \infty \Leftrightarrow
	\forall M > 0 \ \exists n_0 \in \mathbb N \colon \forall n > n_0 \ |x_n| > M$
	\item $\displaystyle \lim_{n \to \infty} x_n = +\infty \Leftrightarrow
	\forall M > 0 \ \exists n_0 \in \mathbb N \colon \forall n > n_0 \ x_n > M$
	\item $\displaystyle \lim_{n \to \infty} x_n = -\infty \Leftrightarrow
	\forall M > 0 \ \exists n_0 \in \mathbb N \colon \forall n > n_0 \ x_n < -M$
\end{itemize}

\subsection{Элементарные свойства пределов}
\begin{enumerate}
	\item Последовательность может иметь не более одного предела.
	\begin{proof}
	Пусть $a = \lim\limits_{n \to \infty} x_n$, $b = \lim\limits_{n \to \infty} x_n$.
	Тогда
	\begin{equation*}
	\forall \varepsilon > 0 \
	\left( \exists n_1 \in \mathbb N \colon \forall n > n_1 \ |x_n - a| < \frac\varepsilon2 \right) \lAnd
	\left( \exists n_2 \in \mathbb N \colon \forall n > n_2 \ |x_n - b| < \frac\varepsilon2 \right) \Rightarrow
	\end{equation*}
	\begin{equation*}
	\Rightarrow \forall n > \max \{ n_1, n_2 \} \ |a - b| = |a - x_n + x_n - b| \leqslant |a - x_n| + |x_n - b| < \varepsilon
	\end{equation*}
	
	Значит, $a = b$.
	\end{proof}
	
	\item \index{Теорема!о двух милиционерах}
	\begin{theorem}[о~двух милиционерах]
	Пусть $\forall n \in \mathbb N \ x_n \leqslant y_n \leqslant z_n$.
	Если $\lim\limits_{n \to \infty} x_n = \lim\limits_{n \to \infty} z_n = a$, то $\lim\limits_{n \to \infty} y_n = a$.
	\end{theorem}
	\begin{proof}
	\begin{equation*}
	\forall \varepsilon > 0 \
	\left( \exists n_1 \in \mathbb N \colon \forall n > n_1 \ |x_n - a| < \varepsilon \right) \lAnd
	\left( \exists n_2 \in \mathbb N \colon \forall n > n_2 \ |z_n - a| < \varepsilon \right) \Rightarrow
	\end{equation*}
	\begin{equation*}
	\Rightarrow \forall n > \max \{ n_1, n_2 \} \ a - \varepsilon < x_n \leqslant y_n \leqslant z_n < a + \varepsilon \Rightarrow |y_n - a| < \varepsilon
	\end{equation*}
	\end{proof}
	
	\item Если $\forall n \in \mathbb N \ x_n \geqslant 0 \lAnd \lim\limits_{n \to \infty} x_n = a$, то $a \geqslant 0$.
	\begin{proofcontra}
	Пусть $a < 0$.
	\begin{equation*}
	\exists n_0 \in \mathbb N \colon \forall n > n_0 \ |x_n - a| < -\frac{a}2 \Rightarrow
	\frac{a}2 < x_n - a < -\frac{a}2 \Rightarrow
	x_n < \frac{a}2 < 0
	\end{equation*}
	
	Противоречие.
	\end{proofcontra}
	
	\item Если $\lim\limits_{n \to \infty} x_n = a$, то $\lim\limits_{n \to \infty} |x_n| = |a|$.
	\begin{proof}
	\begin{equation*}
	\begin{cases}
	|a| - |x_n| \leqslant |a - x_n| \\
	|x_n| - |a| \leqslant |x_n - a|
	\end{cases}
	\Rightarrow \bigl| |x_n| - |a| \bigr| \leqslant |x_n - a|
	\end{equation*}
	
	Тогда
	\begin{equation*}
	\lim_{n \to \infty} x_n = a \Rightarrow
	\forall \varepsilon > 0 \ \exists n_0 \in \mathbb N \colon \forall n > n_0 \ \bigl| |x_n| - |a| \bigr| \leqslant |x_n - a| < \varepsilon \Rightarrow
	\lim_{n \to \infty} |x_n| = |a|
	\end{equation*}
	\end{proof}
	
	\item Если последовательность~$(x_n)$ сходится, то она ограничена.
	\begin{proof}
	Пусть $\lim\limits_{n \to \infty} x_n = a \Rightarrow \lim\limits_{n \to \infty} |x_n| = |a|$.
	Получим:
	\begin{equation*}
	\exists n_0 \in \mathbb N \colon \forall n > n_0 \ ||x_n| - |a|| < 1 \Rightarrow |x_n| < |a| + 1
	\end{equation*}
	
	Тогда $\forall n \in \mathbb N \ |x_n| < \max \{ |x_1| + 1, |x_2| + 1, \ldots, |x_{n_0}| + 1, |a| + 1 \}$.
	\end{proof}
	
	\item Если последовательности $(x_n)$ и $(y_n)$~--- ограниченная и бесконечно малая величины соответственно, то $z_n \opbr= x_n y_n$~--- бесконечно малая величина.
	\begin{proof}
	Пусть $a > 0 \colon \forall n \in \mathbb N \ |x_n| < a$, тогда
	\begin{equation*}
	\forall \varepsilon > 0 \ \exists n_0 \in \mathbb N \colon
	\forall n > n_0 \ |y_n| < \frac\varepsilon{a} \Rightarrow
	|x_n y_n| < \varepsilon \Rightarrow
	\lim_{n \to \infty} z_n = 0
	\end{equation*}
	\end{proof}
	
	\item Если последовательности $(x_n)$ и $(y_n)$~--- бесконечно малые величины, то $z_n = x_n + y_n$~--- тоже бесконечно малая величина.
	\begin{proof}
	\begin{equation*}
	\forall \varepsilon > 0 \
	\left( \exists n_1 \in \mathbb N \colon \forall n > n_1 \ |x_n| < \frac\varepsilon2 \right) \lAnd
	\left( \exists n_2 \in \mathbb N \colon \forall n > n_2 \ |y_n| < \frac\varepsilon2 \right) \Rightarrow
	\end{equation*}
	\begin{equation*}
	\Rightarrow \forall n > \max \{ n_1, n_2 \} \ |x_n + y_n| \leqslant |x_n| + |y_n| < \varepsilon \Rightarrow
	\lim_{n \to \infty} z_n = 0
	\end{equation*}
	\end{proof}
\end{enumerate}

\subsection{Арифметические свойства пределов}
Пусть даны сходящиеся последовательности~$x_n = a + \alpha_n$ и~$y_n = b + \beta_n$, где $(\alpha_n), (\beta_n)$~--- бесконечно малые величины.
\begin{enumerate}
	\item $\displaystyle \lim_{n \to \infty} (x_n + y_n) = a + b$
	\begin{proof}
	\begin{equation*}
	\lim_{n \to \infty} (x_n + y_n) = \lim_{n \to \infty} (a + b + \alpha_n + \beta_n) = a + b
	\end{equation*}
	\end{proof}
	
	\item $\displaystyle \lim_{n \to \infty} x_n y_n = ab$
	\begin{proof}
	\begin{equation*}
	\lim_{n \to \infty} x_n y_n = \lim_{n \to \infty} (ab + a\beta_n + b\alpha_n + \alpha_n \beta_n) = ab
	\end{equation*}
	\end{proof}
	
	\item Если $a \neq 0$, то $\lim\limits_{n \to \infty} \frac1{x_n} = \frac1a$
	\begin{proof}
	Покажем, что $\displaystyle \left| \frac1{x_n} - \frac1a \right| = \frac{|\alpha_n|}{|a||\alpha_n + a|}$~--- бесконечно малая величина.
	\begin{equation*}
	\exists n_0 \in \mathbb N \colon \forall n > n_0 \ |x_n - a| < \frac{|a|}2 \Rightarrow
	\end{equation*}
	\begin{equation*}
	\Rightarrow \forall n > n_0 \ |a| = |a - x_n + x_n| \leqslant |a - x_n| + |x_n| < \frac{|a|}2 + |x_n| \Rightarrow |x_n| > \frac{|a|}2 \Rightarrow \frac1{|x_n|} < \frac2{|a|}
	\end{equation*}
	
	Тогда $\dfrac1{|x_n|} = \dfrac1{|\alpha_n + a|}$~--- ограниченная величина, значит,
	$\displaystyle \frac{|\alpha_n|}{|a||\alpha_n + a|} = \left| \frac1{x_n} - \frac1a \right|$~--- бесконечно малая величина.
	Отсюда $\displaystyle \lim_{n \to \infty} \left( \frac1{x_n} - \frac1a \right) = 0 \Leftrightarrow \lim_{n \to \infty} \frac1{x_n} = \frac1a$.
	\end{proof}
	
	\item Если $b \neq 0$, то $\lim\limits_{n \to \infty} \frac{x_n}{y_n} = \frac{a}b$.
	\begin{proof}
	\begin{equation*}
	\lim_{n \to \infty} \frac{x_n}{y_n} =
	\lim_{n \to \infty} \left( x_n \cdot \frac1{y_n} \right) =
	a \cdot \frac1b = \frac{a}b
	\end{equation*}
	\end{proof}
\end{enumerate}

\subsection{Основные свойства пределов последовательностей}
\begin{enumerate}
	\item Из ограниченной последовательности можно выбрать сходящуюся подпоследовательность.
	\begin{proof}
	Пусть $A$~--- множество значений, принимаемых членами ограниченной последовательности~$(x_n)$.
	\begin{enumerate}
		\item Пусть $A$ конечно.
		Тогда бесконечное множество членов последовательности~$(x_n)$ принимает хотя~бы одно значение из~$A$, значит, подпоследовательность, состоящая из~них, сходится к этому значению.
		
		\item Пусть $A$ бесконечно, тогда оно ограничено, значит, по \hyperref[th:Weierstrass]{теореме Вейерштрасса} оно имеет предельную точку~$a$.
		В окрестности~$\breve U_1(a)$ содержится хотя~бы одна точка из множества~$A$, а соответствующее значение принимает член~$x_{n_1}$.
		
		Рассмотрим множество~$A_1$, полученное из~$A$ удалением значений, принимаемых членами $x_1, x_2, \ldots, x_{n_1}$.
		$A_1$ бесконечно и имеет предельную точку~$a$, поэтому в окрестности~$\breve U_\frac12(a)$ найдётся значение, принимаемое членом~$x_{n_2}$, причём $n_1 < n_2$.
		
		Рассмотрим множество~$A_2$, полученное из~$A_1$ удалением значений, принимаемых членами $x_{n_1 + 1}, x_{n_1 + 2}, \allowbreak \ldots, x_{n_2}$.
		$A_2$ бесконечно и имеет предельную точку~$a$, поэтому в окрестности~$\breve U_\frac13(a)$ найдётся значение, принимаемое членом~$x_{n_3}$, причём $n_2 < n_3$.
		
		Продолжая, получим последовательность~$(x_{n_k}) \colon |x_{n_k} - a| < \frac1k$. По следствию~\ref{conseq:small_rational_exists} $\lim\limits_{k \to \infty} x_{n_k} = a$.
	\end{enumerate}
	\end{proof}
	
	\item \label{st:monotonic_bounded_sequence} Монотонная ограниченная последовательность~$(x_n)$ сходится.
	\begin{proof}
	Для опредёленности предположим, что $\forall n \in \mathbb N \ x_n \leqslant x_{n+1}$.
	Последовательность ограничена, поэтому множество~$A$ её значений имеет супремум~$a = \sup A$.
	По утверждению~\ref{st:inequality_of_supremum}
	\begin{equation*}
	\forall \varepsilon > 0 \ \exists k \in \mathbb N \colon a - \varepsilon < x_k \leqslant a \Rightarrow
	\forall n > k \ a - \varepsilon < x_k \leqslant x_n \leqslant a \Rightarrow
	|x_n - a| < \varepsilon \Rightarrow \lim_{n \to \infty} x_n = a
	\end{equation*}
	\end{proof}
	
	\item \index{Лемма!о~вложенных отрезках}
	\begin{lemma}[о~вложенных отрезках]
	\label{lemma:about_nested_intervals}
	Пусть $(a_n), (b_n)$~--- последовательности концов последовательно вложенных друг в~друга отрезков (т.\,е. $[a_n; b_n] \subset [a_{n-1}; b_{n-1}]$), причём $\lim\limits_{n \to \infty} (b_n - a_n) = 0$.
	Тогда $\bigcap\limits_{k=1}^\infty [a_k; b_k] = \{ a \}$.
	\end{lemma}
	\begin{proof}
	Очевидно, что $(a_n)$ монотонна и ограничена сверху, $(b_n)$ монотонна и ограничена снизу, тогда $\lim\limits_{n \to \infty} a_n = a$, $\lim\limits_{n \to \infty} b_n = b$.
	Имеем:
	\begin{equation*}
	b = \lim_{n \to \infty} b_n = \lim_{n \to \infty} (b_n - a_n + a_n) = \lim_{n \to \infty} (b_n - a_n) + \lim_{n \to \infty} a_n = a
	\end{equation*}
	
	Отрезки последовательно вложены друг в~друга, поэтому $\bigcap\limits_{k=1}^n [a_k; b_k] = [a_n; b_n]$.
	\begin{equation*}
	\bigcap_{k=1}^\infty [a_k; b_k] = \lim_{n \to \infty} \bigcap_{k=1}^n [a_k; b_k] = \lim_{n \to \infty} [a_n; b_n] = \{ a \}
	\end{equation*}
	\end{proof}
\end{enumerate}

\subsection{Число Эйлера}
\begin{statement}
$\exists \lim\limits_{n \to \infty} \left( 1 + \frac1n \right)^n$.
\end{statement}
\begin{proof}
Рассмотрим последовательность~$(x_n) \colon$
\begin{equation*}
x_n = \left( 1 + \frac1n \right)^n = 1 + \frac{n}n + \frac{n(n - 1)}{2n^2} + \frac{n(n - 1)(n - 2)}{2 \cdot 3n^3} + \ldots + \frac{n!}{n!n^n} =
\end{equation*}
\begin{equation}
\label{eq:Euler's_number_1}
= 2 + \frac1{2!} \left( 1 - \frac1n \right) + \frac1{3!} \left( 1 - \frac1n \right) \left( 1 - \frac2n \right) + \ldots + \frac1{n!} \left( 1 - \frac1n \right) \left( 1 - \frac2n \right) \cdot \ldots \cdot \left( 1 - \frac{n - 1}n \right) <
\end{equation}
\begin{equation*}
< 2 + \frac1{2!} + \frac1{3!} + \ldots + \frac1{n!} < 2 + \frac1{2^1} + \frac1{2^2} + \ldots + \frac1{2^{n-1}} = 2 + 1 - \frac1{2^{n-1}} < 3
\end{equation*}

Значит, $(x_n)$ ограничена.
Кроме того, из выражения~(\ref*{eq:Euler's_number_1}) ясно, что $(x_n)$ монотонна.
Тогда $(x_n)$ сходится.
\end{proof}

\index{e@$e$} \index{Число!Эйлера} Число~$e = \lim\limits_{n \to \infty} \left( 1 + \frac1n \right)^n = 2{,}718281828\ldots$ называется \textbf{числом Эйлера} (иногда \textbf{числом Непера}, или \textbf{неперовым числом}).

\subsection{Критерий Коши}
\index{Последовательность!фундаментальная} Последовательность~$(x_n)$ называется \textbf{фундаментальной}, если
\begin{equation*}
\forall \varepsilon > 0 \ \exists n_0 \in \mathbb N \colon \forall m, n > n_0 \ |x_n - x_m| < \varepsilon
\end{equation*}

\begin{theorem}[критерий Коши]
\label{th:Cauchy_criterion}
Последовательность сходится $\Leftrightarrow$ она фундаментальна.
\end{theorem}
\begin{proof}
\begin{enumerate}
	\item $\Rightarrow$. Пусть $\lim\limits_{n \to \infty} x_n = a$, тогда
	\begin{equation*}
	\forall \varepsilon > 0 \ \exists n_0 \in \mathbb N \colon \forall n > n_0 \ |x_n - a| < \frac\varepsilon2
	\end{equation*}
	
	Пусть $m, n > n_0$.
	\begin{equation*}
	|x_n - x_m| = |x_n - a + a - x_m| \leqslant |x_n - a| + |a - x_m| < \frac\varepsilon2 + \frac\varepsilon2 = \varepsilon
	\end{equation*}
	
	\item $\Leftarrow$.
	\begin{equation*}
	\exists n_0 \in \mathbb N \colon \forall m, n > n_0 \ |x_n - x_m| < 1 \Rightarrow
	\end{equation*}
	\begin{equation*}
	\Rightarrow |x_n - x_{n_0 + 1}| < 1 \Rightarrow
	||x_n| - |x_{n_0 + 1}|| < 1 \Rightarrow
	|x_n| < |x_{n_0 + 1}| + 1
	\end{equation*}
	
	Значит, $\forall x \in \mathbb N \ |x_n| < \max \{ |x_1| + 1, |x_2| + 1, \ldots, |x_{n_0}| + 1 \}$, т.\,е. $(x_n)$ ограничена.
	
	Выберем из неё сходящуюся подпоследовательность~$(x_{n_k}): \lim\limits_{k \to \infty} x_{n_k} = a$.
	\begin{equation*}
	\forall \varepsilon > 0 \ \exists k_0, n_1 \in \mathbb N \colon \forall k > k_0, n > n_1 \ |x_n - x_{n_k}| < \varepsilon \Rightarrow
	\end{equation*}
	\begin{equation*}
	\left| \text{ При } k \to \infty \text{ получим } \right|
	\end{equation*}
	\begin{equation*}
	\Rightarrow |x_n - a| < \varepsilon \Rightarrow
	\lim_{n \to \infty} x_n = a
	\end{equation*}
\end{enumerate}
\end{proof}