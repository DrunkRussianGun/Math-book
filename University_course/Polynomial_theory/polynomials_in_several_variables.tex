\section{Многочлены от нескольких переменных}
\index{Многочлен!от нескольких переменных}
\begin{enumerate}
	\item В~многочлене $a_n x^n + a_{n-1} x^{n-1} + \ldots + a_0$ подставим $a_i = P_i(y)$, $i = 0, 1, \ldots, n$~--- многочлен от~$y$.
	Получим многочлен от $x$ и $y$.
	
	\item Пусть имеем многочлен от $n$~переменных.
	Подставим вместо его коэффициентов многочлен от одной переменной, получим многочлен от $n + 1$~переменных.
\end{enumerate}

Одночлены многочлена будем записывать в~лексикографическом порядке степеней переменных (члены с б\'{о}ль\-ши\-ми степенями идут раньше).

\begin{statement}
Старший член произведения многочленов равен произведению старших членов множителей.
\end{statement}
\begin{proof}
Перемножая члены с наибольшими показателями старшей переменной, получим член с наибольшим показателем при этой переменной.
Проведя аналогичные рассуждения для остальных переменных, придём к выводу, что полученный член является старшим.
\end{proof}

Аналогично доказывается следующее утверждение.
\begin{statement}
Младший член произведения многочленов равен произведению младших членов множителей.
\end{statement}