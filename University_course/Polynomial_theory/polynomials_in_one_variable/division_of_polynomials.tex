\subsection{Деление многочленов}
\textbf{Общим делителем} многочленов $f(x)$ и $g(x)$ называется многочлен~$h(x)$, на который и~$f$, и~$g$ делятся без остатка: $f = ph$, $g = qh$.

\textbf{Наибольшим} называется общий делитель наибольшей степени и обозначается $\NOD$.

\index{Алгоритм!Евклида}
\begin{theorem}[алгоритм Евклида]
Любые два многочлена имеют единственный $\NOD$.
\end{theorem}
\begin{proof}
Будем делить многочлены следующим образом:
\begin{equation*}
f = q_1 g + r_1, \
g = q_2 r_1 + r_2, \
r_1 = q_3 r_2 + r_3, \ \ldots,
\end{equation*}
\begin{equation*}
r_{n-1} = q_{n+1} r_n + r_{n+1}, \
r_n = q_{n+2} r_{n+1} + r_{n+2} = q_{n+2} r_{n+1},
\end{equation*}
\begin{equation*}
\deg g > \deg r_1 > \deg r_2 > \ldots > \deg r_{n+1} > \deg r_{n+2} = -\infty
\end{equation*}

Докажем, что $r_{n+1}$~--- общий делитель $f$ и $g$.
\begin{equation*}
r_n \mult r_{n+1} \Rightarrow
r_{n-1} \mult r_{n+1} \Rightarrow
\ldots \Rightarrow
r_1 \mult r_{n+1} \Rightarrow
g \mult r_{n+1} \Rightarrow
f \mult r_{n+1}
\end{equation*}

Докажем, что $\forall h \ (f \mult h \lAnd g \mult h \Rightarrow r_{n+1} \mult h)$.
\begin{equation*}
f \mult h \lAnd g \mult h \Rightarrow r_1 \mult h, \
g \mult h \lAnd r_1 \mult h \Rightarrow r_2 \mult h, \
r_1 \mult h \lAnd r_2 \mult h \Rightarrow r_3 \mult h, \ \ldots, \
r_{n-1} \mult h \lAnd r_n \mult h \Rightarrow r_{n+1} \mult h
\end{equation*}

Значит, $r_{n+1} = \NOD(f, g)$.
\end{proof}