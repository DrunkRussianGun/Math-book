\section{Недетерминированные конечные автоматы}
\index{Конечный автомат!недетерминированный} \textbf{Недетерминированным конечным автоматом} называется набор~$(S, X, \delta, s_1, F)$, где\\
$S$~--- конечное множество состояний;\\
$X$~--- конечный алфавит;\\
$\delta \colon S \times (X \cup \{ \lambda \}) \to Y$~--- функция перехода, где $Y \subseteq 2^S$;\\
$s_1 \in S$~--- начальное состояние;\\
$F \subseteq S$~--- множество допускающих состояний.

Говорят, что недетерминированный автомат \textbf{распознаёт язык~$L$}, если при чтении слова из языка~$L$ хотя бы один из получившихся путей приводит в допускающее состояние.

\index{Конкатенация} \textbf{Конкатенацией языков $L_1$ и $L_2$} называется язык~$\{ \alpha \beta \mid \alpha \in L_1 \lAnd \beta \in L_2 \}$ и обозначается $L_1 L_2$.

\index{Степень} \textbf{$n$-й степенью языка~$L$} называется $L^n = \underbrace{L L \ldots L}_n$.

\index{Итерация} \index{Звёздочка Клини} \textbf{Итерацией}, или \textbf{звёздочкой Клини}, языка $L$ называется $L^* = \bigcup\limits_{n=0}^\infty L^n$.

Если языки $L_1$ и $L_2$ распознаются автоматами $(S_1, X, \delta_1, s_1, F_1)$ и $(S_2, X, \delta_2, s_2, F_2)$ соответственно, тогда автомат~$(S_1 \cup S_2, X, \delta, s_1, F_2)$ распознаёт язык $L_1 L_2$, где
\begin{equation*}
s \delta(x) =
\begin{cases}
s \delta_1(x), \ s \in S_1 \\
s \delta_2(x), \ s \in S_2 \\
s_2, \ s \in F_1 \lAnd x = \lambda
\end{cases}
\end{equation*}

Если $L$ распознаётся автоматом~$(S, X, \delta, s_1, F)$, то $(S \cup \{ s_0 \}, X, \delta_1, s_0, F \cup \{ s_0 \})$ распознаёт $L^*$, где
\begin{equation*}
s(\delta_1) =
\begin{cases}
s \delta(x), \ s \in S \\
s_1, \ s \notin S \lAnd x = \lambda
\end{cases}
\end{equation*}