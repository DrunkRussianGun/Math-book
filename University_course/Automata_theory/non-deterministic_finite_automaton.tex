\section{Недетерминированные конечные автоматы}
\index{Конечный автомат!недетерминированный} \index{Сокращения!НКА} \textbf{Недетерминированным конечным автоматом}, или \textbf{НКА}, называется набор~$(S, X, \delta, s_1, F)$, где\\
$S$~--- конечное множество состояний;\\
$X$~--- конечный алфавит;\\
$\delta \colon S \times (X \cup \{ \lambda \}) \to Y$~--- функция перехода, где $Y \subseteq 2^S$;\\
$s_1 \in S$~--- начальное состояние;\\
$F \subseteq S$~--- множество допускающих состояний.

Функция~$\delta$ определяет функцию~$\delta^* \colon S \times X^* \to Y$:
\begin{enumerate}
	\item $\delta^*(s, x) = \delta(s, x)$
	\item $\delta^*(s, \alpha x) = \delta^*(\delta^*(s, \alpha), x)$
\end{enumerate}
где $s \in S$, $x \in X \cup \{ \lambda \}$, $\alpha \in X^*$, $Y \subseteq 2^S$.

Допустима запись $s \delta(\alpha)$ вместо $\delta^*(s, \alpha)$.

\index{Конкатенация!языков} \textbf{Конкатенацией языков $L_1$ и $L_2$} называется язык~$\{ \alpha \beta \mid \alpha \in L_1 \lAnd \beta \in L_2 \}$ и обозначается $L_1 L_2$.

\index{Степень!языка} \textbf{$n$-й степенью языка~$L$} называется $L^n = \underbrace{L L \ldots L}_n$.

\index{Итерация} \index{Звёздочка Клини} \textbf{Итерацией}, или \textbf{звёздочкой Клини}, \textbf{языка~$L$} называется $L^* = \bigcup\limits_{n=0}^\infty L^n$.

\subsection{Распознаваемость языков}
Говорят, что НКА \textbf{распознаёт язык~$L$}, если $\alpha \in L \Leftrightarrow s_1 \delta(\alpha) \cap F \neq \varnothing$.

Если языки $L_1$ и $L_2$ распознаются НКА $(S_1, X, \delta_1, s_1, F_1)$ и $(S_2, X, \delta_2, s_2, F_2)$ соответственно, тогда автомат~$(S_1 \opbr\cup S_2, \allowbreak X, \allowbreak \delta, \allowbreak s_1, \allowbreak F_2)$ распознаёт язык $L_1 L_2$, где
\begin{equation*}
s \delta(x) =
\begin{cases}
s \delta_1(x), \ s \in S_1 \\
s \delta_2(x), \ s \in S_2 \\
s_2, \ s \in F_1 \lAnd x = \lambda
\end{cases}
\end{equation*}

Если $L$ распознаётся НКА~$(S, X, \delta, s_1, F)$, то $(S \cup \{ s_0 \}, X, \delta_1, s_0, F \cup \{ s_0 \})$ распознаёт $L^*$, где
\begin{equation*}
s(\delta_1) =
\begin{cases}
s \delta(x), \ s \in S \\
s_1, \ s \notin S \lAnd x = \lambda
\end{cases}
\end{equation*}

\subsection{Построение ДКА по НКА}
\index{Замыкание} \textbf{Замыканием состояния~$s$} некоторого НКА называется множество~$\{ s' \mid s \delta(\lambda^k) = s' \}$ и обозначается~$[s]$.

Пусть язык~$L$ распознаётся НКА~$(S, X, \delta, s_1, F)$.
Тогда ДКА~$(T, X, \delta', [s_1], F')$ тоже распознаёт $L$, где\\
$T = \{ M \mid M = \{ m_1, \ldots, m_k \} \subseteq 2^S \lAnd \bigcup\limits_{i=1}^k [m_i] = M \}$,\\
$\delta'(\{ m_1, \ldots, m_k \}, x) = \bigcup\limits_{i=1}^k [\delta(m_i, x)]$,\\
$F' = \{ M \mid M \in T \lAnd M \cap F \neq \varnothing \}$.
\begin{proofmathind}
	\indbase $s_1 \delta(\lambda^k) = [s_1] \lAnd [s_1] \delta'(\lambda^k) = [s_1] \Rightarrow
	s_1 \delta(\lambda^k) = [s_1] \delta'(\lambda^k)$
	\indstep Пусть доказано для слов~$\alpha$ длины~$n$, $s_1 \delta(\alpha) = \{ m_1, \ldots, m_k \}$.
	\begin{equation*}
	s_1 \delta(\alpha x) = \bigcup_{i=1}^k [\delta(m_i, x)] \lAnd
	[s_1] \delta'(\alpha x) = \bigcup_{i=1}^k [\delta(m_i, x)] \Rightarrow
	s_1 \delta(\alpha x) = [s_1] \delta'(\alpha x)
	\end{equation*}
	\indend
	
Т.\,о., $\forall \alpha \in X^* \ s_1 \delta(\alpha) = [s_1] \delta'(\alpha)$.
\end{proofmathind}