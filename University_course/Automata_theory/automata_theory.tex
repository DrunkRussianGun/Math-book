\index{Алфавит} \textbf{Алфавитом} называется конечное непустое множество и обозначим через~$X$.
\index{Буква} Его элементы называются \textbf{буквами}.
\index{Слово} Конечная последовательность букв называется \textbf{словом}, а его \textbf{длиной}~--- количество букв в слове с учётом повторений.

Слово, не содержащее букв, называется \textbf{пустым} и обозначается~$\lambda$, $\varepsilon$ или $e$.

Множество из всех слов алфавита~$X$ обозначается~$X^*$.

\index{Конкатенация!слов} \textbf{Конкатенацией слов} $\alpha = x_1 x_2 \ldots x_n$ и $\beta = y_1 y_2 \ldots y_m$ называется слово $\alpha \cdot \beta = x_1 \ldots x_n y_1 \ldots y_m$.

\index{Степень!слова} \textbf{Степенью слова}~$\alpha = x_1 \ldots x_n$ называется слово~$\alpha^n = \alpha \cdot \alpha \cdot \ldots \cdot \alpha$, где $n \in \mathbb N$.
$\alpha^0 = \lambda$.

\index{Язык} \textbf{Языком} называется множество $L \subseteq X^*$.
Язык, не содержащий слов, называется \textbf{пустым}.