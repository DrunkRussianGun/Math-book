\section{Детерминированные конечные автоматы}
\index{Алфавит} \textbf{Алфавитом} называется конечное непустое множество и обозначим через~$X$.
\index{Буква} Его элементы называются \textbf{буквами}.
\index{Слово} Конечная последовательность букв называется \textbf{словом}, а его \textbf{длиной}~--- количество букв в слове с учётом повторений.

Слово, не содержащее букв, называется \textbf{пустым} и обозначается~$\lambda$, $\varepsilon$ или $e$.

Множество из всех слов алфавита~$X$ обозначается~$X^*$.

\textbf{Конкатенацией слов} $\alpha = x_1 x_2 \ldots x_n$ и $\beta = y_1 y_2 \ldots y_m$ называется слово $\alpha \cdot \beta = x_1 \ldots x_n y_1 \ldots y_m$.

\textbf{Степенью слова}~$\alpha = x_1 \ldots x_n$ называется слово~$\alpha^n = \alpha \cdot \alpha \cdot \ldots \cdot \alpha$, где $n \in \mathbb N$.
$\alpha^0 = \lambda$.

\index{Язык} \textbf{Языком} называется множество $L \subseteq X^*$.
Язык, не содержащий слов, называется \textbf{пустым}.

\index{Конечный автомат!детерминированный} \textbf{Детерминированным конечным автоматом} называется набор~$(S, X, \delta, s_1, F)$, где\\
$S$~--- конечное множество \textbf{состояний};\\
$X$~--- алфавит;\\
$\delta \colon S \times X \to S$~--- \textbf{функция перехода};\\
$s_1$~--- \textbf{начальное состояние};\\
$F$~--- множество \textbf{допускающих состояний}.

Функция перехода по буквам определяет функцию~$\delta^* \colon S \times X^* \to S$ перехода по словам:
\begin{enumerate}
	\item $\delta^*(s, x) = \delta(s, x)$
	\item $\delta^*(s, \alpha x) = \delta(\delta(s, \alpha), x)$
\end{enumerate}
где $s$~--- состояние, $x$~--- буква, $\alpha$~--- слово.

Запись $\delta^*(s, \alpha)$ несколько громоздка, поэтому вместо неё обычно используется запись $s \delta(\alpha)$.

Говорят, что конечный автомат~$(S, X, \delta, s_1, F)$ \textbf{распознаёт язык~$L$}, если $\alpha \in L \Leftrightarrow s_1 \delta(\alpha) \in F$.

\begin{statement}
Любой конечный язык распознаётся конечным автоматом.
\end{statement}
\begin{proof}
Пусть $L$~--- конечный язык, множество~$S$ состояний состоит из префиксов слов~$L$, а также включает дополнительное состояние~$s'$, $\alpha \delta(x) =
\begin{cases}
\alpha x, \ \alpha x \in S \\
s', \ \alpha x \notin S
\end{cases}$

Рассмотрим автомат~$(S, X, \delta, \lambda, L)$.
\begin{equation*}
\lambda \delta(\alpha) =
\begin{cases}
\alpha, \ \alpha \in S \setminus \{ s' \} \\
s', \ \alpha \notin S \setminus \{ s' \}
\end{cases} \Rightarrow
(s_1 \delta(\alpha) \in F \Leftrightarrow \alpha \in L)
\end{equation*}
\end{proof}

\begin{theorem}
Язык~$L = \{ a^k b^k \mid k \geqslant 0 \}$ не распознаётся конечным автоматом.
\end{theorem}
\begin{proofcontra}
Пусть $L$ распознаётся конечным автоматом~$A = (S, X, \delta, s_1, F)$ с $n$~состояниями.
Тогда какие-то из состояний $s_1, s_1 \delta(a), s_1 \delta(aa), \ldots, s_1 \delta(a^{n-1}), s_1 \delta(a^n)$ совпадают.
Пусть $s_1 \delta(a^i) \opbr= s_1 \delta(a^j)$, тогда $s_1 \delta(a^i) \delta(b^i) \in F \Rightarrow
s_1 \delta(a^j) \delta(b^i) \in F$.
Значит, $a^j b^i \in L$.
Противоречие.
\end{proofcontra}

Слова $\alpha$ и $\beta$ называются \textbf{различимыми словом~$\gamma \in X^*$ относительно языка~$L$}, если $\alpha \gamma \in L \opbr\lAnd \beta \gamma \notin L \opbr\lOr \alpha \gamma \notin L \opbr\lAnd \beta \gamma \in L$.
Различимость обозначается $\alpha \not\sim_L \beta$.

Слова $\alpha$ и $\beta$ называются \textbf{неразличимыми относительно языка~$L$}, если $\forall \gamma \in X^* \ \alpha \gamma \in L \Leftrightarrow \beta \gamma \in L$.
Неразличимость обозначается $\alpha \sim_L \beta$.

\begin{statement}
Отношение неразличимости слов относительно языка является отношением эквивалентности.
\end{statement}
\begin{proof}
Очевидно, что $\alpha \sim \alpha$ и $\alpha \sim \beta \Rightarrow \beta \sim \alpha$.

Пусть $\alpha \sim \beta \lAnd \beta \sim \gamma$, тогда $\forall \Theta \in X^* \ 
\alpha \Theta \in L \Leftrightarrow
\beta \Theta \in L \Leftrightarrow
\gamma \Theta \in L \Rightarrow
\alpha \sim \gamma$.
\end{proof}

\begin{statement}
$\alpha \sim \beta \Rightarrow \forall \gamma \in X^* \ \alpha \gamma \sim \beta \gamma$.
\end{statement}
\begin{proof}
$\forall \Theta \in X^* \ (\alpha \gamma) \Theta \in L \Leftrightarrow
\alpha (\gamma \Theta) \in L \Leftrightarrow
\beta (\gamma \Theta) \in L \Leftrightarrow
(\beta \gamma) \Theta \in L \Rightarrow
\alpha \gamma \sim \beta \gamma$
\end{proof}

\textbf{Рангом языка~$L$} называется количество элементов в фактор"=множестве относительно неразличимости слов и языка~$L$ и обозначается~$\rank L$.

\begin{theorem}
\begin{enumerate}
	\item Язык~$L$ распознаётся конечным автоматом с $n$~состояниями $\Leftrightarrow$ $\rank L \leqslant n$.
	\item Если $\rank L = n$, то существует конечный автомат с $n$~состояниями, который распознаёт~$L$, и никакой конечный автомат с меньшим числом состояний не распознаёт~$L$.
\end{enumerate}
\end{theorem}
\begin{proof}
\begin{enumerate}
	\item Язык~$L$ распознаётся конечным автоматом~$A = (X, S, \delta, s_1, F)$ с $n$~состояниями.
	Рассмотрим слова $\alpha_1, \ldots, \alpha_{n+1} \opbr\in X^*$.
	Хотя бы два из состояний $s_1 \delta(\alpha_1), \ldots, s_1 \delta(\alpha_{n+1})$ совпадают.
	
	Пусть $s_1 \delta(\alpha_i) = s_1 \delta(\alpha_j)$, где $i \neq j$, тогда
	\begin{equation*}
	s_1 \delta(\alpha_i \gamma) =
	s_1 \delta(\alpha_i) \delta(\gamma) =
	s_1 \delta(\alpha_j) \delta(\gamma) =
	s_1 \delta(\alpha_j \gamma) \Rightarrow
	(\alpha_i \gamma \in L \Leftrightarrow \alpha_j \gamma \in L)
	\end{equation*}
	
	Т.\,о., среди $n + 1$~состояний всегда найдётся пара неразличимых, значит, $\rank L \leqslant n$.
	
	\item Пусть $A = (X, S, \delta, s_1, F)$, где $S = \{ [\alpha] \mid \alpha \in X^* \}$, $\delta \colon [\alpha] \delta(x) = [\alpha x]$, $s_1 = [\lambda]$, $F = \{ [\alpha] \mid \alpha \in L \}$,
	тогда $s_1 \delta(\alpha) = [\alpha] \in F \Leftrightarrow \alpha \in L$.
	
	Пусть существует конечный автомат с $k$~состояниями, где $k < n$.
\end{enumerate}
\end{proof}

\textbf{Базисом языка~$L$} называется множество~$W \subseteq X^*$ такое, что:
\begin{enumerate}
	\item Все слова из~$W$ попарно различимы.
	\item Любое другое слово неотличимо от одного из слов множества~$W$.
\end{enumerate}

\begin{theorem}
Множество~$W$~--- базис языка $\Leftrightarrow$
\begin{enumerate}
	\item Все слова из~$W$ попарно различимы.
	\item $\lambda \in W$
	\item $\forall \alpha \in W \ \forall x \in X \ \exists \beta \in W \colon \alpha x \sim \beta$
\end{enumerate}
\end{theorem}
\begin{proof}
\begin{enumerate}
	\item $\Leftarrow$. Очевидно.
	\item $\Rightarrow$. Докажем пункт~3 по индукции.
		\indbase Пусть $\alpha = \lambda$, тогда $\forall x \in X \ \exists \beta \in W \colon x \sim \beta$, т.\,к. $W$~--- базис.
		\indstep Пусть доказано для $\alpha \colon |\alpha| \leqslant k$, тогда $\alpha \sim \beta \in W \Rightarrow \alpha x \sim \beta x \sim \gamma \in W$.
		\indend
\end{enumerate}
\end{proof}

Два состояния $s$ и $s'$ называются \textbf{эквивалентными относительно автомата~$A = (X, S, \delta, s_1, F)$}, если $\forall \alpha \opbr\in X^* \ s \delta(\alpha) \in F \Leftrightarrow s' \delta(\alpha) \in F$.

\index{Конечный автомат!связный} Автомат называется \textbf{связным}, если $\forall s \in S \ \exists \alpha \in X^* \ s_1 \delta(\alpha) = s$.

\index{Конечный автомат!приведённый} Автомат называется \textbf{приведённым}, если в нём нет эквивалентных состояний.

Пусть задан автомат~$A = (X, S, \delta, s_1, F)$.
Рассмотрим автомат~$A_m = (X, S_m, \delta_m, s_m, F_m)$, где $S_m = \{ [s] \mid s \in S \}$, $\delta_m \colon [s] \delta(x) = [s \delta(x)]$, $s_m = [s_1]$, $F_m = \{ [s] \mid s \in F \}$, тогда $[s_1] \delta(\alpha) = [s_1 \delta(\alpha)] \in F_m$, т.\,к. $s_1 \delta(\alpha) \in F$, значит, $A_m$ приведённый.