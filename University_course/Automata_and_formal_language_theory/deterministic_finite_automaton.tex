\section{Детерминированные конечные автоматы}
\index{Конечный автомат!детерминированный} \index{Сокращения!ДКА} \textbf{Детерминированным конечным автоматом}, или \textbf{ДКА}, называется набор~$(S, X, \delta, s_1, F)$, где\\
$S$~--- конечное множество \textbf{состояний};\\
$X$~--- алфавит;\\
$\delta \colon S \times (X \cup \{ \lambda \}) \to S$~--- \textbf{функция перехода}, причём $\delta(s, \lambda) = s$;\\
$s_1 \in S$~--- \textbf{начальное состояние};\\
$F \subseteq S$~--- множество \textbf{допускающих состояний}.

Конечный автомат~$(S, X, \delta, s_1, F)$ удобно представлять в виде графа, в котором вершины представляют состояния автомата, а дуги показывают, между какими состояниями возможен переход и по каким буквам.

Функция перехода по буквам определяет функцию~$\delta^* \colon S \times X^* \to S$ перехода по словам:
\begin{enumerate}
	\item $\delta^*(s, x) = \delta(s, x)$
	\item $\delta^*(s, \alpha x) = \delta(\delta^*(s, \alpha), x)$
\end{enumerate}
где $\alpha \in X^*$, $x \in X \cup \{ \lambda \}$.

Запись $\delta^*(s, \alpha)$ несколько громоздка, поэтому вместо неё обычно используется запись $s \delta(\alpha)$.

\subsection{Распознаваемость языков}
Говорят, что \textbf{ДКА~$(S, X, \delta, s_1, F)$ распознаёт язык~$L$}, если $\alpha \in L \Leftrightarrow s_1 \delta(\alpha) \in F$.

\begin{statement}
Любой конечный язык распознаётся конечным автоматом.
\end{statement}
\begin{proof}
Пусть $L$~--- конечный язык, множество~$S$ состояний состоит из префиксов слов~$L$, а также включает дополнительное состояние~$s'$, $\alpha \delta(x) =
\begin{cases}
\alpha x, \ \alpha x \in S \\
s', \ \alpha x \notin S
\end{cases}$

Рассмотрим автомат~$(S, X, \delta, \lambda, L)$.
\begin{equation*}
\lambda \delta(\alpha) =
\begin{cases}
\alpha, \ \alpha \in S \setminus \{ s' \} \\
s', \ \alpha \notin S \setminus \{ s' \}
\end{cases} \Rightarrow
(s_1 \delta(\alpha) \in F \Leftrightarrow \alpha \in L)
\end{equation*}
\end{proof}

\begin{theorem}
Язык~$L = \{ a^k b^k \mid k \geqslant 0 \}$ не распознаётся конечным автоматом.
\end{theorem}
\begin{proofcontra}
Пусть $L$ распознаётся конечным автоматом~$A = (S, X, \delta, s_1, F)$ с $n$~состояниями.
Тогда какие"~то из состояний $s_1, s_1 \delta(a), s_1 \delta(aa), \ldots, s_1 \delta(a^{n-1}), s_1 \delta(a^n)$ совпадают.
Пусть $s_1 \delta(a^i) \opbr= s_1 \delta(a^j)$, тогда $s_1 \delta(a^i) \delta(b^i) \in F \Rightarrow
s_1 \delta(a^j) \delta(b^i) \in F$.
Значит, $a^j b^i \in L$.
Противоречие.
\end{proofcontra}

Слова $\alpha$ и $\beta$ называются \textbf{различимыми словом~$\gamma \in X^*$ относительно языка~$L$}, если $\alpha \gamma \in L \opbr\lAnd \beta \gamma \notin L \opbr\lOr \alpha \gamma \notin L \opbr\lAnd \beta \gamma \in L$.
Различимость обозначается $\alpha \not\sim_L \beta$.

Слова $\alpha$ и $\beta$ называются \textbf{неразличимыми относительно языка~$L$}, если $\forall \gamma \in X^* \ \alpha \gamma \in L \Leftrightarrow \beta \gamma \in L$.
Неразличимость обозначается $\alpha \sim_L \beta$.

\begin{statement}
Отношение неразличимости слов относительно языка является отношением эквивалентности.
\end{statement}
\begin{proof}
Очевидно, что $\alpha \sim \alpha$ и $\alpha \sim \beta \Rightarrow \beta \sim \alpha$.

Пусть $\alpha \sim \beta \lAnd \beta \sim \gamma$, тогда $\forall \Theta \in X^* \ 
\alpha \Theta \in L \Leftrightarrow
\beta \Theta \in L \Leftrightarrow
\gamma \Theta \in L \Rightarrow
\alpha \sim \gamma$.
\end{proof}

\begin{statement}
$\alpha \sim \beta \Rightarrow \forall \gamma \in X^* \ \alpha \gamma \sim \beta \gamma$.
\end{statement}
\begin{proof}
$\forall \Theta \in X^* \ (\alpha \gamma) \Theta \in L \Leftrightarrow
\alpha (\gamma \Theta) \in L \Leftrightarrow
\beta (\gamma \Theta) \in L \Leftrightarrow
(\beta \gamma) \Theta \in L \Rightarrow
\alpha \gamma \sim \beta \gamma$.
\end{proof}

Фактор"=класс слова~$\alpha$ относительно неразличимости обозначается~$[\alpha]$.

\textbf{Рангом языка~$L$} называется количество элементов в фактор"=множестве относительно неразличимости слов и языка~$L$ и обозначается~$\rank L$.

\index{Теорема!Майхилла---Нероуда}
\begin{theorem}[Майхилла"--~Нероуда]
\begin{enumerate}
	\item Язык~$L$ распознаётся конечным автоматом с $n$~состояниями $\Leftrightarrow$ $\rank L \leqslant n$.
	\item Если $\rank L = n$, то существует конечный автомат с $n$~состояниями, который распознаёт~$L$, и никакой конечный автомат с меньшим числом состояний не распознаёт~$L$.
\end{enumerate}
\end{theorem}
\begin{proof}
\begin{enumerate}
	\item Пусть язык~$L$ распознаётся конечным автоматом~$A = (S, X, \delta, s_1, F)$ с $n$~состояниями.
	Рассмотрим слова $\alpha_1, \ldots, \alpha_{n+1} \opbr\in X^*$.
	Хотя бы два из состояний $s_1 \delta(\alpha_1), \ldots, s_1 \delta(\alpha_{n+1})$ совпадают.
	
	Пусть $s_1 \delta(\alpha_i) = s_1 \delta(\alpha_j)$, где $i \neq j$, тогда
	\begin{equation*}
	s_1 \delta(\alpha_i \gamma) =
	s_1 \delta(\alpha_i) \delta(\gamma) =
	s_1 \delta(\alpha_j) \delta(\gamma) =
	s_1 \delta(\alpha_j \gamma) \Rightarrow
	(\alpha_i \gamma \in L \Leftrightarrow \alpha_j \gamma \in L)
	\end{equation*}
	
	Т.\,о., среди любых $n + 1$~состояний всегда найдётся пара неразличимых, значит, $\rank L \leqslant n$.
	
	\item Пусть $A = (S, X, \delta, [\lambda], F)$, где\\
	$S = \{ [\alpha] \mid \alpha \in X^* \}$,\\
	$\delta \colon [\alpha] \delta(x) = [\alpha x]$,\\
	$F = \{ [\alpha] \mid \alpha \in L \}$,\\
	тогда $s_1 \delta(\alpha) = [\alpha] \in F \Leftrightarrow \alpha \in L$.
	
	Пусть $L$ распознаётся конечным автоматом с $k$~состояниями, где $k < n$.
	Тогда $n = \rank L \leqslant k$.
	Противоречие.
\end{enumerate}
\end{proof}

\index{Базис!языка} \textbf{Базисом языка~$L$} называется множество~$W \subseteq X^*$ такое, что все слова в нём попарно различимы, а любое другое слово неотличимо от одного из слов множества~$W$.

\begin{theorem}
Множество~$W$~--- базис языка $\Leftrightarrow$
\begin{enumerate}
	\item Все слова из~$W$ попарно различимы.
	\item $\lambda \in W$.
	\item $\forall \alpha \in W \ \forall x \in X \ \exists \beta \in W \colon \alpha x \sim \beta$.
\end{enumerate}
\end{theorem}
\begin{proof}
\begin{enumerate}
	\item $\Leftarrow$. Очевидно.
	\item $\Rightarrow$. Докажем пункт~3 по индукции.
		\indbase Пусть $\alpha = \lambda$, тогда $\forall x \in X \ \exists \beta \in W \colon x \sim \beta$, т.\,к. $W$~--- базис.
		\indstep Пусть доказано для $\alpha \colon |\alpha| \leqslant k$, тогда $\alpha \sim \beta \in W \Rightarrow \alpha x \sim \beta x \sim \gamma \in W$.
		\indend
\end{enumerate}
\end{proof}

Два состояния $s$ и $s'$ называются \textbf{эквивалентными относительно автомата~$A = (S, X, \delta, s_1, F)$}, если $\forall \alpha \in X^* \ \allowbreak s \delta(\alpha) \in F \Leftrightarrow s' \delta(\alpha) \in F$.

\index{Конечный автомат!связный} Автомат называется \textbf{связным}, если $\forall s \in S \ \exists \alpha \in X^* \ s_1 \delta(\alpha) = s$.

\index{Конечный автомат!приведённый} \index{Конечный автомат!минимальный} Автомат называется \textbf{приведённым}, или \textbf{минимальным}, если в нём нет эквивалентных состояний.

Пусть задан автомат~$A = (S, X, \delta, s_1, F)$, распознающий язык~$L$.
Рассмотрим автомат~$A_m = (S_m, X, \delta_m, [s_1], F_m)$, где\\
$S_m = \{ [s] \mid s \in S \}$,\\
$\delta_m \colon [s] \delta(x) = [s \delta(x)]$,\\
$F_m = \{ [s] \mid s \in F \}$,\\
тогда $\forall \alpha \in L \ [s_1] \delta(\alpha) = [s_1 \delta(\alpha)] \in F_m$, т.\,к. $s_1 \delta(\alpha) \in F$, значит, $A_m$ распознаёт~$L$.

\index{Конечный автомат!изоморфный} Два автомата $(S, X, \delta, s_1, F)$ и $(S', X, \delta', s_2, F')$ называются \textbf{изоморфными}, если существует биекция~$\varphi \colon S \to S'$ такая, что:
\begin{enumerate}
	\item $s \delta(x) = t \Leftrightarrow \varphi(s) \delta'(x) = \varphi(t)$;
	\item $\varphi(s_1) = s_2$;
	\item $t \in F \Leftrightarrow \varphi(t) \in F'$.
\end{enumerate}

\begin{statement}
Если два минимальных автомата распознают один и тот же язык, то они изоморфны.
\end{statement}
\begin{proof}
Пусть $\alpha \in \{ \alpha_1, \ldots, \alpha_n \}$, причём $S = \{ s_1 \delta(\alpha) \mid \alpha \in \{ \alpha_1, \ldots, \alpha_n \} \} \opbr\lAnd S' \opbr= \{ s_2 \delta'(\alpha) \mid \alpha \opbr\in \{ \alpha_1, \ldots, \alpha_n \} \}$.
В каждом из автоматов нет эквивалентных состояний, поэтому можно построить биекцию $\varphi(s_1 \delta(\alpha)) \opbr= s_2 \delta'(\alpha)$.
Тогда $s_1 \delta(\alpha) \in F \Leftrightarrow \alpha \in L \Leftrightarrow s_1' \delta'(\alpha) \in F$.
\end{proof}

Исследуем распознаваемость дополнения, пересечения и объединения языков.
Пусть $L_1$, $L_2$~--- языки, распознаваемые некоторыми конечными автоматами, $\rank L_1 = m$ и $\rank L_2 = n$.
\begin{enumerate}
	\item $\rank L_1 = \rank \overline L_1$.
	\begin{proof}
	$\alpha \not\sim_{L_1} \beta \Leftrightarrow
	\exists \gamma \colon \alpha \gamma \in L_1 \lAnd \beta \gamma \notin L_1 \Leftrightarrow
	\alpha \gamma \notin \overline L_1 \lAnd \beta \gamma \in \overline L_1 \Leftrightarrow
	\alpha \not\sim_{\overline L_1} \beta$.
	\end{proof}
	
	\item $\rank L_1 \cap L_2 \leqslant mn$.
	\begin{proof}
	Пусть $\alpha_1, \ldots, \alpha_m$ и $\beta_1, \ldots, \beta_n$~--- базисы $L_1$ и $L_2$ соответственно.
	Тогда
	\begin{equation*}
	\forall \gamma \in X^* \ \exists \alpha_i, \beta_j \colon \alpha_i \sim_{L_1} \gamma \lAnd \beta_j \sim_{L_2} \gamma
	\end{equation*}
	
	Пусть $\gamma_{ij} \colon \alpha_i \sim_{L_1} \gamma_{ij} \lAnd \beta_j \sim_{L_2} \gamma_{ij}$, тогда $\forall \Theta \in X^*$
	\begin{itemize}
		\item $\gamma \Theta \in L_1 \Leftrightarrow
		\alpha_i \Theta \in L_1 \Leftrightarrow
		\gamma_{ij} \Theta \in L_1$;
		\item $\gamma \Theta \in L_2 \Leftrightarrow
		\beta_j \Theta \in L_2 \Leftrightarrow
		\gamma_{ij} \Theta \in L_2$.
	\end{itemize}
	
	Отсюда $\gamma \Theta \in L_1 \cap L_2 \Leftrightarrow
	\alpha_i \Theta \in L_1 \lAnd \beta_j \in L_2 \Leftrightarrow
	\gamma_{ij} \Theta \in L_1 \cap L_2$, тогда $y_{11}, y_{12}, \ldots, y_{mn}$~--- базис языка~$L_1 \cap L_2$.
	\end{proof}
	
	\item Аналогично доказывается, что $\rank L_1 \cup L_2 \leqslant mn$.
\end{enumerate}

\index{Лемма!о накачке}
\begin{lemma}[о накачке]
Если $L$ распознаётся конечным автоматом, то $\exists n \in \mathbb N \colon (\forall \alpha \in X^* \colon |\alpha| \geqslant n) \ \allowbreak \exists \alpha_1, \alpha_2, \alpha_3 \colon \alpha = \alpha_1 \alpha_2 \alpha_3$, причём
\begin{enumerate}	
	\item $\alpha_2 \neq \lambda$;
	\item $\alpha \in L \Rightarrow \forall i \ \alpha_1 \alpha_2^i \alpha_3 \in L$;
	\item $|\alpha_1 \alpha_2| \leqslant n$.
\end{enumerate}
\end{lemma}
\begin{proof}
Пусть в конечном автомате $n$~состояний, $|\alpha| = k \geqslant n \lAnd \alpha = x_1 \ldots x_k$.
Среди состояний $s_1 \delta(\lambda), s_1 \delta(x_1), s_1 \delta(x_1 x_2), \ldots, s_1 \delta(x_1 \ldots x_n)$ найдутся два совпадающих, т.\,е.
$\exists l, m \colon 0 \leqslant l < m \leqslant n \lAnd s_1 \delta(x_1 \ldots x_l) \opbr= s_1 \delta(x_1 \ldots x_m) \opbr= s'$.

Пусть $\alpha_1 = x_1 \ldots x_l$, $\alpha_2 = x_{l+1} \ldots x_m$, $\alpha_3 = x_{m+1} \ldots x_k$, $s_1 \delta(\alpha_1) = s'$, тогда $s_1 \delta(\alpha_1 \alpha_2) = s' \opbr\Rightarrow s_1 \delta(\alpha_1 \alpha_2^i) = s' \opbr\Rightarrow s_1 \delta(\alpha) = s_1 \delta(\alpha_1 \alpha_2 \alpha_3) = s' \delta(\alpha_3) = s_1 \delta(\alpha_1 \alpha_2^i \alpha_3)$.
\end{proof}

\begin{consequent}
Если $\forall n \in \mathbb N \ \exists \alpha \colon |\alpha| \in n$, причём для любых слов $\alpha_1, \alpha_2, \alpha_3$:
\begin{enumerate}
	\item $\alpha_2 \neq \lambda$;
	\item $\alpha \in L \lAnd \exists i \ \alpha_1 \alpha_2^i \alpha_3 \notin L$;
	\item $|\alpha_1 \alpha_2| \leqslant n$.
\end{enumerate}
то $L$ не распознаётся конечным автоматом.
\end{consequent}

\begin{statement}
Если непустой язык распознаётся автоматом с $n$~состояниями, то он содержит слово длины не больше~$n$.
\end{statement}
\begin{proof}
Если автомат распознаёт слово, то в соответствующем графе есть путь из начального состояния в допускающее, а значит, есть и простой путь.
В графе $n$~вершин, тогда длина простого пути не больше~$n$, а ему соответствует слово длины не больше~$n$.
\end{proof}