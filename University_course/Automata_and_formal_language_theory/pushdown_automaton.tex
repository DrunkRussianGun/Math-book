\section{Автоматы с магазинной памятью}
\index{Конечный автомат!с магазинной памятью} \index{Сокращения!МП-автомат} \textbf{Конечным автоматом с магазинной памятью} (\textbf{МП"=автоматом}) называется набор~$(S, X, \Gamma, \delta, s_1, Z_0, F)$, где\\
$S$~--- конечное множество состояний;\\
$X$~--- алфавит;\\
$\Gamma$~--- конечное множество магазинных символов, причём $\lambda \in \Gamma$;\\
$\delta \colon S \times (X \cup \{ \lambda \}) \times \Gamma \to 2^{(S \times \Gamma^*)}$~--- функция перехода;\\
$s_1 \in S$~--- начальное состояние;\\
$Z_0 \in \Gamma$~--- начальный магазинный символ (называемый \textbf{маркером дна});\\
$F \subseteq S$~--- множество допускающих состояний.

Функция~$\delta$ определяет функцию~$\delta^* \colon S \times X^* \times \Gamma \to 2^{(S \times \Gamma^*)}$:
\begin{enumerate}
	\item $\delta^*(s, x, Z) = \delta(s, x, Z)$
	\item Пусть $\delta^*(s, \alpha, Z) = \{ (s_1, Y_1 \xi_1), \ldots, (s_m, Y_m \xi_m) \}$,
	$\delta(s_i, x, Y_i) = \{ (s_{i1}, \xi_{i1}), \ldots, (s_{in}, \xi_{in}) \}$,
	тогда
	$\delta^*(s, \alpha x, Z) \opbr= \bigcup\limits_{i=1}^m \bigcup\limits_{j=1}^n (s_{ij}, \xi_{ij} \xi_i)$
\end{enumerate}
где $s \in S$, $x \in X \cup \{ \lambda \}$, $\alpha \in X^*$, $Z \in \Gamma$.

Иными словами, при каждом переходе текущий магазинный символ стирается, вместо него записывается последовательность новых символов, определяемая функцией перехода, и в качестве текущего берётся её первый символ.

Говорят, что автомат с магазинной памятью \textbf{распознаёт язык~$L$}:
\begin{itemize}
	\item \textbf{по конечному состоянию}, если $\alpha \in L \Leftrightarrow \delta(s_1, \alpha, Z_0) = (s_f, \gamma)$, где $s_f \in F$, $\gamma \in \Gamma^*$;
	\item \textbf{по пустому магазину}, если $\alpha \in L \Leftrightarrow \exists \beta, \gamma \colon \alpha = \beta \gamma \lAnd \delta(s_1, \beta, Z_0) = (s, \lambda)$, где $s \in S$.
	
	Т.\,е. $\alpha$ распознаётся автоматом, если в процессе его чтения при одном из переходов магазин стал пустым.
\end{itemize}

\begin{statement}
Множество языков, распознаваемых МП"=автоматами по конечному состоянию, совпадает со множеством языков, распознаваемых МП"=автоматами по пустому магазину.
\end{statement}

\subsection{Распознаваемость языков}
\begin{statement}
Язык~$L = \{ a^n b^n \mid n \geqslant 0 \}$ распознаётся МП"=автоматом.
\end{statement}
\begin{proof}
Автомат~$(S, X, \Gamma, \delta, s_0, Z, \{ s_2 \})$ распознаёт~$L$ по конечному состоянию, где\\
$S = \{ s_0, s_1, s_2 \}$,\\
$X = \{ a, b \}$,\\
$\Gamma = \{ A, Z \}$,\\
$\delta$ задаётся переходами\\
$s_0 \delta(a, Z) = (s_0, AZ)$,
$s_0 \delta(a, A) = (s_0, AA)$,
$s_0 \delta(b, A) = (s_1, \lambda)$,
$s_0 \delta(\lambda, Z) = (s_1, Z)$,\\
$s_1 \delta(b, A) = (s_1, \lambda)$,
$s_1 \delta(\lambda, Z) = (s_2, Z)$.
\end{proof}