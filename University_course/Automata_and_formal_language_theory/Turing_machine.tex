\section{Машина Тьюринга}
\index{Машина Тьюринга} \textbf{Машиной Тьюринга} называется набор~$(Q, X, B, \delta, q_1, q_0)$, где\\
$Q$~--- множество состояний;\\
$X$~--- алфавит;\\
$B \notin X$~--- пустой символ;\\
$\delta \colon Q \times (X \cup \{ B \}) \to Q \times (X \cup \{ B \}) \times \{ L, S, R \}$~--- функция перехода;\\
$q_1 \in Q$~--- начальное состояние;\\
$q_0 \in Q$~--- состояние остановки.

$x_1 x_2 \ldots x_{i-1} q x_i x_{i+1} \ldots x_n$ называется \textbf{конфигурацией машины Тьюринга}, где $x_1 x_2 \ldots x_n$~--- текущее слово, $q$~--- текущее состояние, $x_i$~--- текущий символ.
Считается, что слева и справа от слова бесконечное число пустых символов.

Пусть $\delta(q, x_i) = (q', y_i, D)$, тогда на следующем шаге конфигурация меняется на
\begin{itemize}
	\item $x_1 \ldots x_{i-2} q' x_{i-1} y_i x_{i+1} \ldots x_n$, если $D = L$;
	\item $x_1 \ldots x_{i-1} q' y_i x_{i+1} \ldots x_n$, если $D = S$;
	\item $x_1 \ldots x_{i-1} y_i q' x_{i+1} x_{i+2} \ldots x_n$, если $D = R$.
\end{itemize}

Попав в состояние остановки, машина Тьюринга прекращает работу.

\index{Язык!вычислимый} Язык называется \textbf{вычислимым}, если существует машина Тьюринга, которая при чтении слова из этого языка останавливается на не пустом символе, а при чтении слова не из этого языка~--- на пустом.

\begin{theorem}
Любой регулярный язык вычислим.
\end{theorem}
\begin{proof}
Пусть регулярный язык распознаётся автоматом~$(Q, X, \delta_A, q_1, F)$.
Тогда его распознаёт машина Тьюринга $(Q \cup \{ q_0 \}, X, \lambda, \delta_M, q_1, q_0)$, где
\begin{equation*}
\delta_M(q, x) =
\begin{cases}
(\delta_A(q, x), \lambda, R), \ x \neq \lambda \\
(q_0, a, S), \ q \in F \\
(q_0, \lambda, S), \ q \notin F
\end{cases}
\end{equation*}
\end{proof}

\begin{statement}
Язык $\{ a^n b^n \mid n \geqslant 0 \}$ вычислим.
\end{statement}
\begin{proof}
Данный язык распознаёт машина Тьюринга~$(Q, X, \lambda, \delta, q_1, q_0)$, где\\
$Q = \{ q_0, q_1, \ldots, q_4 \}$,\\
$X = \{ a, b \}$,\\
$\delta$ задаётся таблицей\\
$\begin{array}{c|c|c|c}
  & a & b & \lambda \\
\hline
q_1 & q_2 \lambda R & q_0 \lambda S & q_0 a S \\
q_2 & q_2 a R & q_2 b R & q_3 \lambda L \\
q_3 & q_0 \lambda S & q_4 \lambda L & q_0 \lambda S \\
q_4 & q_4 a L & q_4 b L & q_1 \lambda R
\end{array}$\\
\end{proof}

\index{Функция!вычислимая} Функция~$f(x_1, \ldots, x_n)$ называется \textbf{вычислимой}, если существует машина Тьюринга~$(Q, \{ 1 \}, 0, \delta, q_1, q_0)$, переводящая конфигурацию~$q_1 1^{x_1 + 1} 0 1^{x_2 + 1} 0 \ldots 0 1^{x_n + 1}$ в $q_0 1^{f(x_1, \ldots, x_n)}$.

\index{Язык!вычислимо-перечислимый} Язык~$L$ называется \textbf{вычислимо"=перечислимым}, если некоторая машина Тьюринга на словах из~$L$ останавливается на непустом символе, а на словах не из~$L$ работает бесконечно долго.

\begin{statement}
Языки $L$ и $\overline L$ вычислимо"=перечислимы $\Leftrightarrow$ $L$ вычислим.
\end{statement}
\begin{proof}
\begin{enumerate}
	\item $\Rightarrow$.
	$L$ и $\overline L$ вычислимо"=перечислимы $\Rightarrow$ существуют соответствующие машины Тьюринга $M$ и $M'$.
	Построим новую машину, которая дублирует входное слово, а затем на нечётных шагах выполняет команды $M$, а на чётных~--- команды $M'$.
	Тогда получим машину, которая запускает на данном слове и $M$, и $M'$.
	Одна из них остановится, поэтому $L$ вычислим.
	
	\item $\Leftarrow$. Очевидно.
\end{enumerate}
\end{proof}

\begin{statement}
$L$ вычислим $\Leftrightarrow$ $\overline L$ вычислим.
\end{statement}

Элементы множества~$\{ L, S, R \}$ можно закодировать числами~$0, 1, 2$.
Пусть команда $c = (q_i, x_j, q_k, x_l, d)$, означающая, что $\delta(q_i, x_j) = (q_k, x_l, d)$, кодируется числом $|c| = 2^{i+1} \cdot 3^{j+1} \cdot 5^{k+1} \cdot 7^{l+1} \cdot 11^{d+1}$.
Тогда машина Тьюринга с $r$~командами кодируется числом $2^{|c_1| + 1} \cdot 3^{|c_2| + 1} \cdot \ldots \cdot p_r^{|c_r| + 1}$.
Также можно закодировать слова.

Используя обратный алгоритм, любое число можно раскодировать в машину Тьюринга.
Если из некоторого числа нельзя получить машину Тьюринга, используя этот алгоритм, то сопоставляем этому числу машину, вычисляющую пустой язык.

\begin{theorem}
Язык $L_0 = \{ i \mid \alpha_i \in L_{M_i} \}$, где $\alpha_i$~--- слово с кодом $i$, $L_{M_i}$~--- язык, вычисляемый машиной Тьюринга с кодом~$i$, вычислимо"=перечислим, но невычислим.
\end{theorem}
\begin{proof}
\begin{enumerate}
	\item Если $L_0$ вычислим, то $\overline L_0$ также вычислим, тогда существует машина Тьюринга $M_k$ с кодом~$k$, вычисляющая $\overline L_0$.
	Значит,
	\begin{equation*}
	\alpha_k \in L_{M_k} \Leftrightarrow
	k \in L_0 \Leftrightarrow
	k \notin \overline L_0 \Leftrightarrow
	\alpha_k \notin L_{M_k}
	\end{equation*}
	
	Противоречие, тогда $L_0$ не вычислим.
	
	\item Построим машину Тьюринга, которая на 1"~м шаге запускает~$M_1$, на 2"~м~--- $M_2$ и т.\,д.
	Если какая"~то из машин остановилась, то соответствующее слово принадлежит $L_0$.
	Т.\,о., $L_0$ вычислимо"=перечислим.
\end{enumerate}
\end{proof}

\begin{consequent}
$\overline L_0$ не вычислимо"=перечислим.
\end{consequent}

\subsection{Недетерминированная машина Тьюринга}
\index{Машина Тьюринга!недетерминированная} Машина Тьюринга называется \textbf{недетерминированной}, если из одной её конфигурации возможен переход в несколько других.

\index{Тезис!Чёрча---Тьюринга}
\begin{statement}[тезис Чёрча"--~Тьюринга]
Всё, что можно сделать алгоритмом, можно сделать и на машине Тьюринга.
\end{statement}

\begin{statement}
Языки, распознаваемые недетерминированной машиной Тьюринга, распознаются детерминированной машиной Тьюринга.
\end{statement}
\begin{proof}
При каждом переходе в недетерминированной машине Тьюринга получаются сразу несколько конфигураций.
В детерминированной машине Тьюринга эти конфигурации можно записывать друг за другом и по очереди совершать из них переходы.
\end{proof}