\section{Обратные матрицы}
Матрица~$B$ называется \textbf{левой обратной} к квадратной матрице~$A$, если $BA = E$.

Матрица~$C$ называется \textbf{правой обратной} к квадратной матрице~$A$, если $AC = E$.

Заметим, что обе матрицы $B$ и $C$~--- квадратные того же порядка, что и~$A$.

\begin{statement}
Если существуют левая и правая обратные к $A$ матрицы $B$ и $C$, то они совпадают.
\end{statement}
\begin{proof}
$B = BE = BAC = EC = C$.
\end{proof}

\index{Матрица!обратная} Т.\,о., матрица~$A^{-1}$ называется \textbf{обратной} к матрице~$A$, если $A^{-1} A = A A^{-1} = E$.

Приведём методы вычисления обратной матрицы.

\index{Метод!присоединённой матрицы}
\begin{theorem}[метод присоединённой матрицы]
\label{th:inverse_matrix_by_matrix_of_cofactors}
Пусть даны матрицы $A = \|a_{ij}\|_{\begin{smallmatrix}
i = \overline{1, n} \\
j = \overline{1, n}
\end{smallmatrix}}$, $\hat A = \|A_{ij}\|_{\begin{smallmatrix}
i = \overline{1, n} \\
j = \overline{1, n}
\end{smallmatrix}}$, где $A_{ij}$~--- алгебраическое дополнение $a_{ij}$.

Если $|A| \neq 0$, то
\begin{equation*}
A^{-1} = \frac1{|A|} \cdot \hat A^T
\end{equation*}
\end{theorem}
\begin{proof}
\begin{equation*}
A \cdot \left( \frac1{|A|} \cdot \hat A^T \right) =
\frac1{|A|} \cdot A \cdot \hat A^T =
\frac1{|A|} \cdot
\begin{Vmatrix}
a_{11} & a_{12} & \cdots & a_{1n} \\
a_{21} & a_{22} & \cdots & a_{2n} \\
\vdots & \vdots & \ddots & \vdots \\
a_{n1} & a_{n2} & \cdots & a_{nn}
\end{Vmatrix} \cdot
\begin{Vmatrix}
A_{11} & A_{21} & \cdots & A_{n1} \\
A_{12} & A_{22} & \cdots & A_{n2} \\
\vdots & \vdots & \ddots & \vdots \\
A_{1n} & A_{2n} & \cdots & A_{nn}
\end{Vmatrix} =
\end{equation*}
\begin{equation*}
= \frac1{|A|} \cdot
\begin{Vmatrix}
\sum\limits_{k=1}^n a_{1k} A_{1k} & \sum\limits_{k=1}^n a_{1k} A_{2k} & \cdots & \sum\limits_{k=1}^n a_{1k} A_{nk} \\
\sum\limits_{k=1}^n a_{2k} A_{1k} & \sum\limits_{k=1}^n a_{2k} A_{2k} & \cdots & \sum\limits_{k=1}^n a_{2k} A_{nk} \\
\vdots & \vdots & \ddots & \vdots \\
\sum\limits_{k=1}^n a_{nk} A_{1k} & \sum\limits_{k=1}^n a_{nk} A_{2k} & \cdots & \sum\limits_{k=1}^n a_{nk} A_{nk} \\
\end{Vmatrix} =
\frac1{|A|} \cdot
\begin{Vmatrix}
|A| & 0 & \cdots & 0 \\
0 & |A| & \cdots & 0 \\
\vdots & \vdots & \ddots & \vdots \\
0 & 0 & \cdots & |A|
\end{Vmatrix} = E \Rightarrow
\end{equation*}
\begin{equation*}
\Rightarrow \frac1{|A|} \cdot \hat A^T = A^{-1}
\end{equation*}
\end{proof}

\index{Метод!Гаусса---Жордана}
\begin{theorem}[метод Гаусса"--~Жордана]
Пусть дана невырожденная матрица $A = \|a_{ij}\|_{\begin{smallmatrix}
i = \overline{1, n} \\
j = \overline{1, n}
\end{smallmatrix}}$.

Присоединим к~ней единичную матрицу:
\begin{equation*}
B = \begin{Vmatrix}
a_{11} & a_{12} & \cdots & a_{1n} & 1 & 0 & \cdots & 0 \\
a_{21} & a_{22} & \cdots & a_{2n} & 0 & 1 & \cdots & 0 \\
\vdots & \vdots & \ddots & \vdots & \vdots & \vdots & \ddots & \vdots \\
a_{n1} & a_{n2} & \cdots & a_{nn} & 0 & 0 & \cdots & 1
\end{Vmatrix}
\end{equation*}
и с помощью элементарных преобразований только над строками полученной матрицы (или только над столбцами) приведём её левую часть к единичной матрице.
Тогда правая часть будет обратной к~$A$ матрицей.
\end{theorem}
\begin{proof}
Каждое элементарное преобразование квадратной матрицы~$A$ эквивалентно её умножению на некоторую матрицу~$T$ того~же порядка:
\begin{itemize}
	\item \begin{equation*}
	\begin{Vmatrix}
	1 & 0 & \ldots & 0 & \ldots & 0 & \ldots & 0 \\
	0 & 1 & \ldots & 0 & \ldots & 0 & \ldots & 0 \\
	\vdots & \vdots & \ddots & \vdots & \ddots & \vdots & \ddots & \ldots \\
	0 & 0 & \ldots & 0 & \ldots & 1 & \ldots & 0 \\
	\vdots & \vdots & \ddots & \vdots & \ddots & \vdots & \ddots & \ldots \\
	0 & 0 & \ldots & 1 & \ldots & 0 & \ldots & 0 \\
	\vdots & \vdots & \ddots & \vdots & \ddots & \vdots & \ddots & \ldots \\
	0 & 0 & \ldots & 0 & \ldots & 0 & \ldots & 1
	\end{Vmatrix} \cdot
	\begin{Vmatrix}
	a_{11} & a_{12} & \ldots & a_{1i} & \ldots & a_{1j} & \ldots & a_{1n} \\
	a_{21} & a_{22} & \ldots & a_{2i} & \ldots & a_{2j} & \ldots & a_{2n} \\
	\vdots & \vdots & \ddots & \vdots & \ddots & \vdots & \ddots & \ldots \\
	a_{i1} & a_{i2} & \ldots & a_{ii} & \ldots & a_{ij} & \ldots & a_{in} \\
	\vdots & \vdots & \ddots & \vdots & \ddots & \vdots & \ddots & \ldots \\
	a_{j1} & a_{j2} & \ldots & a_{ji} & \ldots & a_{jj} & \ldots & a_{jn} \\
	\vdots & \vdots & \ddots & \vdots & \ddots & \vdots & \ddots & \ldots \\
	a_{1n} & a_{2n} & \ldots & a_{in} & \ldots & a_{jn} & \ldots & a_{nn}
	\end{Vmatrix} =
	\end{equation*}
	\begin{equation*}
	= \begin{Vmatrix}
	a_{11} & a_{12} & \ldots & a_{1i} & \ldots & a_{1j} & \ldots & a_{1n} \\
	a_{21} & a_{22} & \ldots & a_{2i} & \ldots & a_{2j} & \ldots & a_{2n} \\
	\vdots & \vdots & \ddots & \vdots & \ddots & \vdots & \ddots & \ldots \\
	a_{j1} & a_{j2} & \ldots & a_{ji} & \ldots & a_{jj} & \ldots & a_{jn} \\
	\vdots & \vdots & \ddots & \vdots & \ddots & \vdots & \ddots & \ldots \\
	a_{i1} & a_{i2} & \ldots & a_{ii} & \ldots & a_{ij} & \ldots & a_{in} \\
	\vdots & \vdots & \ddots & \vdots & \ddots & \vdots & \ddots & \ldots \\
	a_{1n} & a_{2n} & \ldots & a_{in} & \ldots & a_{jn} & \ldots & a_{nn}
	\end{Vmatrix}
	\end{equation*}
	
	\item \begin{equation*}
	\begin{Vmatrix}
	1 & 0 & \ldots & 0 & \ldots & 0 \\
	0 & 1 & \ldots & 0 & \ldots & 0 \\
	\vdots & \vdots & \ddots & \vdots & \ddots & \ldots \\
	0 & 0 & \ldots & \lambda & \ldots & 0 \\
	\vdots & \vdots & \ddots & \vdots & \ddots & \ldots \\
	0 & 0 & \ldots & 0 & \ldots & 1
	\end{Vmatrix} \cdot
	\begin{Vmatrix}
	a_{11} & a_{12} & \ldots & a_{1i} & \ldots & a_{1n} \\
	a_{21} & a_{22} & \ldots & a_{2i} & \ldots & a_{2n} \\
	\vdots & \vdots & \ddots & \vdots & \ddots & \ldots \\
	a_{i1} & a_{i2} & \ldots & a_{ii} & \ldots & a_{in} \\
	\vdots & \vdots & \ddots & \vdots & \ddots & \ldots \\
	a_{1n} & a_{2n} & \ldots & a_{in} & \ldots & a_{nn}
	\end{Vmatrix} =
	\begin{Vmatrix}
	a_{11} & a_{12} & \ldots & a_{1i} & \ldots & a_{1n} \\
	a_{21} & a_{22} & \ldots & a_{2i} & \ldots & a_{2n} \\
	\vdots & \vdots & \ddots & \vdots & \ddots & \ldots \\
	\lambda a_{i1} & \lambda a_{i2} & \ldots & \lambda a_{ii} & \ldots & \lambda a_{in} \\
	\vdots & \vdots & \ddots & \vdots & \ddots & \ldots \\
	a_{1n} & a_{2n} & \ldots & a_{in} & \ldots & a_{nn}
	\end{Vmatrix}
	\end{equation*}
	
	\item \begin{equation*}
	\begin{Vmatrix}
	1 & 0 & \ldots & 0 & \ldots & 0 & \ldots & 0 \\
	0 & 1 & \ldots & 0 & \ldots & 0 & \ldots & 0 \\
	\vdots & \vdots & \ddots & \vdots & \ddots & \vdots & \ddots & \ldots \\
	0 & 0 & \ldots & 1 & \ldots & 0 & \ldots & 0 \\
	\vdots & \vdots & \ddots & \vdots & \ddots & \vdots & \ddots & \ldots \\
	0 & 0 & \ldots & \lambda & \ldots & 1 & \ldots & 0 \\
	\vdots & \vdots & \ddots & \vdots & \ddots & \vdots & \ddots & \ldots \\
	0 & 0 & \ldots & 0 & \ldots & 0 & \ldots & 1
	\end{Vmatrix} \cdot
	\begin{Vmatrix}
	a_{11} & a_{12} & \ldots & a_{1i} & \ldots & a_{1j} & \ldots & a_{1n} \\
	a_{21} & a_{22} & \ldots & a_{2i} & \ldots & a_{2j} & \ldots & a_{2n} \\
	\vdots & \vdots & \ddots & \vdots & \ddots & \vdots & \ddots & \ldots \\
	a_{i1} & a_{i2} & \ldots & a_{ii} & \ldots & a_{ij} & \ldots & a_{in} \\
	\vdots & \vdots & \ddots & \vdots & \ddots & \vdots & \ddots & \ldots \\
	a_{j1} & a_{j2} & \ldots & a_{ji} & \ldots & a_{jj} & \ldots & a_{jn} \\
	\vdots & \vdots & \ddots & \vdots & \ddots & \vdots & \ddots & \ldots \\
	a_{1n} & a_{2n} & \ldots & a_{in} & \ldots & a_{jn} & \ldots & a_{nn}
	\end{Vmatrix} =
	\end{equation*}
	\begin{equation*}
	= \begin{Vmatrix}
	a_{11} & a_{12} & \ldots & a_{1i} & \ldots & a_{1j} & \ldots & a_{1n} \\
	a_{21} & a_{22} & \ldots & a_{2i} & \ldots & a_{2j} & \ldots & a_{2n} \\
	\vdots & \vdots & \ddots & \vdots & \ddots & \vdots & \ddots & \ldots \\
	a_{i1} & a_{i2} & \ldots & a_{ii} & \ldots & a_{ij} & \ldots & a_{in} \\
	\vdots & \vdots & \ddots & \vdots & \ddots & \vdots & \ddots & \ldots \\
	\lambda a_{i1} + a_{j1} & \lambda a_{i2} + a_{j2} & \ldots & \lambda a_{ii} + a_{ji} & \ldots & \lambda a_{ij} + a_{jj} & \ldots & \lambda a_{in} + a_{jn} \\
	\vdots & \vdots & \ddots & \vdots & \ddots & \vdots & \ddots & \ldots \\
	a_{1n} & a_{2n} & \ldots & a_{in} & \ldots & a_{jn} & \ldots & a_{nn}
	\end{Vmatrix}
	\end{equation*}
\end{itemize}

Т.\,о., результат последовательных элементарных преобразований матрицы~$B$ можно представить в виде $T_k \opbr\cdot \ldots \opbr\cdot T_1 \opbr\cdot B$.
Рассматривая отдельно левую и правую части матрицы~$B$, получим:
\begin{equation*}
\begin{cases}
T_k \cdot \ldots \cdot T_1 \cdot A = E \\
T_k \cdot \ldots \cdot T_1 \cdot E = A_1
\end{cases}
\Rightarrow A_1 \cdot A = E \Rightarrow A_1 = A^{-1}
\end{equation*}
\end{proof}

\begin{theorem}
Если $A$~--- квадратная матрица, то $\exists A^{-1} \Leftrightarrow \det A \neq 0$.
\end{theorem}
\begin{proof}
\begin{enumerate}
	\item $\Rightarrow$. $A \cdot A^{-1} = E \Rightarrow
	\det A \cdot \det A^{-1} = 1 \Rightarrow
	\det A \neq 0$
	
	\item $\Leftarrow$. $\exists A^{-1}$ по теореме~\ref*{th:inverse_matrix_by_matrix_of_cofactors}.
\end{enumerate}
\end{proof}