\section{Эйлеровы графы}
Цикл, содержащий все рёбра графа, называется \textbf{эйлеровым}.

\index{Графы!эйлеровы} Граф, содержащий эйлеров цикл, называется \textbf{эйлеровым}.

\begin{theorem}
Связный граф эйлеров $\Leftrightarrow$ степени всех вершин чётны.
\end{theorem}
\begin{proof}
\begin{enumerate}
	\item $\Rightarrow$. Пусть в~графе есть эйлеров цикл.
	Выберем вершину~$v_0$ в~этом цикле и начнём обходить его.
	При каждом посещении вершины~$v \neq v_0$ её степень увеличивается на~$2$.
	Т.\,о., если посетить её $k$~раз, то $\deg v = 2k \mult 2$.
	
	Для $v_0$ степень увеличивается на~$1$ в~начале обхода, на~$1$ в~конце обхода и на~$2$ при промежуточных посещениях.
	Т.\,о., её степень чётна.
	
	\item $\Leftarrow$. Пусть степени всех вершин чётны.
	Выберём цепь~$C = (v_0, e_0, v_1, e_1, \ldots, e_{k-1}, v_k)$ наибольшей длины.
	Все рёбра, инцидентные~$v_k$, присутствуют в~этой цепи, иначе её можно было~бы удлинить.
	
	Докажем методом от противного, что $v_0 = v_k$.
	Пусть $v_0 \neq v_k$.
	При прохождении вершины~$v_i = v_k$, $i = 1, 2, \ldots, k - 1$, степень~$v_k$ увеличивается на~$2$.
	Также проходим по ребру~$e_{k-1}$, тогда степень~$v_k$ нечётна.
	Противоречие.
	
	Докажем методом от противного, что $C$ содержит все рёбра.	
	Пусть найдётся ребро~$e = \{ u, v \}$, не входящее в~$C$.
	Возьмём первое ребро~$e' = \{ v_i, v' \}$ из $(v_0, u)$-маршрута, не входящее в~$C$.
	Тогда цепь~$(v', e', v_i, e_i, \ldots, e_{k-1}, \allowbreak v_k = v_0, \allowbreak e_0, v_1, e_1, \ldots, v_{i-1})$ длиннее, чем~$C$.
	Противоречие.
\end{enumerate}
\end{proof}

\subsubsection{Алгоритмы нахождения эйлерова цикла}
\begin{enumerate}
	\item \index{Алгоритм!Флёри} \textbf{Алгоритм Флёри.}
	
	В качестве текущей вершины выберем произвольную.
	\begin{enumerate}
		\item Выбираем ребро, инцидентное текущей вершине.
		Оно не должно быть мостом, если есть другие рёбра, не являющиеся мостами.
		\item Проходим по выбранному ребру и вычёркиваем его.
		Вершина, в~которой теперь находимся,~--- текущая.
		\item Повторяем с шага~(a), пока есть рёбра.
	\end{enumerate}
	
	\item \index{Алгоритм!объединения циклов} \textbf{Алгоритм объединения циклов.}
	\begin{enumerate}
		\item Выбираем произвольную вершину.
		\item Выбираем любое непосещённое ребро и идём по нему.
		\item Повторяем шаг~(b), пока не вернёмся в~начальную вершину.
		\item Получили цикл~$C$.
		Если он не эйлеров, то $\exists u \in C, \ e = \{ u, u' \} \colon u' \notin C$.
		Повторяем шаги~(b)--(c), начиная с вершины~$u$.
		Получили цикл~$C'$, рёбра которого не совпадают с рёбрами~$C$.
		Объединим эти циклы и получим новый.
		Повторяем шаг~(d).
	\end{enumerate}
\end{enumerate}

Цепь называется \textbf{эйлеровым путём}, если она не является циклом и содержит все рёбра графа.

\index{Графы!полуэйлеровы} Граф называется \textbf{полуэйлеровым}, если в~нём есть эйлеров путь.

\begin{theorem}
Связный граф полуэйлеров $\Leftrightarrow$ степени двух вершин нечётны, а остальных~--- чётны.
\end{theorem}
\begin{proof}
\begin{enumerate}
	\item $\Rightarrow$. Пусть в~графе есть эйлеров путь.
	Соединив его концы ребром, получим эйлеров цикл.
	Степени соединённых вершин увеличились каждая на~$1$, значит, они были нечётными, а степени остальных вершин~--- чётными.
	\item $\Leftarrow$. Пусть степени двух вершин нечётны, а остальных~--- чётны.
	Соединим нечётные вершины ребром, тогда можно получить эйлеров цикл.
	Убрав из него добавленное ребро, получим эйлеров путь.
\end{enumerate}
\end{proof}