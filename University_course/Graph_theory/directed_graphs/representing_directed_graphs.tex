\subsection{Способы задания ориентированного графа}
Пронумеруем вершины графа~$G = (V, E)$, т.\,е. зададим биекцию $\varphi \colon V \to \{ 1, 2, \ldots, n \}$, где $n = |V|$, и будем их обозначать $1, 2, \ldots, n$.

\subsubsection{Матрица смежности}
\index{Матрица!смежности} \textbf{Матрицей смежности графа~$G$} называется матрица $A = \|a_{ij}\|_{\begin{smallmatrix}
i = \overline{1,n} \\
j = \overline{1,n}
\end{smallmatrix}}$, где $a_{ij}$ равно числу рёбер~$(i, j)$ в графе.

\begin{theorem}
Если $A = \|a_{ij}\|_{\begin{smallmatrix}
i = \overline{1,n} \\
j = \overline{1,n}
\end{smallmatrix}}$~--- матрица смежности графа~$G$,
$A^k = \|b_{ij}\|_{\begin{smallmatrix}
i = \overline{1,n} \\
j = \overline{1,n}
\end{smallmatrix}}$, то $b_{ij}$ равно числу $(i, j)$-путей длины~$k$.
\end{theorem}
\begin{proofmathind}
	\indbase Для $k = 1$ истинность следует из определения.
	\indstep Пусть теорема верна для $k$, $A^{k+1} = \|c_{ij}\|_{\begin{smallmatrix}
	i = \overline{1,n} \\
	j = \overline{1,n}
	\end{smallmatrix}}$.
	\begin{equation*}
	A^{k+1} = A A^k \Rightarrow c_{ij} = \sum_{l=1}^n a_{il} b_{lj}
	\end{equation*}
	
	Значение выражения $a_{il} b_{lj}$, очевидно, равно числу $(i, j)$-путей длины~$k + 1$, проходящих по дуге~$(i, l)$ (если таких дуг нет в графе, то выражение равно $0$).
	Тогда, суммируя эти выражения по всем~$l$, получим число всех $(i, j)$-путей длины $k + 1$. \indend
\end{proofmathind}

\subsubsection{Матрица инцидентности}
Так же, как и вершины, пронумеруем дуги графа~$G$ и будем их обозначать $1, 2, \ldots, n$.

\index{Матрица!инцидентности} \textbf{Матрицей инцидентности графа~$G$} называется матрица $A = \|a_{ij}\|_{\begin{smallmatrix}
i = \overline{1,n} \\
j = \overline{1,n}
\end{smallmatrix}}$, где
\begin{equation*}
a_{ij} =
\begin{cases}
-1, \ \text{вершина~$i$~--- конец дуги~$j$} \\
0, \ \text{вершина~$i$ не инцидентна дуге~$j$} \\
1, \ \text{вершина~$i$~--- начало дуги~$j$}
\end{cases}
\end{equation*}