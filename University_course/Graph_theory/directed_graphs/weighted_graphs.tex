\subsection{Взвешенные графы}
Ориентированный граф называется \textbf{взвешенным}, если каждой его дуге соответствует число, называемое \textbf{весом дуги}.

\textbf{Весом пути} называется сумма весов входящих в него дуг.

$(u, v)$-путь называется \textbf{кратчайшим}, если он имеет наименьший вес среди всех $(u, v)$-путей.

Наименьший вес среди всех $(u, v)$-путей называется \textbf{длиной пути между вершинами $u$ и $v$} и обозначается $d(u, v)$.

\begin{statement}
\begin{equation*}
d(u, v) \leqslant d(u, k) + d(k, v)
\end{equation*}
причём $d(u, v) = d(u, k) + d(k, v)$ $\Leftrightarrow$ $k$ лежит на одном из кратчайших $(u, v)$-путей.
\end{statement}
\begin{proof}
Пусть $p_1$ и $p_2$~--- кратчайшие $(u, k)$-путь и $(k, v)$-путь соответственно.
Тогда $p_1 \cup p_2$~--- $(u, v)$-путь.
Значит, либо этот путь кратчайший и его вес минимален (в таком случае $d(u, v) = d(u, k) + d(k, v)$), либо его вес больше, чем $d(u, v)$ (в этом случае ни один из кратчайших $(u, v)$-путей не проходит через вершину~$k$).
\end{proof}