\subsection{Связность ориентированных графов}
Ориентированный граф называется \textbf{сильно связным}, если для любых его вершин $u$ и $v$ существуют $(u, v)$-путь и $(v, u)$-путь.

Ориентированный граф называется \textbf{слабо связным}, если связно его основание.

\begin{theorem}
Связный неориентированный граф обладает сильно связной ориентацией $\Leftrightarrow$ он не содержит мостов.
\end{theorem}
\begin{proof}
\begin{enumerate}
	\item $\Rightarrow$. Если граф содержит мост~$\{ u, v \}$, то при его ориентации можно получить либо дугу~$(u, v)$, либо дугу~$(v, u)$.
	В таком случае одна из компонент связности, соединяемых мостом, будет недостижима из другой.
	
	\item $\Leftarrow$. Любое ребро принадлежит некоторому циклу~$C$, так как оно не является мостом.
	Ориентируем все рёбра цикла в одну сторону.
	Если остались другие рёбра, то в силу связности графа можно получить ещё один цикл, часть которого является частью цикла~$C$.
	Ориентируем все рёбра полученного цикла в одну сторону, не изменяя уже ориентированные рёбра.
	Повторяя, ориентируем все рёбра графа.
	
	Для любых двух вершин $u$ и $v$ исходного графа существует маршрут $(u = v_1, e_1, \ldots, e_{k-1}, v_k = v)$.
	Чтобы получить $(u, v)$-путь, идём по дугам $(v_1, v_2)$, $(v_2, v_3)$ и т.\,д.
	Если дуги $(v_i, v_{i+1})$ нет, то есть дуга $(v_{i+1}, v_i)$, состоящая в некотором цикле, пройдя по которому, можно из вершины~$v_i$ попасть в вершину~$v_{i+1}$.
	
	Аналогично получим $(v, u)$-путь.
\end{enumerate}
\end{proof}

\index{Компонента связности} \textbf{Компонентой сильной связности} ориентированного графа называется максимальный относительно включения сильно связный подграф.
Аналогично определяется \textbf{компонента слабой связности}.

\textbf{Конденсацией ориентированного графа} называется ориентированный граф, вершинами которого являются компоненты сильной связности исходного графа, а дуга между вершинами показывает наличие пути между вершинами компонент.

\begin{statement}
Конденсация не содержит контуров.
\end{statement}
\begin{proofcontra}
Предположим, что в конденсации существует контур.
Тогда очевидно, что вершины различных компонент, входящих в него, взаимодостижимы, а значит, лежат в одной компоненте сильной связности.
Противоречие.
\end{proofcontra}