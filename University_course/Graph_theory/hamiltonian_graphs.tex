\section{Гамильтоновы графы}
Простой цикл, содержащий все вершины графа, называется \textbf{гамильтоновым}.

\index{Графы!гамильтоновы} Граф называется \textbf{гамильтоновым}, если в~нём есть гамильтонов цикл.

\index{Теорема!Оре}
\begin{theorem}[Оре]
Если в~графе с $n \geqslant 3$~вершинами для любых двух несмежных вершин $u$ и $v$ $\deg u + \deg v \geqslant n$, то граф гамильтонов.
\end{theorem}
\begin{proof}
\begin{enumerate}
	\item Докажем методом от противного, что граф связный.
	Пусть он несвязный, тогда в~нём найдутся хотя~бы две компоненты связности $G_1(V_1, E_1)$ и $G_2(V_2, E_2)$.
	Пусть $u \in V_1$, $v \in V_2$. $u$ и $v$ несмежные, тогда
	\begin{equation*}
	\deg u \leqslant |V_1| - 1, \ \deg v \leqslant |V_2| - 1 \opbr\Rightarrow \deg u + \deg v \opbr\leqslant |V_1| + |V_2| - 2 \opbr\leqslant n - 2
	\end{equation*}
	
	Противоречие с условием.
	
	\item Докажем, что граф гамильтонов.
	Выберем цепь~$W = (v_0, e_0, v_1, \ldots, e_{k-1}, v_k)$ наибольшей длины.
	В~ней содержатся все вершины, соседние с~$v_0$ или с~$v_k$.
	Т.\,о., среди вершин $v_1, \ldots, v_k$ $\deg v_0$ соседних с~$v_0$.
	Аналогично для $v_k$.
	
	$\deg v_0 + \deg v_k \geqslant n$, тогда найдутся $v_i$ и $v_{i+1}$ такие, что $v_i$ соседняя с~$v_k$, а $v_{i+1}$~--- с~$v_0$.
	
	Докажем, что $(v_{i+1}, e_{i+1}, \ldots, v_k, e, v_i, e_{i-1}, v_{i-1}, \ldots, e_0, v_0, e', v_{i+1})$~--- гамильтонов цикл, методом от противного.
	Предположим обратное, тогда есть вершина~$u$, не входящая в~цикл, и существует $(v_0, u)$-маршрут.
	Значит, существует ребро, инцидентное одной из вершин цикла, но не входящее в~него, и можно получить более длинную цепь.
	Противоречие, значит, $G$~--- гамильтонов граф.
\end{enumerate}
\end{proof}

\index{Теорема!Дирака}
\begin{theorem}[Дирака]
\label{th:Dirac}
Если в графе~$G = (V, E)$ с $n \geqslant 3$~вершинами $\forall u \in V \ \deg u \geqslant \frac{n}2$, то граф гамильтонов.
\end{theorem}
\begin{proof}
Пусть $u$, $v$~--- несвязные вершины в~$G$, тогда $\deg u \geqslant \frac{n}2 \lAnd \deg v \geqslant \frac{n}2 \opbr\Rightarrow \deg u \opbr+ \deg v \opbr\geqslant n \Rightarrow$ по теореме Оре $G$ гамильтонов.
\end{proof}

Цепь называется \textbf{гамильтоновым путём}, если она не является циклом и содержит все вершины графа.

\index{Графы!полугамильтоновы} Граф называется \textbf{полугамильтоновым}, если в нём есть гамильтонов путь.