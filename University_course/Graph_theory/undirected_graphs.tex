\section{Неориентированные графы}
\index{Графы!неориентированные} \textbf{Неориентированным графом} называется пара $(V, E)$, где $V$~--- непустое конечное множество вершин графа, $E$~--- совокупность множеств $\{ u, v \}$, где $u, v \in V$.
Элементы~$V$ называются \textbf{вершинами графа}.
Элементы~$E$ называются \textbf{рёбрами графа}.

В~этом разделе будем рассматривать только неориентированные графы.

На рисунках вершины графа изображают точками, а рёбра $e = \{ u, v \}$~--- кривыми, соединяющими точки, которые изображают вершины $u$ и $v$.

Если $e = \{ u, v \} \in E$, то говорят, что:
\begin{itemize}
	\item ребро~$e$ соединяет вершины $u$ и $v$;
	\item $u$ и $v$~--- концы ребра~$e$;
	\item ребро~$e$ инцидентно вершинам $u$ и $v$;
	\item вершины $u$ и $v$ инцидентны ребру~$e$.
\end{itemize}

Вершины называются \textbf{соседними}, или \textbf{смежными}, если их соединяет ребро, иначе~--- \textbf{несоседними}, или \textbf{несмежными}.

\index{deg} Число рёбер, инцидентных вершине~$u$, называется \textbf{степенью вершины} и обозначается $\deg u$.

Если степень вершины равна $0$, то она называется \textbf{изолированной}, а если $1$~--- то \textbf{висячей}.

\begin{floatingfigure}[r]{30mm} \centering
\noindent
$\begin{xy} /r0.4mm/:
(30, 60) = "1" *{\bullet}; % (r, 2r)
(58.532, 39.271) = "2" *{\bullet}; % (r * (1 + cos (pi/10)), r * (1 + sin (pi/10)))
(47.634, 5.729) = "3" *{\bullet}; % (r * (1 + cos (3pi/10)), r * (1 - sin (3pi/10)))
(12.366, 5.729) = "4" *{\bullet}; % (r * (1 - cos (3pi/10)), r * (1 - sin (3pi/10)))
(1.468, 39.271) = "5" *{\bullet}; % (r * (1 - cos (pi/10)), r * (1 + sin (pi/10)))
"5"; "1" **@{-}; "2" **@{-}; "3" **@{-}; "4" **@{-}; "5" **@{-};
"2" **@{-}; "4" **@{-}; "1" **@{-}; "3" **@{-}; "5" **@{-};
\end{xy}$
\caption{Граф $K_5$}
\end{floatingfigure}

\begin{lemma}[о~рукопожатиях]
\index{Лемма!о~рукопожатиях}
\begin{equation*}
\sum_{u \in V} \deg u = 2|E|
\end{equation*}
где $(V, E)$~--- граф.
\end{lemma}
\begin{proof}
Достаточно заметить, что каждое ребро увеличивает степени двух некоторых вершин на~$1$.
\end{proof}

\index{Петля} Ребро вида $e = \{ u, u \}$ называется \textbf{петлёй}.

Рёбра, инцидентные одним и тем~же вершинам, называются \textbf{кратными}.

Граф называется \textbf{простым}, если он не содержит петель и кратных рёбер.

Граф, в~котором любые две вершины соединены ребром, называется \textbf{полным} и обозначается $K_n$, где $n$~--- число вершин в~нём.

Графы $G_1 = (V_1, E_1)$ и $G_2 = (V_2, E_2)$ называются \textbf{изоморфными}, если существует биекция~$\varphi \colon V_1 \to V_2$ такая, что
$\{ u, v \} \in E_1 \Leftrightarrow \{ \varphi(u), \varphi(v) \} \in E_2$, иначе~--- \textbf{неизоморфными}.
$\varphi$ называется \textbf{изоморфизмом}.

\index{Маршрут} \textbf{Маршрутом} в графе называется последовательность вершин и рёбер вида
$(v_1, e_1, v_2, \ldots, e_k, v_{k+1})$, где $e_i \opbr= \{ v_i, v_{i+1} \}$.

\index{Цепь} Маршрут, в~котором все рёбра различны, называется \textbf{цепью}.

Цепь, в~которой все вершины, за исключением, может быть, первой и последней, различны, называется \textbf{простой}.

Маршрут, в~котором первая и последняя вершины совпадают, называется \textbf{замкнутым}.

\index{Цикл} Замкнутая цепь называется \textbf{циклом}.

Маршрут, соединяющий вершины $u$ и $v$, называется \textbf{$(u, v)$-маршрутом}.

\begin{lemma}
\label{lemma:walk_contains_simple_chain}
$(u, v)$-маршрут содержит $(u, v)$-простую цепь.
\end{lemma}
\begin{proof}
Пусть $(u = v_1, e_1, v_2, \ldots, e_k, v_{k+1} = v)$~--- не простая цепь, тогда $\exists i < j \colon v_i = v_j$.
Уберём из маршрута подпоследовательность $(e_i, v_{i+1}, \ldots, e_{j-1}, v_j)$ и получим маршрут, в~котором совпадающих вершин на одну меньше.
Повторяя, получим простую цепь, являющуюся частью данного маршрута.
\end{proof}

\begin{lemma}
\label{lemma:cycle_contains_simple_one}
Любой цикл содержит простой цикл, причём каждая вершина и ребро цикла принадлежат некоторому простому циклу.
\end{lemma}
\begin{proof}
Пусть $(u = v_1, e_1, v_2, \ldots, e_k, v_{k+1} = u)$~--- не простой цикл, тогда $\exists i < j \colon v_i = v_j$.
Уберём из цикла подпоследовательность $(e_i, v_{i+1}, \ldots, e_{j-1}, v_j)$ и получим цикл, в~котором совпадающих вершин на одну меньше.
Повторяя, получим простой цикл, являющийся частью данного цикла.

Заметим, что подпоследовательность $(v_i, e_i, v_{i+1}, \ldots, e_{j-1}, v_j = v_i)$ также является циклом, из которого можно получить простой цикл.
Значит, некоторые вершины и рёбра этой подпоследовательности принадлежат простому циклу, остальные же снова образуют некоторые циклы, из которых можно получить простые.
Продолжая рассуждать таким образом, приходим к выводу, что любая вершина и ребро исходного цикла принадлежат некоторому простому циклу.
\end{proof}
	
\begin{lemma}
\label{lemma:existence_of_simple_cycle}
Если в~графе есть две различные простые цепи, соединяющие одни и те~же вершины, то в~этом графе есть простой цикл.
\end{lemma}
\begin{proof}
Пусть $(u = v_1, e_1, v_2, \ldots, e_n, v_{n+1} = v)$, $(u = v_1', e_1', v_2', \ldots, e_m', v_{m+1}' = v)$~--- простые цепи.
Найдём наименьшее~$i \colon e_i \neq e_i'$, тогда $(v_i, e_i, v_{i+1}, \ldots, e_n, v_{n+1} = v_{m+1}', e_m', \ldots, e_i', v_i' = v_i)$~--- цикл, значит, можно получить простой цикл.
\end{proof}