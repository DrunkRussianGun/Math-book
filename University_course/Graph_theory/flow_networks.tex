\section{Транспортные сети}
\index{Транспортная сеть} \index{Источник} \index{Сток} \textbf{Транспортной сетью} называется ориентированный граф, в котором выделены две вершины, одна из которых называется \textbf{источником} и обозначается~$s$, а другая~--- \textbf{стоком} и обозначается~$t$.
Источник имеет нулевую полустепень захода, а сток~--- нулевую полустепень исхода.
\index{Пропускная способность} Кроме того, каждой дуге графа сопоставлено положительное число, называемое \textbf{пропускной способностью}, т.\,е. задана функция~$q \colon E \to R^+$.

\index{Поток} \textbf{Потоком в сети~$(V, E)$} называется функция~$p \colon E \to R^+$ такая, что
\begin{itemize}
	\item $\forall (i, j) \in E \ p(i, j) \leqslant q(i, j)$;
	\item $\forall k \in V \setminus \{ s, t \} \ \sum\limits_{(i, k) \in E} p(i, k) = \sum\limits_{(k, j) \in E} p(k, j)$.
\end{itemize}

\index{Разрез} \textbf{Разрезом в сети~$(V, E)$} называется разбиение множества вершин на два подмножества $V$ и $\overline V$ таких, что $s \in X$, $t \in \overline X$.

\textbf{Пропускной способностью разреза} называется сумма $\sum\limits_{i \in X, j \in \overline X} q(i, j)$.

\textbf{Величиной потока~$p$ через разрез~$(X, \overline X)$} называется сумма $\sum\limits_{i \in X, j \in \overline X} p(i, j) -
\sum\limits_{i \in X, j \in \overline X} p(j, i)$ и обозначается $p(X, \overline X)$.

\begin{lemma}
Величина потока через любой разрез одна и та же.
\end{lemma}
\begin{proof}
Пусть $(X, \overline X)$~--- разрез сети~$(V, E)$, причём $s \neq j \in X$.
Рассмотрим разрез~$(X \setminus \{ j \}, \overline X \cup \{ j \})$.
\begin{equation*}
p(X \setminus \{ j \}, \overline X \cup \{ j \}) = p(X, \overline X) +
\sum_{i \in X} p(i, j) + \sum_{i \in \overline X} p(i, j) -
\sum_{i \in X} p(j, i) - \sum_{i \in \overline X} (j, i) =
\end{equation*}
\begin{equation*}
= p(X, \overline X) + \sum_{i \in V} p(i, j) - \sum_{i \in V} p(j, i) =
p(X, \overline X)
\end{equation*}
\end{proof}

Т.\,о., \textbf{величиной потока~$p$} называется его величина через любой разрез и обозначается $|p|$.

\begin{consequent}
$\sum\limits_{i \in V} p(s, i) = \sum\limits_{i \in V} p(i, t)$.
\end{consequent}

\begin{lemma}
Для любого потока~$p$ верно $\forall (X, \overline X) \ p(X, \overline X) \leqslant q(X, \overline X)$.
\end{lemma}
\begin{proof}
\begin{equation*}
p(X, \overline X) =
\sum_{i \in X, j \in \overline X} p(i, j) - \sum_{i \in X, j \in \overline X} p(j, i) \leqslant
\sum_{i \in X, j \in \overline X} p(i, j) \leqslant
\sum_{i \in X, j \in \overline X} q(i, j) =
q(X, \overline X)
\end{equation*}
\end{proof}

Разрез с минимальной пропускной способностью называется \textbf{минимальным разрезом}.

Поток максимальной величины называется \textbf{максимальным потоком}.

\begin{lemma}
Если $p_0(X_0, \overline X_0) = q(X_0, \overline X_0)$, то $(X_0, \overline X_0)$~--- минимальный разрез, а $p_0$~--- максимальный поток.
\end{lemma}
\begin{proof}
Пусть $p$~--- произвольный поток, $(X, \overline X)$~--- произвольный разрез.
\begin{itemize}
	\item $|p_0| = q(X_0, \overline X_0) \geqslant |p|$
	\item $q(X_0, \overline X_0) = |p_0| \leqslant q(X, \overline X)$
\end{itemize}
\end{proof}

\subsection{Алгоритмы нахождения максимального потока}
\subsubsection{Алгоритм Форда"--~Фалкерсона}
\index{Алгоритм!Форда---Фалкерсона}
Изначально считаем, что $\forall i, j \in V \ p(i, j) = 0$.
Также пометим источник~$s$ меткой~$(-, \infty)$.
\begin{enumerate}
	\item Пусть вершина~$i$ имеет метку~$(x^\pm, \varepsilon)$.
	\begin{itemize}
		\item Если $(i, j) \in E$ и $q(i, j) - p(i, j) > 0$, то присваиваем $j$ метку~$(i^+, \min \{\varepsilon, q(i, j) - p(i, j)\})$.
		\item Если $(j, i) \in E$ и $p(i, j) > 0$, то присваиваем $j$ метку~$(i^-, \min \{\varepsilon, p(i, j)\})$.
	\end{itemize}
	
	Если сток не достигнут, то поток максимален, иначе сток имеет метку$(x^+, \delta)$.
	
	\item Идём в обратном направлении.
	Если вершина~$u$ имеет метку~$(x^+, \varepsilon)$, то увеличиваем поток~$p(i, u)$ на~$\delta$, а если $(x^-, \varepsilon)$, то уменьшаем $p(i, u)$ на~$\delta$.
	Т.\,о., величина потока увеличилась на~$\delta > 0$.
\end{enumerate}

Повторяя, увеличим поток до максимума, т.\,е. в какой-то момент не сможем пометить следующую вершину.
Получим некоторый разрез~$(X, \overline X)$.

\begin{lemma}
Найденный поток максимален, а разрез минимален.
\end{lemma}
\begin{proof}
\begin{equation*}
|p| = \sum_{i \in X, j \in \overline X} p(i, j) - \sum_{i \in X, j \in \overline X} p(j, i)
\end{equation*}

Т.\,к. из $i$ нельзя попасть в~$j$ и из~$j$ нельзя попасть в~$i$, где $i \in X$, $j \in \overline X$, то $p(i, j) = q(i, j)$, $p(j, i) = 0$, тогда
\begin{equation*}
|p| = \sum_{i \in X, j \in \overline X} q(i, j)
\end{equation*}
\end{proof}

\subsubsection{Алгоритм Эдмондса"--~Карпа}
\index{Алгоритм!Эдмондса---Карпа}
Путь~$(s = v_1, \ldots, v_{k+1} = t)$ в сети~$(V, E)$ называется \textbf{увеличивающим}, если $\forall i \in \{ 1, \ldots, k \} \ \varepsilon(v_i, v_{i+1}) > 0$, где $\varepsilon(v_i, v_{i+1}) = \begin{cases}
q(v_i, v_{i+1}) - p(v_i, v_{i+1}), \ (v_i, v_{i+1}) \in E \\
p(v_{i+1}, v_i), \ (v_{i+1}, v_i) \in E
\end{cases}$

Введём $\delta = \min\limits_{0 \leqslant i \leqslant k} \varepsilon(v_i, v_{i+1})$, тогда новое значение потока в сети равно
$\begin{cases}
p(v_i, v_{i+1}) + \delta, \ (v_i, v_{i+1}) \in E \\
p(v_{i+1}, v_i) - \delta, \ (v_{i+1}, v_i) \in E
\end{cases}$

Рассмотрим случай, когда до~$t$ не существует увеличивающего пути.
Пусть $X$~--- множество вершин, до которых существует увеличивающий путь, $u \in X$, $v \notin X$.
Если $(u, v) \in E$, то $q(u, v) = p(u, v)$, а если $(v, u) \in E$, то $p(v, u) = 0$, тогда
\begin{equation*}
p(X, \overline X) =
\sum_{u \in V, v \in \overline X} p(u, v) - \sum_{u \in V, v \in \overline X} p(v, u) =
\sum_{u \in V, v \in \overline X} q(u, v)
\end{equation*}

\begin{lemma}
В ходе работы алгоритма Эдмондса"--~Карпа кратчайший $(s, t)$"=путь не уменьшается.
\end{lemma}
\begin{proofcontra}
Рассмотрим самую близкую к~$s$ вершину~$v$, для которой кратчайший путь~$(s, \ldots, u, v)$ уменьшается, тогда для вершины~$u$ кратчайший $(s, u)$"=путь не уменьшается.
Пусть $d_u$ и $d_v$~--- длины кратчайших $(s, u)$- и $(s, v)$"=путей соответственно на предыдущем шаге, а $d_u'$ и $d_v'$~--- на текущем.
\begin{equation*}
d_v > d_v' = d_u' + 1 \geqslant d_u + 1 \Rightarrow d_v \geqslant d_u + 2
\end{equation*}

Значит, на предыдущем шаге не было дуги~$(u, v)$, тогда не было и кратчайшего $(s, v)$"=пути.
Противоречие.
\end{proofcontra}

Назовём дугу~$(v_i, v_{i+1})$ \textbf{критической}, если $\varepsilon(v_i, v_{i+1}) = \delta$.

\begin{lemma}
Каждая дуга может быть критической на увеличивающем пути порядка $\frac{|V|}2$~раз.
\end{lemma}
\begin{proof}
Пусть дуга~$(u, v)$ критическая на шагах $t_1$ и $t_2$.
Если она была использована как прямая два раза, то между этими использованиями она должна была быть использована как обратная (на шаге~$t_3$), тогда
\begin{equation*}
d_v(t_2) = d_u(t_2) + 1 \geqslant
d_u(t_3) + 1 = d_v(t_3) + 2 \geqslant
d_v(t_1) + 2
\end{equation*}
\end{proof}