\section{Транспортные сети}
\index{Транспортная сеть} \index{Источник} \index{Сток} \textbf{Транспортной сетью} называется ориентированный граф, в котором выделены две вершины, одна из которых называется \textbf{источником} и обозначается~$s$, а другая~--- \textbf{стоком} и обозначается~$t$.
Источник имеет нулевую полустепень захода, а сток~--- нулевую полустепень исхода.
\index{Пропускная способность} Кроме того, каждой дуге графа сопоставлено положительное число, называемое \textbf{пропускной способностью}, т.\,е. задана функция~$q \colon E \to R^+$.

\index{Поток} \textbf{Потоком в сети~$(V, E)$} называется функция~$p \colon E \to R^+$ такая, что
\begin{itemize}
	\item $\forall (i, j) \in E \ p(i, j) \leqslant q(i, j)$;
	\item $\displaystyle \forall k \in V \setminus \{ s, t \} \ \sum_{(i, k) \in E} p(i, k) = \sum_{(k, j) \in E} p(k, j)$, где $s$~--- источник, $t$~--- сток.
\end{itemize}

\index{Разрез} \textbf{Разрезом в сети~$(V, E)$} называется разбиение множества вершин на два подмножества $V$ и $\overline V$ таких, что $s \in X$, $t \in \overline X$, где $s$~--- источник, $t$~--- сток.

\textbf{Пропускной способностью разреза} называется сумма $\displaystyle \sum_{\begin{smallmatrix}
i \in X \\
j \in \overline X
\end{smallmatrix}} q(i, j)$.

\textbf{Потоком разреза} называется сумма $\displaystyle \sum_{\begin{smallmatrix}
i \in X \\
j \in \overline X
\end{smallmatrix}} p(i, j) -
\sum_{\begin{smallmatrix}
i \in X \\
j \in \overline X
\end{smallmatrix}} p(j, i)$.