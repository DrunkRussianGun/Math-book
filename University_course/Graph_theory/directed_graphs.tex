\section{Ориентированные графы}
\index{Графы!ориентированные} \textbf{Ориентированным графом} называется пара~$(V, E)$, где $V$~--- непустое конечное множество, $E$~--- совокупность элементов множества~$V^2$.
Элементы~$V$ называются \textbf{вершинами графа}.
Элементы~$E$ называются \textbf{дугами графа}.

В~этом разделе будем рассматривать только ориентированные графы, если не оговорено иное.

На рисунках ориентированные графы изображаются так же, как неориентированные, за тем исключением, что на дуги дополнительно наносятся стрелки, направленные от начальной вершины к конечной.

Если $e = (u, v) \in E$, то говорят, что:
\begin{itemize}
	\item дуга~$e$ выходит из вершины~$u$ и входит в вершину~$v$;
	\item $u$~--- начало дуги~$e$, а $v$~--- её конец;
	\item дуга~$e$ инцидентна вершинам $u$ и $v$;
	\item вершины $u$ и $v$ инцидентны дуге~$e$.
\end{itemize}

\index{deg} Количество выходящих из вершины~$v$ дуг называется \textbf{полустепенью исхода вершины} и обозначается $\deg_+ v$.

Количество входящих в вершину~$v$ дуг называется \textbf{полустепенью захода вершины} и обозначается $\deg_- v$.

Количество инцидентных вершине~$v$ дуг называется \textbf{степенью вершины} и обозначается $\deg v$.
Очевидно, что $\deg v = \deg_+ v + \deg_- v$.

\begin{statement}
\begin{equation*}
\sum_{v \in V} \deg_+ v = \sum_{v \in V} \deg_- v = |E|
\end{equation*}
где $(V, E)$~--- граф.
\end{statement}
\begin{proof}
Достаточно заметить, что каждая дуга увеличивает полустепень исхода некоторой вершины на $1$ и полустепень захода некоторой вершины на $1$.
\end{proof}

\index{Петля} Дуга~$e = (u, u)$ называется \textbf{петлёй}.

Если в графе есть несколько дуг~$(u, v)$, то они называются \textbf{кратными}.

Дуги $(u, v)$ и $(v, u)$ называются \textbf{противоположно направленными}.

Граф называется \textbf{простым}, если в нём нет пётель и кратных дуг.

Граф~$(V_1, E_1)$ называется \textbf{подграфом графа~$(V, E)$}, если $V_1 \subseteq V \lAnd E_1 \subseteq E$.

Неориентированный граф, полученный из ориентированного графа~$G$ заменой дуг на рёбра, называется \textbf{основанием графа~$G$}. 

Графы $G_1 = (V_1, E_1)$ и $G_2 = (V_2, E_2)$ называются \textbf{изоморфными}, если существует биекция~$\varphi \colon V_1 \to V_2$ такая, что
$(u, v) \in E_1 \Leftrightarrow (\varphi(u), \varphi(v)) \in E_2$, иначе~--- \textbf{неизоморфными}.

\index{Путь} \textbf{Путём} в графе называется последовательность $(v_1, e_1, v_2, e_2, \ldots, v_{k-1}, e_k, v_k)$ его вершин и дуг такая, что $e_i \opbr= (v_i, v_{i+1})$.

Путь называется \textbf{простым}, если в нём все вершины, кроме, возможно, первой и последней, различны.

Путь называется \textbf{замкнутым}, если в нём первая и последняя вершины совпадают.

\index{Контур} Замкнутый путь называется \textbf{контуром}.

Путь, соединяющий вершины $u$ и $v$, называется \textbf{$(u, v)$-путём}.

Если в графе существует $(u, v)$-путь, то говорят, что вершина~$v$ \textbf{достижима} из вершины~$u$.
Если также существует $(v, u)$-путь, то говорят, что вершины $u$ и $v$ \textbf{взаимодостижимы}.

\begin{lemma}
Любой путь содержит простой путь.
\end{lemma}%
Доказательство аналогично доказательству леммы~\ref{lemma:walk_contains_simple_chain}.

\begin{lemma}
Любой контур содержит простой контур, причём каждая вершина и дуга контура принадлежат некоторому простому контуру.
\end{lemma}%
Доказательство аналогично доказательству леммы~\ref{lemma:cycle_contains_simple_one}.