\section{Ориентированные графы}
\index{Графы!ориентированные} \textbf{Ориентированным графом} называется пара~$(V, E)$, где $V$~--- непустое конечное множество, $E$~--- совокупность элементов множества~$V^2$.
Элементы~$V$ называются \textbf{вершинами графа}.
Элементы~$E$ называются \textbf{дугами графа}.

На рисунках ориентированные графы изображаются так же, как неориентированные, за тем исключением, что на дуги дополнительно наносятся стрелки, направленные от начальной вершины к конечной.

Если $e = (u, v) \in E$, то говорят, что:
\begin{itemize}
	\item дуга~$e$ выходит из вершины~$u$ и входит в вершину~$v$;
	\item $u$~--- начало дуги~$e$, а $v$~--- её конец;
	\item дуга~$e$ инцидентна вершинам $u$ и $v$;
	\item вершины $u$ и $v$ инцидентны дуге~$e$.
\end{itemize}

\index{deg} Количество выходящих из вершины~$v$ дуг называется \textbf{полустепенью исхода вершины} и обозначается $\deg_+ v$.

Количество входящих в вершину~$v$ дуг называется \textbf{полустепенью захода вершины} и обозначается $\deg_- v$.

Количество инцидентных вершине~$v$ дуг называется \textbf{степенью вершины} и обозначается $\deg v$.
Очевидно, что $\deg v = \deg_+ v + \deg_- v$.

\begin{statement}
\begin{equation*}
\sum_{v \in V} \deg_+ v = \sum_{v \in V} \deg_- v = |E|
\end{equation*}
где $(V, E)$~--- граф.
\end{statement}
\begin{proof}
Достаточно заметить, что каждая дуга увеличивает полустепень исхода некоторой вершины на $1$ и полустепень захода некоторой вершины на $1$.
\end{proof}

\index{Петля} Дуга~$e = (u, u)$ называется \textbf{петлёй}.

Если в графе есть несколько дуг~$(u, v)$, то они называются \textbf{кратными}.

Дуги $(u, v)$ и $(v, u)$ называются \textbf{противоположно направленными}.

Граф называется \textbf{простым}, если в нём нет пётель и кратных дуг.

Граф~$(V_1, E_1)$ называется \textbf{подграфом графа~$(V, E)$}, если $V_1 \subseteq V \lAnd E_1 \subseteq E$.

Неориентированный граф, полученный из ориентированного графа~$G$ заменой дуг на рёбра, называется \textbf{основанием графа~$G$}. 

Графы $G_1 = (V_1, E_1)$ и $G_2 = (V_2, E_2)$ называются \textbf{изоморфными}, если существует биекция~$\varphi \colon V_1 \to V_2$ такая, что
$(u, v) \in E_1 \Leftrightarrow (\varphi(u), \varphi(v)) \in E_2$, иначе~--- \textbf{неизоморфными}.

\index{Путь} \textbf{Путём} в графе называется последовательность $(v_1, e_1, v_2, e_2, \ldots, v_{k-1}, e_k, v_k)$ его вершин и дуг такая, что $e_i \opbr= (v_i, v_{i+1})$.

Путь называется \textbf{простым}, если в нём все вершины, кроме, возможно, первой и последней, различны.

Путь называется \textbf{замкнутым}, если в нём первая и последняя вершины совпадают.

\index{Контур} Замкнутый путь называется \textbf{контуром}.

Путь, соединяющий вершины $u$ и $v$, называется \textbf{$(u, v)$"=путём}.

Если в графе существует $(u, v)$"=путь, то говорят, что вершина~$v$ \textbf{достижима} из вершины~$u$.
Если также существует $(v, u)$"=путь, то говорят, что вершины $u$ и $v$ \textbf{взаимодостижимы}.

\begin{lemma}
Любой путь содержит простой путь.
\end{lemma}%
Доказательство аналогично доказательству леммы~\ref{lemma:walk_contains_simple_chain}.

\begin{lemma}
Любой контур содержит простой контур, причём каждая вершина и дуга контура принадлежат некоторому простому контуру.
\end{lemma}%
Доказательство аналогично доказательству леммы~\ref{lemma:cycle_contains_simple_one}.

\subsection{Связность ориентированных графов}
Ориентированный граф называется \textbf{сильно связным}, если для любых его вершин $u$ и $v$ существуют $(u, v)$"=путь и $(v, u)$"=путь.

Ориентированный граф называется \textbf{слабо связным}, если связно его основание.

\index{Теорема!Роббинса}
\begin{theorem}[Роббинса]
Связный неориентированный граф обладает сильно связной ориентацией $\Leftrightarrow$ он не содержит мостов.
\end{theorem}
\begin{proof}
\begin{enumerate}
	\item $\Rightarrow$. Если граф содержит мост~$\{ u, v \}$, то при его ориентации можно получить либо дугу~$(u, v)$, либо дугу~$(v, u)$.
	В таком случае одна из компонент связности, соединяемых мостом, будет недостижима из другой.
	
	\item $\Leftarrow$. Любое ребро принадлежит некоторому циклу~$C$, так как оно не является мостом.
	Ориентируем все рёбра цикла в одну сторону.
	Если остались другие рёбра, то в силу связности графа можно получить ещё один цикл, часть которого является частью цикла~$C$.
	Ориентируем все рёбра полученного цикла в одну сторону, не изменяя уже ориентированные рёбра.
	Повторяя, ориентируем все рёбра графа.
	
	Для любых двух вершин $u$ и $v$ исходного графа существует маршрут $(u = v_1, e_1, \ldots, e_{k-1}, v_k = v)$.
	Чтобы получить $(u, v)$"=путь, идём по дугам $(v_1, v_2)$, $(v_2, v_3)$ и т.\,д.
	Если дуги $(v_i, v_{i+1})$ нет, то есть дуга $(v_{i+1}, v_i)$, состоящая в некотором цикле, пройдя по которому, можно из вершины~$v_i$ попасть в вершину~$v_{i+1}$.
	
	Аналогично получим $(v, u)$"=путь.
\end{enumerate}
\end{proof}

\index{Компонента связности} \textbf{Компонентой сильной связности} ориентированного графа называется максимальный относительно включения сильно связный подграф.
Аналогично определяется \textbf{компонента слабой связности}.

\textbf{Конденсацией ориентированного графа} называется ориентированный граф, вершинами которого являются компоненты сильной связности исходного графа, а дуга между вершинами показывает наличие пути между вершинами компонент.

\begin{statement}
Конденсация не содержит контуров.
\end{statement}
\begin{proofcontra}
Предположим, что в конденсации существует контур.
Тогда очевидно, что вершины различных компонент, входящих в него, взаимодостижимы, а значит, лежат в одной компоненте сильной связности.
Противоречие.
\end{proofcontra}

\subsection{Способы задания ориентированного графа}
Пронумеруем вершины графа~$G = (V, E)$, т.\,е. зададим биекцию $\varphi \colon V \to \{ 1, 2, \ldots, n \}$, где $n = |V|$, и будем их обозначать $1, 2, \ldots, n$.

\subsubsection{Матрица смежности}
\index{Матрица!смежности} \textbf{Матрицей смежности графа~$G$} называется матрица $A = \|a_{ij}\|_{\begin{smallmatrix}
i = \overline{1,n} \\
j = \overline{1,n}
\end{smallmatrix}}$, где $a_{ij}$ равно числу рёбер~$(i, j)$ в графе.

\begin{theorem}
Если $A = \|a_{ij}\|_{\begin{smallmatrix}
i = \overline{1,n} \\
j = \overline{1,n}
\end{smallmatrix}}$~--- матрица смежности графа~$G$,
$A^k = \|b_{ij}\|_{\begin{smallmatrix}
i = \overline{1,n} \\
j = \overline{1,n}
\end{smallmatrix}}$, то $b_{ij}$ равно числу $(i, j)$-путей длины~$k$.
\end{theorem}
\begin{proofmathind}
	\indbase Для $k = 1$ истинность следует из определения.
	\indstep Пусть теорема верна для $k$, $A^{k+1} = \|c_{ij}\|_{\begin{smallmatrix}
	i = \overline{1,n} \\
	j = \overline{1,n}
	\end{smallmatrix}}$.
	\begin{equation*}
	A^{k+1} = A A^k \Rightarrow c_{ij} = \sum_{l=1}^n a_{il} b_{lj}
	\end{equation*}
	
	Значение выражения $a_{il} b_{lj}$, очевидно, равно числу $(i, j)$"=путей длины~$k + 1$, проходящих по дуге~$(i, l)$ (если таких дуг нет в графе, то выражение равно $0$).
	Тогда, суммируя эти выражения по всем~$l$, получим число всех $(i, j)$"=путей длины $k + 1$. \indend
\end{proofmathind}

\subsubsection{Матрица инцидентности}
Так же, как и вершины, пронумеруем дуги графа~$G$ и будем их обозначать $1, 2, \ldots, m$.

\index{Матрица!инцидентности} \textbf{Матрицей инцидентности графа~$G$} называется матрица $A = \|a_{ij}\|_{\begin{smallmatrix}
i = \overline{1,n} \\
j = \overline{1,m}
\end{smallmatrix}}$, где
\begin{equation*}
a_{ij} =
\begin{cases}
-1, \ \text{вершина~$i$~--- конец дуги~$j$} \\
0, \ \text{вершина~$i$ не инцидентна дуге~$j$} \\
1, \ \text{вершина~$i$~--- начало дуги~$j$}
\end{cases}
\end{equation*}

\subsection{Взвешенные графы}
Граф~$(V, E)$ называется \textbf{взвешенным}, если задана функция~$w \colon E \to R^+$, называемая \textbf{весом}.

\textbf{Весом дуги~$e \in E$} называется $w(e)$.

\textbf{Весом}, или \textbf{длиной}, \textbf{пути} называется сумма весов входящих в него дуг.

$(u, v)$"=путь называется \textbf{кратчайшим}, если он имеет наименьший вес среди всех $(u, v)$"=путей.

Наименьший вес среди всех $(u, v)$"=путей называется \textbf{расстоянием между вершинами $u$ и $v$} и обозначается $d(u, v)$.

\begin{statement}
$d(u, v) \leqslant d(u, k) + d(k, v)$, причём $d(u, v) = d(u, k) + d(k, v)$ $\Leftrightarrow$ $k$ лежит на одном из кратчайших $(u, v)$"=путей.
\end{statement}
\begin{proof}
Пусть $p_1$ и $p_2$~--- кратчайшие $(u, k)$"=путь и $(k, v)$"=путь соответственно.
Тогда $p_1 \cup p_2$~--- $(u, v)$"=путь.
Значит, либо этот путь кратчайший и его вес минимален (в таком случае $d(u, v) = d(u, k) + d(k, v)$), либо его вес больше, чем $d(u, v)$ (в этом случае ни один из кратчайших $(u, v)$"=путей не проходит через вершину~$k$).
\end{proof}

Пусть дан граф~$(V, E)$ и вес~$w \colon E \to R^+$.

\subsubsection{Алгоритм Дейкстры}
\index{Алгоритм!Дейкстры} Алгоритм Дейкстры ищет длины кратчайших путей между некоторой вершиной~$u$ и всеми остальными.
Каждой вершине~$v$ на шаге~$i$ сопоставим метки~$l_i(v) \geqslant d(u, v)$.
\begin{enumerate}
	\item[0.] $l_0(u) = 0$, $\forall v \in V \ v \neq u \Rightarrow l_0(v) = \infty$
	\item[k.] Пусть $m$~--- непосещённая вершина с минимальным~$l_{k-1}(m)$.
	Отметим $m$ как посещённую.
	\begin{equation*}
	\forall v \in V \ l_k(v) = \min \{ l_{k-1}(v), l_{k-1}(m) + w(m, v) \}
	\end{equation*}
\end{enumerate}

\begin{theorem}
Если вершина~$v$ становится посещённой на $k$"~м шаге, то $d(u, v) = l_k(v)$.
\end{theorem}
\begin{proofmathind}
	\indbase $n = 0$: $v = u$, $0 = l_0(v) = d(u, u) = 0$.
	\indstep Пусть для $k \leqslant n$ $l_k(v) = d(u, v)$, $(u, \ldots, x, y, \ldots, z)$~--- кратчайший $(u, z)$"=путь, причём $y$~--- первая непосещённая вершина, $z$~--- вершина, посещённая на шаге~$k + 1$, тогда
	\begin{equation*}
	l_{k+1}(z) \leqslant l_{k+1}(y) = \min \{ l_k(y), l_k(x) + w(x, y) \} \leqslant l_k(x) + w(x, y) = d(u, x) + w(x, y) = d(u, y) \leqslant d(u, z)
	\end{equation*}
	\begin{equation*}
	l_{k+1}(z) \leqslant d(u, z) \lAnd d(u, z) \leqslant l_{k+1}(z) \Rightarrow l_{k+1}(z) = d(u, z)
	\end{equation*}
	\indend
\end{proofmathind}

\subsubsection{Алгоритм Флойда"--~Уоршелла}
\index{Алгоритм!Флойда---Уоршелла} Обозначим через~$d_k(i, j)$ длину кратчайшего $(i, j)$"=пути с промежуточными вершинами из множества~$\{ 1, 2, \ldots, k \}$.
Тогда
\begin{equation*}
d_0(i, j) =
\begin{cases}
w(i, j), \ (i, j) \in E \\
\infty, \ (i, j) \notin E
\end{cases}, \
d_{k+1}(i, j) = \min(d_k(i, j), d_k(i, k + 1) + d_k(k + 1, j))
\end{equation*}