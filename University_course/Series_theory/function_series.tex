\section{Функциональные ряды}
\index{Ряд!функциональный} \textbf{Функциональным} называется ряд, все члены которого~--- функции.

\index{Признак!мажорантный п. сходимости}
\begin{theorem}[мажорантный признак сходимости]
Если существует последовательность~$(a_n) \colon \forall n \in \mathbb N \ \allowbreak |f_n(x)| \leqslant a_n$, то $\series a_k$ сходится $\Rightarrow$ $\series f_k(x)$ сходится абсолютно.
\end{theorem}

\subsection{Степенные ряды}
\index{Ряд!степенной} \textbf{Степенным} называется ряд вида $\series[0] c_k (x - x_0)^k$.

Из ряда~$\series[0] c_k (x - x_0)^k$ заменой $x' = x - x_0$ можно получить ряд~$\series[0] c_k (x')^k$, поэтому далее будем рассматривать только ряды вида $\series[0] c_k x^k$.

\index{Теорема!Абеля}
\begin{theorem}[Абеля]
\begin{itemize}
	\item Если ряд $\series[0] c_k x^k$ сходится при $x = x_0$, то $\forall x \colon |x| < |x_0|$ он сходится абсолютно.
	\item Если ряд $\series[0] c_k x^k$ расходится при $x = x_0$, то $\forall x \colon |x| > |x_0|$ он расходится.
\end{itemize}
\end{theorem}
\begin{proof}
\begin{itemize}
	\item Пусть $\series[0] c_k x_0^k$ сходится, тогда
	\begin{equation*}
	\lim_{n \to \infty} c_n x_0^n = 0 \Rightarrow
	\exists M > 0 \colon \forall n \in \mathbb N \ |c_n x_0^n| \leqslant M \Rightarrow
	|c_n x^n| \leqslant M \left| \frac{x}{x_0} \right|^n
	\end{equation*}
	
	$|x| < |x_0| \Rightarrow \left| \frac{x}{x_0} \right| < 1$ $\Rightarrow$ $\series[0] M \left| \frac{x}{x_0} \right|^k$ сходится $\Rightarrow$ $\series[0] c_k x^k$ сходится абсолютно.
	
	\item Пусть $\series[0] c_k x_0^k$ расходится.
	Если $\exists x_1 \colon |x_1| > |x_0|$ $\lAnd$ $\series[0] c_k x_1^k$ сходится, то $\series[0] c_k x_0^k$ сходится.
	Противоречие.
\end{itemize}
\end{proof}

\begin{consequent}
Существует $R > 0$ такое, что $\series[0] c_k x^k$ сходится при $|x| < R$ и расходится при $|x| > R$.
\end{consequent}

$R$ называется \textbf{радиусом сходимости степенного ряда}.

Рассмотрим ряд~$\series[0] |c_k x^k|$.
\begin{itemize}
	\item $\lim\limits_{n \to \infty} \left| \frac{c_{n+1} x^{n+1}}{c_n x^n} \right| =
	\lim\limits_{n \to \infty} \left| \frac{c_{n+1}}{c_n}\,x \right|$
	
	По \hyperref[th:d'Alembert's_ratio_test]{признаку Д'Аламбера} $\lim\limits_{n \to \infty} \left| \frac{c_{n+1}}{c_n}\,x \right| < 1$, если ряд сходится, и $\lim\limits_{n \to \infty} \left| \frac{c_{n+1}}{c_n}\,x \right| > 1$, если расходится.
	Тогда $\lim\limits_{n \to \infty} \left| \frac{c_{n+1}}{c_n}\,R \right| = 1$.
	Отсюда получим
	\begin{equation*}
	R = \lim_{n \to \infty} \frac{|c_n|}{|c_{n+1}|}
	\end{equation*}
	
	\item Аналогично по \hyperref[th:Cauchy's_radical_test]{радикальному признаку Коши} получим $\lim\limits_{n \to \infty} \sqrt[n]{c_n R^n} = 1$, тогда
	\begin{equation*}
	R = \lim_{n \to \infty} \frac1{\sqrt[n]{c_n}}
	\end{equation*}
\end{itemize}

\subsection{Ряды Тейлора}
\begin{theorem}
Если $\series[0] c_k x^k = S(x)$, где $|x| < R$, то
\begin{itemize}
	\item $\series[0] c_k \int\limits_a^b x^k\,dx = \int\limits_a^b S(x)\,dx$, где $|a|, |b| < R$;
	\item $\series c_k (x^k)' = S'(x)$ при $|x| < R$.
\end{itemize}
\end{theorem}

\begin{consequent}
$c_n = \frac{S^{(n)}(0)}{n!}$.
\end{consequent}
\begin{proof}
Дифференцируя ряд $n$~раз, получим
\begin{equation*}
\series[n] c_k (k - n + 1) \cdot \ldots \cdot k x^{k-n} = S^{(n)}(x)
\end{equation*}

Подставляя~$x = 0$, получим
\begin{equation*}
n! c_n = S^{(n)}(0) \Leftrightarrow
c_n = \frac{S^{(n)}(0)}{n!}
\end{equation*}
\end{proof}

Т.\,о., $S(x) = \series[0] \frac{S^{(k)}(0)}{k!}\,x^k$ при $|x| < R$.

\index{Ряд!Тейлора} \textbf{Рядом Тейлора функции~$f(x)$} называется ряд~$\series[0] \frac{f^{(k)}(0)}{k!}\,x^k$.

По \hyperref[eq:Maclaurin_series]{формуле Маклорена}
\begin{equation*}
f(x) = \sum_{k=0}^n \frac{f^{(k)}}{k!}\,x^k + R_n(x), \
R_n(x) = \frac{f^{(n + 1)}(\Theta x)}{(n + 1)!} x^{n + 1}, \
\Theta \in (0; 1)
\end{equation*}

Тогда, если $\lim\limits_{n \to \infty} R_n(x) = 0$, то $f(x)$ представима в виде ряда Тейлора:
\begin{equation*}
f(x) = \series[0] \frac{f^{(k)}}{k!}\,x^k
\end{equation*}

\subsubsection{Разложение некоторых функций в ряд Тейлора}
\begin{itemize}
	\item $f(x) = e^x$.
	Для $\series[0] \frac{x^k}{k!}$
	\begin{equation*}
	R = \lim_{n \to \infty} \frac{(n + 1)!}{n!} =
	\lim_{n \to \infty} (n + 1) = \infty
	\end{equation*}
	\begin{equation*}
	\lim_{n \to \infty} e^{|x|}\,\frac{|x|^{n+1}}{(n + 1)!} = 0 \Rightarrow
	\lim_{n \to \infty} e^{\Theta x}\,\frac{x^{n+1}}{(n + 1)!} = 0 \Rightarrow
	\lim_{n \to \infty} R_n(x) = 0
	\end{equation*}
	
	Тогда при $x \in \mathbb R$
	\begin{equation*}
	e^x = \series[0] \frac{x^k}{k!}
	\end{equation*}
	
	\item $f(x) = \sin x$.
	Для $\series \frac{(-1)^{k-1}}{(2k - 1)!}\,x^{2k-1}$
	\begin{equation*}
	R = \lim_{n \to \infty} \frac
	{\frac1{(2n - 1)!}}
	{\frac1{(2n + 1)!}} =
	\lim_{n \to \infty} 2n (2n + 1) = \infty
	\end{equation*}
	\begin{equation*}
	\lim_{n \to \infty} \frac{|x|^{2n+1}}{(2n + 1)!} = 0 \Rightarrow
	\lim_{n \to \infty} \frac{|x|^{2n+1}}{(2n + 1)!}\,\sin \left( \Theta x + \frac\pi2 (2n + 1) \right) = 0 \Rightarrow
	\lim_{n \to \infty} R_n(x) = 0
	\end{equation*}
	
	Тогда при $x \in \mathbb R$
	\begin{equation*}
	\sin x = \series \frac{(-1)^{k-1}}{(2k - 1)!}\,x^{2k-1}
	\end{equation*}
	
	\item $f(x) = \cos x$.
	Для $\series[0] \frac{(-1)^k}{(2k)!}\,x^{2k}$
	\begin{equation*}
	R = \lim_{n \to \infty} \frac
	{\frac1{(2n)!}}
	{\frac1{(2n + 2)!}} =
	\lim_{n \to \infty} (2n + 1) (2n + 2) = \infty
	\end{equation*}
	\begin{equation*}
	\lim_{n \to \infty} \frac{x^{2n}}{(2n)!} = 0 \Rightarrow
	\lim_{n \to \infty} \frac{x^{2n}}{(2n)!}\,\cos \left( \Theta x + \pi n \right) = 0 \Rightarrow
	\lim_{n \to \infty} R_n(x) = 0
	\end{equation*}
	
	Тогда при $x \in \mathbb R$
	\begin{equation*}
	\cos x = \series[0] \frac{(-1)^k}{(2k)!}\,x^{2k}
	\end{equation*}
	
	\item $f(x) = (1 + x)^\alpha$ при $\alpha \notin \mathbb N$.
	Для $\series \frac{\alpha (\alpha - 1) \cdot \ldots \cdot (\alpha - k + 1)}{k!}\,x^k$
	\begin{equation*}
	R = \lim_{k \to \infty} \frac
	{|\alpha (\alpha - 1) \cdot \ldots \cdot (\alpha - k + 1)| (k + 1)!}
	{k! |\alpha (\alpha - 1) \cdot \ldots \cdot (\alpha - k)|} =
	\lim_{k \to \infty} \frac{k + 1}{|\alpha - k|} = 1
	\end{equation*}
	\begin{equation*}
	R_n(x) = \frac{\alpha (\alpha - 1) \cdot \ldots \cdot (\alpha - n) (1 + \Theta x)^{\alpha-n-1}}{(n + 1)!}\,x^{n+1} =
	\frac{\alpha (\alpha - 1) \cdot \ldots \cdot (\alpha - n)}{(n + 1)!} \cdot
	\frac{x^{n+1}}{(1 + \Theta x)^{n+1-\alpha}}
	\end{equation*}
	
	Если $0 < x < 1$, то $|R_n(x)| \leqslant \frac{|x|^{n+1}}{(n + 1)!} \Rightarrow \lim\limits_{n \to \infty} R_n(x) = 0$.
	
	Для $-1 < x < 0$ также можно показать, что $\lim\limits_{n \to \infty} R_n(x) = 0$, используя другую форму представления~$R_n(x)$.
	
	Тогда при $x \in (-1; 1)$
	\begin{equation*}
	(1 + x)^\alpha = \series \frac{\alpha (\alpha - 1) \cdot \ldots \cdot (\alpha - k + 1)}{k!}\,x^k
	\end{equation*}
	
	\item $\displaystyle \frac1{1 + x} = \series[0] (-1)^k x^k$ при $|x| < 1$
	
	\item $\displaystyle \frac1{1 - x} = \series[0] x^k$ при $|x| < 1$
	
	\item При $|x| < 1$
	\begin{equation*}
	\ln (1 + x) =
	\int_0^x \frac{dt}{1 + t} =
	\int_0^x \series[0] (-1)^k t^k\,dt =
	\series[0] \frac{(-1)^k x^{k+1}}{k + 1} =
	\series \frac{(-1)^{k-1} x^k}{k}
	\end{equation*}
	
	\item При $|x| < 1$
	\begin{equation*}
	\arctg x =
	\int_0^x \frac{dt}{1 + t^2} =
	\int_0^x \series[0] (-1)^k t^{2k}\,dt =
	\series[0] \frac{(-1)^k}{2k + 1}\,x^{2k+1}
	\end{equation*}
\end{itemize}

\subsection{Ряды Фурье}
\index{Гармоника} Функции $\sin \frac{2\pi k}T\,x$ и $\cos \frac{2\pi k}T\,x$, где $k \in \mathbb Z$, $T$~--- период, называются \textbf{гармониками}.

Пусть $f(x)$ имеет период~$T$ и непрерывна на отрезке~$\left[ -\frac{T}2; \frac{T}2 \right]$.
\index{Ряд!Фурье} \textbf{Рядом Фурье функции~$f(x)$} называется ряд
\begin{equation}
\label{eq:Fourier_series}
\frac{a_0}2 + \series \left( a_k \cos \frac{2\pi k}T\,x + b_k \sin \frac{2\pi k}T\,x \right) = f(x)
\end{equation}

Пусть $m, n \in \mathbb Z$, $m \neq n$.
Найдём следующие интегралы:
\begin{itemize}
	\item $\displaystyle \int_{-\frac{T}2}^{\frac{T}2} \cos \frac{2 \pi n}T\,x\,dx =
	\frac{T}{2 \pi n} \left. \sin \frac{2 \pi n}T\,x \right|_{-\frac{T}2}^{\frac{T}2} = 0$
	
	\item $\displaystyle \int_{-\frac{T}2}^{\frac{T}2} \sin \frac{2 \pi n}T\,x\,dx =
	-\frac{T}{2 \pi n} \left. \cos \frac{2 \pi n}T\,x \right|_{-\frac{T}2}^{\frac{T}2} = 0$
	
	\item $\displaystyle \int_{-\frac{T}2}^{\frac{T}2} \cos \frac{2 \pi m}T\,x \cos \frac{2 \pi n}T\,x\,dx =
	\frac12 \int_{-\frac{T}2}^{\frac{T}2} \cos \frac{2 \pi (m - n)}T\,x +
	\frac12 \int_{-\frac{T}2}^{\frac{T}2} \cos \frac{2 \pi (m + n)}T\,x = 0$
	
	\item $\displaystyle \int_{-\frac{T}2}^{\frac{T}2} \sin \frac{2 \pi m}T\,x \sin \frac{2 \pi n}T\,x\,dx =
	\frac12 \int_{-\frac{T}2}^{\frac{T}2} \cos \frac{2 \pi (m - n)}T\,x -
	\frac12 \int_{-\frac{T}2}^{\frac{T}2} \cos \frac{2 \pi (m + n)}T\,x = 0$
	
	\item $\displaystyle \int_{-\frac{T}2}^{\frac{T}2} \cos \frac{2 \pi m}T\,x \sin \frac{2 \pi n}T\,x\,dx =
	\frac12 \int_{-\frac{T}2}^{\frac{T}2} \sin \frac{2 \pi (m + n)}T\,x +
	\frac12 \int_{-\frac{T}2}^{\frac{T}2} \sin \frac{2 \pi (m - n)}T\,x = 0$
	
	\item $\displaystyle \int_{-\frac{T}2}^{\frac{T}2} 1^2\,dx = T$
	
	\item $\displaystyle \int_{-\frac{T}2}^{\frac{T}2} \cos^2 \frac{2 \pi n}T\,x\,dx = \frac{T}2$
	
	\item $\displaystyle \int_{-\frac{T}2}^{\frac{T}2} \sin^2 \frac{2 \pi n}T\,x\,dx = \frac{T}2$
\end{itemize}

Тогда, умножив обе части равенства~(\ref*{eq:Fourier_series}) на выражение~$M$ и проинтегрировав на отрезке~$\left[ -\frac{T}2; \frac{T}2 \right]$, получим:
\begin{itemize}
	\item Для $M = 1$:
	\begin{equation*}
	a_0 = \frac2T \int_{-\frac{T}2}^{\frac{T}2} f(x)\,dx
	\end{equation*}
	
	\item Для $M = \cos \frac{2 \pi n}T\,x$:
	\begin{equation*}
	a_k = \frac2T \int_{-\frac{T}2}^{\frac{T}2} f(x) \cos \frac{2 \pi k}T\,x\,dx
	\end{equation*}
	
	\item Для $M = \sin \frac{2 \pi n}T\,x$:
	\begin{equation*}
	b_k = \frac2T \int_{-\frac{T}2}^{\frac{T}2} f(x) \sin \frac{2 \pi k}T\,x\,dx
	\end{equation*}
\end{itemize}

Если $f(x)$ чётна, то
\begin{equation*}
a_0 = \frac4T \int_0^{\frac{T}2} f(x)\,dx, \
a_k = \frac4T \int_0^{\frac{T}2} f(x) \cos \frac{2 \pi k}T\,x\,dx, \
b_k = 0
\end{equation*}

Если $f(x)$ нечётна, то
\begin{equation*}
a_0 = a_k = 0, \
b_k = \frac4T \int_0^{\frac{T}2} f(x) \sin \frac{2 \pi k}T\,x\,dx
\end{equation*}