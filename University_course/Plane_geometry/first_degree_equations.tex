\section{Уравнения первой степени}
Уравнение вида~$Ax + By + C = 0$, где $x, y$~--- переменные, называется \textbf{уравнением первой степени}.

\begin{theorem}
\begin{itemize}
	\item Любое уравнение первой степени задаёт прямую на плоскости.
	\item Любая прямая на плоскости задаётся уравнением первой степени.
\end{itemize}
\end{theorem}

\begin{wrapfigure}[3]{r}{0pt}\noindent
\shorthandoff{"}
\begin{tikzpicture}[>=latex]
\drawaxis{-0.5}{4}{-0.5}{3};

\def\nangle{45}
\draw[->] (0, 0) coordinate (O) -- node[above left] {$\overline n$} (\nangle:1.5) coordinate (N);
\draw (1, 0) coordinate (X)
	pic[draw, "$\Theta$", angle eccentricity=1.5, angle radius=5mm] {angle = X--O--N};

\draw[->] (1.5, \top_y) --
	++(\nangle - 90:0.5) node {$\bullet$} node[above right] {$\scriptstyle (x_0, y_0)$} --
	node[above right] {$\overline m$}
	++(\nangle - 90:2) coordinate (X) node {$\bullet$} node[above right] {$\scriptstyle (x, y)$};
\draw (X) -- ++(\nangle - 90:1);
\end{tikzpicture}
\shorthandon{"}
\end{wrapfigure}

\begin{proof}
\begin{itemize}
	\item Пусть $(x_0, y_0)$~--- решение уравнения $Ax + By + C = 0$, $\overline m = (x - x_0, y - y_0)$, $\overline n = (A, B)$, тогда $A(x - x_0) + B(y - y_0) = 0 \Leftrightarrow \overline n \perp \overline m$.
	Т.\,о., $\overline m$ независимо от значений $x, y$ всегда лежит на одной и той же прямой.
	
	\item Пусть $(x_0, y_0)$~--- некоторая точка прямой~$l$, вектор~$\overline n = (A, B) \perp l$, тогда $\forall (x, y) \in l \ \allowbreak A(x - x_0) + B(y - y_0) = 0 \Leftrightarrow Ax + By - (Ax_0 + By_0) = 0$.
\end{itemize}
\end{proof}

\index{Уравнения!прямой} Т.\,о., уравнение
\begin{equation*}
Ax + By + C = 0
\end{equation*}
называется \textbf{общим уравнением прямой}.
Если хотя бы один из коэффициентов равен нулю, то уравнение называется \textbf{неполным}.

\index{Нормаль} Вектор, перпендикулярный прямой, называется \textbf{нормальным вектором} (\textbf{нормалью}) прямой.

Вектор, лежащий на прямой, называется \textbf{направляющим вектором прямой}.

Пусть $(a, b)$~--- направляющий вектор прямой, $(x_0, 0), (0, y_0), (x_1, y_1), (x_2, y_2)$~--- точки, лежащие на ней, прямая образует угол~$\varphi$ с положительным направлением оси~$Ox$, $\Theta$~--- угол между нормалью прямой и положительным направлением оси~$Ox$

Если $(A, B)$~--- нормаль прямой, то она задаётся уравнением
\begin{equation*}
A(x - x_1) + B(y - y_1) = 0
\end{equation*}

\textbf{Каноническим уравнением прямой} называется уравнение
\begin{equation*}
\frac{x - x_1}a = \frac{y - y_1}b
\end{equation*}

Также прямую можно задать уравнениями
\begin{equation*}
\frac{x - x_1}{x_2 - x_1} = \frac{y - y_1}{y_2 - y_1} \Leftrightarrow
\begin{vmatrix}
x - x_0 & y - y_0 \\
x_1 - x_0 & y_1 - y_0
\end{vmatrix} = 0
\end{equation*}

\textbf{Параметрическим уравнением прямой} называется система
\begin{equation*}
\begin{cases}
x = x_1 + at \\
y = y_1 + bt
\end{cases}
\end{equation*}
\begin{proof}
Каноническое уравнение можно записать в виде $(x - x_1, y - y_1) = t \cdot (a, b)$.
Записав его покоординатно, получим параметрическое уравнение.
\end{proof}

\textbf{Уравнением прямой в отрезках} называется уравнение
\begin{equation*}
\frac{x}{x_0} + \frac{y}{y_0} = 1
\end{equation*}

\textbf{Уравнением прямой с угловым коэффициентом} называется уравнение
\begin{equation*}
y = kx + y_0
\end{equation*}
где $k = \tg \varphi$.
\begin{proof}
Если $B \neq 0$, то $Ax + B(y - y_0) = 0 \Leftrightarrow
y = -\frac{A}B\,x + y_0$.

Докажем, что $-\frac{A}B = \tg \varphi$:
\begin{equation*}
\tg \Theta = \frac{B}A \Leftrightarrow
\tg \left(\varphi - \frac\pi2\right) = \frac{B}A \Leftrightarrow
-\ctg \varphi = \frac{B}A \Leftrightarrow
\tg \varphi = -\frac{A}B
\end{equation*}
\end{proof}

\textbf{Нормальным} (\textbf{нормированным}) \textbf{уравнением прямой} называется уравнение
\begin{equation*}
x \cos \Theta + y \sin \Theta - p = 0
\end{equation*}
где $p$~--- расстояние от начала координат до прямой.
\begin{proof}
Пусть $\overline n = (\cos \Theta, \sin \Theta)$, $OM$~--- перпендикуляр, опущенный из начала координат~$O$ на прямую, тогда $M = (p \cos \Theta, p \sin \Theta)$.
\begin{equation*}
(x - p \cos \Theta) \cos \Theta + (y - p \sin \Theta) \sin \Theta = 0 \Leftrightarrow
x \cos \Theta + y \sin \Theta - p = 0
\end{equation*}
\end{proof}

Пусть две прямые заданы уравнениями $l_1 \colon A_1 x + B_1 y + C_1 = 0$ и $l_2 \colon A_2 x + B_2 y + C_2 = 0$, тогда:
\begin{itemize}
	\item $\displaystyle l_1 = l_2 \Leftrightarrow
	\frac{A_1}{A_2} = \frac{B_1}{B_2} = \frac{C_1}{C_2}$;
	\item $\displaystyle l_1 \parallel l_2 \Leftrightarrow
	\frac{A_1}{A_2} = \frac{B_1}{B_2} \Leftrightarrow
	\begin{vmatrix}
	A_1 & B_1 \\
	A_2 & B_2
	\end{vmatrix} = 0$;
	\item $\displaystyle l_1 \perp l_2 \Leftrightarrow
	A_1 A_2 + B_1 B_2 = 0$;
	\item $\displaystyle \angle(l_1, l_2) = \arccos \frac{|A_1 A_2 + B_1 B_2|}{\sqrt{A_1^2 + B_1^2} \cdot \sqrt{A_2^2 + B_2^2}}$.
\end{itemize}

Пусть задана прямая $x \cos \varphi + y \sin \varphi - p = 0$ и точка~$M(x_0, y_0)$.

\index{Отклонение} \textbf{Отклонением точки от прямой} называется расстояние между ними со знаком плюс, если начало координат и точка находятся с разных сторон прямой, иначе со знаком минус, и равно $\delta = x_0 \cos \varphi + y_0 \sin \varphi - p$.

Расстояние между точкой и прямой равно $|\delta|$.