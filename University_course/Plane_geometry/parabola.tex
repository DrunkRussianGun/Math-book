\section{Парабола}
\index{Парабола} \textbf{Параболой} называется геометрическое место точек~$M$ таких, что расстояния от~$M$ до некоторой прямой, называемой \textbf{директрисой}, и до некоторой точки, называемой \textbf{фокусом}, равны.

Найдём уравнение параболы.
Пусть директриса задаётся уравнением $x = -\frac{p}2$, а фокус имеет координаты~$\left(\frac{p}2, 0\right)$.
\begin{equation*}
\left(x + \frac{p}2\right)^2 = \left(x - \frac{p}2\right)^2 + y^2 \Leftrightarrow
y^2 = 2px
\end{equation*}

Найдём уравнение касательной, проходящей через точку~$(x_0, y_0)$ параболы.
\begin{equation*}
y = \pm\sqrt{2px} \Rightarrow
y' = \pm\frac{p}{\sqrt{2px}} =
\pm\frac{\sqrt{2px}}{2x} =
\frac{y}{2x}
\end{equation*}
\begin{equation*}
y = y_0 + \frac{y_0}{2x_0} (x - x_0) \Leftrightarrow
2 x_0 (y - y_0) = y_0 (x - x_0) \Leftrightarrow
\end{equation*}
\begin{equation*}
\Leftrightarrow 2 x_0 y = y_0 (x + x_0) \Leftrightarrow
2 x_0 y_0 y = y_0^2 (x + x_0) \Leftrightarrow
2 y y_0 = 2p(x + x_0)
\end{equation*}