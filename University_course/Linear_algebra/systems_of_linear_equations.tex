\section{Системы линейных алгебраических уравнений}
Система линейных алгебраических уравнений имеет вид
\begin{equation*}
\begin{cases}
a_{11} x_1 + a_{12} x_2 + \dots + a_{1n} x_n = b_1 \\
a_{21} x_1 + a_{22} x_2 + \dots + a_{2n} x_n = b_2 \\
\ldots \\
a_{m1} x_1 + a_{m2} x_2 + \dots + a_{mn} x_n = b_m
\end{cases}
\end{equation*}
где $x_1, \ldots, x_n$~--- переменные.

$a_{11}, a_{12}, \ldots, a_{mn}$ называются \textbf{коэффициентами при переменных}, $b_1, b_2, \dots, b_m$~--- \textbf{свободными членами}.

Система линейных уравнений называется \textbf{однородной}, если все её свободные члены равны~$0$, иначе~--- \textbf{неоднородной}.

Система линейных уравнений называется \textbf{совместной}, если она имеет хотя~бы одно решение, иначе~--- \textbf{несовместной}.

Система линейных уравнений называется \textbf{определённой}, если она имеет единственное решение.
Если система имеет более одного решения, то она называется \textbf{неопределённой}.

Две системы линейных уравнений называются \textbf{эквивалентными}, если их решения совпадают или обе не имеют решений.

Если применить к системе линейных уравнений одно из следующих преобразований, называемых \textbf{элементарными}, то получим систему, эквивалентную исходной, что элементарно проверяется подстановкой:
\begin{enumerate}
	\item Перестановка двух уравнений.
	\item Умножение одного из уравнений на ненулевое число.
	\item Сложение одного уравнения с другим, умноженным на некоторое число.
\end{enumerate}

\subsection{Матричная форма системы линейных уравнений}
Систему линейных уравнений можно представить в матричной форме:
\begin{equation*}
\begin{Vmatrix}
a_{11}x_1 + a_{12}x_2 + \dots + a_{1n}x_n \\
a_{21}x_1 + a_{22}x_2 + \dots + a_{2n}x_n \\
\vdots \\
a_{m1}x_1 + a_{m2}x_2 + \dots + a_{mn}x_n
\end{Vmatrix} =
\begin{Vmatrix}
b_1 \\
b_2 \\
\vdots \\
b_m
\end{Vmatrix}
\Leftrightarrow
\end{equation*}
\begin{equation*}
\Leftrightarrow
\begin{Vmatrix}
a_{11} & a_{12} & \cdots & a_{1n} \\
a_{21} & a_{22} & \cdots & a_{2n} \\
\vdots & \vdots & \ddots & \vdots \\
a_{m1} & a_{m2} & \cdots & a_{mn}
\end{Vmatrix} \cdot
\begin{Vmatrix}
x_1 \\
x_2 \\
\vdots \\
x_n
\end{Vmatrix} =
\begin{Vmatrix}
b_1 \\
b_2 \\
\vdots \\
b_m
\end{Vmatrix}
\Leftrightarrow
\end{equation*}
\begin{equation*}
\Leftrightarrow
A \cdot X = B
\end{equation*}

$A$ называется \textbf{основной матрицей системы}, $X$~--- \textbf{столбцом переменных}, $B$~--- \textbf{столбцом свободных членов}.
Если к основной матрице справа приписать столбец свободных членов, то получится \textbf{расширенная матрица системы}:
\begin{equation*}
\begin{Vmatrix}
a_{11} & a_{12} & \cdots & a_{1n} & \vline & b_1 \\
a_{21} & a_{22} & \cdots & a_{2n} & \vline & b_2 \\
\vdots & \vdots & \ddots & \vdots & \vline & \vdots \\
a_{m1} & a_{m2} & \cdots & a_{mn} & \vline & b_m
\end{Vmatrix}
\end{equation*}

\subsection{Линейная независимость}
Уравнение системы линейных уравнений называется \textbf{линейно зависимым}, если соответствующая ему строка расширенной матрицы является нетривиальной линейной комбинацией других строк, иначе~--- \textbf{линейно независимым}.

Система линейных уравнений называется \textbf{линейно зависимой}, если существует нетривиальная линейная комбинация строк расширенной матрицы, в~результате которой получается нулевая строка, иначе~--- \textbf{линейно независимой}.

\begin{statement}
Система линейных уравнений линейно зависима $\Leftrightarrow$ одно из её уравнений линейно зависимо.
\end{statement}
\begin{proof}
\begin{enumerate}
	\item $\Rightarrow$. Пусть система строк~$A_1, \ldots, A_n$ линейно зависима:
	\begin{equation*}
	\sum_{i=1}^n \alpha_i A_i = O \lAnd \sum_{i=1}^n \alpha_i^2 \neq 0
	\end{equation*}
	где $O$~--- нулевая строка. Без ограничения общности можно считать, что $\alpha_1 \neq 0$, тогда
	\begin{equation*}
	A_1 = -\sum_{i=2}^n \frac{\alpha_i}{\alpha_1} A_i
	\end{equation*}
	
	Значит, $A_1$~--- линейно зависимая строка.
	
	\item $\Leftarrow$. Пусть одна из строк линейно зависима:
	\begin{equation*}
	A_1 = \sum_{i=2}^n \alpha_i A_i \Leftrightarrow
	1 \cdot A_1 - \alpha_2 A_2 - \ldots - \alpha_n A_n = O
	\end{equation*}
	
	Значит, система линейно зависима.
\end{enumerate}
\end{proof}

\subsection{Решение систем линейных уравнений}
\begin{lemma}
\label{lemma:linear_independency}
Пусть система строк~$A_1, \ldots, A_n$ линейно независима и $A_{n+1}$ не является линейной комбинацией $A_1, \ldots, A_n$. Тогда система строк~$A_1, \ldots, A_n, A_{n+1}$ линейно независима.
\end{lemma}
\begin{proofcontra}
Пусть система строк~$A_1, \ldots, A_n, A_{n+1}$ линейно зависима:
\begin{equation*}
\sum_{i=1}^{n+1} \alpha_i A_i = O \lAnd
\sum_{i=1}^{n+1} \alpha_i^2 \neq 0
\end{equation*}
где $O$~--- нулевая строка.
Система строк~$A_1, \ldots, A_n$ линейно независима по условию, тогда
\begin{equation*}
\alpha_{n+1} \neq 0 \Rightarrow A_{n+1} = -\sum_{i=1}^n \frac{\alpha_i}{\alpha_{n+1}} A_i
\end{equation*}

Значит, $A_{n+1}$~--- линейная комбинация $A_1, \ldots, A_n$.
Противоречие с условием.
\end{proofcontra}

\index{Теорема!Кронекера~---~Капелли}
\begin{theorem}[Кронекера~---~Капелли]
Система линейных уравнений совместна $\Leftrightarrow$ ранг основной матрицы~$A$ совпадает с рангом расширенной матрицы.
\end{theorem}
\begin{proof}
\begin{enumerate}
	\item $\Rightarrow$. Пусть $(a_1, \ldots, a_n)$~--- решение системы, $B$~--- столбец свободных членов системы.
	Тогда $\sum\limits_{i=1}^n a_i A^i = B$, значит, $B$~--- линейная комбинация столбцов~$A^1, \ldots, A^n$, поэтому ранг расширенной матрицы совпадает с рангом основной.
	
	\item $\Leftarrow$. Пусть ранг основной матрицы равен рангу расширенной.
	Предположим, что система несовместна, тогда $B$ не является линейной комбинацией столбцов~$A^1, \ldots, A^n$, значит, по лемме~\ref*{lemma:linear_independency} система строк $A^1, \ldots, A^n, B$ линейно независима.
	Получили, что ранг расширенной матрицы больше ранга основной.
	Противоречие.
\end{enumerate}
\end{proof}

\subsubsection{Метод Гаусса}
Пусть дана система линейных уравнений
\begin{equation}
\label{eq:Gaussian_elimination(1)}
\begin{cases}
a_{11} x_1 + a_{12} x_2 + \dots + a_{1n} x_n = b_1 \\
a_{21} x_1 + a_{22} x_2 + \dots + a_{2n} x_n = b_2 \\
\ldots \\
a_{m1} x_1 + a_{m2} x_2 + \dots + a_{mn} x_n = b_m
\end{cases}
\end{equation}

Её расширенную матрицу можно привести к ступенчатому виду, т.\,е. (\ref*{eq:Gaussian_elimination(1)}) эквивалентна
\begin{equation}
\label{eq:Gaussian_elimination(2)}
\begin{cases}
a_{1\, j_1} x_{j_1} + \ldots + a_{1\, j_n} x_{j_n} = b_1 \\
a_{2\, j_2} x_{j_2} + \ldots + a_{2\, j_n} x_{j_n} = b_2 \\
\ldots \\
a_{r\, j_r} x_{j_r} + \ldots + a_{r\, j_n} x_{j_n} = b_r \\
0 = b_{r+1} \\
\ldots \\
0 = b_m
\end{cases}
\end{equation}
где $a_{1\, j_1}, \ldots, a_{r\, j_r} \neq 0$.
Без ограничения общности можно считать, что в~базисный минор основной матрицы системы~(\ref*{eq:Gaussian_elimination(2)}) входят только коэффициенты при переменных~$x_{j_1}, \ldots, x_{j_r}$, называемых \textbf{главными (зависимыми)}.
Остальные переменные называются \textbf{свободными (независимыми)}.

Если $\exists i > r \colon b_i \neq 0$, то система несовместна.
Пусть $\forall i > r \ b_i = 0$. Тогда получим систему
\begin{equation*}
\begin{cases}
\displaystyle x_{j_1} = \frac{b_1}{a_{1\, j_1}} - \frac{a_{1\, j_2}}{a_{1\, j_1}} x_{j_2} - \ldots - \frac{a_{1\, j_n}}{a_{1\, j_1}} x_{j_n} \\
\displaystyle x_{j_2} = \frac{b_2}{a_{2\, j_2}} - \frac{a_{2\, j_3}}{a_{2\, j_2}} x_{j_3} - \ldots - \frac{a_{2\, j_n}}{a_{2\, j_2}} x_{j_n} \\
\ldots \\
\displaystyle x_{j_r} = \frac{b_r}{a_{r\, j_r}} - \frac{a_{r\, j_{r+1}}}{a_{r\, j_r}} x_{j_{r+1}} - \ldots - \frac{a_{r\, j_n}}{a_{r\, j_r}} x_{j_n} \\
\end{cases}
\end{equation*}

Если свободным переменным полученной системы придавать все возможные значения и решать новую систему относительно главных неизвестных от нижнего уравнения к верхнему, то получим все решения данной системы.

\subsubsection{Метод Крамера}
\begin{theorem}[Крамера]
\label{th:Cramer}
Пусть дана система линейно независимых уравнений
\begin{equation*}
\begin{cases}
a_{11}x_1 + a_{12}x_2 + \dots + a_{1n}x_n = b_1 \\
a_{21}x_1 + a_{22}x_2 + \dots + a_{2n}x_n = b_2 \\
\ldots \\
a_{n1}x_1 + a_{n2}x_2 + \dots + a_{nn}x_n = b_n
\end{cases}
\end{equation*}

Если определитель её основной матрицы не равен~$0$, то система имеет единственное решение.
\end{theorem}
\begin{proof}
Запишем систему в~матричной форме:
\begin{equation*}
AX = B \Leftrightarrow
A^{-1}AX = A^{-1}B \Leftrightarrow
X = A^{-1}B \Leftrightarrow
\end{equation*}
\begin{equation*}
\Leftrightarrow
\begin{Vmatrix}
x_1 \\
x_2 \\
\vdots \\
x_n
\end{Vmatrix} =
\begin{Vmatrix}
\dfrac{A_{11}}{|A|} & \dfrac{A_{21}}{|A|} & \cdots & \dfrac{A_{n1}}{|A|} \\
\dfrac{A_{12}}{|A|} & \dfrac{A_{22}}{|A|} & \cdots & \dfrac{A_{n2}}{|A|} \\
\vdots & \vdots & \ddots & \vdots \\
\dfrac{A_{1n}}{|A|} & \dfrac{A_{2n}}{|A|} & \cdots & \dfrac{A_{nn}}{|A|}
\end{Vmatrix} \cdot
\begin{Vmatrix}
b_1 \\
b_2 \\
\vdots \\
b_n
\end{Vmatrix}
\end{equation*}
где $A_{ij}$~--- алгебраическое дополнение $a_{ij}$.

\index{Формула!Крамера}
Т.\,о., получим решение системы:
\begin{equation}
\label{eq:Cramer's_formula}
x_i = \frac{\displaystyle \sum_{j=1}^n A_{ji} b_j}{|A|} =
\frac1{{|A|}} \cdot
\begin{vmatrix}
a_{11} & \cdots & a_{1\, i-1} & b_{1} & a_{1\, i+1} & \cdots & a_{1n} \\
a_{21} & \cdots & a_{2\, i-1} & b_{2} & a_{2\, i+1} & \cdots & a_{2n} \\
\vdots & \ddots & \vdots & \vdots & \vdots & \ddots & \vdots \\
a_{n1} & \cdots & a_{n\, i-1} & b_{n} & a_{n\, i+1} & \cdots & a_{nn} \\
\end{vmatrix}, \ i = 1, 2, \ldots, n
\end{equation}
\end{proof}

Полученные формулы~(\ref*{eq:Cramer's_formula}) называется \textbf{формулами Крамера}.

% beta
\subsection{Фундаментальная система решений}
\begin{statement}
\label{st:homogeneous_system_sets_vector_space}
Однородная линейно независимая система уравнений
\begin{equation*}
\begin{cases}
\displaystyle \sum_{i=1}^n a_{1i} x_i = 0 \\
\displaystyle \sum_{i=1}^n a_{2i} x_i = 0 \\
\ldots \\
\displaystyle \sum_{i=1}^n a_{mi} x_i = 0
\end{cases}
\end{equation*}
задаёт векторное пространство.
\end{statement}
\begin{proof}
Пусть $(\alpha_1,  \ldots, \alpha_n), (\beta_1,  \ldots, \beta_n)$~--- решения данной системы, $\lambda \neq 0$.
\begin{itemize}
	\item \begin{equation*}
	\begin{cases}
	\displaystyle \sum_{i=1}^n a_{1i} (\alpha_i + \beta_i) = 0 \\
	\displaystyle \sum_{i=1}^n a_{2i} (\alpha_i + \beta_i) = 0 \\
	\ldots \\
	\displaystyle \sum_{i=1}^n a_{mi} (\alpha_i + \beta_i) = 0
	\end{cases}
	\Leftrightarrow
	\begin{cases}
	\displaystyle \sum_{i=1}^n a_{1i} \alpha_i + \sum_{i=1}^n a_{1i} \beta_i = 0 \\
	\displaystyle \sum_{i=1}^n a_{2i} \alpha_i + \sum_{i=1}^n a_{2i} \beta_i = 0 \\
	\ldots \\
	\displaystyle \sum_{i=1}^n a_{mi} \alpha_i + \sum_{i=1}^n a_{mi} \beta_i = 0
	\end{cases}
	\Leftrightarrow
	\begin{cases}
	0 = 0 \\
	0 = 0 \\
	\ldots \\
	0 = 0
	\end{cases}
	\end{equation*}
	
	Значит, $(\alpha_1 + \beta_1, \ldots, \alpha_n + \beta_n)$ тоже является решением системы.
	
	\item \begin{equation*}
	\begin{cases}
	\displaystyle \sum_{i=1}^n a_{1i} \lambda \alpha_i = 0 \\
	\displaystyle \sum_{i=1}^n a_{2i} \lambda \alpha_i = 0 \\
	\ldots \\
	\displaystyle \sum_{i=1}^n a_{mi} \lambda \alpha_i = 0
	\end{cases}
	\Leftrightarrow
	\begin{cases}
	\displaystyle \lambda \sum_{i=1}^n a_{1i} \alpha_i = 0 \\
	\displaystyle \lambda \sum_{i=1}^n a_{2i} \alpha_i = 0 \\
	\ldots \\
	\displaystyle \lambda \sum_{i=1}^n a_{mi} \alpha_i = 0
	\end{cases}
	\Leftrightarrow
	\begin{cases}
	0 = 0 \\
	0 = 0 \\
	\ldots \\
	0 = 0
	\end{cases}
	\end{equation*}
	
	Значит, $(\lambda \alpha_1, \ldots, \lambda \alpha_n)$ тоже является решением системы.
\end{itemize}

Тогда множество решений данной системы~--- векторное пространство.
\end{proof}

\textbf{Фундаментальной системой решений однородной системы линейных уравнений} называется базис множества всех её решений.

Пусть дана однородная линейно независимая система уравнений:
\begin{equation}
\label{eq:homogeneous_system1}
\begin{cases}
\displaystyle \sum_{i=1}^n a_{1i} x_i = 0 \\
\displaystyle \sum_{i=1}^n a_{2i} x_i = 0 \\
\ldots \\
\displaystyle \sum_{i=1}^n a_{mi} x_i = 0
\end{cases} \Leftrightarrow
\begin{cases}
\displaystyle \sum_{i=1}^m a_{1i} x_i = -\sum_{i=m+1}^n a_{1i} x_i \\
\displaystyle \sum_{i=1}^m a_{2i} x_i = -\sum_{i=m+1}^n a_{2i} x_i \\
\ldots \\
\displaystyle \sum_{i=1}^m a_{mi} x_i = -\sum_{i=m+1}^n a_{mi} x_i
\end{cases}
\end{equation}

Присваивая переменным~$x_{m+1}, \ldots, x_n$ произвольные значения, получаем систему уравнений, которая по \hyperref[th:Cramer]{теореме Крамера} имеет единственное решение.
Тогда решения системы~(\ref*{eq:homogeneous_system1})
\begin{equation*}
\overline e_1 = (x_{11}, x_{21}, \ldots, x_{m1}, 1, 0, \ldots, 0), \
\overline e_2 = (x_{12}, x_{22}, \ldots, x_{m2}, 0, 1, \ldots, 0), \ \ldots,
\end{equation*}
\begin{equation*}
\overline e_{n-m} = (x_{1\, n-m}, x_{2\, n-m}, \ldots, x_{m\, n-m}, 0, 0, \ldots, 1)
\end{equation*}

образуют фундаментальную систему решений.
\begin{proof}
Пусть $\overline r = (r_1, \ldots, r_n)$~--- решение системы~(\ref*{eq:homogeneous_system1}), $\overline p = (p_1, \ldots, p_n) = r_{m+1} \overline e_1 \opbr+ r_{m+2} \overline e_2 \opbr+ \ldots \opbr+ r_n \overline e_{n-m} \opbr- \overline r$.
По утверждению~\ref*{st:homogeneous_system_sets_vector_space} $\overline p$~--- решение системы~(\ref*{eq:homogeneous_system1}).
Легко проверить, что $p_{m+1} \opbr= \ldots \opbr= p_n \opbr= 0$.
Подставим эти значения в систему~(\ref*{eq:homogeneous_system1}), тогда по \hyperref[th:Cramer]{теореме Крамера} она имеет единственное решение~--- нулевое.
Значит, $\overline r = r_{m+1} \overline e_1 + r_{m+2} \overline e_2 + \ldots + r_n \overline e_{n-m}$, т.\,е. $\overline e_1, \ldots, \overline e_{n-m}$~--- фундаментальная система решений.
\end{proof}

\begin{theorem}
Пусть дана линейно независимая система уравнений:
\begin{equation}
\label{eq:nonhomogeneous_system}
\begin{cases}
\displaystyle \sum_{i=1}^n a_{1i} x_i = b_1 \\
\displaystyle \sum_{i=1}^n a_{2i} x_i = b_2 \\
\ldots \\
\displaystyle \sum_{i=1}^n a_{mi} x_i = b_m
\end{cases}
\end{equation}

Если $\overline e_0$~--- её решение, а векторы~$\overline e_1, \ldots, \overline e_{n-m}$~--- фундаментальная система решений системы уравнений
\begin{equation}
\label{eq:homogeneous_system2}
\begin{cases}
\displaystyle \sum_{i=1}^n a_{1i} x_i = 0 \\
\displaystyle \sum_{i=1}^n a_{2i} x_i = 0 \\
\ldots \\
\displaystyle \sum_{i=1}^n a_{mi} x_i = 0
\end{cases}
\end{equation}

то любое решение системы~(\ref*{eq:nonhomogeneous_system}) можно найти по формуле
\begin{equation*}
\lambda_1 \overline e_1 + \ldots + \lambda_{n-m} \overline e_{n-m} + \overline e_0
\end{equation*}
где $\lambda_1, \ldots, \lambda_{n-m}$~--- произвольные числа.
\end{theorem}
\begin{proof}
Пусть $\overline v$~--- решение системы~(\ref*{eq:nonhomogeneous_system}).
Убедимся подстановкой, что $\overline v - \overline u_0$~--- решение системы~(\ref*{eq:homogeneous_system2}).
Тогда
\begin{equation*}
\overline v - \overline u_0 = \sum_{i=1}^{n-m} \lambda_i \overline u_i \Leftrightarrow
\overline v = \sum_{i=1}^{n-m} \lambda_i \overline u_i + \overline u_0
\end{equation*}
\end{proof}