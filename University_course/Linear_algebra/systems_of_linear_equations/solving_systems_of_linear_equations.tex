\subsection{Решение систем линейных уравнений}
\begin{lemma}
\label{lemma:linear_independency}
Пусть система строк~$A_1, \ldots, A_n$ линейно независима и $A_{n+1}$ не является линейной комбинацией $A_1, \ldots, A_n$. Тогда система строк~$A_1, \ldots, A_n, A_{n+1}$ линейно независима.
\end{lemma}
\begin{proofcontra}
Пусть система строк~$A_1, \ldots, A_n, A_{n+1}$ линейно зависима:
\begin{equation*}
\sum_{i=1}^{n+1} \alpha_i A_i = O \lAnd
\sum_{i=1}^{n+1} \alpha_i^2 \neq 0
\end{equation*}
где $O$~--- нулевая строка.
Система строк~$A_1, \ldots, A_n$ линейно независима по условию, тогда
\begin{equation*}
\alpha_{n+1} \neq 0 \Rightarrow A_{n+1} = -\sum_{i=1}^n \frac{\alpha_i}{\alpha_{n+1}} A_i
\end{equation*}

Значит, $A_{n+1}$~--- линейная комбинация $A_1, \ldots, A_n$.
Противоречие с условием.
\end{proofcontra}

\index{Теорема!Кронекера~---~Капелли}
\begin{theorem}[Кронекера~---~Капелли]
Система линейных уравнений совместна $\Leftrightarrow$ ранг основной матрицы~$A$ совпадает с рангом расширенной матрицы.
\end{theorem}
\begin{proof}
\begin{enumerate}
	\item $\Rightarrow$. Пусть $(a_1, \ldots, a_n)$~--- решение системы, $B$~--- столбец свободных членов системы.
	Тогда $\displaystyle \sum_{i=1}^n a_i A^i = B$, значит, $B$~--- линейная комбинация столбцов~$A^1, \ldots, A^n$, поэтому ранг расширенной матрицы совпадает с рангом основной.
	
	\item $\Leftarrow$. Пусть ранг основной матрицы равен рангу расширенной.
	Предположим, что система несовместна, тогда $B$ не является линейной комбинацией столбцов~$A^1, \ldots, A^n$, значит, по лемме~\ref*{lemma:linear_independency} система строк $A^1, \ldots, A^n, B$ линейно независима.
	Получили, что ранг расширенной матрицы больше ранга основной.
	Противоречие.
\end{enumerate}
\end{proof}

\subsubsection{Метод Гаусса}
Пусть дана система линейных уравнений
\begin{equation}
\label{eq:Gaussian_elimination(1)}
\begin{cases}
a_{11} x_1 + a_{12} x_2 + \dots + a_{1n} x_n = b_1 \\
a_{21} x_1 + a_{22} x_2 + \dots + a_{2n} x_n = b_2 \\
\ldots \\
a_{m1} x_1 + a_{m2} x_2 + \dots + a_{mn} x_n = b_m
\end{cases}
\end{equation}

Её расширенную матрицу можно привести к ступенчатому виду, т.\,е. (\ref*{eq:Gaussian_elimination(1)}) эквивалентна
\begin{equation}
\label{eq:Gaussian_elimination(2)}
\begin{cases}
a_{1\, j_1} x_{j_1} + \ldots + a_{1\, j_n} x_{j_n} = b_1 \\
a_{2\, j_2} x_{j_2} + \ldots + a_{2\, j_n} x_{j_n} = b_2 \\
\ldots \\
a_{r\, j_r} x_{j_r} + \ldots + a_{r\, j_n} x_{j_n} = b_r \\
0 = b_{r+1} \\
\ldots \\
0 = b_m
\end{cases}
\end{equation}
где $a_{1\, j_1}, \ldots, a_{r\, j_r} \neq 0$.
Без ограничения общности можно считать, что в~базисный минор основной матрицы системы~(\ref*{eq:Gaussian_elimination(2)}) входят только коэффициенты при переменных~$x_{j_1}, \ldots, x_{j_r}$, называемых \textbf{главными (зависимыми)}.
Остальные переменные называются \textbf{свободными (независимыми)}.

Если $\exists i > r \colon b_i \neq 0$, то система несовместна.
Пусть $\forall i > r \ b_i = 0$. Тогда получим систему
\begin{equation*}
\begin{cases}
\displaystyle x_{j_1} = \frac{b_1}{a_{1\, j_1}} - \frac{a_{1\, j_2}}{a_{1\, j_1}} x_{j_2} - \ldots - \frac{a_{1\, j_n}}{a_{1\, j_1}} x_{j_n} \\
\displaystyle x_{j_2} = \frac{b_2}{a_{2\, j_2}} - \frac{a_{2\, j_3}}{a_{2\, j_2}} x_{j_3} - \ldots - \frac{a_{2\, j_n}}{a_{2\, j_2}} x_{j_n} \\
\ldots \\
\displaystyle x_{j_r} = \frac{b_r}{a_{r\, j_r}} - \frac{a_{r\, j_{r+1}}}{a_{r\, j_r}} x_{j_{r+1}} - \ldots - \frac{a_{r\, j_n}}{a_{r\, j_r}} x_{j_n} \\
\end{cases}
\end{equation*}

Если свободным переменным полученной системы придавать все возможные значения и решать новую систему относительно главных неизвестных от нижнего уравнения к верхнему, то получим все решения данной системы.

\subsubsection{Метод Крамера}
\begin{theorem}[Крамера]
\label{th:Cramer}
Пусть дана система линейно независимых уравнений
\begin{equation*}
\begin{cases}
a_{11}x_1 + a_{12}x_2 + \dots + a_{1n}x_n = b_1 \\
a_{21}x_1 + a_{22}x_2 + \dots + a_{2n}x_n = b_2 \\
\ldots \\
a_{n1}x_1 + a_{n2}x_2 + \dots + a_{nn}x_n = b_n
\end{cases}
\end{equation*}

Если определитель её основной матрицы не равен~$0$, то система имеет единственное решение.
\end{theorem}
\begin{proof}
Запишем систему в~матричной форме:
\begin{equation*}
AX = B \Leftrightarrow
A^{-1}AX = A^{-1}B \Leftrightarrow
X = A^{-1}B \Leftrightarrow
\end{equation*}
\begin{equation*}
\Leftrightarrow
\begin{Vmatrix}
x_1 \\
x_2 \\
\vdots \\
x_n
\end{Vmatrix} =
\begin{Vmatrix}
\dfrac{A_{11}}{|A|} & \dfrac{A_{21}}{|A|} & \cdots & \dfrac{A_{n1}}{|A|} \\
\dfrac{A_{12}}{|A|} & \dfrac{A_{22}}{|A|} & \cdots & \dfrac{A_{n2}}{|A|} \\
\vdots & \vdots & \ddots & \vdots \\
\dfrac{A_{1n}}{|A|} & \dfrac{A_{2n}}{|A|} & \cdots & \dfrac{A_{nn}}{|A|}
\end{Vmatrix} \cdot
\begin{Vmatrix}
b_1 \\
b_2 \\
\vdots \\
b_n
\end{Vmatrix}
\end{equation*}
где $A_{ij}$~--- алгебраическое дополнение $a_{ij}$.

\index{Формула!Крамера}
Т.\,о., получим решение системы:
\begin{equation}
\label{eq:Cramer's_formula}
x_i = \frac{\displaystyle \sum_{j=1}^n A_{ji} b_j}{|A|} =
\frac1{{|A|}} \cdot
\begin{vmatrix}
a_{11} & \cdots & a_{1\, i-1} & b_{1} & a_{1\, i+1} & \cdots & a_{1n} \\
a_{21} & \cdots & a_{2\, i-1} & b_{2} & a_{2\, i+1} & \cdots & a_{2n} \\
\vdots & \ddots & \vdots & \vdots & \vdots & \ddots & \vdots \\
a_{n1} & \cdots & a_{n\, i-1} & b_{n} & a_{n\, i+1} & \cdots & a_{nn} \\
\end{vmatrix}, \ i = 1, 2, \ldots, n
\end{equation}
\end{proof}

Полученные формулы~(\ref*{eq:Cramer's_formula}) называется \textbf{формулами Крамера}.