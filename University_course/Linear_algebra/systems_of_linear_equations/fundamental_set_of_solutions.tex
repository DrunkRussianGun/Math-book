% beta
\subsection{Фундаментальная система решений}
\begin{statement}
\label{st:homogeneous_system_sets_vector_space}
Однородная линейно независимая система уравнений
\begin{equation*}
\begin{cases}
\displaystyle \sum_{i=1}^n a_{1i} x_i = 0 \\
\displaystyle \sum_{i=1}^n a_{2i} x_i = 0 \\
\ldots \\
\displaystyle \sum_{i=1}^n a_{mi} x_i = 0
\end{cases}
\end{equation*}
задаёт векторное пространство.
\end{statement}
\begin{proof}
Пусть $(\alpha_1,  \ldots, \alpha_n), (\beta_1,  \ldots, \beta_n)$~--- решения данной системы, $\lambda \neq 0$.
\begin{itemize}
	\item \begin{equation*}
	\begin{cases}
	\displaystyle \sum_{i=1}^n a_{1i} (\alpha_i + \beta_i) = 0 \\
	\displaystyle \sum_{i=1}^n a_{2i} (\alpha_i + \beta_i) = 0 \\
	\ldots \\
	\displaystyle \sum_{i=1}^n a_{mi} (\alpha_i + \beta_i) = 0
	\end{cases}
	\Leftrightarrow
	\begin{cases}
	\displaystyle \sum_{i=1}^n a_{1i} \alpha_i + \sum_{i=1}^n a_{1i} \beta_i = 0 \\
	\displaystyle \sum_{i=1}^n a_{2i} \alpha_i + \sum_{i=1}^n a_{2i} \beta_i = 0 \\
	\ldots \\
	\displaystyle \sum_{i=1}^n a_{mi} \alpha_i + \sum_{i=1}^n a_{mi} \beta_i = 0
	\end{cases}
	\Leftrightarrow
	\begin{cases}
	0 = 0 \\
	0 = 0 \\
	\ldots \\
	0 = 0
	\end{cases}
	\end{equation*}
	
	Значит, $(\alpha_1 + \beta_1, \ldots, \alpha_n + \beta_n)$ тоже является решением системы.
	
	\item \begin{equation*}
	\begin{cases}
	\displaystyle \sum_{i=1}^n a_{1i} \lambda \alpha_i = 0 \\
	\displaystyle \sum_{i=1}^n a_{2i} \lambda \alpha_i = 0 \\
	\ldots \\
	\displaystyle \sum_{i=1}^n a_{mi} \lambda \alpha_i = 0
	\end{cases}
	\Leftrightarrow
	\begin{cases}
	\displaystyle \lambda \sum_{i=1}^n a_{1i} \alpha_i = 0 \\
	\displaystyle \lambda \sum_{i=1}^n a_{2i} \alpha_i = 0 \\
	\ldots \\
	\displaystyle \lambda \sum_{i=1}^n a_{mi} \alpha_i = 0
	\end{cases}
	\Leftrightarrow
	\begin{cases}
	0 = 0 \\
	0 = 0 \\
	\ldots \\
	0 = 0
	\end{cases}
	\end{equation*}
	
	Значит, $(\lambda \alpha_1, \ldots, \lambda \alpha_n)$ тоже является решением системы.
\end{itemize}

Тогда множество решений данной системы~--- векторное пространство.
\end{proof}

\textbf{Фундаментальной системой решений однородной системы линейных уравнений} называется базис множества всех её решений.

Пусть дана однородная линейно независимая система уравнений:
\begin{equation}
\label{eq:homogeneous_system1}
\begin{cases}
\displaystyle \sum_{i=1}^n a_{1i} x_i = 0 \\
\displaystyle \sum_{i=1}^n a_{2i} x_i = 0 \\
\ldots \\
\displaystyle \sum_{i=1}^n a_{mi} x_i = 0
\end{cases} \Leftrightarrow
\begin{cases}
\displaystyle \sum_{i=1}^m a_{1i} x_i = -\sum_{i=m+1}^n a_{1i} x_i \\
\displaystyle \sum_{i=1}^m a_{2i} x_i = -\sum_{i=m+1}^n a_{2i} x_i \\
\ldots \\
\displaystyle \sum_{i=1}^m a_{mi} x_i = -\sum_{i=m+1}^n a_{mi} x_i
\end{cases}
\end{equation}

Присваивая переменным~$x_{m+1}, \ldots, x_n$ произвольные значения, получаем систему уравнений, которая по \hyperref[th:Cramer]{теореме Крамера} имеет единственное решение.
Тогда решения системы~(\ref*{eq:homogeneous_system1})
\begin{equation*}
\overline e_1 = (x_{11}, x_{21}, \ldots, x_{m1}, 1, 0, \ldots, 0), \
\overline e_2 = (x_{12}, x_{22}, \ldots, x_{m2}, 0, 1, \ldots, 0), \ \ldots,
\end{equation*}
\begin{equation*}
\overline e_{n-m} = (x_{1\, n-m}, x_{2\, n-m}, \ldots, x_{m\, n-m}, 0, 0, \ldots, 1)
\end{equation*}

образуют фундаментальную систему решений.
\begin{proof}
Пусть $\overline r = (r_1, \ldots, r_n)$~--- решение системы~(\ref*{eq:homogeneous_system1}), $\overline p = (p_1, \ldots, p_n) = r_{m+1} \overline e_1 \opbr+ r_{m+2} \overline e_2 \opbr+ \ldots \opbr+ r_n \overline e_{n-m} \opbr- \overline r$.
По утверждению~\ref*{st:homogeneous_system_sets_vector_space} $\overline p$~--- решение системы~(\ref*{eq:homogeneous_system1}).
Легко проверить, что $p_{m+1} \opbr= \ldots \opbr= p_n \opbr= 0$.
Подставим эти значения в систему~(\ref*{eq:homogeneous_system1}), тогда по \hyperref[th:Cramer]{теореме Крамера} она имеет единственное решение~--- нулевое.
Значит, $\overline r = r_{m+1} \overline e_1 + r_{m+2} \overline e_2 + \ldots + r_n \overline e_{n-m}$, т.\,е. $\overline e_1, \ldots, \overline e_{n-m}$~--- фундаментальная система решений.
\end{proof}

\begin{theorem}
Пусть дана линейно независимая система уравнений:
\begin{equation}
\label{eq:nonhomogeneous_system}
\begin{cases}
\displaystyle \sum_{i=1}^n a_{1i} x_i = b_1 \\
\displaystyle \sum_{i=1}^n a_{2i} x_i = b_2 \\
\ldots \\
\displaystyle \sum_{i=1}^n a_{mi} x_i = b_m
\end{cases}
\end{equation}

Если $\overline e_0$~--- её решение, а векторы~$\overline e_1, \ldots, \overline e_{n-m}$~--- фундаментальная система решений системы уравнений
\begin{equation}
\label{eq:homogeneous_system2}
\begin{cases}
\displaystyle \sum_{i=1}^n a_{1i} x_i = 0 \\
\displaystyle \sum_{i=1}^n a_{2i} x_i = 0 \\
\ldots \\
\displaystyle \sum_{i=1}^n a_{mi} x_i = 0
\end{cases}
\end{equation}

то любое решение системы~(\ref*{eq:nonhomogeneous_system}) можно найти по формуле
\begin{equation*}
\lambda_1 \overline e_1 + \ldots + \lambda_{n-m} \overline e_{n-m} + \overline e_0
\end{equation*}
где $\lambda_1, \ldots, \lambda_{n-m}$~--- произвольные числа.
\end{theorem}
\begin{proof}
Пусть $\overline v$~--- решение системы~(\ref*{eq:nonhomogeneous_system}).
Убедимся подстановкой, что $\overline v - \overline u_0$~--- решение системы~(\ref*{eq:homogeneous_system2}).
Тогда
\begin{equation*}
\overline v - \overline u_0 = \sum_{i=1}^{n-m} \lambda_i \overline u_i \Leftrightarrow
\overline v = \sum_{i=1}^{n-m} \lambda_i \overline u_i + \overline u_0
\end{equation*}
\end{proof}