\section{Квадратичные формы}
\index{Квадратичная форма} \textbf{Квадратичной формой} называется многочлен, все одночлены в~котором второй степени:
\begin{equation*}
f(x_1, \ldots, x_n) = \sum_{i=1}^n \sum_{j=1}^n a_{ij} x_i x_j
\end{equation*}
Для определённости полагают $a_{ij} = a_{ji}$.

Квадратичной форме можно сопоставить \textbf{матрицу квадратичной формы}, составленную из коэффициентов:
\begin{equation*}
\begin{Vmatrix}
a_{11} & a_{12} & \cdots & a_{1n} \\
a_{21} & a_{22} & \cdots & a_{2n} \\
\vdots & \vdots & \ddots & \vdots \\
a_{n1} & a_{n2} & \cdots & a_{nn}
\end{Vmatrix} =
\begin{Vmatrix}
a_{11} & a_{12} & \cdots & a_{1n} \\
a_{12} & a_{22} & \cdots & a_{2n} \\
\vdots & \vdots & \ddots & \vdots \\
a_{1n} & a_{2n} & \cdots & a_{nn}
\end{Vmatrix}
\end{equation*}

\textbf{Каноническим видом квадратичной формы} называется её представление в~виде суммы квадратов с некоторыми коэффициентами.

\begin{theorem}[метод Лагранжа]
Любая квадратичная форма может быть приведена к каноническому виду.
\end{theorem}
\begin{proof}
Пусть дана квадратичная форма~$f(x_1, \ldots, x_n) = \sum\limits_{i=1}^n \sum\limits_{j=1}^n a_{ij} x_i x_j$.
Возможны два случая:
\begin{enumerate}
	\item $\exists i \colon a_{ii} \neq 0$.
	Без ограничения общности будем считать, что $a_{11} \neq 0$, тогда
	\begin{equation*}
	f(x_1, \ldots, x_n) = a_{11}\left( x_1^2 + \frac{2 x_1}{a_{11}} \sum_{i=2}^n a_{1i} x_i + \frac1{a_{11}^2} \left( \sum_{i=2}^n a_{1i} x_i \right)^2 \right) + \sum_{i=2}^n \sum_{j=2}^n a_{ij} x_i x_j - \frac1{a_{11}} \left( \sum_{i=2}^n a_{1i} x_i \right)^2 =
	\end{equation*}
	\begin{equation*}
	= a_{11}\left( x_1 + \frac1{a_{11}} \sum_{i=2}^n a_{1i} x_i \right)^2 + f_1(x_2, \ldots, x_n)
	\end{equation*}
	
	\item $\forall i \ a_{ii} = 0$.
	Тогда $\exists i, j \colon a_{ij} \neq 0$.
	Без ограничения общности будем считать, что $a_{12} \neq 0$, тогда заменой переменных $x_1 = y_1 + y_2$, $x_2 = y_1 - y_2$, $x_i = y_i$, $i = 3, 4, \ldots, n$ этот случай сводится к первому.
\end{enumerate}

$f_1(x_2, \ldots, x_n)$~--- квадратичная форма от $n - 1$~переменных.
Применяя к ней описанные действия, получим квадратичную форму от $n - 2$~переменных.
Продолжая таким образом, получим канонический вид $f(x_1, \ldots, x_n)$.
\end{proof}

\textbf{Нормальным видом квадратичной формы} называется её канонический вид, коэффициенты в~котором равны $-1$ или $1$.

\textbf{Рангом квадратичной формы} называется количество переменных в~её каноническом виде.
Количество положительных коэффициентов в~каноническом виде квадратичной формы называется её \textbf{положительным индексом}, а отрицательных~--- \textbf{отрицательным индексом}.
\textbf{Сигнатурой квадратичной формы} называется модуль разности положительного и отрицательного индексов.

Ранг, положительный и отрицательный индексы и сигнатура одинаковы для всех канонических видов квадратичной формы.