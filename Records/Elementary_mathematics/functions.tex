\section{Функции}
\index{Функция} Пусть $A$ и $B$~--- множества.
\textbf{Функцией~$f$} называется правило, ставящее в~соответствие каждому элементу~$a \in A$ единственный элемент~$f(a) \in B$.
$A$ называется \textbf{областью определения функции~$f$} и обозначается $D(f)$, $B$~--- \textbf{областью значений функции~$f$} и обозначается $E(f)$.
$a$ называется \textbf{прообразом~$f(a)$}, $f(a)$~--- \textbf{образом~$a$}.

\textbf{Нулём функции~$f \colon X \to \mathbb R$}, где $X \subseteq \mathbb R$, называется элемент $x \in X \colon f(x) = 0$.

\index{Последовательность} \textbf{Последовательностью} называется функция, заданная на множестве~$X \subseteq \mathbb N$, и обозначается~$(x_n)$.

\textbf{Подпоследовательностью последовательности~$(x_n)$} называется последовательность~$(x_{n_k})$, если $\forall k \in \mathbb N \allowbreak n_k < n_{k+1}$.

\subsection{Возрастающие и убывающие функции}
\index{Функция!возрастающая} Функция~$f$ называется \textbf{возрастающей} (\textbf{монотонной}) \textbf{на множестве~$X$}, если $\forall x_1, x_2 \in X \ x_1 < x_2 \opbr\Rightarrow f(x_1) < \nobreak f(x_2)$.

\index{Функция!убывающая} Функция~$f$ называется \textbf{убывающей} (\textbf{монотонной}) \textbf{на множестве~$X$}, если $\forall x_1, x_2 \in X \ x_1 < x_2 \opbr\Rightarrow f(x_1) > \nobreak f(x_2)$.

Если функция возрастает/убывает на всей области определения, то её называют \textbf{возрастающей/убывающей} (\textbf{монотонной}).

Свойства монотонных функций:
\begin{enumerate}
	\item Если $f(x)$ и $g(x)$~--- возрастающие/убывающие на множестве~$X$ функции, то $h(x) = f(x) + g(x)$~--- воз\-ра\-ста\-ющ\-ая/убывающая на~$X$ функция.
	\begin{proof}
	Пусть $f(x)$ и $g(x)$ возрастают на~$X$, $x_1, x_2 \in X \colon x_1 < x_2$.
	\begin{equation*}
	x_1 < x_2 \Rightarrow
	f(x_1) < f(x_2) \lAnd g(x_1) < g(x_2) \Rightarrow
	f(x_1) + g(x_1) < f(x_2) + g(x_2) \Rightarrow
	h(x_1) < h(x_2)
	\end{equation*}
	
	Значит, $h(x)$ возрастает на~$X$.
	
	Доказательство для случая убывания аналогично.
	\end{proof}
	
	\item Если $f(x)$~--- возрастающая/убывающая на множестве~$X$ функция, то:
	\begin{itemize}
		\item при $C < 0$ $h(x) = Cf(x)$~--- убывающая/возрастающая на~$X$ функция;
		\item при $C > 0$ $h(x) = Cf(x)$~--- возрастающая/убывающая на~$X$ функция.
	\end{itemize}
	\begin{proof}
	Пусть $f(x)$ возрастает на~$X$, $C < 0$, $x_1, x_2 \in X \colon x_1 < x_2$.
	\begin{equation*}
	x_1 < x_2 \Rightarrow
	f(x_1) < f(x_2) \Rightarrow
	C f(x_1) > C f(x_2) \Rightarrow
	h(x_1) > h(x_2)
	\end{equation*}
	
	Значит, $h(x)$ убывает на~$X$.
	
	Доказательства для остальных трёх случаев аналогичны.
	\end{proof}
	
	\item Если $f(x)$ и $g(x)$~--- возрастающие/убывающие на множестве~$X$ функции и
	\begin{itemize}
		\item $f(x), g(x) < 0$ на~$X$, то $h(x) = f(x)g(x)$~--- убывающая/возрастающая на~$X$ функция;
		\item $f(x), g(x) > 0$ на~$X$, то $h(x) = f(x)g(x)$~--- возрастающая/убывающая на~$X$ функция.
	\end{itemize}
	\begin{proof}
	Пусть $f(x)$ и $g(x)$ возрастают на~$X$, $f(x), g(x) < 0$ на~$X$, $x_1, x_2 \in X \colon x_1 < x_2$.
	\begin{equation*}
	x_1 < x_2 \Rightarrow
	f(x_1) < f(x_2) \lAnd g(x_1) < g(x_2) \Rightarrow
	\end{equation*}
	\begin{equation*}
	\Rightarrow -f(x_1) > -f(x_2) \And -g(x_1) > -g(x_2) \Rightarrow
	f(x_1)g(x_1) > f(x_2)g(x_2) \Rightarrow
	h(x_1) > h(x_2)
	\end{equation*}
	
	Значит, $h(x)$ убывает на~$X$.
	
	Доказательства для остальных трёх случаев аналогичны.
	\end{proof}
	
	\item Если $f(x) \neq 0$~--- функция, возрастающая/убывающая на множестве~$X$ и сохраняющая на нём знак, то $h(x) = \frac1{f(x)}$~--- убывающая/возрастающая на~$X$ функция.
	\begin{proof}
	Пусть $f(x)$ возрастает на~$X$, $x_1, x_2 \in X \colon x_1 < x_2$.
	\begin{equation*}
	x_1 < x_2 \Rightarrow
	f(x_1) < f(x_2) \Rightarrow
	\frac{f(x_1)}{f(x_1)f(x_2)} < \frac{f(x_2)}{f(x_1)f(x_2)} \Rightarrow
	\frac1{f(x_1)} > \frac1{f(x_2)} \Rightarrow
	h(x_1) > h(x_2)
	\end{equation*}
	
	Значит, $h(x)$ убывает на~$X$.
		
	Доказательство для случая убывания аналогично.
	\end{proof}
	
	\item Если $f(x)$ и $g(x)$~--- обе возрастающие или обе убывающие на множестве~$X$ функции, то $h(x) = g(f(x))$~--- возрастающая на~$X$ функция.
	\begin{proof}
	Пусть $f(x)$ и $g(x)$ возрастают на~$X$, $x_1, x_2 \in X \colon x_1 < x_2$.
	\begin{equation*}
	x_1 < x_2 \Rightarrow
	f(x_1) < f(x_2) \Rightarrow
	g(f(x_1)) < g(f(x_2)) \Rightarrow
	h(x_1) < h(x_2)
	\end{equation*}
	
	Значит, $h(x)$ возрастает на~$X$.
	
	Доказательство для случая убывания аналогично.
	\end{proof}
\end{enumerate}

\subsection{Чётные и нечётные функции}
\index{Функция!чётная} Функция~$f(x)$ называется \textbf{чётной}, если $\forall x \in D(f) \ {-x} \in D(f) \lAnd f(-x) = f(x)$.

\index{Функция!нечётная} Функция~$f(x)$ называется \textbf{нечётной}, если $\forall x \in D(f) \ {-x} \in D(f) \lAnd f(-x) = -f(x)$.

Свойства чётных и нечётных функций:
\begin{enumerate}
	\item Функция~$f(x) \colon \forall x \in D(f) \ {-x} \in D(f)$ единственным образом может быть представлена в виде суммы чётной и нечётной функций.
	\begin{proof}
	\begin{itemize}
		\item Докажем представимость.
		Пусть
		\begin{equation*}
		g(x) = \frac{f(x) + f(-x)}2, \ h(x) = \frac{f(x) - f(-x)}2
		\end{equation*}
		
		Тогда
		\begin{equation*}
		g(-x) = \frac{f(-x) + f(x)}2 = g(x), \ h(-x) = \frac{f(-x) - f(x)}2 = -h(x)
		\end{equation*}
		
		Значит, $g(x)$ чётна, $h(x)$ нечётна.
		\begin{equation*}
		f(x) = \frac12 f(x) + \frac12 f(x) + \frac12 f(-x) - \frac12 f(-x) =
		\frac{f(x) + f(-x)}2 + \frac{f(x) - f(-x)}2 =
		g(x) + h(x)
		\end{equation*}
		
		\item Докажем единственность представления.
		Пусть $f(x) = g(x) + h(x) = g_1(x) + h_1(x)$, где $g, g_1$~--- чётные функции, $h, h_1$~--- нечётные функции.
		\begin{equation*}
		g(x) + h(x) = g_1(x) + h_1(x) \Leftrightarrow
		g(x) - g_1(x) = h_1(x) - h(x)
		\end{equation*}
		
		Подставляя $-x$, получим
		\begin{equation*}
		g(-x) - g_1(-x) = h_1(-x) - h(-x) \Leftrightarrow
		g(x) - g_1(x) = h(x) - h_1(x)
		\end{equation*}
		
		Тогда
		\begin{equation*}
		h_1(x) - h(x) = g(x) - g_1(x) = h(x) - h_1(x) \Rightarrow
		\end{equation*}
		\begin{equation*}
		\Rightarrow g(x) - g_1(x) = 0 \And h_1(x) - h(x) = 0 \Rightarrow
		g_1(x) = g(x) \lAnd h_1(x) = h(x)
		\end{equation*}
	\end{itemize}
	\end{proof}
	
	\item Если $f(x)$ и $g(x)$~--- чётные/нечётные функции, то $h(x) = f(x) + g(x)$ чётна/нечётна.
	\begin{proof}
	\begin{itemize}
		\item Пусть $f(x)$ и $g(x)$ чётны, тогда $h(-x) = f(-x) + g(-x) = f(x) + g(x) = h(x)$, значит, $h(x)$ чётна.
		\item Пусть $f(x)$ и $g(x)$ нечётны, тогда $h(-x) = f(-x) + g(-x) = -f(x) - g(x) = -h(x)$, значит, $h(x)$ нечётна.
	\end{itemize}
	\end{proof}
	
	\item Если $f(x)$~--- чётная/нечётная функция, то $h(x) = C f(x)$ чётна/нечётна.
	\begin{proof}
	\begin{itemize}
		\item Пусть $f(x)$ чётна, тогда $h(-x) = C f(-x) = C f(x) = h(x)$, значит, $h(x)$ чётна.
		\item Пусть $f(x)$ нечётна, тогда $h(-x) = C f(-x) = -C f(x) = -h(x)$, значит, $h(x)$ нечётна.
	\end{itemize}
	\end{proof}
	
	\item Если $f(x)$ и $g(x)$~--- обе чётные или обе нечётные функции, то $h(x) = f(x)g(x)$ чётна.
	\begin{proof}
	\begin{itemize}
		\item Пусть $f(x)$ и $g(x)$ чётны, тогда $h(-x) = f(-x)g(-x) = f(x)g(x) = h(x)$, значит, $h(x)$ чётна.
		\item Пусть $f(x)$ и $g(x)$ нечётны, тогда $h(-x) = f(-x)g(-x) = (-f(x)) (-g(x)) = h(x)$, значит, $h(x)$ нечётна.
	\end{itemize}
	\end{proof}
	
	\item Если $f(x)$ и $g(x)$~--- нечётная и чётная функции, то $h(x) = f(x)g(x)$ нечётна.
	\begin{proof}
	$h(-x) = f(-x)g(-x) = -f(x)g(x) = -h(x)$, значит, $h(x)$ нечётна.
	\end{proof}
	
	\item Если $f(x)$ и $g(x)$~--- нечётные функции, то $h(x) = g(f(x))$ нечётна.
	\begin{proof}
	$h(-x) = g(f(-x)) = g(-f(x)) = -g(f(x)) = -h(x)$, значит, $h(x)$ нечётна.
	\end{proof}
	
	\item Если $f(x)$ и $g(x)$~--- нечётная и чётная функции соответственно, то $h(x) = g(f(x))$ чётна.
	\begin{proof}
	$h(-x) = g(f(-x)) = g(-f(x)) = g(f(x)) = h(x)$, значит, $h(x)$ чётна.
	\end{proof}
	
	\item Если $f(x)$ и $g(x)$~--- функции, причём $f(x)$ чётна, то $h(x) = g(f(x))$ чётна.
	\begin{proof}
	$h(-x) = g(f(-x)) = g(f(x)) = h(x)$, значит, $h(x)$ чётна.
	\end{proof}
\end{enumerate}

\subsection{Ограниченные функции}
\index{Функция!ограниченная} Функция~$f(x)$ называется \textbf{ограниченной сверху на множестве~$X$}, если $\exists M \colon \forall x \in X \ f(x) \leqslant M$.
Если $X \opbr= D(f)$, то $f(x)$ называется \textbf{ограниченной сверху}.

Функция~$f(x)$ называется \textbf{ограниченной снизу на множестве~$X$}, если $\exists m \colon \forall x \in X \ f(x) \geqslant m$.
Если $X = D(f)$, то $f(x)$ называется \textbf{ограниченной снизу}.

Функция называется \textbf{ограниченной на множестве}, если она ограничена и сверху, и снизу на этом множестве.
Если данное множество совпадает с областью определения этой функции, то она называется \textbf{ограниченной}.

\subsection{Обратные функции}
\index{Функция!обратимая} Функция называется \textbf{обратимой}, если $\forall x_1, x_2 \in D(f) \ (x_1 \neq x_2 \Rightarrow f(x_1) \neq f(x_2))$.

\index{Функция!обратная} Функция~$g \colon Y \to X$ называется \textbf{обратной к} обратимой \textbf{функции~$f \colon X \to Y$}, если $\forall x \in X \ g(f(x)) = x \opbr\lAnd \forall y \in Y \ f(g(y)) = y$, и обозначается $f^{-1}$.

\begin{statement}
$(f^{-1})^{-1} = f$.
\end{statement}
\begin{proof}
Пусть даны функции $f \colon X \to Y$ и $g \colon Y \to X$, причём $g = f^{-1}$, тогда
$\forall x \in Y \ f(g(x)) = x \opbr\lAnd \forall y \in X \ g(f(y)) = y \Rightarrow f = g^{-1} = (f^{-1})^{-1}$.
\end{proof}

Функции $f$ и $g = f^{-1}$ называются \textbf{взаимно обратными}.

\begin{theorem}
Графики взаимно обратных функций симметричны относительно прямой~$y = x$.
\end{theorem}
\begin{proof}
Пусть даны функции $f \colon X \to Y$ и $g \colon Y \to X$, причём $g = f^{-1}$, тогда $f(a) = b$, $g(b) \opbr= g(f(a)) \opbr= a$, где $a \in X$, $b \in Y$.
Т.\,о., точка~$(a, b)$ принадлежит графику функции~$f$ $\Leftrightarrow$ точка~$(b, a)$ принадлежит графику функции~$g$.

Найдём расстояния от точек~$(a, b)$ и $(b, a)$ до произвольной точки~$(c, c)$ прямой~$y = x$:
\begin{equation*}
d_1 = \sqrt{(a - c)^2 + (b - c)^2}, \
d_2 = \sqrt{(b - c)^2 + (a - c)^2}
\end{equation*}

$d_1 = d_2$, значит, прямая~$y = x$~--- серединный перпендикуляр к отрезку, концами которого являются точки $(a, b)$ и $(b, a)$, поэтому они симметричны относительно $y = x$.
В силу того, что $a$ может принимать любое значение из множества~$X$, графики функций $f$ и $g$ симметричны относительно прямой~$y = x$.
\end{proof}

\subsection{Линейная функция}
\index{Функция!линейная} \textbf{Линейной} называется функция вида $y = kx + b$.
$k$ называется \textbf{угловым коэффициентом}.

Очевидно, что $D(y) = E(y) = \mathbb R$.

Докажем монотонность линейной функции при $k \neq 0$.
\begin{enumerate}
	\item Пусть $k < 0, x_1 < x_2$, тогда
	\begin{equation*}
	f(x_1) - f(x_2) =
	k x_1 + b - k x_2 - b =
	k(x_1 - x_2) > 0 \Rightarrow
	f(x_1) > f(x_2)
	\end{equation*}
	
	Значит, функция убывает.
	
	\item Аналогичным образом легко доказать, что при~$k > 0$ функция возрастает.
\end{enumerate}

График линейной функции~--- прямая.
\begin{center}
\shorthandoff{"}
\begin{tikzpicture}
\drawaxis{-3}{3}{-2}{3};

\def\func#1{-1/2*#1 + 0.5}
\draw[name path=plot] (\left_x, \func{\left_x}) coordinate (start)
	-- (\right_x, \func{\right_x}) coordinate[pos=0.3](point)
	node[below left] {$y = kx + b$, $k < 0$};

% указываем угол \alpha
\coordinate (x) at (\right_x, 0);
\draw[name intersections={of=plot and x_axis, by=zero}]
	pic[draw, "$\alpha$", angle eccentricity=1.5, angle radius=3mm] {angle = x--zero--start};

\printcoordsonaxis{point}{x_0}{y_0}[right];
\end{tikzpicture}
\begin{tikzpicture}
\drawaxis{-3}{3}{-2}{3};

\def\func#1{#1 - 0.5}
\draw[name path=plot] (\left_x, \func{\left_x})
	-- (\right_x, \func{\right_x}) coordinate (end)
	coordinate[pos=0.8](point)
	node[above left] {$y = kx + b$, $k > 0$};

% указываем угол \alpha
\coordinate (x) at (\right_x, 0);
\draw[name intersections={of=plot and x_axis, by=zero}]
	pic[draw, "$\alpha$", angle eccentricity=1.6, angle radius=3mm] {angle = x--zero--end};

\printcoordsonaxis{point}{x_0}{y_0};
\end{tikzpicture}
\shorthandon{"}
\end{center}

Докажем, что $k = \tg \alpha$.
Пусть $y_0 = k x_0 + b$.
Заметим, что при $k \neq 0$ график линейной функции пересекает ось абсцисс в точке~$(-\frac{b}k, 0)$.
\begin{enumerate}
	\item Если $k < 0$, то для определённости предположим, что $x_0 < -\frac{b}k$.
	\begin{equation*}
	\tg \alpha =
	-\tg (\pi - \alpha) =
	-\frac{y_0}{-\frac{b}k - x_0} =
	\frac{k y_0}{k x_0 + b} =
	k
	\end{equation*}
	
	\item При $k = 0$ $\tg \alpha = 0$.
	
	\item Если $k > 0$, то для определённости предположим, что $x_0 > -\frac{b}k$.
	\begin{equation*}
	\tg \alpha =
	\frac{y_0}{x_0 - (-\frac{b}k)} =
	\frac{k y_0}{k x_0 + b} =
	k
	\end{equation*}
\end{enumerate}

\subsection{Квадратичная функция}
\index{Функция!квадратичная} \textbf{Квадратичной} называется функция вида $y = ax^2 + bx + c$, где $a \neq 0$.

Очевидно, что $D(y) = \mathbb R$.

Для исследования функции выделим в правой части уравнения полный квадрат:
\begin{equation*}
y = a \left( x^2 + \frac{b}a x \right) + c \Leftrightarrow
y = a \left( x + \frac{b}{2a} \right)^2 + c - \frac{b^2}{4a} \Leftrightarrow
y = a \left( x + \frac{b}{2a} \right)^2 - \frac{b^2 - 4ac}{4a}
\end{equation*}
\begin{itemize}
	\item Пусть $a < 0$.
	
	$\displaystyle E(y) = \left( -\infty; -\frac{b^2 - 4ac}{4a} \right]$
	
	Функция возрастает на~$\bigl( -\infty; -\frac{b}{2a} \bigr]$ и убывает на~$\bigl[ -\frac{b}{2a}; +\infty \bigr)$.
	
	\item Пусть $a > 0$.
	
	$\displaystyle E(y) = \left[ -\frac{b^2 - 4ac}{4a}; +\infty \right)$
	
	Функция убывает на~$\bigl( -\infty; -\frac{b}{2a} \bigr]$ и возрастает на~$\bigl[ -\frac{b}{2a}; +\infty \bigr)$.
\end{itemize}

Найдём нули функции:
\begin{equation*}
a \left( x + \frac{b}{2a} \right)^2 - \frac{b^2 - 4ac}{4a} = 0 \Leftrightarrow
\left( x + \frac{b}{2a} \right)^2 = \frac{b^2 - 4ac}{4a^2} \Leftrightarrow
x + \frac{b}{2a} = \pm\frac{\sqrt{b^2 - 4ac}}{2a} \Leftrightarrow
\end{equation*}
\begin{equation*}
\Leftrightarrow x = \frac{-b \pm \sqrt{b^2 - 4ac}}{2a}, \ b^2 - 4ac \geqslant 0
\end{equation*}

Если $b^2 - 4ac < 0$, то график квадратичной функции не пересекает ось абсцисс.
\index{Дискриминант} Выражение $D = b^2 - 4ac$ называется \textbf{дискриминантом квадратного многочлена}.

\index{Парабола} График квадратичной функции называется \textbf{параболой}.
При $a < 0$ её \textbf{ветви} направлены вниз, а при $a > 0$~--- вверх.
Точка наименьшего или наибольшего значения функции называется \textbf{вершиной параболы} (на рисунках обозначена через $M$), которая имеет координаты~$\bigl( -\frac{b}{2a}, -\frac{b^2 - 4ac}{4a} \bigr)$.
\begin{center}
\begin{tikzpicture}
\drawaxis{-2}{2}{-3}{3};
\draw (-0.25, 1) coordinate (M) node {$\bullet$} node[above] {$M$}
	(\left_x, \bottom_y) parabola bend (M) (1.5, \bottom_y)
	node[above left,rotate=-75] {$\scriptstyle y = ax^2 + bx + c$, $\scriptstyle a < 0$};
\end{tikzpicture}
\begin{tikzpicture}
\drawaxis{-2}{2}{-3}{3};
\draw (0.25, 0.5) coordinate (M) node {$\bullet$} node[below] {$M$}
	(-1, \top_y) parabola bend (M) (1.5, \top_y)
	node[below left, rotate=70] {$\scriptstyle y = ax^2 + bx + c$, $\scriptstyle a > 0$};
\end{tikzpicture}
\end{center}

\subsection{Рациональная функция}
\index{Функция!рациональная} \textbf{Рациональной} называется функция вида $y = \dfrac{a_n x^n + \ldots + a_1 x + a_0}{b_m x^m + \ldots + b_1 x + b_0}$.

$D(y) = \{ x \in \mathbb R \mid b_m x^m + \ldots + b_0 \neq 0 \}$

Выражение~$\frac{P(x)}{Q(x)}$, где $P(x)$ и $Q(x)$~--- многочлены, называется \textbf{правильной рациональной дробью}, если $\deg P(x) < \deg Q(x)$, иначе~--- \textbf{неправильной рациональной дробью}.

Делением многочленов любую рациональную функцию можно представить в виде суммы многочлена и правильной рациональной дроби.