\section{Многочлены от нескольких переменных}
\index{Многочлен!от нескольких переменных}
\begin{enumerate}
	\item В~многочлене $a_n x^n + a_{n-1} x^{n-1} + \ldots + a_0$ подставим $a_i = P_i(y)$, $i = 0, 1, \ldots, n$~--- многочлен от~$y$.
	Получим многочлен от $x$ и $y$.
	
	\item Пусть имеем многочлен от $n$~переменных.
	Подставим вместо его коэффициентов многочлен от одной переменной, получим многочлен от $n + 1$~переменных.
\end{enumerate}

Одночлены многочлена будем записывать в~лексикографическом порядке степеней переменных (члены с б\'{о}ль\-ши\-ми степенями идут раньше).

\begin{statement}
Старший член произведения многочленов равен произведению старших членов множителей.
\end{statement}
\begin{proof}
Перемножая члены с наибольшими показателями старшей переменной, получим член с наибольшим показателем при этой переменной.
Проведя аналогичные рассуждения для остальных переменных, придём к выводу, что полученный член является старшим.
\end{proof}

Аналогично доказывается следующее утверждение.
\begin{statement}
Младший член произведения многочленов равен произведению младших членов множителей.
\end{statement}

\subsection{Симметрические многочлены}
\index{Многочлен!симметрический} Многочлен называется \textbf{симметрическим}, если при перестановке переменных он не изменяется.

\begin{statement}
Если $f(x_1, \ldots, x_n) = a x_1^{i_1} x_2^{i_2} \cdot \ldots \cdot x_n^{i_n} + \ldots$~--- симметрический многочлен, то $i_1 \opbr\geqslant i_2 \opbr\geqslant \ldots \opbr\geqslant i_n$.
\end{statement}
\begin{proofcontra}
Пусть $\exists r < q \colon i_r < i_q$, тогда $f$ содержит
$b x_1^{i_1} \opbr\cdot x_2^{i_2} \opbr\cdot \ldots \opbr\cdot x_r^{i_q} \opbr\cdot \ldots \opbr\cdot x_q^{i_r} \opbr\cdot \ldots \opbr\cdot x_n^{i_n}$, который старше, чем
$a x_1^{i_1} \cdot x_2^{i_2} \cdot \ldots \cdot x_n^{i_n}$.
Противоречие.
\end{proofcontra}

\textbf{Элементарными} симметрическими многочленами от $n$~переменных называются многочлены
\begin{equation*}
\sigma_1(x_1, \ldots, x_n) = \sum_{i=1}^n x_i, \
\sigma_2(x_1, \ldots, x_n) = \sum_{i=1}^n \sum_{j=i+1}^n x_i x_j, \ \ldots, \
\sigma_n(x_1, \ldots, x_n) = x_1 x_2 \cdot \ldots \cdot x_n
\end{equation*}

\index{Теорема!основная т. о симметрических многочленах}
\begin{theorem}[основная теорема о симметрических многочленах]
Любой симметрический многочлен может быть представлен в~виде многочлена от элементарных симметрических многочленов.
\end{theorem}
\begin{proof}
Пусть $f(x_1, \ldots, x_n) = a x_1^{k_1} \cdot \ldots \cdot x_n^{k_n} + \ldots$~--- симметрический многочлен. Введём
\begin{equation*}
g_1(\sigma_1, \ldots, \sigma_n) = a \sigma_1^{k_1 - k_2} \sigma_2^{k_2 - k_3} \cdot \ldots \cdot \sigma_{n-1}^{k_{n-1} - k_n} \sigma_n^{k_n} =
\end{equation*}
\begin{equation*}
= a(x_1 + \ldots)^{k_1 - k_2} (x_1 x_2 + \ldots)^{k_2 - k_3} \cdot \ldots \cdot (x_1 x_2 \cdot \ldots \cdot x_{n-1} + \ldots)^{k_{n-1} - k_n} (x_1 x_2 \cdot \ldots \cdot x_n)^{k_n} =
\end{equation*}
\begin{equation*}
= a x_1^{k_1} \cdot \ldots \cdot x_n^{k_n} + \ldots
\end{equation*}

Тогда старший член многочлена $f_1 = f - g_1$ младше старшего члена многочлена~$f$.
Повторим те~же действия с многочленом~$f_1$.
Будем продолжать таким образом, пока не получим ноль.
В итоге получим $f = g_1 + g_2 + \ldots + g_m$, где $g_1, g_2, \ldots, g_m$~--- многочлены от элементарных симметрических многочленов.
\end{proof}