\section{Многочлены от одной переменной}
\index{Одночлен} \index{Моном} \textbf{Одночленом} (\textbf{мономом}) называется произведение числового множителя и нуля и более переменных, взятых каждая в~неотрицательной степени.

\textbf{Степенью одночлена} называется сумма степеней входящих в~него переменных.
Степень тождественного нуля равна~$-\infty$.

\index{Многочлен} \index{Полином} \textbf{Многочленом} (\textbf{полиномом}) \textbf{от одной переменной} называется сумма вида
\begin{equation*}
a_n x^n + a_{n-1} x^{n-1} + \ldots + a_1 x + a_0, \ a_n \neq 0
\end{equation*}
где $x_1, \ldots, x_n$~--- переменные.

\index{deg} \textbf{Степенью многочлена~$f$} называется максимальная из степеней его одночленов и обозначается $\deg f$.
Многочлен $1$-й степени называется \textbf{линейным}, $2$-й степени~--- \textbf{квадратным}, $3$-й степени~--- \textbf{кубическим}.

\begin{statement}
Пусть $f$ и $g$~--- многочлены, тогда $\deg (f + g) \leqslant \max \{ \deg f, \deg g \}$.
\end{statement}
\begin{statement}
Пусть $f$ и $g$~--- многочлены, тогда $\deg fg = \deg f + \deg g$.
\end{statement}

\subsection{Деление многочленов}
\begin{theorem}
Пусть $f(x)$ и $g(x) \neq 0$~--- многочлены, тогда существуют единственные многочлены $q(x)$ и $r(x)$ такие, что $f = qg + r$, причём $\deg r < \deg g$.
\end{theorem}
\begin{proof}
Пусть $\deg f = n$, $\deg g = m$.
\begin{enumerate}
	\item Докажем существование.
	Если $m = 0$, то $q = \frac{f}g$, $r = 0$.
	Пусть $m > 0$.
	Если $n < m$, то $q = 0$, $r = f$.
	Пусть $n \geqslant m$.
	В~таком случае докажем существование $q$ и $r$ методом математической индукции.
		\indbase $n = m$.
		Пусть $f = a x^n + f_1$, $g = b x^n + g_1$, тогда $\deg f_1, \deg g_1 < n$.
		$\deg (f_1 - \frac{a}b g_1) \opbr\leqslant \max \{ \deg f_1, \deg g_1 \} \opbr< \deg g$, тогда $q = \frac{a}b$, $r = f_1 - \frac{a}b g_1$.
		\indstep Пусть $n > m$, теорема верна для $k < n$ и $f = a x^n + f_1$, $g = b x^m + g_1$.
		Рассмотрим
		\begin{equation*}
		h_1(x) = \frac{a}b x^{n-m} \Rightarrow
		h_1 g = a x^n + h_1 g_1 \Rightarrow
		f - h_1 g = a x^n + f_1 - a x^n - h_1 g_1 = f_1 - h_1 g_1
		\end{equation*}
		
		Тогда
		\begin{equation*}
		\deg h_1 g_1 = \deg h_1 + \deg g_1 < (n - m) + m = n \Rightarrow
		\end{equation*}
		\begin{equation*}
		\Rightarrow \deg (f - h_1 g) = \deg (f_1 - h_1 g_1) \leqslant \max \{ \deg f_1, \deg h_1 g_1 \} < n
		\end{equation*}
		
		По предположению индукции
		\begin{equation*}
		f - h_1 g = q_1 g + r \Rightarrow
		f = (h_1 + q_1)g + r, \ \deg r < \deg g
		\end{equation*}
		\indend
		
	\item Докажем единственность.
	Пусть
	\begin{equation*}
	f = q_1 g + r_1 = q_2 g + r_2 \Rightarrow
	(q_1 - q_2)g = r_2 - r_1, \ \deg r_1, \deg r_2 < \deg g
	\end{equation*}
	
	Возможны два случая:
	\begin{enumerate}
		\item $q_1 \neq q_2 \Rightarrow
		\deg (r_1 - r_2) \leqslant \max \{ \deg r_1, \deg r_2 \} < \deg g \leqslant \deg (q_1 - q_2)g$
		
		Противоречие.
		
		\item $q_1 = q_2 \Rightarrow r_1 = r_2$
	\end{enumerate}
\end{enumerate}
\end{proof}

Многочлен~$q$ называется \textbf{частным}, а $r$~--- \textbf{остатком от деления $\frac{f}g$}.
Если $r = 0$, то говорят, что \textbf{$f$ делится на $g$ без остатка}, и пишут $f \mult g$.

\textbf{Общим делителем} многочленов $f(x)$ и $g(x)$ называется многочлен~$h(x)$, на который и~$f$, и~$g$ делятся без остатка: $f = ph$, $g = qh$.

\textbf{Наибольшим} называется общий делитель наибольшей степени и обозначается $\NOD$.

\index{Алгоритм!Евклида}
\begin{theorem}[алгоритм Евклида]
Любые два многочлена имеют единственный $\NOD$.
\end{theorem}
\begin{proof}
Будем делить многочлены следующим образом:
\begin{equation*}
f = q_1 g + r_1, \
g = q_2 r_1 + r_2, \
r_1 = q_3 r_2 + r_3, \ \ldots,
\end{equation*}
\begin{equation*}
r_{n-1} = q_{n+1} r_n + r_{n+1}, \
r_n = q_{n+2} r_{n+1} + r_{n+2} = q_{n+2} r_{n+1},
\end{equation*}
\begin{equation*}
\deg g > \deg r_1 > \deg r_2 > \ldots > \deg r_{n+1} > \deg r_{n+2} = -\infty
\end{equation*}

Докажем, что $r_{n+1}$~--- общий делитель $f$ и $g$.
\begin{equation*}
r_n \mult r_{n+1} \Rightarrow
r_{n-1} \mult r_{n+1} \Rightarrow
\ldots \Rightarrow
r_1 \mult r_{n+1} \Rightarrow
g \mult r_{n+1} \Rightarrow
f \mult r_{n+1}
\end{equation*}

Докажем, что $\forall h \ f \mult h \lAnd g \mult h \Rightarrow r_{n+1} \mult h$.
\begin{equation*}
f \mult h \lAnd g \mult h \Rightarrow r_1 \mult h, \
g \mult h \lAnd r_1 \mult h \Rightarrow r_2 \mult h, \
r_1 \mult h \lAnd r_2 \mult h \Rightarrow r_3 \mult h, \ \ldots, \
r_{n-1} \mult h \lAnd r_n \mult h \Rightarrow r_{n+1} \mult h
\end{equation*}

Значит, $r_{n+1} = \NOD(f, g)$.
\end{proof}

\subsection{Корень многочлена}
\index{Корень} \textbf{Корнем многочлена~$f(x)$} называется такое число~$a$, что $f(a) = 0$.

\index{Теорема!Безу}
\begin{theorem}[Безу]
Остаток от деления многочлена~$f(x)$ на двучлен~$x - a$ равен $f(a)$.
\end{theorem}
\begin{proof}
\begin{equation*}
f(x) = g(x)(x - a) + r \Rightarrow f(a) = g(a)(a - a) + r \Leftrightarrow r = f(a)
\end{equation*}
\end{proof}

\begin{consequent}
\label{conseq:division_by_binomial}
Если $a$~--- корень~$f(x)$, то $f(x)$ делится на~$x - a$ без остатка.
\end{consequent}

\textbf{Кратностью корня~$a$} многочлена~$f(x)$ называется число~
$m \colon f(x) \mult (x - a)^m \lAnd f(x) \notmult (x - a)^{m+1}$.

\begin{theorem}
Если многочлен~$P(x) = a_n x^n + a_{n-1} x^{n-1} + \ldots + a_1 x + a_0$, где $a_0, \ldots, a_n \in \mathbb Z$, имеет рациональный корень, то этот корень равен частному делителя числа~$a_0$ и делителя числа~$a_n$.
\end{theorem}
\begin{proof}
Пусть $\frac{p}q$~--- несократимая дробь, являющаяся корнем $P(x)$.
Тогда
\begin{equation*}
a_n \left( \frac{p}q \right)^n + a_{n-1} \left( \frac{p}q \right)^{n-1} + \ldots + a_1 \frac{p}q + a_0 = 0 \Leftrightarrow
a_n p^n + a_{n-1} p^{n-1} q + \ldots + a_1 pq^{n-1} a_0 q^n = 0
\end{equation*}

Учитывая, что $\NOD(p, q) = 1$, получим
\begin{equation*}
a_n p^n = -q (a_{n-1} p^{n-1} + \ldots + a_1 pq^{n-2} + a_0 q^{n-1}) \Rightarrow
a_n \mult q
\end{equation*}
\begin{equation*}
a_0 q^n = -p (a_n p^{n-1} + a_{n-1} p^{n-2} q + \ldots + a_1 q^{n-1}) \Rightarrow
a_0 \mult p
\end{equation*}
\end{proof}

\index{Теорема!основная т. алгебры}
\begin{theorem}[основная теорема алгебры]
\label{th:fundamental_th_of_algebra}
Если $f(x)$~--- многочлен, отличный от константы, то он имеет хотя~бы один комплексный корень.
\end{theorem}%
Доказательство теоремы слишком сложно, поэтому здесь не приводится.

\begin{consequent}
\label{conseq:n_roots_of_polynomial}
Многочлен $n$"~й степени имеет ровно $n$ комплексных корней с учётом их кратности.
\end{consequent}
\begin{proof}
Пусть $f(x)$~--- многочлен $n$"~й степени.
По основной теореме алгебры он имеет корень~$a$, тогда по следствию~\ref*{conseq:division_by_binomial} $f(x) = g(x)(x - a)$, где $g(x)$~--- многочлен степени $n - 1$, который также имеет корень.
Будем повторять деление до~тех пор, пока не получим константу.
Т.\,о., получим $n$~корней.
\end{proof}

\begin{consequent}
Любой многочлен~$f(x)$ $n$"~й степени представим в~виде
\begin{equation*}
f(x) = a(x - x_0)(x - x_1) \cdot \ldots \cdot (x - x_{n-1})
\end{equation*}
где $a$~--- число, $x_0, \ldots, x_{n-1}$~--- корни $f(x)$.
\end{consequent}

\begin{lemma}
Если $f(x)$~--- многочлен с действительными коэффициентами, $z \in \mathbb C$, то $\overline{f(z)} = f(\overline z)$.
\end{lemma}
\begin{proof}
Пусть $z_1 = a_1 + b_1 i$, $z_2 = a_2 + b_2 i$, $a_1, b_1, a_2, b_2 \in \mathbb R$.
Многочлен строится при помощи операций сложения и умножения, поэтому достаточно доказать следующее:
\begin{enumerate}
	\item $\overline{z_1 + z_2} = \overline{z_1} + \overline{z_2}$
	\begin{equation*}
	\overline{z_1 + z_2} = \overline{(a_1 + a_2) + (b_1 + b_2)i} = (a_1 + a_2) - (b_1 + b_2)i =
	(a_1 - b_1 i) + (a_2 - b_2 i) = \overline{z_1} + \overline{z_2}
	\end{equation*}
	
	\item $\overline{z_1 z_2} = \overline{z_1} \cdot \overline{z_2}$
	\begin{equation*}
	\overline{z_1 z_2} = \overline{(a_1 a_2 - b_1 b_2) + (a_1 b_2 + a_2 b_1)i} =
	(a_1 a_2 - b_1 b_2) - (a_1 b_2 + a_2 b_1)i = (a_1 - b_1 i)(a_2 - b_2 i) =
	\overline{z_1} \cdot \overline{z_2}
	\end{equation*}
\end{enumerate}
Тогда $\overline{a_n z^n + \ldots + a_1 z + a_0} = a_n \overline z^n + \ldots + a_1 \overline z + a_0$ при $a_0, a_1, \ldots, a_n \in \mathbb R$.
\end{proof}

\begin{theorem}
\label{th:polynomial_factorization}
Любой многочлен с действительными коэффициентами можно разложить на линейные и квадратные множители с действительными коэффициентами.
\end{theorem}
\begin{proof}
Пусть $f(x)$~--- многочлен с действительными коэффициентами, тогда если $f(z) = 0$, то $f(\overline z) \opbr= \overline{f(z)} \opbr= \overline 0 \opbr= 0$.
Значит, если $a + bi$~--- корень~$f(x)$, то $a - bi$~--- тоже корень~$f(x)$.
Имеем:
\begin{equation*}
f(x) = a \prod_{j=1}^{m} (x - x_j) \cdot \prod_{j=1}^{n} (x - (a_j + b_j i))(x - (a_j - b_j i)) =
a \prod_{j=1}^{m} (x - x_j) \cdot \prod_{j=1}^{n} (x^2 - 2a_j x + a_j^2 + b_j^2)
\end{equation*}
где $a, x_1, \ldots, x_m, a_1, \ldots, a_n, b_1, \ldots, b_n \in \mathbb R$,
$x_1, \ldots, x_m, a_1 + b_1 i, \ldots, a_n + b_n i$~--- корни~$f(x)$.
\end{proof}

\index{Формула!Виета}
\begin{theorem}[формулы Виета]
Пусть 
\begin{equation}
\label{eq:formulas_in_Vieta_theorem}
f(x) = a_n x^n + a_{n-1} x^{n-1} + \ldots + a_1 x + a_0 = a_n(x - x_0)(x - x_1) \cdot \ldots \cdot (x - x_{n-1})
\end{equation}
тогда
\begin{equation*}
a_{n-1} = -a_n \sum_{i=0}^{n-1} x_i, \
a_{n-2} = a_n \sum_{i=0}^{n-1} \sum_{j=i+1}^{n-1} x_i x_j, \
a_{n-3} = -a_n \sum_{i=0}^{n-1} \sum_{j=i+1}^{n-1} \sum_{k=j+1}^{n-1} x_i x_j x_k, \ \ldots,
\end{equation*}
\begin{equation*}
a_2 = (-1)^{n-1} \cdot a_n \sum_{i=0}^{n-1} x_0 x_1 \cdot \ldots \cdot x_{i-1} x_{i+1} \cdot \ldots \cdot x_{n-1}, \
a_1 = (-1)^n \cdot a_n x_0 x_1 \cdot \ldots \cdot x_{n-1}
\end{equation*}
\end{theorem}%
Для доказательства достаточно раскрыть скобки в~правой части равенства (\ref*{eq:formulas_in_Vieta_theorem}).

\begin{theorem}
Пусть на плоскости даны $n + 1$~точек, никакие две из которых не лежат на прямой, паралелльной оси ординат, тогда через них проходит единственная кривая $n$"~го порядка.
\end{theorem}
\begin{proof}
Пусть данные точки заданы координатами $(a_0, b_0), (a_1, b_1), \ldots, (a_n, b_n)$.
\begin{enumerate}
	\item Докажем существование.
	Рассмотрим многочлен~$f(x)$, называемый \textbf{интерполяционным многочленом Лагранжа}:
	\begin{equation*}
	f(x) = \sum_{i=0}^n b_i \frac
	{(x - a_0) \cdot \ldots \cdot (x - a_{i-1})(x - a_{i+1}) \cdot \ldots \cdot (x - a_n)}
	{(a_i - a_0) \cdot \ldots \cdot (a_i - a_{i-1})(a_i - a_{i+1}) \cdot \ldots \cdot (a_i - a_n)}
	\end{equation*}
	
	Докажем, что кривая, задаваемая функцией~$f$, проходит через все данные точки.
	Рассмотрим точку~$(a_k, b_k)$.
	Подставим $x = a_k$, тогда $k$"~е (считая с нуля) слагаемое равно $b_k$, а остальные~--- 0.
	
	\item Докажем единственность.
	Предположим, что существуют многочлены $f(x)$ и $g(x)$ $n$"~й степени такие, что $f(a_i) = g(a_i) = b_i$, где $i = 0, 1, \ldots, n$.
	Рассмотрим $h(x) = f(x) - g(x) \Rightarrow \deg h \leqslant n \Rightarrow$ $h(x)$ имеет не более $n$~корней.
	При этом $h(x) = 0$ в $n + 1$~точках $\Rightarrow$ $h(x)$ тождественно равен нулю $\Rightarrow$ $f(x) = g(x)$.
\end{enumerate}
\end{proof}