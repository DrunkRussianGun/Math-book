\section{Геометрическая интерпретация булевых функций}
Пусть дана булева функция~$f \colon B^n \to B$.
$B^n$ можно отождествить с вершинами единичного куба в~$\mathbb R^n$.

С каждой булевой функцией~$f$ можно связать множество~$N_f$, состоящее из вершин, в которых~$f = 1$.

Пусть дан $n$-мерный куб.
\index{Грань} Множество его вершин, для которых значения~$x_{i_1}, \ldots, x_{i_k}$ совпадают, называется \textbf{гранью ранга $n - k$}.
Очевидно, что каждая грань однозначно соответствует некоторому элементарному конъюнкту, который принимает значение~$1$ в точности на вершинах грани.

Грань~$N_K$ называется \textbf{максимальной относительно булевой функции~$f$}, если $N_K \subseteq N_f \lAnd \forall N_{K'} \ N_K \opbr\subseteq N_{K'} \opbr\subseteq N_f \Rightarrow N_K = N_K'$.

\begin{statement}
Каждой максимальной относительно булевой функции~$f$ грани соответствует простой импликант $f$.
\end{statement}
\begin{proof}
Пусть $N_K$~--- максимальная грань.
\begin{enumerate}
	\item $K(\alpha_1, \ldots, \alpha_n) = 1 \Rightarrow
	(\alpha_1, \ldots, \alpha_n) \in N_K \subseteq N_f \Rightarrow
	f(\alpha_1, \ldots, \alpha_n) = 1 \Rightarrow$
	$K$~--- импликант $f$.
	
	\item Докажем методом от противного простоту $K$.
	Пусть $K$ не простой, тогда можно получить простой импликант~$K'$ из $K$ вычёркиванием переменных, поэтому $N_K \subset N_{K'} \subseteq N_f$, значит, $N_K$ не является максимальной гранью.
	Противоречие.
\end{enumerate}
\end{proof}

\begin{statement}
Если $K$~--- простой импликант, то $N_K$~--- максимальная грань.
\end{statement}
\begin{proofcontra}
Пусть $N_K$ не является максимальной гранью, тогда $N_K \subset N_K'$, где $N_K'$~--- максимальная грань, значит, можно получить простой импликант~$K'$ из $K$ вычёркиванием переменных, поэтому $K$ не является простым импликантом.
Противоречие.
\end{proofcontra}

\index{Покрытие} Набор граней $N_{K_1}, \ldots, N_{K_m}$ называется \textbf{покрытием булевой функции~$f$}, если $N_f = N_{K_1} \cup \ldots \cup N_{K_m}$.

\begin{statement}
Каждое покрытие булевой функции~$f$ однозначно соответствует ДНФ для~$f$.
\end{statement}
\begin{proof}
Каждому конъюнкту в ДНФ однозначно соответствует некоторая грань в покрытии, тогда покрытие из этих граней однозначно соответствует данной ДНФ.
\end{proof}

\begin{consequent}
Сокращённой ДНФ соответствует покрытие из всех максимальных граней.
\end{consequent}

ДНФ булевой функции~$f$, содержащая наименьшее число литералов, считая повторяющиеся, называется \textbf{минимальной}.

ДНФ булевой функции~$f$, содержащая наименьшее число элементарных конъюнктов, называется \textbf{кратчайшей}.

Покрытие булевой функции~$f$ называется \textbf{неприводимым}, если оно состоит только из максимальных граней и при удалении любой грани из него оно перестаёт быть покрытием, а ДНФ, соответствующая ему, называется \textbf{тупиковой}.

\begin{statement}
Кратчайшая или минимальная ДНФ является тупиковой.
\end{statement}
\begin{proofcontra}
Пусть кратчайшая/минимальная ДНФ не является тупиковой, тогда из соответствующего ей покрытия можно удалить грань.
Но в таком случае исходная ДНФ не является кратчайшей/минимальной.
Противоречие.
\end{proofcontra}