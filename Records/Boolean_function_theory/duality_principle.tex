\section{Принцип двойственности}
\index{Функция!булева!двойственная} Если $f(x_1, \ldots, x_n)$~--- булева функция, то \textbf{двойственной к~$f$ функцией} называется функция $\overline f(\overline{x_1}, \overline{x_2}, \ldots, \overline{x_n})$, обозначаемая $f^*(x_1, x_2, \ldots, x_n)$.

\begin{statement}
$(f^*)^* = f$.
\end{statement}
\begin{proof}
$\bigl( f^*(x_1, x_2, \ldots, x_n) \bigr)^* =
\bigl( \overline f(\overline{x_1}, \overline{x_2}, \ldots, \overline{x_n}) \bigr)^* =
f(x_1, x_2, \ldots, x_n)$.
\end{proof}

\index{Принцип!двойственности}
\begin{theorem}[принцип двойственности]
Если формула~$\Phi_1 = f_0(f_1, f_2, \ldots, f_n)$ задаёт некоторую функцию~$f$, то формула~$\Phi_2 = f_0^*(f_1^*, f_2^*, \ldots, f_n^*)$ задаёт $f^*$.
\end{theorem}
\begin{proof}
Без ограничения общности можно считать, что $f_1, \ldots, f_n$~--- функции от $m$~переменных, т.\,к. можно добавить несущественные переменные.
\begin{equation*}
f^*(x_1, \ldots, x_m) =
\overline f(\overline{x_1}, \ldots, \overline{x_m}) =
\overline{f_0}(\overline{\overline{f_1}}(\overline{x_1}, \ldots, \overline{x_m}), \ldots, \overline{\overline{f_n}}(\overline{x_1}, \ldots, \overline{x_m})) =
\end{equation*}
\begin{equation*}
= \overline{f_0}(\overline{f_1^*}(x_1, \ldots, x_m), \ldots, \overline{f_n^*}(x_1, \ldots, x_m)) =
f_0^*(f_1^*(x_1, \ldots, x_m), \ldots, f_n^*(x_1, \ldots, x_m)) =
\Phi_2
\end{equation*}
\end{proof}