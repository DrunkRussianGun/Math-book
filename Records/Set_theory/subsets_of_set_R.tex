\section{Подмножества множества \texorpdfstring{$\mathbb R$}{}}
\subsection{Промежутки}
\index{Промежуток} \textbf{Промежутком} называется множество вещественных чисел, которое вместе с любыми двумя числами содержит любое число между ними.
Типы промежутков:
\begin{itemize}
	\item \index{Отрезок} $[a; b] = \{ x \in \mathbb R \mid a \leqslant x \leqslant b \}$~--- \textbf{отрезок}
	\item \index{Интервал} $(a; b) = \{ x \in \mathbb R \mid a < x < b \}$~--- \textbf{интервал}
	\item $[a; b) = \{ x \in \mathbb R \mid a \leqslant x < b \}$~--- \textbf{полуинтервал}
	\item $(a; b] = \{ x \in \mathbb R \mid a < x \leqslant b \}$~--- \textbf{полуинтервал}
\end{itemize}

Также определяются \textbf{бесконечные} промежутки:
\begin{itemize}
	\item $[a; +\infty) = \{ x \in \mathbb R \mid x \geqslant a \}$
	\item $(a; +\infty) = \{ x \in \mathbb R \mid x > a \}$
	\item $(-\infty; a] = \{ x \in \mathbb R \mid x \leqslant a \}$
	\item $(-\infty; a) = \{ x \in \mathbb R \mid x < a \}$
	\item $(-\infty; +\infty) = \mathbb R$
\end{itemize}

\subsection{Окрестности}
\index{Окрестность} \textbf{Окрестностью точки~$x \in \mathbb R$} называется интервал~$(a; b) \colon x \in (a; b)$.

\textbf{$\varepsilon$-окрестностью $U_\varepsilon(x)$ точки~$x \in \mathbb R$} называется интервал~$(x - \varepsilon; x + \varepsilon)$.

\textbf{Проколотой $\varepsilon$-окрестностью $\breve U_\varepsilon(x)$ точки~$x \in \mathbb R$} называется $U_\varepsilon(x) \setminus \{ x \}$.