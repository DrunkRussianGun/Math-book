\section{Отношения}
\index{Отношение} Пусть $\circ$~--- некоторое отношение.
Оно может обладать следующими свойствами:
\begin{itemize}
	\item $a \circ a$~--- рефлексивность;
	\item $a \circ b \Rightarrow b \circ a$~--- симметричность;
	\item $a \circ b \lAnd b \circ c \Rightarrow a \circ c$~--- транзитивность.
\end{itemize}

\subsection{Отношение эквивалентности}
\index{Отношение!эквивалентности} Некоторое отношение~$\sim$ называется \textbf{отношением эквивалентности}, если оно обладает свойствами рефлексивности, симметричности и транзитивности.

\index{Класс!фактор-класс} \index{Класс!эквивалентности} \textbf{Классом эквивалентности} (\textbf{фактор"=классом}) \textbf{элемента~$x$} называется множество~$[x] = \{ y \mid y \sim x \}$, где $\sim$~--- отношение эквивалентности.

\index{Множество!фактор-множество} \textbf{Фактор"=множеством} называется множество различных фактор"=классов.