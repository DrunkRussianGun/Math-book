\section{Метод интервалов}
\index{Метод!интервалов} Метод интервалов~--- метод решения рациональных неравенств, т.\,е. неравенств вида $\frac{P(x)}{Q(x)} < 0$ или $\frac{P(x)}{Q(x)} > 0$, где $P(x)$ и $Q(x)$~--- многочлены, причём как строгих, так и нестрогих.
\begin{enumerate}
	\item Найдём корни многочленов $P(x)$ и $Q(x)$ и отметим их на числовой прямой, разбив её таким образом на интервалы.
	\item Найдём знак значения дроби $\frac{P(x)}{Q(x)}$ на каждом из полученных интервалов.
	Для этого достаточно найти знак значения дроби в любой точке, лежащей в рассматриваемом интервале.
	\item Включаем в ответ подходящие интервалы, а также корни многочленов, если они удовлетворяют неравенству.
\end{enumerate}

Заметим, что нахождение знаков значений дроби можно упростить.
\begin{itemize}
	\item Если старшие коэффициенты $P(x)$ и $Q(x)$ оба положительны или оба отрицательны, то дробь положительна на интервале $(a; +\infty)$, где $a$~--- наибольший из найденных корней.
	\item Дробь меняет знак при переходе через корень нечётной кратности и не меняет при переходе через корень чётной кратности.
\end{itemize}