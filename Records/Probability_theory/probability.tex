\section{Вероятность}
\index{Вероятность} Существует несколько определений вероятности:
\begin{enumerate}
	\item \textbf{Классическое}
	
	Рассмотрим испытание с конечным числом~$n$ равновероятных элементарных исходов.
	\textbf{Вероятностью события~$A$} называется число~$\frac{n_A}n$, где $n_A$~--- число элементарных исходов, при которых происходит событие~$A$, и обозначается~$P(A)$.
	
	\item \textbf{Геометрическое}
	
	Рассмотрим испытание с бесконечным числом равновероятных элементарных исходов, пространство которых имеет меру~$\mu(\Omega)$.
	Пусть событие~$A$ происходит, когда случайная точка попадает в подпространство~$B$.
	\textbf{Вероятностью события~$A$} называется число~$\frac{\mu(B)}{\mu(\Omega)}$ и обозначается~$P(A)$.
	
	\item \textbf{Статистическое}
	
	Рассмотрим серию из $m$ независимых подобных испытаний.
	\textbf{Вероятностью события~$A$} называется число~$\lim\limits_{m \to \infty} \frac{m_A}m$, где $m_A$~--- число испытаний, в результате которых произошло событие~$A$, и обозначается~$P(A)$.
	
	Число~$\frac{m_A}m$ называется \textbf{частотой события~$A$}.
	
	\item \textbf{Аксиоматическое}
	
	Пусть $\Omega$~--- пространство элементарных исходов испытания, $S$~--- соответствующая алгебра событий.
	\textbf{Вероятностью} называется функция~$P(A)$, заданная на~$S$ и удовлетворяющая требованиям:
	\begin{itemize}
		\item $\forall A \in S \ P(A) \geqslant 0$;
		\item $P(A + B) = P(A) + P(B)$, где $A$ и $B$~--- несовместные события;
		\item $P(\Omega) = 1$.
	\end{itemize}
	
	\index{Модель!вероятностная} Тройка $(\Omega, S, P)$ называется \textbf{вероятностной моделью испытания}.
\end{enumerate}

Свойства вероятности (доказательства даны для классического определения):
\begin{enumerate}
	\item $\forall A \in S \ 0 \leqslant P(A) \leqslant 1$
	\begin{proof}
	$0 \leqslant n_A \leqslant n \Rightarrow
	\frac0n \leqslant \frac{n_A}n \leqslant \frac{n}n \Rightarrow
	0 \leqslant P(A) \leqslant 1$.
	\end{proof}
	
	\item $P(\Omega) = 1$
	\begin{proof}
	$n_A = n \Rightarrow P(\Omega) = \frac{n}n = 1$.
	\end{proof}
	
	\item $P(\varnothing) = 0$
	\begin{proof}
	$n_A = 0 \Rightarrow P(\Omega) = \frac0n = 0$.
	\end{proof}
	
	\item $P(\overline A) = 1 - P(A)$
	\begin{proof}
	$P(\overline A) = \frac{n - n_A}n = 1 - \frac{n_A}n = 1 - P(A)$.
	\end{proof}
\end{enumerate}

\index{Теорема!сложения вероятностей}
\begin{theorem}[сложения вероятностей]
$P(A + B) = P(A) + P(B) - P(AB)$, где $A$ и $B$~--- события.
\end{theorem}
\begin{proof}[для классического определения вероятности]
Заметим, что $n_{A + B} = n_A + n_B - n_{AB}$, тогда
\begin{equation*}
\frac{n_{A + B}}n = \frac{n_A}n + \frac{n_B}n - \frac{n_{AB}}n \Rightarrow
P(A + B) = P(A) + P(B) - P(AB)
\end{equation*}
\end{proof}

\begin{consequent}
$P(A + B) = P(A) + P(B)$, где $A$ и $B$~--- несовместные события.
\end{consequent}

\begin{theorem}[обобщённая т. сложения вероятностей]
$P(A_1 + A_2 + \ldots + A_k) = 1 - P(A_1 \cdot A_2 \cdot \ldots \cdot A_k)$, где $A_1, A_2, \ldots, A_k$~--- события.
\end{theorem}