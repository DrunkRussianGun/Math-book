\section{Вероятность}
\index{Вероятность} Существует несколько определений вероятности:
\begin{enumerate}
	\item \textbf{Классическое}
	
	Рассмотрим испытание с конечным числом~$n$ равновероятных элементарных исходов.
	\textbf{Вероятностью события~$A$} называется число~$\frac{n_A}n$, где $n_A$~--- число элементарных исходов, при которых происходит событие~$A$, и обозначается~$P(A)$.
	
	\item \textbf{Геометрическое}
	
	Рассмотрим испытание с бесконечным числом равновероятных элементарных исходов, пространство которых имеет меру~$\mu(\Omega)$.
	Пусть событие~$A$ происходит, когда случайная точка попадает в подпространство~$B$.
	\textbf{Вероятностью события~$A$} называется число~$\frac{\mu(B)}{\mu(\Omega)}$ и обозначается~$P(A)$.
	
	\item \textbf{Статистическое}
	
	Рассмотрим серию из $m$ независимых подобных испытаний.
	\textbf{Вероятностью события~$A$} называется число~$\lim\limits_{m \to \infty} \frac{m_A}m$, где $m_A$~--- число испытаний, в результате которых произошло событие~$A$, и обозначается~$P(A)$.
	
	Число~$\frac{m_A}m$ называется \textbf{частотой события~$A$}.
	
	\item \textbf{Аксиоматическое}
	
	Пусть $\Omega$~--- пространство элементарных исходов испытания, $S$~--- соответствующая алгебра событий.
	\textbf{Вероятностью} называется функция~$P(A)$, заданная на~$S$ и удовлетворяющая требованиям:
	\begin{itemize}
		\item $\forall A \in S \ P(A) \geqslant 0$;
		\item $P(A + B) = P(A) + P(B)$, где $A$ и $B$~--- несовместные события;
		\item $P(\Omega) = 1$.
	\end{itemize}
	
	\index{Модель!вероятностная} Тройка $(\Omega, S, P)$ называется \textbf{вероятностной моделью испытания}.
\end{enumerate}

Свойства вероятности (доказательства даны для классического определения):
\begin{enumerate}
	\item $\forall A \in S \ 0 \leqslant P(A) \leqslant 1$
	\begin{proof}
	$0 \leqslant n_A \leqslant n \Rightarrow
	\frac0n \leqslant \frac{n_A}n \leqslant \frac{n}n \Rightarrow
	0 \leqslant P(A) \leqslant 1$.
	\end{proof}
	
	\item $P(\Omega) = 1$
	\begin{proof}
	$n_A = n \Rightarrow P(\Omega) = \frac{n}n = 1$.
	\end{proof}
	
	\item $P(\varnothing) = 0$
	\begin{proof}
	$n_A = 0 \Rightarrow P(\Omega) = \frac0n = 0$.
	\end{proof}
	
	\item $P(\overline A) = 1 - P(A)$
	\begin{proof}
	$P(\overline A) = \frac{n - n_A}n = 1 - \frac{n_A}n = 1 - P(A)$.
	\end{proof}
\end{enumerate}

\index{Теорема!сложения вероятностей}
\begin{theorem}[сложения вероятностей]
Пусть $A$ и $B$~--- события, тогда
\begin{equation*}
P(A + B) = P(A) + P(B) - P(AB)
\end{equation*}
\end{theorem}
\begin{proof}[для классического определения вероятности]
Заметим, что $n_{A + B} = n_A + n_B - n_{AB}$, тогда
\begin{equation*}
\frac{n_{A + B}}n = \frac{n_A}n + \frac{n_B}n - \frac{n_{AB}}n \Rightarrow
P(A + B) = P(A) + P(B) - P(AB)
\end{equation*}
\end{proof}

\begin{consequent}
$P(A + B) = P(A) + P(B)$, где $A$ и $B$~--- несовместные события.
\end{consequent}

\begin{theorem}[обобщённая т. сложения вероятностей]
Пусть $A_1, A_2, \ldots, A_k$~--- события, тогда
\begin{equation*}
P(A_1 + A_2 + \ldots + A_k) = 1 - P(A_1 \cdot A_2 \cdot \ldots \cdot A_k)
\end{equation*}
\end{theorem}

\subsection{Условная вероятность}
\index{Вероятность!условная} \textbf{Условной вероятностью~$P(A \vert B)$} называется вероятность события~$A$ при условии, что событие~$B$ уже произошло, при этом $P(A \vert B) = \frac{n_{AB}}{n_B}$.

Очевидно, что $P(A \vert B) = P(A)$ для независимых событий $A$ и $B$.

\index{Теорема!умножения вероятностей}
\begin{theorem}[умножения вероятностей]
Пусть $A$ и $B$~--- события, тогда
\begin{equation*}
P(AB) = P(A \vert B) \cdot P(B) = P(B \vert A) \cdot P(A)
\end{equation*}
\end{theorem}%
Доказательство для классического определения вероятности очевидно.

\begin{consequent}
$P(AB) = P(A) \cdot P(B)$, где $A$ и $B$~--- независимые события.
\end{consequent}

\begin{theorem}[обобщённая т. умножения вероятностей]
Пусть $A_1, A_2, \ldots, A_k$~--- события, тогда
\begin{equation*}
P(A_1 \cdot A_2 \cdot \ldots \cdot A_k) = P(A_1) \cdot P(A_2 \vert A_1) \cdot P(A_3 \vert A_1 \cdot A_2) \cdot \ldots \cdot P(A_k \vert A_1 \cdot A_2 \cdot \ldots \cdot A_{k-1})
\end{equation*}
\end{theorem}

\subsection{Полная группа гипотез}
\index{Гипотеза} \textbf{Гипотезой} называется предположение о результате испытания.

Гипотезы $H_1, \ldots, H_n$ образуют \textbf{полную группу гипотез}, если они охватывают все возможные результаты испытания и в результате испытания подтверждается только одна из гипотез.
Очевидно, что $\sum\limits_{i=1}^n P(H_i) = 1$.

\index{Формула!полной вероятности}
\begin{theorem}[формула полной вероятности]
Пусть $A$~--- событие, $H_1, \ldots, H_n$~--- полная группа гипотез, тогда
\begin{equation*}
P(A) = \sum_{i=1}^n P(A \vert H_i) \cdot P(H_i)
\end{equation*}
\end{theorem}
\begin{proof}
Событие $A$ может произойти только вместе с одной из гипотез, т.\,к. они образуют полную группу, поэтому
\begin{equation*}
A = \sum_{i=1}^n A H_i \Rightarrow
P(A) =
P\left(\sum_{i=1}^n A H_i\right) \;
\left| A H_1, \ldots, A H_n \text{ несовместимы} \right| =
\sum_{i=1}^n P(A H_i) =
\sum_{i=1}^n (P(A \vert H_i) \cdot P(H_i))
\end{equation*}
\end{proof}

\index{Формула!Байеса}
\begin{theorem}[формула Байеса]
Пусть $A$ и $B$~--- события, тогда
\begin{equation*}
P(A \vert B) = \frac{P(B \vert A) \cdot P(A)}{P(B)}
\end{equation*}
\end{theorem}
\begin{proof}
$P(AB) = P(A \vert B) \cdot P(B) = P(B \vert A) \cdot P(A) \Rightarrow
P(A \vert B) = \frac{P(B \vert A) \cdot P(A)}{P(B)}$
\end{proof}

\subsection{Биномиальная вероятность}
\index{Схема!Бернулли} \textbf{Схемой Бернулли для $n$ испытаний} называется серия из $n$ независимых подобных испытаний, в каждом из которых с постоянной вероятностью $p$ может произойти событие~$A = \{ \text{<<успех>>} \}$.

\index{Вероятность!биномиальная} \textbf{Биномиальной} называется вероятность $P(A_{n,k}) = P_n(k)$, где $A_{n,k} = \{ \text{при } n \text{ испытаниях \allowbreak событие } A \allowbreak \text{ произойдёт } \allowbreak k \text{ раз} \}$.

Пусть $q = 1 - p$.
Свойства биномиальной вероятности:
\begin{enumerate}
	\item $P_n(k) = C_n^k \cdot p^k \cdot q^{n-k}$
	\begin{proof}
	$A_{n,k} = \sum \underbrace{A \cdot \ldots \cdot A}_k \cdot \underbrace{\overline A \cdot \ldots \cdot \overline A}_{n-k}$, где сумма берётся по всем перестановкам в произведении.
	Тогда
	\begin{equation*}
	P_n(k) =
	P\left(\sum \underbrace{A \cdot \ldots \cdot A}_k \cdot \underbrace{\overline A \cdot \ldots \cdot \overline A}_{n-k}\right) =
	\sum P(\underbrace{A \cdot \ldots \cdot A}_k \cdot \underbrace{\overline A \cdot \ldots \cdot \overline A}_{n-k}) =
	\sum (P^k(A) \cdot P^{n-k}(\overline A)) =
	C_n^k \cdot p^k \cdot q^{n-k} 
	\end{equation*}
	\end{proof}
	
	\item $\sum\limits_{k=0}^n P_n(k) = 1$
	\begin{proof}
	$\sum\limits_{k=0}^n P_n(k) =
	\sum\limits_{k=0}^n C_n^k \cdot p^k \cdot q^{n-k} =
	(p + q)^n =
	1^n =
	1$.
	\end{proof}
	
	\item Пусть $B = \{ \text{при } n \text{ испытаниях \allowbreak событие } A \allowbreak \text{ произойдёт \allowbreak от } k_1 \allowbreak \text{ до } k_2 \text{ раз} \}$, тогда $P(B) = \sum\limits_{k=k_1}^{k_2} P_n(k)$.
	\begin{proof}
	$B = \sum\limits_{k=k_1}^{k_2} A_{n,k} \Rightarrow
	P(B) = \sum\limits_{k=k_1}^{k_2} P(A_{n,k}) = \sum\limits_{k=k_1}^{k_2} P_n(k)$.
	\end{proof}
	
	\item Пусть $B = \{ \text{при } n \text{ испытаниях \allowbreak событие } A \allowbreak \text{ произойдёт хотя бы один раз} \}$, тогда $P(B) = 1 - q^n$.
	\begin{proof}
	$P(B) = 
	\sum\limits_{k=1}^n P_n(k) =
	\sum\limits_{k=0}^n P_n(k) - P_n(0) =
	1 - C_n^0 \cdot p^0 \cdot q^n =
	1 - q^n$.
	\end{proof}
	
	\item Среднее число <<успехов>> при $n$ испытаниях равно $np$.
	
	\item Наивероятнейшее число~$k_0$ <<успехов>> (имеющее наибольшую вероятность~$P_n(k_0)$) при $n$ испытаниях удовлетворяет неравенству $np - q \opbr\leqslant k_0 \opbr\leqslant np + p$.
\end{enumerate}