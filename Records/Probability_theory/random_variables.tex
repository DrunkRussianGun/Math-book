\section{Случайные величины}
\index{Величина!случайная} \textbf{Случайной величиной} называется числовая переменная, которая принимает свои значения случайным образом.

Случайная величина называется \textbf{дискретной}, если она принимает конечное или счётное количество значений.

Случайная величина называется \textbf{непрерывной}, если она принимает несчётное количество значений.

\textbf{Законом распределения случайной величины~$X$} называется зависимость между её возможными значениями и соответствующими им вероятностями, обозначаемыми $P(X = x_i)$, где $x_i$~--- значение случайной величины.
\begin{enumerate}
	\item Случайная величина имеет \textbf{равномерное} распределение, если она принимает свои значения с одинаковой вероятностью.
	
	\item Рассмотрим схему Бернулли для $n$ испытаний с вероятностью <<успеха>> $p$.
	Случайная величина $X = \{ \text{число } \allowbreak \text{успехов} \}$, принимающая значения $0, 1, \ldots, n$, имеет \textbf{биномиальное} распределение $P(X = k) = C_n^k \opbr\cdot p^k \opbr\cdot (1 \opbr- p)^{n-k}$.
	
	\item Рассмотрим серию независимых подобных испытаний, в каждом из которых с постоянной вероятностью $p$ может произойти событие~$A = \{ \text{<<успех>>} \}$.
	Случайная величина $X = \{ \text{число проведённых испытаний до } \allowbreak \text{наступления события } A \}$, принимающая значения $1, 2, \ldots$, имеет \textbf{геометрическое} распределение $P(X = k) \opbr= p (1 - p)^{k-1}$.
	
	\item Пусть среди $n$ элементов есть $s$ элементов I типа.
	Случайная величина $X = \{ \text{число элементов I типа среди } m \allowbreak \text{ случайно выбранных} \}$, принимающая значения $0, 1, \ldots, \min \{ s, m \}$, имеет \textbf{гипергеометрическое} распределение $P(X = t) \opbr= \dfrac{C_s^t \cdot C_{n-s}^{m-t}}{C_n^m}$.
\end{enumerate}

\subsection{Функция распределения}
\index{Функция!распределения} \textbf{Функцией распределения случайной величины~$X$} называется функция $F(x) = P(X < x)$.

\subsection{Функция плотности вероятности}
\index{Функция!плотности вероятности} \textbf{Функцией плотности вероятности} непрерывной \textbf{случайной величины~$X$}, имеющей дифференцируемую функцию распределения~$F(x)$, называется функция $f(x) = F'(x)$.