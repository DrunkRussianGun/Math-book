\section{События}
\index{Событие} \textbf{Событием} (\textbf{случайным событием}) называется подмножество пространства элементарных исходов.
Говорят, что \textbf{событие~$A$ произошло в результате испытания}, если результатом испытания стал один из элементарных исходов события~$A$.

Событие называется \textbf{элементарным}, если оно происходит только тогда, когда результатом испытания становится определённый элементарный исход.

Событие называется \textbf{достоверным}, если оно всегда происходит в результате испытания.
Ему соответствует~$\Omega$.

Событие называется \textbf{невозможным}, если оно никогда не происходит в результате испытания.
Ему соответствует~$\varnothing$.

События называются \textbf{несовместными}, если они не могут произойти одновременно в результате конкретного испытания.

Два события называются \textbf{независимыми}, если наступление одного из них не влияет на наступление другого.

\textbf{Суммой событий~$A$ и $B$} называется событие, которое происходит, когда происходит $A$ или $B$, и обозначается~$A + B$.

\textbf{Произведением событий~$A$ и $B$} называется событие, которое происходит, когда одновременно происходят $A$ и $B$, и обозначается~$A \cdot B$, или $AB$.

Событие называется \textbf{противоположным событию~$A$}, если оно происходит тогда, когда не происходит событие~$A$, и обозначается~$\overline A$.

\subsection{Алгебра событий}
\index{Алгебра!событий} \textbf{Алгеброй~$S$ событий} называется множество всех событий данного испытания.
Очевидно, что $S = 2^\Omega$, где $\Omega$~--- пространство элементарных исходов испытания.

Легко доказать, что:
\begin{itemize}
	\item алгебра событий замкнута относительно операций сложения, умножения и отрицания;
	\item невозможное событие~--- $0$ относительно сложения;
	\item достоверное событие~--- $1$ относительно умножения.
\end{itemize}