\section{Элементы комбинаторики}
Пусть элемент~$A$ можно выбрать $m$~способами, а элемент~$B$~--- $n$~способами. Существуют основные правила комбинаторики:
\begin{enumerate}
	\item \textbf{Правило суммы:} выбор либо $A$, либо $B$ можно сделать $m + n$~способами.
	\item \textbf{Правило произведения:} выбор $A$ и $B$ можно сделать $m \cdot n$~способами.
\end{enumerate}

\index{!} \index{Факториал} \textbf{Факториалом числа~$n \in \mathbb N$} называется произведение $1 \cdot 2 \cdot \ldots \cdot n$ и обозначается~$n!$.
Также принято считать, что $0! = 1$.

\textbf{Двойным факториалом числа~$n \in \mathbb N$} называется произведение всех натуральных чисел той~же чётности, что и~$n$, и обозначается~$n!!$.
Т.\,о., $(2k)!! = 2 \cdot 4 \cdot \ldots \cdot (2k)$, $(2k - 1)!! = 1 \cdot 3 \cdot \ldots \cdot (2k - 1)$, где $k \in \mathbb N$.
Также принято считать, что $0!! = 1$.

\index{Перестановка} \textbf{Перестановкой} $n$"~элементного \textbf{множества~$A$} называется биекция $f \colon \{ 1, 2, \ldots, n \} \to A$ и записывается в~виде упорядоченного набора $(a_1, a_2, \ldots, a_n)$, где $a_i = f(i), \ i = 1, 2, \ldots, n$.

\index{Размещение} \textbf{Размещением из $n$~элементов по $k$} (\textbf{$k$"~элементным размещением}) $n$"~элементного \textbf{множества~$A$} называется инъекция $f \colon \{ 1, 2, \ldots, n \} \to A$.
Количество различных размещений из $n$~элементов по $k$ обозначается $A_n^k$.

\begin{statement}
$A_n^k = \frac{n!}{(n - k)!}$
\end{statement}
\begin{proofmathind}
	\indbase При $k = 0$ $A_n^0 = 1 = \frac{n!}{n!}$.
	\indstep Пусть теорема верна при $k - 1$.
	Докажем её для $k$, рассматривая размещение из $n$ элементов по $k$.
	Первым элементом размещения может быть любой из $n$ элементов, тогда по предположению индукции на оставшиеся $k - 1$ места можно $A_{n-1}^{k-1}$ способами распределить $n - 1$ элементов.
	Получим $A_n^k \opbr= n \cdot A_{n-1}^{k-1} \opbr= n \cdot \frac{(n - 1)!}{(n - k)!} \opbr= \frac{n!}{(n - k)!}$. \indend
\end{proofmathind}

\begin{consequent}
Количество различных перестановок $n$"~элементного множества равно $n!$.
\end{consequent}
\begin{proof}
Заметим, что искомое количество равно $A_n^n = \frac{n!}{0!} = n!$.
\end{proof}

\index{Сочетание} \textbf{Сочетанием из $n$~элементов по $k$} (\textbf{$k$"~элементным сочетанием}) $n$"~элементного \textbf{множества~$A$} называется множество $X \subseteq A \colon |X| = k$.
Количество различных сочетаний из $n$~элементов по $k$ обозначается $C_n^k$, или $\binom{n}{k}$.

\begin{statement}
$C_n^k = \frac{n!}{k! (n - k)!}$
\end{statement}
\begin{proof}
Рассмотрим сочетание из $n$ элементов по $k$.
Переставляя в нём элементы, получим $k!$ различных размещений из $n$ по $k$.
Проведя аналогичные рассуждения для каждого сочетания, получим все возможные размещения из $n$ по $k$.
Тогда $C_n^k \cdot k! = A_n^k \Leftrightarrow C_n^k = \frac{A_n^k}{k!} = \frac{n!}{k! (n - k)!}$.
\end{proof}

\index{Перестановка!с повторениями} Пусть множество $A = \{ a_1, a_2, \ldots, a_n \}$, $c_1, c_2, \ldots, c_n$~--- некоторые целые неотрицательные числа, $k = \sum\limits_{i=1}^n c_i$.
\textbf{Перестановками с повторениями} называются размещения с повторениями из $n$ по $k$ множества~$A$, в которых элемент~$a_i$ повторяется ровно $c_i$~раз.
Количество таких различных перестановок с повторениями обозначается $P_k(c_1, c_2, \ldots, c_n)$.

\begin{statement}
$P_k(c_1, c_2, \ldots, c_n) = \dfrac{k!}{c_1! \cdot c_2! \cdot \ldots \cdot c_n!}$.
\end{statement}
\begin{proof}
Если считать все $k$ элементов перестановки с повторениями различными, то всего имеем $k!$ различных вариантов перестановок.
Однако среди этих перестановок не все различны.
В самом деле, элементы $a_1$ можно переставлять местами друг с другом $c_1!$ способами, и от этого перестановка не изменится.
Аналогично можно переставлять элементы $a_2, \ldots, a_n$.

Т.\,о., любая перестановка с повторениями может быть записана $c_1! \cdot c_2! \cdot \ldots \cdot c_n!$ способами, которые будут считаться разными, если считать все $k$ элементов разными.
Получим:
\begin{equation*}
P_k(c_1, c_2, \ldots, c_n) \cdot c_1! \cdot c_2! \cdot \ldots \cdot c_n! = k! \Leftrightarrow
P_k(c_1, c_2, \ldots, c_n) = \frac{k!}{c_1! \cdot c_2! \cdot \ldots \cdot c_n!}
\end{equation*}
\end{proof}

\index{Размещение!с повторениями} \textbf{Размещением с повторениями из $n$~элементов по $k$} (\textbf{$k$"~элементным размещением с повторениями}) $n$"~элементного \textbf{множества~$A$} называется упорядоченный набор~$(a_1, a_2, \ldots, a_k)$, элементы которого принадлежат множеству~$A$ и могут повторяться.
Количество различных размещений с повторениями из $n$~элементов по $k$ обозначается $\overline A_n^k$.

\begin{statement}
$\overline A_n^k = n^k$.
\end{statement}
\begin{proof}
По правилу произведения $\overline A_n^k = \underbrace{n \cdot n \cdot \ldots \cdot n}_k = n^k$.
\end{proof}

\index{Сочетание!с повторениями} \textbf{Сочетанием с повторениями из $n$~элементов по $k$} (\textbf{$k$"~элементным сочетанием с повторениями}) $n$"~элементного \textbf{множества~$A$} называется неупорядоченный набор~$(a_1, a_2, \ldots, a_k)$, элементы которого принадлежат множеству~$A$ и могут повторяться.
Количество различных сочетаний с повторениями из $n$~элементов по $k$ обозначается $\overline C_n^k$.

\begin{statement}
$\overline C_n^k = C_{n+k-1}^k = C_{n+k-1}^{n-1}$.
\end{statement}
\begin{proof}
Рассмотрим сочетание с повторениями из $n$~элементов по $k$ множества $A = \{ a_1, a_2, \ldots, a_n \}$.
Пусть в данном сочетании элемент $a_i$ встречается $c_i$~раз.
Очевидно, что $\sum\limits_{i=1}^n c_i = k$.

Поставим такому сочетанию во взаимо"=однозначное соответствие набор $(\underbrace{1, 1, \ldots, 1}_{c_1}, 0, \underbrace{1, \ldots, 1}_{c_2}, 0, \ldots, 0, \underbrace{1, \ldots, 1}_{c_n})$ из $n - 1$ нулей и $k$ единиц.
Тогда, вычислив число таких различных наборов, мы получим число различных сочетаний с повторениями:
\begin{equation*}
\overline C_n^k = P_{n + k - 1}(k, n - 1) = \frac{(n + k - 1)!}{k!(n - 1)!} = C_{n+k-1}^k = C_{n+k-1}^{n-1}
\end{equation*}
\end{proof}

\index{Инверсия} \textbf{Инверсией в перестановке~$\pi$} называется пара индексов $i, j \colon i < j \lAnd \pi(i) > \pi(j)$.
\textbf{Чётность перестановки} определяется чётностью числа инверсией в ней.

\begin{statement}
\label{st:parity_of_permutation}
Если в перестановке~$(a_1, a_2, \ldots, a_n)$ поменять местами два элемента, то её чётность изменится.
\end{statement}
\begin{proof}
\begin{enumerate}
	\item Пусть переставлены соседние элементы.
	Если они образовывали инверсию, то после обмена местами не образуют, и наоборот.
	При этом наличие инверсий с остальными элементами остаётся неизменным.
	Значит, количество инверсий в перестановке изменилось на~$1$, т.\,е. чётность числа инверсий изменилась, тогда изменилась и чётность перестановки.
	
	\item Поменяем местами элементы $a_i$ и $a_{i+d}$, где $d > 0$.
	Для этого последовательно поменяем местами элементы, имеющие индексы $i+d$ и $i+d-1$, $i+d-1$ и $i+d-2$, \ldots, $i+2$ и $i+1$, $i+1$ и $i$, $i+1$ и $i+2$, $i+2$ и $i+3$, \ldots, $i+d-2$ и $i+d-1$, $i+d-1$ и $i+d$.
	Всего совершили $2d - 1$~обменов соседних элементов местами, тогда перестановка изменила чётность, т.\,к. $2d - 1 \notmult 2$.
\end{enumerate}
\end{proof}