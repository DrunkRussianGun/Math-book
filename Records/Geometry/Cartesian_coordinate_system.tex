\section{Прямоугольная система координат}
\subsection{Преобразования плоскости и пространства}
\index{Преобразование} \textbf{Преобразованием пространства~$\mathbb R^n$} называется функция $f \colon \mathbb R^n \to \mathbb R^n$.

Преобразование~$f \colon \mathbb R^n \to \mathbb R^n$ называется \textbf{линейным}, если $\forall \overline a, \overline b \in \mathbb R^n,\, \alpha \in \mathbb R \ f(\overline a) + f(\overline b) = f(\overline a + \overline b) \lAnd \alpha f(\overline a) \opbr= f(\alpha \overline a)$.

Далее подразумевается, что $f(x, y, z) = (x', y', z')$.

\subsubsection{Параллельный перенос}
\index{Параллельный перенос} \textbf{Параллельным переносом на вектор~$(a, b, c)$} называется преобразование, при котором каждая точка пространства перемещается на вектор~$(a, b, c)$.
\begin{equation*}
\begin{cases}
x' = x + a \\
y' = y + b \\
z' = z + c
\end{cases} \Leftrightarrow
\begin{cases}
x = x' - a \\
y = y' - b \\
z = z' - c
\end{cases}
\end{equation*}

Аналогично определяется параллельный перенос на плоскости.

\subsubsection{Поворот}
\begin{wrapfigure}{r}{0pt}\noindent
\shorthandoff{"}
\begin{tikzpicture}[scale=0.7]
\drawaxis{-3}{6}[]{-1}{6}[];
\def\rightanglesize{7.142857pt}

% рисуем точку M
\def\angleM{53.130102}
\def\radM{5}
\draw (0, 0) coordinate (O) node [below left] {$O$}
	(\angleM:\radM) coordinate (M) node[above right] {$M(x, y)$}
	(M |- O) coordinate (K);
\printcoordsonaxis[solid]{M}{\scriptstyle K(x, 0)}{y};

% проставляем координаты на новой системе
\def\angle{30}
\begin{scope}[rotate=\angle]
\drawaxis[]{0}{6}[]{0}{6}[];
\draw (\angleM - \angle:\radM) coordinate (M1) -- node[right] {$y'$} (M1 |- O) node[below right] {$x'$}
	(M1) -- (M1 -| O) node[below left] {$y'$};
\draw (M1 |- O) rectangle +(\rightanglesize, \rightanglesize);
\end{scope}

% рисуем точку L
\draw let
	\n1 = {\radM * cos(\angleM - \angle)}
in
	[dashed] (\angle:\n1) coordinate (L) -- (L |- O) node[below] {$L$}
	(L) -- (L -| M) node[left] {$S$};

% рисуем углы
\draw (1, 0) coordinate (X)
	pic[draw, "$\alpha$", angle eccentricity=1.5, angle radius=5mm] {angle = X--O--L}
	pic[draw, "$\alpha$", angle eccentricity=1.5, angle radius=5mm] {angle = K--M--L}
	(L |- O) rectangle +(\rightanglesize, \rightanglesize)
	(L -| M) rectangle +(\rightanglesize, \rightanglesize);
\end{tikzpicture}
\shorthandon{"}
\end{wrapfigure}
\index{Поворот} \textbf{Поворотом на угол~$\alpha$} называется преобразование, при котором каждый луч, исходящий из начала координат, поворачивается на угол~$\alpha$.
\begin{equation*}
\begin{cases}
x' = x \cos \alpha + y \sin \alpha \\
y' = -x \sin \alpha + y \cos \alpha
\end{cases} \Leftrightarrow
\begin{cases}
x = x' \cos \alpha - y' \sin \alpha \\
y = x' \sin \alpha + y' \cos \alpha
\end{cases}
\end{equation*}
\begin{proof}
\begin{equation*}
x = OL - KL = x' \cos \alpha - y' \sin \alpha \lAnd
y = KS + MS = x' \sin \alpha + y' \cos \alpha
\end{equation*}

Решим получившуюся систему относительно $x'$ и $y'$ \hyperref[th:Cramer]{методом Крамера}:
\begin{equation*}
\Delta = \cos^2 \alpha + \sin^2 \alpha = 1 \lAnd
\Delta_1 = x \cos \alpha + y \sin \alpha \lAnd
\Delta_2 = -x \sin \alpha + y \cos \alpha \Rightarrow
\end{equation*}
\begin{equation*}
\Rightarrow x' = x \cos \alpha + y \sin \alpha \lAnd
y' = -x \sin \alpha + y \cos \alpha
\end{equation*}
\end{proof}