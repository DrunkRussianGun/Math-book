\section{Знакопеременные ряды}
\index{Ряд!знакопеременный} Ряд называется \textbf{знакопеременным}, если он содержит как положительные, так и отрицательные члены.

\index{Ряд!знакочередующийся} Ряд называется \textbf{знакочередующимся}, если его любые два последовательных члена имеют разные знаки.

\begin{theorem}
Если $\series |a_k|$ сходится, то $\series a_k$ тоже сходится.
\end{theorem}
\begin{proof}
Пусть $\sigma_n$ и $S_n$~--- частичные суммы рядов $\series a_k$ и $\series |a_k|$ соответственно.
По \hyperref[th:Cauchy_criterion]{критерию Коши}
\begin{equation*}
\forall \varepsilon > 0 \
\exists N \in \mathbb N \colon
\forall m > N,\, k \geqslant 1 \
|S_{m+k} - S_m| < \varepsilon
\end{equation*}

Тогда
\begin{equation*}
|\sigma_{m+k} - \sigma_m| =
\left| \sum_{i=1}^k a_{m+i} \right| \leqslant
\sum_{i=1}^k |a_{m+i}| =
|S_{m+k} - S_m| < \varepsilon
\end{equation*}

Значит, $\series a_k$ сходится.
\end{proof}

Знакопеременный ряд~$\series a_k$ называется \textbf{абсолютно сходящимся}, если $\series |a_k|$ сходится.

Знакопеременный ряд~$\series a_k$ называется \textbf{условно сходящимся}, если $\series a_k$ сходится, но $\series |a_k|$ расходится.

\index{Признак!Лейбница}
\begin{theorem}[признак Лейбница]
Пусть $\series (-1)^{k-1} a_k$~--- знакочередующийся ряд, где $a_k > 0$.
Если $\lim\limits_{n \to \infty} a_n = 0$, причём $(a_n)$~--- монотонно убывающая последовательность, то ряд сходится.
\end{theorem}
\begin{proof}
\begin{equation*}
S_{2n} = (a_1 - a_2) + (a_3 - a_4) + \ldots + (a_{2n-1} - a_{2n}) < S_{2n+2}
\end{equation*}
\begin{equation*}
S_{2n} = a_1 - (a_2 - a_3) - (a_4 - a_5) - \ldots - (a_{2n-2} - a_{2n-1}) - a_{2n} < a_1
\end{equation*}

Тогда по свойству~\ref{st:monotonic_bounded_sequence} предела последовательности
\begin{equation*}
\lim_{n \to \infty} S_{2n} = S \Rightarrow
\lim_{n \to \infty} S_{2n+1} =
\lim_{n \to \infty} S_{2n} + \lim_{n \to \infty} a_{2n+1} =
S \Rightarrow
\lim_{n \to \infty} S_n = S
\end{equation*}

Значит, ряд сходится.
\end{proof}

\begin{consequent}
Ряд~$\series \frac{(-1)^k}{k^\alpha}$ расходится при $\alpha \leqslant 0$, условно сходится при $0 < \alpha \leqslant 1$ и абсолютно сходится при $\alpha > 1$.
\end{consequent}
\begin{proof}
Это следует из следствия~\ref{conseq:series_1_div_k}.
\end{proof}