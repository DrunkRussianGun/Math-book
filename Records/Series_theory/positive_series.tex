\section{Знакоположительные ряды}
\index{Ряд!знакоположительный} Ряд называется \textbf{знакоположительным}, если все его члены положительны.

\begin{lemma}
\label{lemma:convergence_of_sign-positive_series}
Знакоположительный ряд сходится $\Leftrightarrow$ последовательность его частных сумм ограничена.
\end{lemma}
\begin{proof}
\begin{enumerate}
	\item $\Rightarrow$. Пусть $\exists \lim\limits_{n \to \infty} S_n = S$, тогда $\forall n \in N \ S_n \leqslant S$.
	\item $\Leftarrow$. Пусть $\exists T > 0 \colon \forall n \in \mathbb N \ S_n \leqslant T$, тогда по свойству~\ref{st:monotonic_bounded_sequence} предела последовательности ряд сходится.
\end{enumerate}
\end{proof}

\subsection{Признаки сходимости}
\begin{theorem}[признак сравнения]
Пусть даны знакоположительные ряды $\series a_k$ и $\series b_k$, $\forall n \in \mathbb N \ a_n \leqslant b_n$.
\begin{enumerate}
	\item Если $\series b_k$ сходится, то $\series a_k$ тоже сходится.
	\item Если $\series a_k$ расходится, то $\series b_k$ тоже расходится.
\end{enumerate}
\end{theorem}
\begin{proof}
Пусть $(\sigma_n)$ и $(S_n)$~--- частичные суммы рядов $\series a_k$ и $\series b_k$ соответственно.
\begin{enumerate}
	\item Пусть $\series b_k$ сходится, тогда $\exists T > 0 \colon \forall n \in N \ S_n \leqslant T \Rightarrow
	\sigma_n \leqslant S_n < T$.
	По лемме~\ref*{lemma:convergence_of_sign-positive_series} $\series a_k$ сходится.
	
	\item Очевидно.
\end{enumerate}
\end{proof}

\index{Признак!предельный п. сравнения}
\begin{theorem}[предельный признак сравнения]
Пусть даны знакоположительные ряды $\series a_k$ и $\series b_k$, $\lim\limits_{n \to \infty} \frac{a_n}{b_n} = c$.
Если $0 < c < \infty$, то $\series a_k$ сходится $\Leftrightarrow$ $\series b_k$ сходится.
\end{theorem}
\begin{proof}
\begin{equation*}
\forall \varepsilon > 0 \ \exists N \in \mathbb N \colon \forall k > N \ \left| \frac{a_k}{b_k} - c \right| < \varepsilon
\end{equation*}

Пусть $\varepsilon = \frac{c}2$, тогда
\begin{equation*}
\exists N_0 \in \mathbb N \colon \forall k > N_0 \ c - \frac{c}2 < \frac{a_k}{b_k} < c + \frac{c}2 \Rightarrow
a_k < \frac{3c}2 b_k \lAnd b_k < \frac2c a_k
\end{equation*}

По признаку сравнения получим:
\begin{itemize}
	\item Если $\series a_k$ сходится, то $\series \frac2c a_k$ сходится $\Rightarrow$ $\series b_k$ сходится.
	\item Если $\series b_k$ сходится, то $\series \frac{3c}2 b_k$ сходится $\Rightarrow$ $\series a_k$ сходится.
\end{itemize}
\end{proof}

\index{Признак!Д'Аламбера}
\begin{theorem}[признак Д'Аламбера]
\label{th:d'Alembert's_ratio_test}
Пусть $\series a_k$~--- знакоположительный ряд, $\lim\limits_{n \to \infty} \frac{a_{n+1}}{a_n} = p$.
Если $p < 1$, то $\series a_k$ сходится, а если $p > 1$, то расходится.
\end{theorem}
\begin{proof}
\begin{enumerate}
	\item Пусть $p < 1$.
	Выберём $\varepsilon > 0 \colon p + \varepsilon < 1$, тогда
	\begin{equation*}
	\exists N \in \mathbb N \colon \forall n > N \ \left| \frac{a_{n+1}}{a_n} - p \right| < \varepsilon \Rightarrow
	\frac{a_{n+1}}{a_n} < p + \varepsilon \Rightarrow
	\end{equation*}
	\begin{equation*}
	\Rightarrow \frac{a_{N+2}}{a_{N+1}} < p + \varepsilon \lAnd \frac{a_{N+3}}{a_{N+2}} < p + \varepsilon \Rightarrow
	a_{N+3} < (p + \varepsilon) a_{N+2} < (p + \varepsilon)^2 a_{N+1} \Rightarrow
	a_{N+k} < (p + \varepsilon)^{k-1} a_{N+1}
	\end{equation*}
	
	$\series (p + \varepsilon)^{k-1} a_{N+1}$ сходится $\Rightarrow$ $\series a_{N+k}$ сходится $\Rightarrow$ $\series a_k$ сходится.
	
	\item Пусть $p > 1$.
	Выберём $\varepsilon > 0 \colon p - \varepsilon > 1$, тогда
	\begin{equation*}
	\exists N \in \mathbb N \colon \forall n > N \ \left| \frac{a_{n+1}}{a_n} - p \right| < \varepsilon \Rightarrow
	\frac{a_{n+1}}{a_n} > p - \varepsilon \Rightarrow
	\end{equation*}
	\begin{equation*}
	\Rightarrow \frac{a_{N+2}}{a_{N+1}} > p - \varepsilon \lAnd \frac{a_{N+3}}{a_{N+2}} > p - \varepsilon \Rightarrow
	a_{N+3} > (p - \varepsilon) a_{N+2} > (p - \varepsilon)^2 a_{N+1} \Rightarrow
	a_{N+k} > (p - \varepsilon)^{k-1} a_{N+1}
	\end{equation*}
	
	$\series (p - \varepsilon)^{k-1} a_{N+1}$ расходится $\Rightarrow$ $\series a_{N+k}$ расходится $\Rightarrow$ $\series a_k$ расходится.
\end{enumerate}
\end{proof}

\index{Признак!радикальный п. Коши}
\begin{theorem}[радикальный признак Коши]
\label{th:Cauchy's_radical_test}
Пусть $\series a_k$~--- знакоположительный ряд и $\exists \lim\limits_{n \to \infty} \sqrt[n]{a_n} = p$.
Если $p < 1$, то $\series a_k$ сходится, а если $p > 1$, то расходится.
\end{theorem}
\begin{proof}
\begin{enumerate}
	\item Пусть $p < 1$.
	Выберём $\varepsilon > 0 \colon p + \varepsilon < 1$, тогда
	\begin{equation*}
	\exists N \in \mathbb N \colon \forall n > N \ \left| \sqrt[n]{a_n} - p \right| < \varepsilon \Rightarrow
	\sqrt[n]{a_n} < p + \varepsilon \Rightarrow
	a_n < (p + \varepsilon)^n \Rightarrow
	\frac{a_{n+1}}{a_n} < p + \varepsilon \Rightarrow
	\lim_{n \to \infty} \frac{a_{n+1}}{a_n} < 1
	\end{equation*}
	
	Тогда по признаку Д'Аламбера $\series a_k$ сходится.
	
	\item Пусть $p > 1$.
	Выберём $\varepsilon > 0 \colon p - \varepsilon > 1$, тогда
	\begin{equation*}
	\exists N \in \mathbb N \colon \forall n > N \ \left| \sqrt[n]{a_n} - p \right| < \varepsilon \Rightarrow
	\sqrt[n]{a_n} > p - \varepsilon \Rightarrow
	a_n > (p - \varepsilon)^n \Rightarrow
	\frac{a_{n+1}}{a_n} > p - \varepsilon \Rightarrow
	\lim_{n \to \infty} \frac{a_{n+1}}{a_n} > 1
	\end{equation*}
	
	Тогда по признаку Д'Аламбера $\series a_k$ расходится.
\end{enumerate}
\end{proof}

\index{Признак!интегральный п. Коши}
\begin{theorem}[интегральный признак Коши]
Пусть $\series a_k$~--- знакоположительный ряд.
Если существует монотонная функция~$f(x) \colon f(n) = a_n \lAnd \lim\limits_{x \to +\infty} f(x) = 0$, то $\series a_k$ сходится $\Leftrightarrow$ $\exists \int\limits_1^{+\infty} f(x)\,dx$.
\end{theorem}
\begin{proof}
Рассмотрим отрезок~$[1; n + 1]$.
Площадь синей фигуры на нём равна $a_1 + a_2 + \ldots + a_n = S_n$, красной~--- $a_2 + \ldots + a_{n+1} = S_{n+1} - a_1$, а криволинейной трапеции~--- $S = \int\limits_1^{n+1} f(x)\,dx$, тогда $S_{n+1} - a_1 < S < S_n$.
\begin{enumerate}
	\item \begin{minipage}[t]{105mm}\noindent
	Пусть $\exists \int\limits_1^{+\infty} f(x)\,dx$, тогда
	\begin{equation*}
	\forall n \in \mathbb N \ S_{n+1} - a_1 < S \Rightarrow
	S_n < S_{n+1} < a_1 + S < a_1 + \int\limits_1^{+\infty} f(x)\,dx
	\end{equation*}
	
	Получили, что $S_n$ ограничена, значит, $\exists \lim\limits_{n \to \infty} S_n$, т.\,к. $S_n$ возрастает.
	Отсюда $\series a_k$ сходится.
	\end{minipage}
	\hfill
	\begin{minipage}[t]{68mm}\noindent
	\begin{flushright}
	\begin{tikzpicture}
	\drawaxis{-0.5}{6}{-0.5}{3};
	\draw[name path=curve] (0.5, 2.5) coordinate (a) .. controls (1.75, 1.2) .. (6, 0.7) coordinate (b);
	
	% Рисуем прямоугольники в начале
	\coordinate (prev) at (a);
	\foreach \i/\x in {1/1, 2/1.5} {
		\getordinate{x\i}{\x}{curve};
		\draw[red] (0, 0 -| x\i) -- (x\i) -- (x\i -| prev);
		\draw[blue] (x\i) -- (prev -| x\i) -- (prev);
		\coordinate (prev) at (x\i);
	}
	\getordinate{x3}{2}{curve};
	\draw[blue] (0, 0 -| a) -- (a)
		(x2) -- (x2 -| x3);
	\draw[red] (x2 |- x3) -- (x3);
	
	% Рисуем чёрточки на Ox
	\foreach \x in {1, 2, 3}
		\draw (\x/2, 0) -- +(0, -0.1) node[below] {$\x$} -- +(0, 0.1);
	
	% Рисуем прямоугольник в середине
	\getordinate{xn1}{3.5}{curve};
	\getordinate{xn2}{4}{curve};
	\draw[red] (xn1) -- (0, 0 -| xn1)
		(0, 0 -| xn2) -- (xn2) -- (xn1 |- xn2);
	\draw[blue] (xn1) -- (xn1 -| xn2) -- (xn2);
	
	% Рисуем чёрточки на Ox
	\draw (xn1 |- 0, 0) -- +(0, -0.1) node[below] {$\scriptstyle n \vphantom1$} -- +(0, 0.1)
		(xn2 |- 0, 0) -- +(0, -0.1) node[below] {$\scriptstyle n + 1$} -- +(0, 0.1);
	
	% Подписываем точки
	\draw (a) node[above] {$\scriptstyle a_1$}
		(xn1) node[above] {$\scriptstyle a_n$};
	\end{tikzpicture}
	\end{flushright}
	\end{minipage}
	
	\item Пусть $\int\limits_1^{+\infty} f(x)\,dx = \infty$, тогда
	\begin{equation*}
	\forall n \in \mathbb N \ S_n > S \;
	\left| \lim\limits_{n \to \infty} S = +\infty \right| \Rightarrow
	\lim\limits_{n \to \infty} S_n = +\infty
	\end{equation*}
	
	Значит, $\series a_k$ расходится.
\end{enumerate}
\end{proof}

\begin{consequent}
\label{conseq:series_1_div_k}
Ряд~$\series \frac1{k^\alpha}$ расходится при~$\alpha \leqslant 1$ и сходится при~$\alpha > 1$.
\end{consequent}
\begin{proof}
Это следует из утверждения~\ref{st:improper_int_1_div_x}.
\end{proof}