\section{Несобственный интеграл}
\index{Интеграл!несобственный}

\subsection{Несобственный интеграл I рода}
Пусть функция~$f(x)$ непрерывна на~$[a; +\infty)$ и $\forall b > a \ \exists \int\limits_a^b f(x)\,dx$.
\textbf{Несобственным интегралом Римана I рода} называется интеграл $\int\limits_a^{+\infty} f(x)\,dx = \lim\limits_{b \to +\infty} \int\limits_a^b f(x)\,dx$.

Пусть функция~$f(x)$ непрерывна на~$(-\infty; b]$ и $\forall a < b \ \exists \int\limits_a^b f(x)\,dx$.
\textbf{Несобственным интегралом Римана I рода} называется интеграл $\int\limits_{-\infty}^b f(x)\,dx = \lim\limits_{a \to -\infty} \int\limits_a^b f(x)\,dx$.

Несобственный интеграл называется \textbf{сходящимся}, если он существует и конечен, иначе~--- \textbf{расходящимся}.

Если функция~$f(x)$ непрерывна на~$\mathbb R$, то
$\int\limits_{-\infty}^{+\infty} f(x)\,dx =
\lim\limits_{a \to -\infty} \int\limits_a^c f(x)\,dx + \lim\limits_{b \to +\infty} \int\limits_c^b f(x)\,dx$

\begin{statement}
\label{st:improper_int_1_div_x}
$\int\limits_1^{+\infty} \frac{dx}{x^\alpha}$ расходится при~$\alpha \leqslant 1$ и сходится при~$\alpha > 1$.
\end{statement}
\begin{proof}
\begin{enumerate}
	\item Пусть $\alpha \neq 1$, тогда
	\begin{equation*}
	\int_1^{+\infty} \frac{dx}{x^\alpha} =
	\left.\left( \frac{x^{1-\alpha}}{1 - \alpha} \right)\right|_1^{+\infty} =
	\lim_{x \to +\infty} \frac{x^{1-\alpha} - 1}{1 - \alpha}
	\end{equation*}
	
	Отсюда следует, что $\int\limits_1^{+\infty} \frac{dx}{x^\alpha}$ расходится при~$\alpha < 1$ и сходится при~$\alpha > 1$.
	
	\item Пусть $\alpha = 1$, тогда
	\begin{equation*}
	\int_1^{+\infty} \frac{dx}{x} =
	\lim_{x \to +\infty} (\ln x - \ln 1) = +\infty
	\end{equation*}
	
	Значит, $\nexists \int\limits_1^{+\infty} \frac{dx}x$.
\end{enumerate}
\end{proof}

\subsection{Несобственный интеграл II рода}
Пусть функция~$f(x)$ непрерывна на~$[a; b)$, терпит разрыв II рода в точке~$b$ и $\forall \varepsilon > 0 \ \exists \int\limits_a^{b-\varepsilon} f(x)\,dx$.
\textbf{Несобственным интегралом Римана II рода} называется интеграл $\int\limits_a^b f(x)\,dx = \lim\limits_{\varepsilon \to +0} \int\limits_a^{b-\varepsilon} f(x)\,dx$.

Пусть функция~$f(x)$ непрерывна на~$(a; b]$, терпит разрыв II рода в точке~$a$ и $\forall \varepsilon > 0 \ \exists \int\limits_{a+\varepsilon}^b f(x)\,dx$.
\textbf{Несобственным интегралом Римана II рода} называется интеграл $\int\limits_a^b f(x)\,dx = \lim\limits_{\varepsilon \to +0} \int\limits_{a+\varepsilon}^b f(x)\,dx$.

Если функция~$f(x)$ терпит разрыв II рода в точке $c \in (a; b)$, то $\int\limits_a^b f(x)\,dx = \int\limits_a^c f(x)\,dx + \int\limits_c^b f(x)\,dx$.