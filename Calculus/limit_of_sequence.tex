\section{Предел последовательности}
\hypertarget{def:limit_of_sequence}{}\hypertarget{def:convergence_of_sequence}{} Число~$a$ называется \textbf{пределом последовательности~$(x_n)$}, если
\begin{equation*}
\forall \varepsilon > 0 \ \exists n_0 \in \mathbb N \colon \forall n > n_0 \ |x_n - a| < \varepsilon
\end{equation*}
и~обозначается $\displaystyle \lim_{n \to \infty} x_n$.
Говорят, что последовательность~$(x_n)$ \textbf{сходится}, если $\displaystyle \exists\lim_{n \to \infty} x_n$, иначе говорят, что $(x_n)$ \textbf{расходится}.

\hypertarget{def:bounded_quantity}{} Последовательность~$(x_n)$ называется \textbf{ограниченной}, или~\textbf{ограниченной величиной}, если $\exists a > 0 \colon \forall n \in \mathbb N \ |x_n| < a$.

\hypertarget{def:infinitesimal}{\textbf{Бесконечно малой величиной}} называется последовательность~$(x_n)$, если
\begin{equation*}
\forall \varepsilon > 0 \ \exists n_0 \in \mathbb N \colon \forall n > n_0 \ |x_n| < \varepsilon
\end{equation*}

Можно определить предел последовательности, используя понятие бесконечно малой величины.

Число~$a$ называется \textbf{пределом последовательности~$(x_n)$}, если $x_n = a + \alpha_n$, где $\alpha_n$~--- бесконечно малая величина.

Докажем эквивалентность этих определений.
\begin{proof}
\begin{enumerate}
	\item Пусть дана последовательность~$(x_n)$ такая, что
	\begin{equation*}
	\forall \varepsilon > 0 \ \exists n_0 \in \mathbb N \colon \forall n > n_0 \ |x_n - a| < \varepsilon
	\end{equation*}
	
	Докажем, что $x_n = a + \alpha_n$.
	В~самом деле, $\alpha_n = x_n - a$~--- бесконечно малая величина.
	Т.\,о., $x_n = a + \alpha_n$.
	
	\item Проведя те~же самые рассуждения в~обратную сторону, докажем обратное утверждение.
\end{enumerate}
\end{proof}