\subsection{Предел функции в точке}
Пусть $a$~--- предельная точка области определения функции~$f(x)$.
Следующие определения эквивалентны:
\begin{enumerate}
	\item\label{def:limit_of_function_by_Heine}\textbf{Определение по Гейне}
	
	Число~$b$ называется \textbf{пределом функции~$f(x)$ в~точке~$a$}, если $\displaystyle \lim_{n \to \infty} f(x_n) = b$ для~любой последовательности~$\displaystyle (x_n): \lim_{n \to \infty} x_n = a$.
	
	\item\label{def:limit_of_function_by_Cauchy}\textbf{Определение по Коши}
	
	Число~$b$ называется \textbf{пределом функции~$f(x)$ в~точке~$a$}, если
	\begin{equation*}
	\forall \varepsilon > 0 \ \exists \delta > 0 \colon (\forall x \colon |x - a| < \delta) \ |f(x) - b| < \varepsilon
	\end{equation*}
\end{enumerate}

Предел функции~$f(x)$ в~точке~$a$ обозначается $\displaystyle \lim_{x \to a} f(x)$.
\begin{proof}
\begin{enumerate}
	\item (\ref{def:limit_of_function_by_Cauchy}) $\Rightarrow$ (\ref{def:limit_of_function_by_Heine}).
	Пусть 
	\begin{equation*}
	\lim_{n \to \infty} x_n = a \Rightarrow
	\forall \delta > 0 \ \exists n_0 \in \mathbb N \colon \forall n > n_0 \ |x_n - a| < \delta \Rightarrow
	\end{equation*}
	\begin{equation*}
	\Rightarrow \forall \varepsilon > 0 \ \exists k > 0 \colon \forall x_n > k \ |f(x_n) - b| < \varepsilon \Rightarrow
	\lim_{n \to \infty} f(x_n) = b
	\end{equation*}
	
	\item (\ref{def:limit_of_function_by_Heine}) $\Rightarrow$ (\ref{def:limit_of_function_by_Cauchy}).
	Докажем методом от~противного, что условия определения~(\ref{def:limit_of_function_by_Cauchy}) выполняются.
	Пусть 
	\begin{equation*}
	\exists \varepsilon_0 > 0 \colon (\forall \delta > 0 \ \exists x_0 \colon |x_0 - a| < \delta, \ |f(x_0) - b| \geqslant \varepsilon_0 \Rightarrow
	\forall n \in \mathbb N \ \exists x_n \colon |x_n - a| < \frac1n, \ |f(x_n) - b| \geqslant \varepsilon_0)
	\end{equation*}
	
	Получили последовательность~$\displaystyle (x_n) \colon \lim_{n \to \infty} x_n = a, \ \lim_{n \to \infty} f(x_n) \neq b$.
	Противоречие.
\end{enumerate}
\end{proof}

Из~определения по~Гейне следует, что пределы функции в~точке обладают теми~же свойствами, что и~пределы последовательности.

Также можно определить односторонние пределы.

Число~$b$ называется \textbf{левым пределом}, или \textbf{пределом слева}, функции~$f(x)$ \textbf{в~точке}~$a$, если
\begin{equation*}
\forall \varepsilon > 0 \ \exists \delta > 0 \colon (\forall x \colon 0 < a - x < \delta) \ |f(x) - b| < \varepsilon
\end{equation*}
и обозначается $\displaystyle \lim_{x \to a-0} f(x)$.

Число~$b$ называется \textbf{правым пределом}, или \textbf{пределом справа}, функции~$f(x)$ \textbf{в~точке}~$a$, если
\begin{equation*}
\forall \varepsilon > 0 \ \exists \delta > 0 \colon (\forall x \colon 0 < x - a < \delta) \ |f(x) - b| < \varepsilon
\end{equation*}
и обозначается $\displaystyle \lim_{x \to a+0} f(x)$.

Т.\,о., $\displaystyle \lim_{x \to a} f(x) = b \Leftrightarrow \lim_{x \to a-0} f(x) = \lim_{x \to a+0} f(x) = b$.