\section{Неопределённый интеграл}
\index{Первообразная} Первообразной функции~$f(x)$ называется функция~$F(x) \colon F'(x) = f(x)$.

\begin{theorem}
Если $F'(x) = G'(x) = f(x)$, то $F(x) - G(x) = C$, где $C$~--- некоторая константа.
\end{theorem}
\begin{proof}
Пусть $H(x) = F(x) - G(x)$, тогда по \hyperref[th:mean_value]{теореме Лагранжа}
\begin{equation*}
H(b) - H(a) = H'(c)(b - a) = 0 \Rightarrow
H(b) - H(a) = (F'(c) - G'(c))(b - a) = 0 \Rightarrow
H(x) = C
\end{equation*}
\end{proof}

\index{Интеграл!неопределённый} Множество всех первообразных функции~$f(x)$ называется \textbf{неопределённым интегралом} и обозначается $\int f(x)\,dx = F(x) + C$, где $F(x)$~--- первообразная~$f(x)$, $C$~--- произвольная константа.
$f(x)$ называется \textbf{подынтегральной функцией}, а $f(x)\,dx$~--- \textbf{подынтегральным выражением}.