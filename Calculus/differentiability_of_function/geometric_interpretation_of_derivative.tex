\subsection{Геометрический смысл производной}
\begin{floatingfigure}[r]{43mm}
\noindent
$\begin{xy} /r8mm/:
(-0.5, 0); (4, 0) **@{-} *@{>} *++!U{x};
(0, -0.5); (0, 3) **@{-} *@{>} *++!R{y};
(-0.6, 0.5); (3, 3.1) **\crv{(-0.25, 1.5) & (1.75, 0.75) & (1.75, 3)} *!LU{\scriptstyle y = f(x)};
(-0.5, 0.12); (2.68, 3.3) **@{-};
(2.05, 2.67) *{\bullet} *+!DR{c};
\end{xy}$
\end{floatingfigure}
Пусть дана кривая, заданная уравнением~$y = f(x)$, $f(x)$ непрерывна на~$[a; b]$.
Проведём касательную к~этой кривой в~точке~$c \in (a; b)$.
Заметим, что касательная~--- это прямая, получающаяся в~пределе из~хорд, проходящих через~точки $(c, f(c))$ и~$(c + \Delta x, f(c + \Delta x))$.
Уравнение такой хорды имеет вид
\begin{equation*}
\frac{x - c}{(c + \Delta x) - c} = \frac{y - f(c)}{f(c + \Delta x) - f(c)} \Leftrightarrow
y = f(c) + \frac{f(c + \Delta x) - f(c)}{\Delta x} (x - c)
\end{equation*}

Переходя к~пределу при~$\Delta x \to 0$, получим значение углового коэффициента~$k$ касательной:
\begin{equation*}
k = \lim_{\Delta x \to 0} \frac{f(c + \Delta x) - f(c)}{\Delta x} = f'(c)
\end{equation*}

Т.\,о., $y = f(c) + f'(c)(x - c)$~--- уравнение касательной в~точке~$c$.

Существование касательной означает, что
\begin{equation*}
\displaystyle \exists \lim_{\Delta x \to 0} \frac{f(c + \Delta x) - f(c)}{\Delta x} = f'(c) \Rightarrow
f(c + \Delta x) - f(c) = f'(c) \Delta x + a\Delta x
\end{equation*}
где $\displaystyle \lim_{\Delta x \to 0} a = 0 \Rightarrow a\Delta x = o(\Delta x)$.
Т.\,о., существование касательной к графику функции~$f(x)$ в~точке~$c$ равносильно дифференцируемости функции~$f(x)$ в~точке~$c$.