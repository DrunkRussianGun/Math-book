\subsection{Правило Лопиталя}
\begin{theorem}[правило Лопиталя]
Если
\begin{enumerate}
	\item $\displaystyle \lim_{x \to a} f(x) = \lim_{x \to a} g(x) = 0$ или $\displaystyle \lim_{x \to a} f(x) = \lim_{x \to a} g(x) = \infty$
	\item $g'(x) \neq 0$
	\item $\displaystyle \exists \lim_{x \to a} \dfrac{f'(x)}{g'(x)}$	
\end{enumerate}
то $\displaystyle \lim_{x \to a} \dfrac{f(x)}{g(x)} = \lim_{x \to a} \dfrac{f'(x)}{g'(x)}$.
\end{theorem}

Если $f(x)$ и $g(x)$ непрерывны в~окрестности точки~$a$, то для случая $\displaystyle \lim_{x \to a} f(x) = \lim_{x \to a} g(x) = 0$ можно провести следующее доказательство:
\begin{equation*}
\lim_{x \to a} \frac{f(x)}{g(x)} =
\lim_{x \to a} \frac{f(x) - 0}{g(x) - 0} =
\lim_{x \to a} \frac{f(x) - f(a)}{g(x) - g(a)} \;
\left| \text{По \hyperref[th:Cauchy's_mean_value]{теореме Коши}} \right| =
\lim_{x \to a} \frac{f'(c)}{g'(c)} =
\end{equation*}
\begin{equation*}
\left| c \in (x; a) \text{ или } c \in (a; x) \right| =
\lim_{x \to a} \frac{f'(x)}{g'(x)}
\end{equation*}

Полное доказательство правила Лопиталя слишком сложно, поэтому здесь не приводится.