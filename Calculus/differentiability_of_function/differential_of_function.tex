\subsection{Дифференциал функции}
\hypertarget{def:differential_of_function}{} Пусть $f(x)$~--- функция, дифференцируемая в~точке~$x_0$, тогда \hyperlink{def:differentiability_of_function}{по~определению}
\begin{equation*}
f(x) - f(x_0) = f'(x_0)(x - x_0) + \alpha
\end{equation*}
\noindent где $\alpha = o(x - x_0)$ при $x \to x_0$.
Слагаемое~$f'(x_0)(x - x_0)$ представляет линейную часть приращения функции.
Его называют \textbf{дифференциалом функции~$f(x)$ в~точке~$x_0$} и~обозначают $df(x_0) = f'(x_0)dx$, где $dx = \Delta x$~--- приращение аргумента.

\begin{center}
\noindent
$\begin{xy} /r12mm/:
(-1, 0); (3, 0) **@{-} *@{>} *++!U{x};
(0, -0.5); (0, 3) **@{-} *@{>} *++!R{y};
(-1, 0.5); (1.5, 3) **\crv{(1.5, 0.5)} *+!R{\scriptstyle y = f(x)};
(-0.75, -0.5); (2.75, 3) **@{-}; % касательная
(0.9, 1.15) = "M"; (0.9, 0) **@{--} *++!U{x_0};
(1.7, 1.15); (0, 1.15) **@{--} *++!R{f(x_0)};
(1.49, 2.7) = "X"; (1.49, 0) **@{--} *++!U{x};
"X"; (0, 2.7) **@{--} *++!R{f(x)};
(1.2, 0.2) *{\Delta x};
(1.7, 1.74); (0, 1.74) **@{--} *++!R{f(x_0) + f'(x_0) \Delta x};
(1.6, 1.74); (1.6, 1.15) **{} *@{>}; % стрелка рядом с df(x_0)
(1.6, 1.74) **@{-} *@{>};
(2.1, 1.45) *{df(x_0)};
\end{xy}$
\end{center}

Можно записать производную, используя дифференциал: $\displaystyle f'(x) = \frac{df}{dx}$.