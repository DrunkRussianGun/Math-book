\section{Ограниченные подмножества множества \texorpdfstring{$\mathbb R$}{}}
\index{Множество!ограниченное}\index{Мажоранта} Множество~$X \subset \mathbb R$ называется \textbf{ограниченным сверху}, если $\exists a \in \mathbb R \colon \forall x \in X \ x \leqslant a$.
Число~$a$ называется \textbf{мажорантой множества~$X$}.

\index{Миноранта} Множество~$X \subset \mathbb R$ называется \textbf{ограниченным снизу}, если $\exists a \in \mathbb R \colon \forall x \in X \ a \leqslant x$.
Число~$a$ называется \textbf{минорантой множества~$X$}.

Множество, ограниченное и~сверху, и~снизу, называется \textbf{ограниченным}.

Мажоранта ограниченного сверху множества, принадлежащая ему, называется \textbf{его максимальным элементом}.
Миноранта ограниченного снизу множества, принадлежащая ему, называется \textbf{его минимальным элементом}.

Очевидно, что во~множестве может быть не более одного минимального элемента и не более одного максимального элемента.

\index{Супремум} Минимальный элемент множества мажорант ограниченного сверху множества~$A$ называется \textbf{супремумом} и обозначается $\sup A$.

\begin{statement}
\label{st:single_supremum}
Если множество~$A$ ограничено сверху, то $\exists! \sup A$.
\end{statement}
\begin{proof}
Пусть $B$~--- множество всех мажорант множества~$A$, тогда $\forall a \in A, \ b \in B \ a \leqslant b$.
По~\hyperlink{eq:continuity_axiom}{аксиоме непрерывности} $\exists c \in \mathbb R \colon \forall a \in A, \ b \in B \ a \leqslant c \leqslant b$, тогда $c$~--- минимальная мажоранта множества~$A$.

Единственность следует из единственности минимального элемента.
\end{proof}

\begin{statement}
\label{st:inequality_of_supremum}
Если $a = \sup A$, то
$\forall \varepsilon > 0 \ \exists x \in A \colon a - \varepsilon < x \leqslant a$.
\end{statement}
\begin{proofcontra}
Пусть $\exists \varepsilon_0 \colon \forall x \in A \ x \leqslant a - \varepsilon_0$.
Тогда $a - \varepsilon_0$~--- мажоранта множества~$A$, значит, $a \neq \sup A$.
Противоречие.
\end{proofcontra}

\index{Инфимум} Максимальный элемент множества минорант ограниченного снизу множества~$A$ называется \textbf{инфимумом} и обозначается $\inf A$.

\begin{statement}
Если множество~$A$ ограничено снизу, то $\exists! \inf A$.
\end{statement}%
Доказательство аналогично доказательству утверждения~\ref*{st:single_supremum}.

\begin{statement}
\label{st:inequality_of_infimum}
Если $a = \inf A$, то $\forall \varepsilon > 0 \ \exists x \in A \colon a \leqslant x < a + \varepsilon$.
\end{statement}%
Доказательство аналогично доказательству утверждения~\ref*{st:inequality_of_supremum}.

\begin{theorem}[принцип Архимеда]
Если $h > 0$, то
$\forall x \in \mathbb R \ \exists k \in \mathbb Z \colon (k - 1)h \leqslant x < kh$.
\end{theorem}
\begin{proof}
Рассмотрим множество~$A = \left\{ z \in \mathbb Z \mid z > \dfrac{x}h \right\}$, тогда $\exists a = \inf A$.
По утверждению~\ref*{st:inequality_of_infimum}
$\forall \varepsilon \in (0; 1] \ \exists z_0 \colon a \leqslant z_0 < a + \varepsilon$.
Т.\,к. в~промежутке~$[a; a + 1)$ лежит только одно целое число, то $a = z_0$, тогда $a - 1 \leqslant \dfrac{x}h < a$.
Т.\,о., $a$~--- искомое значение~$k$.
\end{proof}

Из принципа Архимеда следует, что не существует бесконечно больших чисел.

\begin{consequent}
\label{conseq:small_rational_exists}
\begin{equation*}
\forall \varepsilon > 0 \ \exists n \in \mathbb N \colon \frac1n < \varepsilon
\end{equation*}
\end{consequent}
\begin{proof}
По принципу Архимеда для $h = \varepsilon$, $x = 1$ получим:
\begin{equation*}
\forall \varepsilon > 0 \ \exists n \in \mathbb N \colon
(n - 1)\varepsilon \leqslant 1 < n\varepsilon \Leftrightarrow
(1 - \frac1n)\varepsilon \leqslant \frac1n < \varepsilon \Rightarrow
\frac1n < \varepsilon
\end{equation*}
\end{proof}

Отсюда следует, что не существует бесконечно малых чисел.

\begin{consequent}
\begin{equation*}
\forall a, b \in \mathbb R \ \exists c \in \mathbb Q \colon a < c < b
\end{equation*}
\end{consequent}
\begin{proof}
Из следствия~\ref*{conseq:small_rational_exists} для $\varepsilon = b - a$ получим $\exists n \in \mathbb N \colon \dfrac1n < b - a$.
По принципу Архимеда для $h = \dfrac1n$, $x = a$ получим:
\begin{equation*}
\exists k \in \mathbb Z \colon \frac{k - 1}n \leqslant a < \frac{k}n \Rightarrow
a < \frac{k}n = \frac{k - 1}n + \frac1n < a + (b - a) = b
\end{equation*}

Т.\,о., $\dfrac{k}n$~--- искомое значение~$c$.
\end{proof}

\index{Точка!предельная} Точка~$a \in \mathbb R$ называется \textbf{предельной точкой множества~$A \subset \mathbb R$}, если
$\forall \varepsilon > 0 \ \breve U_\varepsilon(a) \cap A \neq \varnothing$.

\index{Точка!дискретная} Точка~$a \in A$ называется \textbf{дискретной точкой множества~$A \subset \mathbb R$}, если
$\exists \varepsilon > 0 \colon \breve U_\varepsilon(a) \cap A = \varnothing$.

\index{Точка!внутренняя} Точка~$a \in A$ называется \textbf{внутренней точкой множества~$A \subset \mathbb R$}, если
$\exists \varepsilon > 0 \colon U_\varepsilon(a) \subset A$.

Множество называется \textbf{открытым}, если состоит только из внутренних точек.

Множество называется \textbf{замкнутым}, если его дополнение~$\overline A$ до $\mathbb R$ является открытым.

\begin{statement}
Множество~$A$ замкнуто $\Leftrightarrow$ оно содержит все свои предельные точки.
\end{statement}
\begin{proof}
\begin{enumerate}
	\item $\Rightarrow$. Докажем методом от противного, что $A$ содержит все свои предельные точки.
	Пусть $\exists a_0 \notin A$~--- предельная точка~$A$, тогда
	\begin{equation*}
	a_0 \in \overline A \Rightarrow
	\exists \varepsilon > 0 \colon U_\varepsilon(a_0) \subset \overline A \Rightarrow
	U_\varepsilon(a_0) \cap A = \varnothing
	\end{equation*}
	
	Значит, $a_0$ не является предельной точкой $A$.
	Противоречие.
	
	\item $\Leftarrow$. Докажем методом от противного, что $\overline A$ открыто.
	Пусть $\exists a \in \overline A \colon \forall \varepsilon > 0 \ U_\varepsilon(a) \cap A \neq \varnothing$, тогда $a \notin A$~--- предельная точка~$A$.
	Противоречие.
\end{enumerate}
\end{proof}

\index{Теорема!Вейерштрасса}
\begin{theorem}[Вейерштрасса]
\label{th:Weierstrass}
Если $A$~--- бесконечное ограниченное множество, то $\exists a \in \mathbb R$~--- предельная точка $A$.
\end{theorem}
\begin{proof}
$A \subseteq [a; b]$, где $a = \inf A$, $b = \sup A$.
Пусть $a$ не является предельной точкой $A$, т.\,е. $\exists \varepsilon_0 > 0 \colon \breve U_{\varepsilon_0}(a) \cap A = \varnothing$, тогда $a \in A$, значит, $a$~--- дискретная точка $A$.

Рассмотрим множество~$B$ точек~$y$ таких, что интервал~$(-\infty; y)$ содержит конечное число точек $A$.
Интервал~$(-\infty; a + \varepsilon_0)$ содержит только одну точку множества~$A$~--- $a$, значит, $\forall k \in (0; 1] \ a + k\varepsilon_0 \in B$.

$A \subset (-\infty; b]$, тогда $b$~--- мажоранта $B$, значит, $\exists c = \sup B$.
\begin{enumerate}
	\item $\forall \varepsilon > 0 \ (-\infty; c - \varepsilon)$ содержит конечное число точек множества~$A$.
	\item $\forall \varepsilon > 0 \ (-\infty; c + \varepsilon)$ содержит бесконечное число точек множества~$A$, т.\,к. $c + \varepsilon \notin B$.
\end{enumerate}

Тогда $\forall \varepsilon > 0 \ \breve U_\varepsilon(c)$ содержит бесконечное число точек множества~$A$, значит, $c$~--- предельная точка множества~$A$.
\end{proof}