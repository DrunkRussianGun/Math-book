\subsection{Свойства непрерывных функций}
\begin{statement}
Если функция~$f(x)$ непрерывна в~точке~$a$, $g(x)$~--- в~точке~$f(a)$, то $h(x) = g(f(x))$ непрерывна в~точке~$a$.
\end{statement}
\begin{proof}
\begin{equation*}
\lim_{x \to a} h(x) = \lim_{x \to a} g(f(x)) = g(\lim_{x \to a} f(x)) = g(f(a))
\end{equation*}
\end{proof}

Пусть функция~$f(x)$ непрерывна на~отрезке~$[a; b]$.
\begin{enumerate}
	\item $f(x)$ ограничена на~$[a; b]$.
	\begin{proofcontra}
	Пусть $\forall n \in \mathbb N \ \exists x_n \in [a; b] \colon |f(x_n)| > n$.
	Получили ограниченную последовательность~$(x_n)$.
	Выберем из~неё сходящуюся подпоследовательность~$\displaystyle (x_{n_k}) \colon \lim_{k \to \infty} x_{n_k} = x_0$.
	Точка~$x_0$ предельная для~$[a; b]$, значит, $x_0 \in [a; b]$, т.\,к. $[a; b]$~--- замкнутое множество.
	Тогда в~силу непрерывности~$f(x)$ $\displaystyle \lim_{k \to \infty} f(x_{n_k}) = f(x_0)$.
	Это противоречит тому, что $|f(x_{n_k})| > n_k$, т.\,к. $\displaystyle \lim_{k \to \infty} n_k = +\infty$.
	\end{proofcontra}
	
	\item\label{st:continuous_function_takes_inf_and_sup} Если $\displaystyle m = \inf_{x \in [a; b]} f(x)$, $\displaystyle M = \sup_{x \in [a; b]} f(x)$, то $\exists x_m, x_M \in [a; b] \colon f(x_m) = m, \ f(x_M) = M$.
	\begin{proof}
	По~утверждению~\ref{st:inequality_of_infimum} $\forall n \in \mathbb N \ \exists x_n \in [a; b] \colon m \leqslant x_n < m + \dfrac1n$.
	Получили ограниченную последовательность~$(x_n)$.
	Выберем из~неё сходящуюся подпоследовательность~$\displaystyle (x_{n_k}) \colon \lim_{k \to \infty} x_{n_k} = x_m$.
	Точка~$x_m$ предельная для~$[a; b]$, значит, $x_m \in [a; b]$, т.\,к. $[a; b]$~--- замкнутое множество.
	Тогда в~силу непрерывности~$f(x)$ $\displaystyle \lim_{k \to \infty} f(x_{n_k}) = f(x_m), \ m \leqslant x_{n_k} < m + \frac1{n_k} \Rightarrow f(x_m) = m$.
	
	$f(x_M) = M$ доказывается аналогично.
	\end{proof}
	
	\item \begin{theorem}[о нуле непрерывной функции]
	\label{th:zero_of_continuous_function}
	Если $f(a) \cdot f(b) < 0$, то $\exists x_0 \in [a; b] \colon f(x_0) = 0$.
	\end{theorem}
	\begin{proof}
	Разделим отрезок~$[a; b]$ пополам.
	Если $f\left( \dfrac{a + b}2 \right) = 0$, то $x_0 = \dfrac{a + b}2$.
	Иначе выберём ту половину отрезка~$[a; b]$, на~границах которой функция принимает разные знаки.
	Разделим её пополам и~проверим значение в~середине.
	Продолжая таким образом, получим либо $0$ в~одной из~середин полученных отрезков, либо последовательность вложенных отрезков~$\displaystyle [a_n; b_n] \colon \lim_{n \to \infty} (b_n - a_n) = 0, \ f(a_n) \cdot f(b_n) < 0$.
	По~\hyperref[lemma:about_nested_intervals]{лемме о~вложенных отрезках} $\displaystyle \lim_{n \to \infty} a_n = \lim_{n \to \infty} b_n = c \in [a; b]$.
	В~силу непрерывности~$f(x)$ $\displaystyle \lim_{n \to \infty} f(a_n) = \lim_{n \to \infty} f(b_n) = f(c) = 0$, т.\,к. $f(a_n)$ и~$f(b_n)$ имеют разные знаки.
	\end{proof}
	
	\item \begin{theorem}[о промежуточном значении]
	Если $f(a) \neq f(b)$, то, без ограничения общности полагая $f(a) < f(b)$, $[f(a); f(b)] \subseteq E(f)$.
	\end{theorem}
	\begin{proof}
	Пусть $C \in (f(a); f(b))$, $g(x) = f(x) - C$.
	$g(a) = f(a) - C < 0$, $g(b) = f(b) - C > 0$, тогда по \hyperref[th:zero_of_continuous_function]{теореме о~нуле непрерывной функции} $\exists c \in [a; b] \colon (g(c) = 0 \Leftrightarrow f(c) = C)$.
	\end{proof}
	
	\item Если $\displaystyle m = \inf_{x \in [a; b]} f(x)$, $\displaystyle M = \sup_{x \in [a; b]} f(x)$, то $[m; M] \subseteq E(f)$.
\end{enumerate}

Пусть $f(x)$ и~$g(x)$ непрерывны на~$[a; b]$.
\begin{enumerate}
	\item $h(x) = f(x) + g(x)$ непрерывна на~$[a; b]$.
\end{enumerate}