\subsection{Локальный экстремум функции}
\begin{theorem}
\label{th:criterion_of_monotonic_function}
Если функция~$f(x)$ дифференцируема на~$(a; b)$, то она не убывает (не возрастает) на~$(a; b)$ $\Leftrightarrow$ $\forall x \in (a; b) \ f'(x) \geqslant 0 \ (f'(x) \leqslant 0)$.
\end{theorem}
\begin{proof}
\begin{enumerate}
	\item $\Rightarrow$. Пусть $f(x)$ не убывает на~$(a; b)$, $x_1, x_2 \in (a; b)$.
	\begin{equation*}
	\frac{f(x_2) - f(x_1)}{x_2 - x_1} \geqslant 0 \Rightarrow
	\lim_{x_2 \to x_1} \frac{f(x_2) - f(x_1)}{x_2 - x_1} \geqslant 0 \Leftrightarrow
	f'(x_1) \geqslant 0
	\end{equation*}
	
	Доказательство в~случае невозрастания $f(x)$ аналогично.
	
	\item $\Leftarrow$. Пусть $\forall x \in (a; b) \ f'(x) \geqslant 0$, $a < x_1 < x_2 < b$.
	По \hyperref[th:mean_value]{теореме Лагранжа}
	\begin{equation*}
	\exists x_3 \in (x_1; x_2) \colon f(x_2) - f(x_1) = f'(x_3)(x_2 - x_1)
	\end{equation*}
	
	$f'(x_3)(x_2 - x_1) \geqslant 0 \Leftrightarrow
	f(x_2) - f(x_1) \geqslant 0 \Leftrightarrow
	f(x)$ не убывает на~$(a; b)$.
	
	Доказательство в~случае $f'(x) \leqslant 0$ аналогично.
\end{enumerate}
\end{proof}

\index{Максимум!функции} Точка~$x_0$ называется \textbf{точкой локального минимума функции}~$f(x)$, если существует проколотая окрестность~$\breve U(x_0) \colon \forall x \in \breve U(x_0) \ f(x) > f(x_0)$.

\index{Минимум!функции} Точка~$x_0$ называется \textbf{точкой локального максимума функции}~$f(x)$, если существует проколотая окрестность~$\breve U(x_0) \colon \forall x \in \breve U(x_0) \ f(x) < f(x_0)$.

\index{Экстремум} Точки локального минимума и максимума называются \textbf{точками локального экстремума}.

\begin{theorem}
Если $x_0$~--- точка локального экстремума функции~$f(x)$, то $\raise 0.4mm\hbox{\lefteqn{\mkern -2mu\not}} \exists f'(x_0)$ или $f'(x_0) = 0$.
\end{theorem}
\begin{proof}
Пусть $x_0$~--- точка локального минимума, $\exists f'(x_0)$, тогда
\begin{equation*}
\begin{cases}
\displaystyle \frac{f(x) - f(x_0)}{x - x_0} < 0, \ x < x_0 \\
\displaystyle \frac{f(x) - f(x_0)}{x - x_0} > 0, \ x > x_0
\end{cases} \Rightarrow
\begin{cases}
\displaystyle \lim_{x \to x_0-0} \frac{f(x) - f(x_0)}{x - x_0} \leqslant 0 \\
\displaystyle \lim_{x \to x_0+0} \frac{f(x) - f(x_0)}{x - x_0} \geqslant 0
\end{cases} \Rightarrow
\lim_{x \to x_0} \frac{f(x) - f(x_0)}{x - x_0} = 0 \Leftrightarrow
f'(x_0) = 0
\end{equation*}

Доказательство для локального максимума аналогично.
\end{proof}

\index{Точка!критическая} Точка, в~которой производная функции не существует или равна нулю, называется \textbf{критической}.

Существуют следующие признаки локального экстремума:
\begin{enumerate}
	\item Если $f_-'(x_0) \leqslant 0 \ (\geqslant 0)$, $f_+'(x_0) \geqslant 0 \ (\leqslant 0)$, то $x_0$~--- точка локального минимума (максимума).
	\begin{proof}
	Пусть
	\begin{equation*}
	f_-'(x_0) \leqslant 0, \ f_+'(x_0) \geqslant 0 \Leftrightarrow
	\lim_{x \to x_0-0} \frac{f(x) - f(x_0)}{x - x_0} \leqslant 0, \
	\lim_{x \to x_0+0} \frac{f(x) - f(x_0)}{x - x_0} \geqslant 0
	\end{equation*}
	
	Значит, в некоторой окрестности точки~$x_0$ $f(x) \geqslant f(x_0)$, тогда $x_0$~--- точка локального минимума.
	
	Аналогичное доказательство для максимума.
	\end{proof}
	
	\item Если $f'(x_0) = f''(x_0) = \ldots = f^{(2n-1)}(x_0) = 0$, то
	\begin{itemize}
		\item $x_0$~--- точка локального максимума при~$f^{(2n)}(x_0) < 0$;
		\item $x_0$~--- точка локального минимума при~$f^{(2n)}(x_0) > 0$;
		\item $x_0$ не является точкой локального экстремума при~$f^{(2n)}(x_0) = 0$, $f^{(2n+1)}(x_0) \neq 0$.
	\end{itemize}
	\begin{proof}
	\begin{itemize}
		\item Пусть $f^{(2n)}(x_0) < 0$.
		По \hyperref[eq:Taylor_series]{формуле Тейлора}
		\begin{equation*}
		f(x) = f(x_0) + f'(x_0)(x - x_0) + \frac{f''(x_0)}{2!} (x - x_0)^2 + \ldots + \frac{f^{(2n)}(x_0)}{(2n)!} (x - x_0)^{2n} + o((x - x_0)^{2n}) \Leftrightarrow
		\end{equation*}
		\begin{equation*}
		\Leftrightarrow f(x) - f(x_0) = \frac{f^{(2n)}(x_0)}{(2n)!} (x - x_0)^{2n} + o((x - x_0)^{2n}) < 0
		\end{equation*}
		
		Тогда $x_0$~--- точка локального максимума.
		
		\item Случай при~$f^{(2n)}(x_0) > 0$ доказывается аналогично.
		
		\item Пусть $f^{(2n)}(x_0) = 0$, $f^{(2n+1)}(x_0) \neq 0$.
		По \hyperref[eq:Taylor_series]{формуле Тейлора}
		\begin{equation*}
		f(x) = f(x_0) + f'(x_0)(x - x_0) + \frac{f''(x_0)}{2!} (x - x_0)^2 + \ldots +
		\end{equation*}
		\begin{equation*}
		\vphantom1 + \frac{f^{(2n)}(x_0)}{(2n)!} (x - x_0)^{2n} + \frac{f^{(2n+1)}(x_0)}{(2n+1)!} (x - x_0)^{2n+1} + o((x - x_0)^{2n+1}) \Leftrightarrow
		\end{equation*}
		\begin{equation*}
		\Leftrightarrow f(x) - f(x_0) = \frac{f^{(2n+1)}(x_0)}{(2n+1)!} (x - x_0)^{2n+1} + o((x - x_0)^{2n+1})
		\end{equation*}
		
		Знак $f(x) - f(x_0)$ зависит от знака $x - x_0$, поэтому в точке~$x_0$ не может быть локального экстремума.
	\end{itemize}
	\end{proof}
\end{enumerate}