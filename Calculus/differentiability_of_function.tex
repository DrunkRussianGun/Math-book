\section{Дифференцируемость функции}
Условие непрерывности функции~$f(x)$ в~точке~$a$ можно сформулировать так:
\begin{equation*}
\lim_{\Delta x \to 0} \Delta f = \lim_{\Delta x \to 0} (f(a + \Delta x) - f(a)) = 0
\end{equation*}
где $\Delta x = x - a$ называется \textbf{приращением аргумента}, $\Delta f = f(x) - f(a)$~--- \textbf{приращением функции}.

\hypertarget{def:differentiability_of_function}{}\hypertarget{def:derivative_of_function}{} Функция~$f(x)$ называется \textbf{дифференцируемой в~точке}~$a$, если $\Delta f = k \Delta x + o(\Delta x)$ при~$\Delta x \to 0$, где $\Delta f = f(x) - f(a)$, $\Delta x = x - a$, $k$~--- константа, называемая \textbf{производной} функции~$f(x)$ \textbf{в~точке}~$a$ и~обозначаемая $f'(a)$.

Из~определения следует, что функция, дифференцируемая в~точке~$a$, непрерывна в~ней.

Функция называется \textbf{дифференцируемой~на} некотором \textbf{множестве}, если она дифференцируема в~каждой точке этого множества.

\hypertarget{def:smoothness_of_function}{} Точки, в~которых функция дифференцируема, называются \textbf{точками гладкости}.
Функция называется \textbf{гладкой}, если она дифференцируема на~всей области определения.

Найдём производную функции~$f(x)$ в~точке~$a$.
\begin{equation*}
f'(a) = \lim_{\Delta x \to 0} \frac{\Delta f}{\Delta x} = \lim_{x \to a} \frac{f(x) - f(a)}{x - a}
\end{equation*}

Т.\,о., производная функции~$f(x)$ является функцией~$f'(x)$.