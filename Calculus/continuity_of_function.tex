\section{Непрерывность функции}
\index{Функция!одной переменной!непрерывная} Пусть функция~$f$ задана на множестве~$D \subseteq \mathbb R$ и $a \in D$.
$f$ называется \textbf{непрерывной в точке~$a$}, если $\displaystyle \lim_{x \to a} f(x) = f(\lim_{x \to a} x)$, что эквивалентно $\displaystyle \lim_{x \to a} f(x) = f(a)$.

\index{Точка!разрыва} \textbf{Точкой разрыва первого рода функции~$f(x)$} называется точка~$a$, в~которой и~левый, и~правый пределы функции~$f(x)$ конечны, причём $f(x)$ не является непрерывной в точке~$a$.

\textbf{Точкой разрыва второго рода функции~$f(x)$} называется предельная точка~$a$ множества~$D(f)$, в~которой левый или правый предел функции~$f(x)$ не~существует или бесконечен.

Функция называется \textbf{непрерывной на} некотором \textbf{множестве}, если она непрерывна в~каждой точке этого множества.